\documentclass{article}

\usepackage{hyperref}

\begin{document}

\begin{abstract}
In this short note we shall describe a curious connection
between non-standard arithmetics and weak equivalences
in higher categories, with special focus on the category of
opetopes. At the heart of the correspondence lies the
connection between opetopic composition and non-standard
multiplication. Weak equivalences of the same composites are
witnessed by universal cells. Universality between two
composite cells is then observed by a vanishing standard
part of the commutator applied to the corresponding formulas.
The presence of a non-vanishing error term (non-standard part)
is the consequence of the broken referential transparency of
operadic composition.
While the connection between operad algebras and
lie algebras appears to be well-known, this particular
case does not appear to be discussed.
\end{abstract}

\section{Introduction}
A very superfical introduction to non-standard analysis
and to weak equivalences in higher categories is due.

\subsection{Non-standard analysis}

\url{http://en.wikipedia.org/wiki/Non-standard_calculus}

Let $\alpha$ be a value $\neq 0$ with $\alpha^2 = 0$. Then
our non-standard reals are a two-dimensional vector space
inhabited by pairs $(s, i\alpha)$ with $s, i \from R$.

We can recover the \emph{standard part} by applying the function
... 
and the \emph{error term} as $nst (s, i\alpha) = v - st \over \alpha$

We also establish a multiplication on these vectors, making
the vector space to a \emph{ring}.

\subsection{Universality in higher categories}

\section{Structure}
This section's content...

\subsection{Top Matter}
This subsection's content...

\subsubsection{Article Information}
This subsubsection's content...


transfer principle in non-standard analysis vs. homotopy theory.

\url{http://en.wikipedia.org/wiki/Transfer_principle}

\end{document}
