\documentclass{article}

\usepackage{hyperref}

\begin{document}

\newcommand{\compose}[3]{$#1{;_#2}#3$}
\newcommand{\nsproduct}[3]{$(#1 + #2)(#3 + #2)$}
\newcommand{\lie}[4]{$#1{_{[#3,#2]}}#4$}

\begin{abstract}
In this short note we shall describe a curious connection
between non-standard arithmetics and weak equivalences
in higher categories, with special focus on the category of
opetopes. At the heart of the correspondence lies the
connection between opetopic composition and non-standard
multiplication. Weak equivalences of the same composites are
witnessed by universal cells. Universality between two
composite cells is then observed by a vanishing standard
part of the commutator applied to the corresponding formulas.
The presence of a non-vanishing error term (non-standard part)
is the consequence of the broken referential transparency of
operadic composition.
While the connection between operad algebras and
lie algebras appears to be well-known, this particular
case does not appear to be discussed.
\end{abstract}

\section{Introduction}
A very superfical introduction to non-standard analysis
and to weak equivalences in higher categories is due.

\subsection{Non-standard analysis}

\url{http://en.wikipedia.org/wiki/Non-standard_calculus}

Let $\alpha$ be a value $\neq 0$ with $\alpha^2 = 0$. Then
our non-standard reals are a two-dimensional vector space
inhabited by pairs $(s, i \alpha)$ with $s, i \in R$.

We can recover the \emph{standard part} by applying the function
... 
and the \emph{error term} as $nst (s, i\alpha) = {v - st \over \alpha}$

We also establish a multiplication on these vectors, making
the vector space to a \emph{ring}.

\subsection{Universality in higher categories}

Composition in weak higher categories is only defined up to
isomorphism. This includes identity morphisms, which can be
seen as empty compositions. Fortunately where composition
is defined a higher cell is available to witness the isomorphism.

We shall write $g \circ f$ and $f;g$ interchangably to stand for
composition of diagrams.

The strict equality between syntactically equal terms $f;g$ and $f;g$
is rooted in the broken referential transparency of the monadic
\emph{bind} operation. This fault is carried over into the operadic
composition and thus appears in the higher multicategory of (pasting)
diagrams. To make the context-sensitive nature of composition explicit
we shall from now on subscript $\circ$ and $;$ like this: ${\circ_\alpha}$
and ${;_\beta}$.

\section{Structure}
This section's content...

\section{The Correspondence}

Our idea is to map a context-annotated diagram composition \compose{f}{\alpha}{g}
to the product formula \nsproduct{f}{\alpha}{g} where $\alpha$ is interpreted as
an infinitesimal. By multiplication we obtain $fg+\alpha(f+g)$. Obviously the
standard part is $fg$, encoding the composition, while the error term $\alpha(f+g)$
absorbs the context dependence.

We are, however, more interested in the universal cell that can be found between
 \compose{f}{\alpha}{g} and \compose{f}{\beta}{g}. To get a hold on it, we form the
commutator, as the distance between \compose{f}{\alpha}{g} and \compose{f}{\beta}{g}:
\nsproduct{f}{\beta}{g}-\nsproduct{f}{\alpha}{g}. This simplifies to $(\beta-\alpha)(f+g)$.
\footnote{We note that this is the lie-bracket $[f+(\beta-\alpha), g+(\beta-\alpha)]$. CHECK THIS!}

Examining this latter term we observe that it is a product of a distance of infinitesimals
and a sum of the standard number $f+g$. The first factor entitles us to regard this as a
universal cell. By abuse of the lie-bracket notation we abbreviate this cell as
\lie{f}{\alpha}{\beta}{g}.

\subsection{Inverse}

Obviously we expect the universal cell to have an inverse (up to a higher universal cell).



\subsubsection{Article Information}
This subsubsection's content...


transfer principle in non-standard analysis vs. homotopy theory.

\url{http://en.wikipedia.org/wiki/Transfer_principle}

\end{document}
