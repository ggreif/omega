\documentclass[11pt,twoside,A4]{llncs}
\usepackage{makeidx,multicol}
\include{psinput}
\include{psfig}
\usepackage{epsf}
\usepackage{epsfig}
\usepackage{graphics,color}
\usepackage{amssymb}
\usepackage{array}
\usepackage{mathpartir}

%\DeclareGraphicsRule{*}{mps}{*}{}

\newcommand{\om}{\emph{$\Omega$}mega}
\newcommand{\plus}[2]{\{plus {#1} {#2}\}}
\newcommand{\plusH}[2]{\begin{tabular}[t]{ll}
                       \{plus \hspace*{-0.1in} &{#1} \\ 
                              & {#2} \}
                      \end{tabular}}
\newcommand{\hide}[1]{}
%\newtheorem{exercise}{Exercise}

\newcommand{\deffbox}[8]{{\begin{tabular}{|l|rcl|} \hline
{\small expected type} & {\small{\tt #1}} & $\rightarrow$ & {\small{\tt #2}} \\ \hline
{\small equation} & {\small{\tt #3}} & =             & {\small{\tt #4}} \\ \hline
{\small computed type} & {\small{\tt #5}} & $\rightarrow$ & {\small{\tt #6}} \\ \hline
{\small equalities}    & {\small{\tt #7}} & $\Rightarrow$ & {\small{\tt #8}} \\ \hline 
\end{tabular}}}

\newcommand{\deffboxShort}[8]{{\begin{tabular}{|l|rcl|} \hline
{\small \begin{tabular}{l} expected \\ type \end{tabular}} & {\small{\tt #1}} & $\rightarrow$ & {\small{\tt #2}} \\ \hline
{\small equation} & {\small{\tt #3}} & =             & {\small{\tt #4}} \\ \hline
{\small \begin{tabular}{l} computed \\ type \end{tabular}} & {\small{\tt #5}} & $\rightarrow$ & {\small{\tt #6}} \\ \hline
{\small equalities}    & {\small{\tt #7}} & $\Rightarrow$ & {\small{\tt #8}} \\ \hline 
\end{tabular}}}

\newcommand{\zero}{0t}
\newcommand{\one}{1t}
\newcommand{\two}{2t}



%\setlength{\textheight}{9.5in}
%\setlength{\textwidth}{6.4in}
%\setlength{\oddsidemargin}{-.2in}
%\setlength{\evensidemargin}{-.2in}
%\setlength{\topmargin}{-1in}


%\setcounter   {topnumber}{3}
%\renewcommand {\topfraction}{1.0}
%\setcounter   {totalnumber}{4}
%\renewcommand {\textfraction}{0.2}
%\renewcommand {\floatpagefraction}{0.99} 
%\renewcommand {\floatsep}{0.1}
 


%\setlength{\linewidth}{0.2cm}

\begin{document}


\title{Programming in \om.}
\author{Tim Sheard \& Nathan Linger}
\institute{Computer Science Department\\
Maseeh College of Engineering and Computer Science\\
Portland State University}{}

%%%%%%%%%%%%%%%%%%%%%%%%%%%%%%%%%%%%%%%%%%


\maketitle

%\tableofcontents

\begin{abstract}
This report was originally prepared as notes for
a short course on \om\ taught at the 
Central-European Functional Programming School held in
Cluj, Romania, between 25-30 June, 2007. It can be viewed
as a tutorial on the use of the \om\ programming language.


\vspace*{0.1in}
It introduces readers to the {\em types as propositions}
notion based upon the Curry-Howard isomorphism.
Such types can express precise
properties of programs. The \om\ language allows us to use a single
language for the specification of designs, the definition of properties,
the implementation of programs, and the production of proofs that
programs adhere to their properties. \om\ bundles all these in a coherent
manner into a single unified system that appears to the user to be a
programming language.

\end{abstract}

\section{Introduction}\label{intro}

\om\ is a language with an infinite hierarchy of computational levels: value,
type, kind, sort, etc. Data, and functions manipulating data, can be introduced
at any level. Data is introduced by declaring the type of constructors, and
functions are introduced by writing (possibly recursive) pattern matching
equations.

Terms at each level are classified by terms at the next level. Thus values are
classified by types, types are classified by kinds, kinds are classified by
sorts, etc. As discussed earlier, programmers are allowed to introduce new terms
and functions at every level, but any particular program will have terms at only
a finite number of levels. We illustrate the level hierarchy for the many of the
examples given in this paper in Figure \ref{hierarchy}.

We maintain a strict phase distinction ---
the classification of a term at level $n$ cannot depend upon terms at
lower levels. For example, no types can depend on values, and no kinds
can depend on types. 
We formalize properties of programs by exploiting the Curry-Howard
isomorphism. Terms at computational level $n$, are used
as proofs about terms at level $n+1$. We use indexed types
to maintain a strict and formal connection between the two levels,
and singleton types to maintain the strict separation between
values and types.


\section{A simple example} \label{simple}

To illustrate the hierarchy of computational levels we give the
following two-level example which uses natural numbers as a type
index to lists that record their length in their type.

First, we introduce tree-like data (the natural
numbers, {\tt Nat}) at the {\em type level} by using the {\tt data}
introduction form. This form is a generalization
over the \verb+data+ declaration in Haskell~\cite{peyton-jones:haskell98}. 


{\small
\begin{verbatim}
data Nat :: *1 where
  Z :: Nat
  S :: Nat ~> Nat
\end{verbatim}}

\noindent
The line ``\verb+data Nat :: *1 where+" indicates that
\verb+Nat+ is classified by \verb+*1+ (rather than \verb+*0+), which tells
the programmer that \verb+Nat+ is a kind (rather than a type), and 
that \verb+Z+ and \verb+S+ are types (rather than values) that are
classified as indicated.  Think of the operator \verb+~>+ as the
operator that classifies functions at the type level. I.e.
it is similar in use to the operator \verb+->+, but used on kinds rather
than types. Thus, \verb+S :: Nat ~> Nat+ indicates
a type constructor that takes a \verb+Nat+ as input and produces
a \verb+Nat+ as output.

The classifiers
\verb+*0+, \verb+*1+, \verb+*2+, etc. indicate the level of a term.
All values are classified by types that are classified by \verb+*0+. All types
are classified by kinds that are classified by \verb+*1+. All kinds are
classified by sorts that are classified by \verb+*2+, etc. This is illustrated
with great detail in Figure \ref{hierarchy}.

Second, we write a function at the type level over this data
({\tt plus}).  At the type level and higher, we distinguish
function application from constructor application by surrounding
function application by braces (\verb+{+ and \verb+}+). For example,
we write \verb+S x+ for constructor application,
and \verb+{plus x y}+ for function application.


{\small
\begin{verbatim}
plus:: Nat ~> Nat ~> Nat
{plus Z m} = m
{plus (S n) m} = S {plus n m}
\end{verbatim}}


Third, using the {\tt data} introduction form at the {\em value level}, we introduce
the algebraic data structure ({\tt Seq}). The types of such values are indexed
by the natural numbers.  These indexes describe an invariant about the
constructed values --- their length appears in their type --- consider the type
of {\tt l1} below. 

{\small
\begin{verbatim}
data Seq:: *0 ~> Nat ~> *0 where
  Snil :: Seq a Z
  Scons:: a -> Seq a n -> Seq a (S n)
  
l1 = (Scons 3 (Scons 5 Snil)) :: Seq Int (S(S Z))
\end{verbatim}}

Finally, we introduce an append function at the value level over {\tt Seq}
values ({\tt app}). The type of {\tt app} describes one of its important
properties --- there is a functional relationship between the lengths of its two
inputs, and the length of its output.

{\small
\begin{verbatim}
app:: Seq a n -> Seq a m -> Seq a {plus n m}
app Snil ys = ys
app (Scons x xs) ys = Scons x (app xs ys)
\end{verbatim}}

\noindent
To see that the {\tt app} is well typed, the type checker does the following.
The expected type is the type given in the function prototype. We
compute the type of both the left- and right-hand-side of the equation
defining a clause. We compare the expected type with the computed type
for both the left- and right-hand-sides. This comparison generates
some necessary equalities (for each side) to make the expected and computed
types equal. We assume the left-hand-side
equalities to prove the right-hand-side equalities. To see this in
action, consider the second clause of the definition of \verb+app+.

\vspace*{.1in}
{\tt
\begin{tabular}{|l|rlclcl|} \hline
{\small {\normalfont expected type}} & &{\small{{Seq a n}}} & $\rightarrow$ & {\small{{Seq a m}}} & $\rightarrow$ &  {\small{{Seq a \plus{n}{m}}}}\\ \hline
{\small {\normalfont equation}}      & {\small{app}}&  {\small{{(Scons x xs)}}} & &  {\small{{ys}}} & = & {\small{{Scons x (app xs ys)}}} \\ \hline
{\small {\normalfont computed type}} & & {\small{{Seq a (S b)}}} & $\rightarrow$ & {\small{{Seq a m}}} & $\rightarrow$ & {\small{{Seq a (S \plus{b}{m})}}}  \\ \hline
{\small {\normalfont equalities}}    & & \multicolumn{3}{r}{\small{{n = (S b)}}} & $\Rightarrow$ & {\small{{\plus{n}{m}= S\plus{b}{m}}}} \\ \hline 
\end{tabular}}
\vspace*{.1in}

\noindent
The expected types are taken from the type declaration accompanying the
function definition. The computed type is computed\footnote{Using an inference
algorithm based upon algorithm-W} from the known types of the constructors and functions in the
definition. The equalities are generasted by equating the expected type
and the computed type. The left-hand-side equalities (to the left of the
$\Rightarrow$) let us assume
\verb+n+ = \verb+S b+. The right-hand-side equalities, require us to
establish that \verb+{plus n m}+ = \verb+S{plus b m}+.   Using the
assumption that \verb+n+ = \verb+S b+, we are left with the requirement
that \verb+{plus (S b) m}+ = \verb+S{plus b m}+, which is easy to prove
using the definition of \verb+plus+.
 
The different levels of the objects introduced in this example (and elsewhere
in the paper) are plotted in Figure \ref{hierarchy}. The reader may wish
to consult the figure to help visualize the relationships involved.

\begin{exercise} \label{lengthEx}

Write an \om\ function that defines the length function over sequences.
\verb+length:: Seq a n -> Int+. You will need to create a file, and paste the
definition for {\tt Seq} into the file, as well as write the length function.
The {\tt Nat} kind is predefined. You will need to include the function
prototype, above, in your file (type inference is limited in \om). How might we
reflect the fact that the resulting {\tt Int} should have size {\tt n}? See
Section \ref{nat'}.

\end{exercise}

\begin{exercise}
After you complete Exercise \ref{lengthEx}, create a table,
as we did for {\tt app} above, with expected type, equations, computed type,
and equations to be discharged. How might we solve the equations produced?
\end{exercise}

\renewcommand{\textfraction}{0.1}
\renewcommand{\topfraction}{.80}
 
\begin{figure}[t]
{\tt
\begin{tabular}{|lclclcl|} \hline
  $\leftarrow$     {\it {\tiny value name space}}&|& {\it {\tiny type name space $\rightarrow$}} & &  && \\ \hline
{\em value}&|& {\em type} &|&  {\em kind} &|& {\em sort} \\ \hline
     & |& Tree        &::& *0 \verb+~>+ *0 &::& *1 \\
Fork &::& Tree a -> Tree a -> Tree a &::& *0       &::& *1 \\
Node &::& a -> Tree a &::& *0       &::& *1 \\
Tip  &::&      Tree a &::& *0       &::& *1 \\ \hline
%     &  &          & |& Nat        &::& *1 \\      
     & |& Z        &::& Nat &::& *1 \\
     & |& S        &::& Nat \verb+~>+ Nat &::& *1 \\   \hline  
     & |& plus     &::& Nat \verb+~>+ Nat \verb+~>+ Nat &::& *1 \\    
     & |& \{plus 1t 3t \} &::& Nat &::& *1 \\    \hline
     &|& Seq       &::& *0 \verb+~>+ Nat \verb+~>+ *0 &::& *1 \\ 
Snil  &::& Seq a Z &::& *0 &::& *1 \\
Scons &::& a -> Seq a b -> Seq a (S b) &::& *0 &::& *1 \\
app   &::& Seq a n -> Seq a m -> && && \\
      &  & Seq a \{plus n m\} &::&*0  &::&  *1\\   \hline  
%      &  &                   & |& TempUnit &::&  *1 \\           
      & |& Tp        &::& Shape &::& *1 \\     
      & |& Nd        &::& Shape &::& *1 \\     
      & |& Fk        &::& Shape &::& *1 \\   \hline       

      & |& Tree   &::& Shape \verb+~>+ *0  \verb+~>+ *0  &::& *1\\      
Tip &::& Tree Tp a  &::& *0 &::& *1\\  
Node &::& a -> Tree Nd a  &::& *0 &::& *1\\ 
Fork &::& Tree x a -> Tree y a -> && && \\
     &  & Tree (Fk x y) a  &::& *0 &::& *1\\ 
find &::& (a -> a -> Bool) -> a -> &&  && \\
     &  & Tree sh a -> [Path sh a] &::& *0 &::& *1 \\ \hline
%      &  &          & |& Boolean &::&  *1 \\
      & |& T        &::& Boolean &::& *1 \\  
      & |& F        &::& Boolean &::& *1 \\ 
    &|& le  &::& Nat \verb+~>+ Nat ~> Boolean &::& *1 \\
    &|& \{le 0t 2t\}   &::& Boolean &::& *1 \\ \hline
    &|&  LE &::& Nat \verb+~>+ Nat ~> *0 &::& *1 \\
LeZ &::& LE Z a            &::&  *0     &::& *1 \\
LeS &::& LE n m -> LE (S n) (S m)  &::&  *0     &::& *1 \\ \hline

      & |&  Even  &::&  Nat \verb+~>+ *0 &::& *1 \\
EvenZ &::& Even Z &::& *0 &::& *1 \\
EvenSS &::& Even n -> Even (S(S n)) &::& *0 &::& *1 \\
\hline
\end{tabular}
}
\caption{The level hierarchy for some of the examples in the paper.} \label{hierarchy}
\end{figure}


\section{Features of the \om\ Language} \label{features}
\om\ is modelled after the Haskell language. There are several
important differences between \om\ and Haskell that
give \om\ its unique power of expression. These include.

\begin{itemize}

\item{\bf Data Structures at All Levels.} Kinds are a type system
for classifying types. Sorts are a type system for classifying
kinds. There is no practical limit to this hierarchy. In \om,
programmers can introduce new tree-like structures at any level. In
Haskell all introduced datatypes are classified by \verb+*0+. I.e.
the introduced types classify only values. In Figure \ref{hierarchy},
Haskell types are illustrated by {\tt Tree}, which is a type
constructor which classifies its constructor functions ({\tt Fork},
{\tt Node}, and {\tt Tip}) which are values. In \om, the {\tt data}
declaration is generalized to all levels.

\item{\bf GADTs.} Generalized Algebraic Datatypes allow constructor
functions to have more general types than the types supported by the
\verb+data+ declaration in Haskell. GADTs are important because the
additional generality allows the programmer to express properties of
types as witness types, proof objects, or singleton types. GADTs are
the machinery that support the Curry-Howard isomorphism in \om.
In Figure \ref{hierarchy}, the types {\tt Seq},
{\tt LE}, and {\tt Even} require the generality introduced by GADTs.


\item{\bf Functions at All Levels.} \om\ supports functions over
tree- structured data at all levels. Such functions are written
by pattern matching equations, in much the same manner one writes
functions over data at the value level in Haskell. We restrict the
form of such definitions to be inductively sequential (See Appendix \ref{IndSeq}). This ensures
a sound and complete strategy for answering certain type-checking
time questions by the use of narrowing. The class of inductively
sequential functions is a large one, in fact every Haskell function
has an inductively sequential definition. The inductively sequential
restriction affects the form of the equations, and not the functions
that can be expressed. In Figure \ref{hierarchy}, {\tt plus} 
and {\tt le} are functions at the type level.

\item{\bf Code Constructing Quasi-Quotes.}  \om\
supports the run-time generation of code, along the lines of
MetaML~\cite{Sheard:1999:UMS} and Template Haskell~\cite{Sheard02}. The meta-programming ability of
code generation allows us to remove a layer of interpretation
from our programs, that makes them efficient as well as
general.

\end{itemize}

Some of the following sections are labeled with {\em Feature} if they
are an addition to Haskell, {\em Pattern} if they are a paradigmatic
use of the features to accomplish a particular end, or {\em Example} if
they illustrate an important concept.

\vspace*{0.1in}

\subsection{Feature: Kinds} \label{shape} We can introduce new tree-like data at any level,
including the type level and higher. The data declaration introduces both the constructors for
tree-like data and the object that classifies these structures. We
indicate the level where these objects reside using \verb+*0+, \verb+*1+,
\verb+*2+, etc. in the \verb+data+ declaration. Consider the kinds {\tt Nat} (introduced earlier), and
{\tt Boolean}:
 
\vspace*{.15in}
\begin{tabular}{l|l}
\begin{minipage}[t]{2.2in}
{\small
\begin{verbatim}
data Shape :: *1 where 
  Tp:: Shape
  Nd:: Shape
  Fk:: Shape ~> Shape ~> Shape

\end{verbatim}}
\end{minipage}
&
\begin{minipage}[t]{2.2in}
{\small
\begin{verbatim}
 data Boolean:: *1 where
   T:: Boolean
   F:: Boolean  
\end{verbatim}}
\end{minipage}
\end{tabular}
\vspace*{0.1in}

\noindent
Like the kind {\tt Nat} defined earlier, {\tt Shape} and {\tt Boolean}
also define new kinds, and new types classified by these kinds. The new
tree-like data at the type level are constructed by the type-constants
({\tt Tp}, {\tt Nd}, {\tt T}, {\tt F}, {\tt Z}), and type constructors
({\tt Fk} and {\tt S}). The kinds {\tt Shape} and {\tt Boolean} classify
these structures, as shown explicitly in the declaration. For example {\tt
T} is classified by {\tt Boolean}, and {\tt Fk} is a constructor from {\tt
Shape} to {\tt Shape} to {\tt Shape}. Note that while {\tt Tp}, {\tt Nd},
{\tt T}, and {\tt F}  live at the type level, there are no values
classified by them. Again, see Figure \ref{hierarchy} to see where these
objects reside in the larger picture.

Even though there are no values classified by the types introduced by {\tt
Nat}, {\tt Shape}, and {\tt Boolean}, they are very useful. Instead of
using them to classify values, we use them as indexes to value level data,
i.e. types like {\tt Proof \{even n\}} and {\tt Seq a (S Z)}. The indexes
like {\tt \{even n\}} and {\tt S z} indicate static (type-checking time)
properties of values. For example, a value with type {\tt Seq a (S Z)} is
statically guaranteed to have length 1.

\begin{exercise}
Write a data declaration introducing a new kind called {\tt Color}
with types {\tt Red} and {\tt Black}. Are there any values with
type {\tt Red}? Now write a data declaration introducing a new
type {\tt Tree} which is indexed by {\tt Color} (this will be
similar to the use of {\tt Nat} in the declaration of {\tt Seq}).
There should be some values classified by the type {\tt (Tree Red)},
and others classified by the type {\tt (Tree Black)}.
\end{exercise}


\subsection{Feature: Type Functions} \label{functions}  Kind declarations allow us to
introduce new tree-like structures at the type level. We can use
these structures to parameterize data at the value level as we did
with {\tt Nat} indexing {\tt Seq}. We may also compute over these tree-like
structures. Such functions are written by pattern matching equations,
in much the same manner one writes functions over data at the value
level. Several useful functions over types defined
earlier are:

\vspace*{.05in}
\begin{tabular}{l|l}
\begin{minipage}[t]{2.2in}
{\small
\begin{verbatim}
even :: Nat ~> Boolean
{even Z} = T
{even (S Z)} = F
{even (S (S n))} = {even n}
 
le:: Nat ~> Nat ~> Boolean 
{le Z n} = T
{le (S n) Z} = F
{le (S n) (S m)} = {le n m}
\end{verbatim}}
\end{minipage}
&
\begin{minipage}[t]{2.8in}
{\small
\begin{verbatim}
 plus:: Nat ~> Nat ~> Nat
 {plus Z m} = m
 {plus (S n) m} = S {plus n m}
 
 and:: Boolean ~> Boolean ~> Boolean
 {and T x} = x
 {and F x} = F
\end{verbatim}}
\end{minipage}
\end{tabular}
\vspace*{0.05in}\\ \noindent
Like functions at the value level, the type functions  {\tt plus}, {\tt and},
{\tt even}, and {\tt le} are expressed using equations. The function {\tt and}
is a binary function that combines two {\tt Booleans}.  The property {\tt even}
is a unary predicate that distinguishes odd from even numbers, and the property
{\tt le} is a binary less-than-or-equal-to predicate. All the functions are
strict total (terminating) functions at the type level. Termination
is a necessary property of type functions, though this is not
currently checked by the system.

\begin{exercise}
Write an \om\ function {\tt mult}, which is the multiplication
function at the type level over natural numbers.
It should be classified by the kind
\verb+mult:: Nat ~> Nat ~> Nat+.
\end{exercise}

\begin{exercise}
Write the {\tt odd} function classified by the kind \verb+Nat ~> Boolean+.
\end{exercise}

\begin{exercise}
Write the {\tt or} and {\tt not'} functions, that are classified by the kinds
\verb+(Boolean ~> Boolean ~> Boolean)+  and  \verb+(Boolean ~> Boolean)+.
Use {\tt not'} rather than {\tt not} since the name {\tt not}
is already predefined.
Which arguments of {\tt or} should you pattern match over?
Does it matter? Experiment, \om\ won't allow some combinations.
See Appendix \ref{IndSeq} 
on inductively sequential definitions and narrowing for the reason why.
\end{exercise}


\subsection{Feature: GADTs} 
Generalized Algebraic Datatypes allow constructor
functions to have more general types than the types supported by
\verb+data+ declaration in Haskell. GADTs are important because the
additional generality allows the programmer to express properties 
of types using type indexes and witnesses (or proof objects). 
The {\tt data} declaration in \om\ defines generalized algebraic datatypes
(GADT). These are characterized by explicitly classifying constructors in a {\tt data}
declaration with their full types. The additional generality arises because
the range of a constructor in a GADT is not constrained to be the type
constructor applied to only type variables. For example consider the value
level GADTs {\tt Seq}, {\tt Path} and {\tt Tree}:

{\small
\begin{verbatim}
data Seq:: *0 ~> Nat ~> *0 where
  Snil :: Seq a Z
  Scons:: a -> Seq a n -> Seq a (S n)

data Path:: Shape ~> *0 ~> *0 where
  None :: Path Tp a
  Here :: b -> Path Nd b
  Left :: Path x a -> Path (Fk x y) a
  Right:: Path y a -> Path (Fk x y) a
  
data Tree :: Shape ~> *0 ~> *0 where
  Tip:: Tree Tp a
  Node:: a -> Tree Nd a
  Fork::  Tree x a -> Tree y a -> Tree (Fk x y) a
\end{verbatim}}

\noindent
Note that instead of ranges like {\tt (Seq a b)}, and {\tt (Path a b)} where
only type variables like {\tt a}, and {\tt b} can be used as parameters, the
ranges contain sophisticated instantiations such as 
{\tt (Seq a (S n))} and {\tt (Path Nd)}. Note that the second index to {\tt Seq} (the one of kind {\tt
Nat}) is used to describe an invariant about the length of the sequence,
and the {\tt Shape} index to {\tt Path}, indicates the shape of a tree
in which that path is legal. This is one of the many uses of GADTs --
to enforce invariants about the structure of data.
Notice how the shape of {\tt tree1} appears in its type.
{\small
\begin{verbatim}
tree1 :: Tree (Fk  (Fk   Tp  Nd)       (Fk   Nd       Nd))        Int
tree1 =       Fork (Fork Tip (Node 4)) (Fork (Node 4) (Node 3))
\end{verbatim}}
\noindent
We can write pattern matching functions
over GADTs just as we can over algebraic datatypes. The only caveat
is that we must specify the type of the function using a prototype.
\om\ does type checking of functions over GADTs rather than type inference.

Suppose we wanted to search a tree, returning all paths that
lead to a particular element. It would be nice to know that
every path returned was a legal path within the tree. For example
{\tt (Left (Here 2))} is not a legal path within the tree {\tt Tip}.
The {\tt Shape} index allows us to specify that our searching function
always returns a value that obeys this legal path invariant.

{\small
\begin{verbatim}
find:: (a -> a -> Bool) -> a -> Tree sh a -> [Path sh a]
find eq n Tip = []
find eq n (Node m) =
  if eq n m then [Here n] else []
find eq n (Fork x y) = 
  map Left (find eq n x) ++ 
  map Right (find eq n y)
\end{verbatim}}

\noindent
The type of {\tt find} guarantees that every path returned is a legal
path within the tree searched, because both the tree and every path
in the list has the same {\tt Shape}, namely {\tt sh}.

\begin{exercise} \label{extract}
Write an \om\ function with type
{\tt extract:: Path sh a -> Tree sh a -> a}, which extracts
the value of type {\tt a}, stored in the tree at the
location pointed to by the path. This function will pattern
match over two arguments simultaneously. Some combinations
of patterns are not necessary. Why? See section
\ref{unreachable} for how you can document this fact.
\end{exercise}

\begin{exercise} 
Replicate the shape index pattern for lists.
Write two \om\ GADTs. One at the kind level which encodes
the shape of lists, and one at the type level for lists
indexed by their shape. Also, write a find function for
your new types. {\tt find:: (a -> a -> Bool) -> a -> List sh a -> Maybe(ListPath sh a)},
which returns the first path, if one exists.

Since every GADT is comprised of a sum of products, can you define
a single shape kind, that could be used for all parametric
datatypes?
\end{exercise}

\begin{exercise} \label{reptype}
Consider the GADT below.
{\small
\begin{verbatim}
data Rep :: *0 ~> *0 where
   Int :: Rep Int
   Prod :: Rep a -> Rep b -> Rep (a,b)
   List :: Rep a -> Rep [a]
\end{verbatim}}
Construct a few terms. Do you note anything interesting
about this type? Write a function with the following prototype:
{\tt showR:: Rep a -> a -> String}, which given values
of type {\tt Rep a} and {\tt a}, displays the second as a string.
Extend this GADT with a few more constructors, then extend your
{\tt showR} function as well.
\end{exercise}


\subsection{Pattern: Witnesses} \label{witness}

GADTs can be used to witness relational properties
between types. This is because the parameters to types introduced
using the GADT mechanism can play different roles.
The natural number argument of the type constructor {\tt
Seq} (from Section \ref{simple}) plays a qualitatively
different role than type arguments in ordinary ADTs.
Consider the declaration for a binary tree datatype in Haskell:

{\small\begin{verbatim} 
data HTree a = HFork (HTree a) (HTree a) | HNode a | HTip
\end{verbatim}}
\noindent
In this declaration the type parameter {\tt a} is used to
indicate that there are sub components of {\tt HTree}s that
are of type {\tt a}. In fact, {\tt HTree}s are parametric.
Any type of value can be placed in the ``sub component" of
type {\tt a}. The type of the value placed there is
reflected in the {\tt HTree}'s type. Contrast this with the
{\tt n} in {\tt (Seq a n)}, and the {\tt sh} in {\tt (Tree sh a)}. Instead, the parameter {\tt n}
is used to stand for an abstract property (the length of the
list represented), and the parameter {\tt sh} is used to stand for the
shape of the tree. When we use a type parameter in this way
we call it a type index~\cite{XiThesis,Xi:1999:DTP} rather than a type parameter.

We can use indexes to GADTs to define value level data that
we can think of as proofs, or witnesses to type level
properties. This is a powerful idea. Consider the
introduction of several new indexed types {\tt Proof}, {\tt Plus}, {\tt LE} and
{\tt Even}. Note that these are ordinary data structures that
exist at the value level, but describe properties at
the type level.
\vspace*{.3in}
\noindent
\begin{tabular}{l|l}
\begin{minipage}[t]{2.3in}
{\small
\begin{verbatim}

data Proof:: Boolean ~> *0 where
  Triv:: Proof T

data Plus:: Nat ~> Nat ~> Nat ~> *0 
      where
  PlusZ:: Plus Z m m
  PlusS:: Plus n m z -> 
          Plus (S n) m (S z)
\end{verbatim}}
\end{minipage}
&
\begin{minipage}[t]{2.6in}
{\small
\begin{verbatim} 

 data LE:: Nat ~> Nat ~> *0 where
   LeZ:: LE Z n
   LeS:: LE n m -> 
         LE (S n) (S m)

 data Even:: Nat ~> *0 where
   EvenZ:: Even Z
   EvenSS:: Even n -> Even (S (S n))
\end{verbatim}}
\end{minipage}
\end{tabular}

\noindent
These declarations introduce value-level constants ({\tt Triv}, {\tt EvenZ}, {\tt PlusZ}, and
{\tt LeZ}) and constructor functions ({\tt EvenSS}, {\tt PlusS}, and {\tt LeS}).
Values of these types can be used as proofs about the natural numbers.

To make it easier to enter and display types of kind \verb+Nat+, 
in \om,  we have special syntactic sugar for them:
{\tt Z} = \verb+0t+,
\verb+S Z+ = \verb+1t+, and
\verb+S(S Z)+ = \verb+2t+, 
etc. We may also write \verb|(1+x)t| for \verb|S x|,
 and  \verb|(2+x)t| for \verb|S(S x)|, etc.
We introduce this notation here (see Section \ref{synext} for more detail)
to emphasize that we should view {\tt LE}, {\tt Plus} and {\tt Even}
as relationships
between natural numbers. To emphasize this,
let's examine the types of several values constructed with these
constructors. 

\vspace*{.2in}
\begin{tabular}{l|l}
\begin{minipage}[t]{2.4in}
{\small
\begin{verbatim}
EvenZ:: Even 0t
(EvenSS EvenZ):: Even 2t
(EvenSS (EvenSS EvenZ)):: Even 4t

p1 ::Plus 2t 3t 5t
p1 = PlusS (PlusS PlusZ)
\end{verbatim}}
\end{minipage}
&
\begin{minipage}[t]{3.0in}
{\small
\begin{verbatim}
 LeZ:: LE 0t a
 (LeS LeZ):: LE 1t (1+a)t
 (LeS (LeS LeZ)):: LE 2t (2+a)t
 
 even2 :: Proof {even 2t}
 even2 = Triv
\end{verbatim}}
\end{minipage}
\end{tabular}
\vspace*{.05in}

\noindent
The important thing to notice is 
that we may view ordinary values with types 
\verb+(LE n m)+, \verb+(Even n)+, and \verb+(Proof {even n})+ as proofs, since the types of all 
legally constructed values witness only true statements about {\tt n} and {\tt m}. 
For example we cannot build a term of type \verb+(Even 1t)+. This is the essence
of the Curry-Howard isomorphism. 

We can view \verb+(EvenSS EvenZ):: Even 2t+
as either the statement that \verb+(EvenSS EvenZ)+ has type \verb+(Even 2t)+,
or that \verb+(EvenSS EvenZ)+ is a proof of the property \verb+(Even 2t)+.
In the same fashion, the type system will reject ill-typed terms that witness false
statements. For example, consider the response when we
try to type the term {\tt Triv} with the type {\tt (Proof \{even 1t\})}


{\small
\begin{verbatim}
bad:: Proof {even 1t}
bad = Triv

While type checking in the scope of:
   bad
We need to prove:
   Equal {even 1t} T
From the truths:
And the rules:S,Z,
But, The equations: (F=T) =>  have no solution
\end{verbatim}}

All this follows directly from the introduction of new types as GADTs
and the ability to define them, and to compute over them, at arbitrary levels. 

\begin{exercise} 
Construct terms with the types {\tt (Plus 2t 3t 5t)}, {\tt (Plus 2t 1t 3t)},
and {\tt (Plus 2t 6t 8t)}. What did you discover?
\end{exercise}

\begin{exercise} \label{summand}
Write an \om\ function with the following prototype:\\
{\tt summandLessThanOrEqualToSum:: Plus a b c -> LE a c}. Hint: it is a recursive function.
Can you write a similar function with type {\tt (Plus a b c -> LE b c)}?
\end{exercise}

\subsection{Pattern: Witness vs. Type Function}
The reader may have noticed that {\tt (Proof \{even n\})} and {\tt (Even n)} are
two different ways to express the same notion.
Either we write a ({\tt Boolean}) function at the type level ({\tt even}), or
introduce a witness type ({\tt Even}) at the value level. 

The general principle of replacing a boolean function at the type level
with a witness object at the value level, can be further
generalized (you can try this in Exercise \ref{EvenWitness}).
The type function does not
have to have {\tt Boolean} as its range. Instead, 
for every $n$-ary function at the type
level, we can build an $(n+1)$-ary witness type. 
We express the equality between a function call and its
result: {\tt \{function a b\} = c} as a relation: {\tt \{Relation a b c\}}.

The witness
type turns the $n$-ary function into an $(n+1)$-ary type constructor.
Each clause in the function definition is named by a constructor
function in the witness. If the right-hand-side of a clause has
$m$ recursive calls, the constructor function becomes
an $m$-ary constructor. The right-hand-side of each clause becomes
the $(n+1)^{st}$ argument of the range, where every recursive call
to the function in the right-hand-side, is replaced with a variable.
Each recursive call becomes one of the $m$ arguments.
The $(n+1)^{st}$ argument to these calls
is the new variable replacing the corresponding recursive
call in the $(n+1)^{st}$ argument of the
range. For example: The clause of the
binary function {\small \verb+{plus (S n) m} = S {plus n m}+}, becomes
a ternary predicate {\small \verb+Plus (S n) m (S {plus n m})+}. By replacing
the recursive call with {\tt z}, and making {\tt z} be the $(n+1)^{st}$ parameter
to the first argument, we get the type of the unary constructor\\
{\small \verb+PlusS:: Plus n m z -> Plus (S n) m (S z)+}.

\begin{exercise}\label{EvenWitness}
Use the pattern above to define a GADT (a type constructor
with 2 arguments) that witnesses the {\tt even} type function.
\end{exercise}

Witnesses and type functions express the same ideas, but can be used
in very different ways.  Type functions are only useful at compile-time 
(they're static) and their structure cannot be observed (they
can only be applied, so we say they are extensional). Witnesses, on
the other hand, are actual data that is manipulated at run time
(they're dynamic). Their structure can also be observed and taken
apart (we say they're intensional). They are true data. A big
difference between the two ways of representing properties is the
computational mechanisms used to ensure that programs adhere to such
properties.


 
\subsection{Pattern: Singleton Types}  \label{singleton} Sometimes
it is useful to direct computation at the type level, by
writing functions at the value level. Even though types
cannot depend on values directly, this can be simulated by the use of
singleton types. The idea is to build a completely separate
isomorphic copy of the type in the value world, but still
retain a connection between the two isomorphic structures.
This connection is maintained by indexing the value-world
type with the corresponding type-world kind. This is best
understood by example. Consider reflecting the kind {\tt
Nat} into the value-world by defining the type constructor
{\tt SNat} using a {\tt data} declaration.

{\small
\begin{verbatim}
data SNat:: Nat ~> *0 where
  Zero:: SNat Z
  Succ:: SNat n -> SNat (S n)
  
three = (Succ (Succ (Succ Zero))):: SNat(S(S(S Z)))
\end{verbatim}}

\noindent
Here, the value constructors of the {\tt data} declaration for {\tt
SNat} mirror the type constructors in the {\tt kind} declaration of
{\tt Nat}.  We maintain the connection
between the two isomorphic structures by the use of {\tt SNat}'s natural
number index. This type index is in one-to-one correspondence with
the shape of the value. Thus, the type index of {\tt SNat} exactly
mirrors its shape. For example, consider the
example {\tt three} above, and pay particular attention to the
structure of the type index, and the structure of the value with that
type.
 
This kind of relationship between values and types
is called a {\it singleton type} because there is only one
element of any singleton type. For example only {\tt Succ (Succ Zero)}
inhabits the type {\tt SNat(S (S Z))}. It is possible to define a
singleton type for any first order type (of any kind). All singleton
types always have kinds of the form \verb+I ~> *0+ where {\tt I} is the
index we are reflecting into the value world. We sometimes call
singleton types {\em representation types}. We cannot over emphasize
the importance of the singleton property. Every singleton type
completely characterizes the structure of its single inhabitant, and
the structure of a value in a singleton type completely characterizes
its type. Thus we can compute over a value of a singleton type, and the
computation at the value level can express a property at the type
level.
By using singleton types we completely avoid the use of dependent
types where types depend on values~\cite{StoneHar00,Shao:2002:TSC}.
The cost associated with this avoidance is the possible duplication
of data structures and functions at several levels.



\subsection{Pattern: A pun: Nat'} \label{nat'}
We now define the type {\tt Nat'},
which is identical structurally to the type {\tt SNat}. As such, the type {\tt Nat'} is
also a singleton type representing the natural numbers, but it relies on a
feature of the \om\ type system. In \om\ (as in Haskell) the name space for values
is separate from the name space for types. Thus it is possible to have the same
name stand for two things. One in the value space, and the other in the type
space. The pun is because we use the names {\tt S} and {\tt Z} in both the
value and type name spaces. We exploit this ability by writing:

{\small
\begin{verbatim}
data Nat':: Nat ~> *0 where
  Z:: Nat' Z
  S:: Nat' n -> Nat' (S n)
\end{verbatim}}

\noindent
The value constructors {\tt Z:: Nat' Z} and {\tt S:: Nat' n ->
Nat' (S n)} are ordinary values whose types mention the type
constructors they pun. The name space partition, and the relationship
between {\tt Nat} and {\tt Nat'} is illustrated below.

\vspace*{.1in}
{\tt {\small
\begin{tabular}{|lclclcl|} \hline
  $\leftarrow$     {\it {\tiny value name space}}&|& {\it {\tiny type name space $\rightarrow$}} & &  && \\ \hline
{\em value}&|& {\em type} &|&  {\em kind} &|& {\em sort} \\ \hline
     & |& Z        &::& Nat &::& *1 \\
     & |& S        &::& Nat \verb+~>+ Nat &::& *1 \\   \hline  
Z &::& Nat' Z &::& *0 &::& *1 \\
S &::& Nat' m -> Nat' (S m) &::& *0 &::& *1 \\
\hline
\end{tabular}}}
\vspace*{.1in}

\noindent
In {\tt Nat'}, the singleton relationship
between a {\tt Nat'} value and its type is emphasized even more
strongly, as witnessed by the example {\tt three'}. 

\vspace*{-.05in}
{\small
\begin{verbatim}
three' = (S(S(S Z))):: Nat'(S(S(S Z)))
\end{verbatim}}

\noindent
Here the
shape of the value, and the type index appear isomorphic.
We further exploit this pun, by extending the syntactic sugar
for writing natural numbers at the type level ({\tt 0t}, {\tt 1t},  etc.) to their
singleton types at the value level. Thus we may write
{\tt (2t:: Nat' 2v)}. See Section \ref{synext} for details.

\begin{exercise} \label{pred}
Write the two \om\ functions with types: {\tt same:: Nat' n -> LE n n},
and {\tt predLE:: Nat' n -> LE n (S n)}. Hint: they are simple
recursive functions.
\end{exercise}

\begin{exercise}  \label{trans}

Write the \om\ function which witnesses the implication stating the
transitivity of the less-than-or-equal-to predicate. {\tt trans:: LE a
b -> LE b c -> LE a c}. By the curry-Howard isomorphism a total function
between two witnesses

Hint: it is a recursive function with pattern
matching over both arguments. One of the cases is not reachable. Which
one? Why? See Section \ref{unreachable} for how you can document this
fact. \end{exercise}


\begin{exercise} 
In Exercise \ref{summand} we proposed writing a function with type
{\tt (Plus a b c -> LE b c)}. This turned out to be not possible given
our current knowledge. But,
it is possible to write a function with type
{\tt (Nat' b -> Plus a b c -> LE b c)}. Write this function. What
benefit does the first {\tt Nat' b} argument provide? Hint:
both the functions {\tt same} and {\tt predLE} come in useful.
\end{exercise}


\subsection{Pattern: Leibniz Equality} \label{equal}
Terminating terms of type \verb+(Equal lhs rhs)+ are values witnessing the equality
of {\tt lhs} and {\tt rhs}. The type constructor {\tt Equal} is defined as:

{\small
\begin{verbatim}
data Equal :: a ~> a ~> *0 where
  Eq:: Equal x x
\end{verbatim}}

The type constructor {\tt Equal} can be applied to any
two types, as long as both are classified by the same
classifier {\tt a}. The classifier
{\tt a} is largely unconstrained. In Section \ref{level} we
discuss this in greater depth. Intuitively,
given a term {\tt w} with type {\tt (Equal x y)}, we can
think of {\tt w} as a proof that {\tt x} and {\tt y}
are equal.

Note that {\tt Equal} is a GADT, since in the type of the constructor {\tt Eq} the two type indexes
are the same, and not just polymorphic variables (i.e. the type of
{\tt Eq} is {\em not} {\tt (Equal x y)} but is rather {\tt (Equal x x)}).
The single constructor ({\tt Eq}) has a polymorphic type {\tt(Equal x x)}.
Ordinarily, if the two arguments of {\tt Equal} are type-constructor terms,
the two arguments must be the same (or they couldn't be equal). But, if we
allow type functions as arguments (see Section \ref{functions}), since many functions may compute the
same result (even with different arguments), the two terms can be
syntactically different (but semantically the same).  
For example \verb+(Equal 2t {plus 1t 1t})+ is a well formed equality type
since 2 is semantically equal to 1+1.
The {\tt Equal} type allows the programmer to reify
the type checkers notion of equality, and to pass this reified evidence
around as a value. The {\tt Equal} type plays a large role
in the {\tt theorem} declaration (see Section \ref{theorem}).


\begin{exercise} \label{samenat}
Singleton types allow us to construct Equal objects at runtime.
Because of the one-to-one relationship between singleton values and their
types, knowing the shape of a value determines its type. In a similar
manner knowing the type of a singleton determines its shape. Write the
function in \om\ that exploits this fact: {\tt sameNat:: Nat' a -> Nat' b -> Maybe(Equal a b)}.
We have written the first clause. You can finish it.
\begin{verbatim}
sameNat:: Nat' a -> Nat' b -> Maybe(Equal a b)
sameNat Z Z = Just Eq
\end{verbatim}
If one wonders how this function is typed,
it is very instructive to construct the typing box (as we did for
{\tt app} in Section \ref{simple}) with expected types, equations,
computed types, and generated equalities.
\end{exercise}

\subsection{Computing Programs and Properties Simultaneously} We can write
programs that compute an indexed value along with a witness that the
value has some additional property. For example, when we add two
static length lists, the resulting list has a length that is
related to the lengths of the two input lists, and we can
simultaneously produce a witness to this relationship.

{\small
\begin{verbatim}
data Plus:: Nat ~> Nat ~> Nat ~> *0 where
  PlusZ:: Plus Z m m
  PlusS:: Plus n m z -> Plus (S n) m (S z)

app1:: Seq a n -> Seq a m -> exists p . (Seq a p,Plus n m p)
app1 Snil ys = Ex(ys,PlusZ)
app1 (Scons x xs) ys = case (app1 xs ys) of  
                        Ex(zs,p) ->  Ex(Scons x zs,PlusS p) 
\end{verbatim}}\vspace*{-.05in}

\noindent The keyword {\tt Ex} is the ``pack" operator of Cardelli and
Wegner~\cite{Cardelli-Wegner85}. Its use turns a normal type 
{\small \verb+(Seq a p,Plus n m p)+} into an existential type 
{\small \verb+exists p.(Seq a p,Plus n m p)+}. The \om\ compiler
uses a bidirectional type checking algorithm to propagate
the existential type in the signature inwards to the
{\tt Ex} tagged expressions. This allows it to
abstract over the correct existentially quantified variables.

In a similar manner, given a proof that $a \leq b$ we can always find a $c$
such that $a+c=b$.
{\small
\begin{verbatim}
smaller :: Proof {le (S a) (S b)} -> Proof {le a b}
smaller Triv = Triv

diff:: Proof {le a b} -> Nat' a -> Nat' b -> 
       exists c .(Nat' c,Equal {plus a c} b)
diff Triv Z m = Ex (m,Eq)
diff Triv (S m) Z = unreachable
diff (q@Triv) (S x) (S y) =  
  case diff (smaller q) x y of
   Ex (m,Eq) -> Ex (m,Eq)
\end{verbatim}}

\begin{exercise}
The filter function drops some elements from a list. Thus, the length
of the resulting list cannot be known statically. But, we can
compute the length of the resulting list along with the list. Write
the \om\ function with prototype:\\
{\small {\tt filter :: (a->Bool) -> Seq a n -> exists m . (Nat' m,Seq a m)}}

Since filter never adds elements to the list, that weren't already
in the list, the result-list is never longer than the original list. We can compute 
a proof of this fact as well. Write
the \om\ function with prototype:\\
{\small {\tt filter :: (a->Bool) -> Seq a n -> exists m . (LE m n,Nat' m,Seq a m)}}\\
Hint: You may find the functions {\tt predLE} from
Exercise \ref{pred} useful.
\end{exercise}

\subsection{Feature: Unreachable Clauses}\label{unreachable}

The keyword {\tt unreachable} in the second clause of the definition
for {\tt diff} states that type considerations
preclude the flow of control ever reaching the clause labeled unreachable.
This is because the type information in the function prototype for {\tt diff}
is propagated into the patterns of each clause. In the second
clause the following information is propagated.

{\small
\begin{verbatim}
Triv  :: Proof {le a b}
(S m) :: Nat' a
Z     :: Nat' b 
\end{verbatim}}

We compute the type of \verb+(S m)+ to be \verb+(Nat' (S m))+, and we compute
the type of \verb+Z+ to be \verb+(Nat' Z)+, combining this with
the propagated type information we see that {\tt a = (S m)} and {\tt b = Z}.
Thus the type of {\tt Triv} must be {\tt Proof \{le (S m) Z\}}.
The type function application {\tt \{le (S m) Z\}} reduce to {\tt F},
but the argument to {\tt Proof} must be {\tt T}.
These sets of assumptions
are inconsistent. So the clause in the
scope of these patterns is unreachable. There are no well-typed
arguments, to which we could apply {\tt diff}, that would
exercise the second clause. The keyword {\tt unreachable}
indicates to the compiler that we recognize this fact. 
The reachability of all unreachable clauses is tested.
If they are in fact reachable, an error is raised. An
unreachable clause, without the {\tt unreachable} keyword
also raises an error.

The point of the unreachable clause is to document that the author of the
code knows that this clause is unreachable, and to help document that the
clauses exhaustively cover all possible cases. The function {\tt extract}
from exercise \ref{extract} and the function {\tt trans} from exercise
\ref{trans} could use an unreachable clause.


\subsection{Feature: Staging} \label{stage}
\om\ supports staging annotations:
brackets ({\small \verb+[| _ |]+}),
escape ({\small \verb+$( _ )+}), and the two staging functions
{ {\tt lift::(forall a . a -> Code a)}} and 
{ {\tt run::(forall a . (Code a) -> a)}} 
for building and manipulating code. \om\ uses
the Template Haskell~\cite{Sheard02} conventions for creating code. 
Brackets  (\verb+[| _ |]+) are a quasi-quotation mechanism, and escape
(\verb+$( _ )+) escapes from the effects of quasi-quotation.
For example.

\vspace*{-.1in}
{\small
\begin{verbatim}
inc x = x + 1
c1a = [| 4 + 3 |]
c2a = [| \ x -> x + $c1a |]
c3 = [| let f x = y - 1 where y = 3 * x in f 4 + 3 |]
c4 = [| inc 3 |]
c5 = [| [| 3 |] |]
c6 = [| \ x -> x |]
\end{verbatim}}
\noindent
In the examples above, {\tt inc} is a normal function. The variable {\tt c1a} names a piece
of code with type {\small \verb+Code Int+}. The variable {\tt c2a} names a piece of code
with type {\small {\small \verb+Code(Int -> Int)+}}. It is constructed by
splicing the code {\tt c1a} into the body of the lambda abstraction.
The variable {\tt c3} names a piece of code with type {\small \verb+Code Int+}. It illustrates
the ability to define rich pieces of code with embedded
{\tt let} and {\tt where} clauses. The variable {\tt c4} names a piece
of code with type {\small \verb+Code Int+}. It illustrates that functions
defined in earlier stages ({\tt inc}) can be lifted (or embedded)
in code. The variable {\tt c5} names a piece
of code with type {\small \verb+Code (Code Int)+}. It illustrates that
code can be nested.

The purpose of the staging mechanism is to have finer control over
evaluation order, which is exactly what
we want to do when removing the interpretive overhead of generic programming. \om\ supports many of the features of
MetaML~\cite{Sheard:1999:UMS,TS00}. 

\begin{exercise}
The traditional staged function is the power function.
The term {\tt (power 3 x)} returns x to the third power. The unstaged
power function can be written as:
{\small 
\begin{verbatim}
power:: Int -> Int -> Int
power 0 x = 1
power n x = x * power (n-1) x
\end{verbatim}}
\noindent
Write a staged power function: {\tt pow:: Int -> Code Int -> Code Int}\\
such that {\tt (pow 3 [|99|])} evaluates to \verb+[| 99 * 99 * 99 * 1 |]+.
This can be written simply by placing staging annotations in the unstaged version.
\end{exercise}


\subsection{Feature: Level Polymorphism} \label{level}
Sometimes we wish to use the same structure at both the value and type level.
One way to do this is to build isomorphic, but different, data structures
at different levels. In \om, we can define a structure to live
at many levels. We call this level polymorphism. For example
a {\tt Tree} type that lives at all levels can be defined by:

{\small
\begin{verbatim}
data Tree :: level n . *n ~> *n where
  Tip :: a ~> Tree a
  Fork :: Tree a ~> Tree a ~> Tree a
\end{verbatim}}
\noindent
Levels are {\it not} types. A level variable can only be used
as an argument to the {\tt *} operator. Level abstraction can only
be introduced in the kind part of a {\tt data} declaration, but level polymorphic
functions can be inferred from their use of constructor functions
introduced in level polymorphic {\tt data} declarations.

In the example above,
\om\ adds the type constructor {\tt Tree} at all type levels,
and the constructors {\tt Tip} and {\tt Fork} at the value level
as well at all type levels. We can illustrate this by evaluating
a tree at the value level, and by asking \om\ for the kind of
a similar term at the type level.

{\small
\begin{verbatim}
prompt> Fork (Tip 3) (Tip 1)
(Fork (Tip 3) (Tip 1)) : Tree Int

prompt> :k Tip Int
Tip Int :: Tree *0 
\end{verbatim}}

Another useful pattern is to define normal ({\tt *0}) datatypes indexable
by types at all levels. For example consider the kind of the type constructor
{\tt Equal} and the type of its constructor {\tt Eq} from Section \ref{equal}.
Its type can be more verbosely expressed as follows where the level polymorphism
is explicit (rather than inferred, as it is in Section \ref{equal}).

{\small
\begin{verbatim}
Equal :: level b . forall (a:*(1+b)).a ~> a ~> *0

Eq :: level b . forall (a:*(1+b)) (c:a:*(1+b)).Equal c c
\end{verbatim}}
\noindent
For all levels {\tt b}, the type {\tt a} is classified by
a star at level {\tt 1+b}. Some legal instances are:
{\small
\begin{verbatim}
Equal :: forall (a:*1).a ~> a ~> *0   -- when b=0
Equal :: forall (a:*2).a ~> a ~> *0   -- when b=1
\end{verbatim}}
\noindent
Without level polymorphism, the {\tt Equal} type constructor could only
witness equality between types at a single level, i.e. types classified by
{\tt a:: *1} but not {\tt a:: *2}. So {\tt (Equal Int Bool)} is well formed
but {\tt (Equal Nat Tag)} would not be, since both {\tt Nat}
and {\tt Tag}\footnote{See Section \ref{tag}.} are classified by {\tt *1:: *2}. For a useful
example, the type of {\tt labelEq} could not be expressed
using a level-monomorphic {\tt Equal} datatype.

{\small
\begin{verbatim}
labelEq:: forall (a:Tag) (b:Tag). Label a -> Label b -> Maybe (Equal a b)
\end{verbatim}}
\noindent
This is because the {\tt a} and {\tt b} are classified by {\tt Tag}, and are
not classified by {\tt *0}.

\begin{exercise} \label{ExRow}
A row is a list-like structure that associates
a pair of objects. In \om\ we write \verb+{`a=Int,`z=Bool}r+ for the
row classified by {\tt (Row Tag *0)}, which associates the {\tt Tag}
\verb+`a+ with \verb+Int+, and \verb+`z+ with \verb+Bool+.
In general we'd like not to restrict rows to any single level.
Level polymorphism comes in handy here. Define a GADT, {\tt MyRow},
that defines a level polymorphic row type at level 1, but which
is indexed by a pair of types from any level. I.e.
{\tt MyRow} should be classified as follows:

{\small
\begin{verbatim}
MyRow :: level d b . forall (a:*(2+b)) (c:*(2+d)). a ~> c ~> *1
\end{verbatim} }
\end{exercise}

\subsection{Feature: Syntactic Extension} \label{synext}
Many languages supply syntactic sugar for constructing homogeneous sequences and
heterogeneous tuples. For example in Haskell lists are often
written with bracketed syntax, \verb+[1,2,3]+, rather than a constructor function syntax, \verb+(Cons 1 (Cons 2 (Cons 3 Nil)))+, and
tuples are often written as \verb+(5,"abc")+ and \verb+(2,True,[])+
rather than \verb+(Pair 5 "abc")+ and \verb+(Triple 2 True [])+. In \om\
we supply special syntax for four different kinds of data, and allow users to
use this syntax for data they define themselves. \om\ has
special syntax for list-like, natural-number-like, pair-like, and record-like types.
Some examples in the supported syntax are: \verb+[4,5]i+, \verb|(2+n)j|,
\verb+(4,True)k+, and \verb+{"a"=5, "b"=6}h+. In general, the syntax starts 
with list-like, natural-number-like, record-like, or pair-like syntax,
and is terminated by a tag. A programmer may specify that 
a user defined type should be displayed using the special syntax with a given tag. Each
tag is associated with a set of functions (a different set for
list-like, natural-number-like, record-like, and pair-like types). Each
term written using the special syntax (with tag {\it i}) expands into a call of the
functions specified by tag {\it i}. For example {\tt 2}{\it i} expands to {\tt S(S Z))}
if the functions associated with {\it i} are {\tt S} and {\tt Z}. We now
explian the details for each case.


The list-like syntax associates two functions with each
tag. These functions play the role of {\tt Nil} and {\tt Cons}.
For example if the tag ``{\tt i}" is associated with
the functions {\tt (C,N)}, then the expansion is as follows.
{\small
\begin{verbatim}
[]i         ---> N
[x,y,z]i    ---> C x(C y (C z N))
[x;xs]i     ---> (C x xs)
[x,y ; zs]i ---> C x (C y zs)
\end{verbatim}}
\noindent
The semicolon may only appear before the last element in the square brackets.
In this case, the last element stands for the tail of the resulting list.

The natural-number-like syntax associates two functions with each
tag. These functions play the role of {\tt Zero} and {\tt Succ}.
For example if the tag ``{\tt i}" is associated with
the functions {\tt (Z,S)}, then the expansion is as follows.
{\small
\begin{verbatim}
4i     ---> S(S(S(S Z)))
0i     ---> Z
(2+x)i ---> S(S x)
\end{verbatim}}
In earlier versions of \om, before the addition of syntactic
extensions, values of the
built in types {\tt Nat} and {\tt Nat'}, could 
be specified using the syntax \verb+#4+. 
For backward compatibility reasons, this is currently still
supported and is equivalent 
to either \verb+4t+ (i.e. {\tt  S(S(S(S Z)))}) in the type name space, and \verb+4v+ 
(i.e. {\tt  S(S(S(S Z)))}) in the value name space.


The tuple-like syntax associates one function with each
tag. This function plays the role of a binary constructor.
For example if the tag ``{\tt i}" is associated with
the function {\tt P}, then the expansion is as follows.

{\small
\begin{verbatim}
(a,b)i      ---> P a b
(a,b,c)i    ---> P a (P b c)
(a,b,c,d)i  ---> P a (P b (P c d))
\end{verbatim}}

The record-like syntax associates two functions with each
tag. These functions play the role of the constant {\tt RowNil} and 
the ternary function {\tt RowCons}.
For example, if the tag ``{\tt i}" is associated with
the functions {\tt (RN,RC)}, then the expansion is as follows.

{\small
\begin{verbatim}
{}i            ---> RN
{a=x,b=y}i    ---> RC a x (RC b y RN)
{a=x;xs}i     ---> (RC a x xs)
{a=x,b=y ; zs}i ---> RC a x (RC b y zs)
\end{verbatim}}

Syntactic extension can be applied to any GADT, at either the value or type level. The
new syntax can be used by the programmer for terms, types, or patterns. \om\ uses the
new syntax to display such terms. The constructor based mechanism can also still
be used. The tags are specified using a deriving clause in a GADT. 
See Section \ref{binding} for an example use of this feature that
makes \om\ code easy to read and understand.

\begin{exercise}
Consider the GADT with syntactic extension ``{\tt i}".
{\small
\begin{verbatim}
data Nsum:: *0 ~> *0 where
    SumZ:: Nsum Int
    SumS:: Nsum x -> Nsum (Int -> x)
  deriving Nat(i)
\end{verbatim}}
\noindent
What is the type of the terms {\tt 0i}, {\tt 1i}, and {\tt 2i}? Can you write
a function with prototype: {\tt add:: Nsum i -> i}, where {\tt (add n)}
is a function that sums
$n$ integers. For example:
\verb+add 3i 1 2 3+ $\longrightarrow$ \verb+6+.
\end{exercise}

\subsection {Feature: Tags and Labels} \label{tag}

Many object languages have a notion of name. To make representing names in the
type system easy we introduce the notion of Tags and Labels. As a {\em first
approximation}, consider the finite kind {\tt Tag} and its singleton type {\tt
Label}:

{\small
\begin{verbatim}
data Tag:: *1 where
  A:: Tag
  B:: Tag
  C:: Tag

data Label:: Tag ~> *0 where
  A:: Label A
  B:: Label B
  C:: Label C
\end{verbatim}}

Here, we again deliberately use the value-name space, type-name space
overloading. The names {\tt A}, {\tt B}, and {\tt C} name different,
but related, objects at both the value and type level.
At the value level, every {\tt Label} has a type index
that reflects its value. I.e. {\tt A::Label A}, and {\tt B::Label B}, and {\tt
C::Label C}. Now consider a countably infinite set of tags and labels. We can't
define this explicitly, but we can build such a type as a primitive inside of
\om. At the type level, every legal identifier whose name is preceded by a
back-tick ({\tt `}) is a type classified by the kind {\tt Tag}. For example the type {\tt
`abc} is classified by {\tt Tag}. At the value level, every such symbol {\tt `abc} is classified
by the type {\tt (Label `abc)}.

There are several functions that operate on labels. The first is
{\tt labelEq} which compares two labels for equality. Since labels
are singletons, a simple true or false answer would be useless.
Instead {\tt labelEq} returns a Leibniz proof of equality 
(see Section \ref{equal}) that the {\tt Tag}
indexes of identical labels are themselves equal.

{\small
\begin{verbatim}
labelEq :: forall (a:Tag) (b:Tag).Label a -> Label b -> Maybe (Equal a b)

prompt> labelEq `w `w
(Just Eq) : Maybe (Equal `w `w)

prompt> labelEq `w `s
Nothing : Maybe (Equal `w `s)
\end{verbatim}}

Fresh labels can be generated by the function {\tt freshLabel}.
Since the {\tt Tag} index for such a label is unknown, the generator
must return a structure where the
{\tt Tag} indexing the label is existentially quantified. Since every call
to {\tt freshLabel} generates a different label, the {\tt freshLabel}
operation must be an action in the {\tt IO} monad. The function
{\tt newLabel} coerces a string into a label. It too, must
existentially hide the Tag indexing the returned label. But,
because it always returns the same label when given the same input
it can be a pure function.

{\small
\begin{verbatim}
freshLabel :: IO HiddenLabel
newLabel:: String -> HiddenLabel

data HiddenLabel :: *0 where 
 Hidden:: Label t -> HiddenLabel
\end{verbatim}}
We illustrate this at the top-level loop. The \om\ top-level loop executes
{\tt IO} actions, and evaluates and prints
out the value of expressions with other types.

{\small
\begin{verbatim}
prompt> freshLabel
Executing IO action               -- An IO action
(Hidden `#cbp) : IO HiddenLabel


prompt> temp <- freshLabel        -- An IO action
Executing IO action
(Hidden `#sbq) : HiddenLabel
prompt> temp
(Hidden `#sbq) : HiddenLabel

prompt> newLabel "a"              -- A pure value
(Hidden `a) : HiddenLabel
\end{verbatim}}

\begin{exercise} \label{existVar}
A common use of labels is to name variables in a data structure used
to represent some object language as data. Consider the GADT
and an evaluation function over that object type.

{\small
\begin{verbatim}
data Expr:: *0 where
  VarExpr :: Label t -> Expr
  PlusExpr:: Expr -> Expr -> Expr

valueOf:: Expr -> [exists t .(Label t,Int)] -> Int
valueOf (VarExpr v) env = lookup v env
valueOf (PlusExpr x y) env = valueOf x env + valueOf y env
\end{verbatim}}
\noindent
Write the function: {\small {\tt lookup:: Label v -> [exists t .(Label t,Int)] -> Int}}.
\end{exercise}


\section{Maintaining Structural Invariants of Data}

Both {\tt Seq} and {\tt Tree} use kinds as indexes ({\tt Nat} for {\tt Seq}, and
{\tt Shape} for {\tt Tree}) to maintain an invariant about the shape of the data.
This is quite common. In this section we illustrate this in more detail by
examining the world of balanced trees.

\subsection{AVL Trees}


%%%%%%%%%%%%%%%%%%%%%%%%%%%%%%%%%%%%%%%%%%%%%%%%%%%%%%%%%%%%%%%%%%%%%%%%%%%%%%%%
% macros

%\newcommand{\avl}[1]{\includegraphics{avl.#1}}         % use with pdflatex
\newcommand{\avl}[1]{\includegraphics{avl.#1.eps}}      % use with latex
\newcommand{\lift}[1]{\raise 0.48cm \hbox{#1}}

%%%%%%%%%%%%%%%%%%%%%%%%%%%%%%%%%%%%%%%%%%%%%%%%%%%%%%%%%%%%%%%%%%%%%%%%%%%%%%%%
%% introduction

Binary search trees are a classic data structure for implementing 
finite maps or sets in a purely functional way.  To guarantee efficient
operations, we want our trees to be somewhat balanced.  There are several
ways to define what it means for a tree to be balanced, each leading to 
different data structures such as Red-Black trees, AVL trees, B-trees, etc.
In this section we implement AVL trees in such a way that \om's type system
guarantees compliance with the balancing invariant.

%%%%%%%%%%%%%%%%%%%%%%%%%%%%%%%%%%%%%%%%%%%%%%%%%%%%%%%%%%%%%%%%%%%%%%%%%%%%%%%%
\subsubsection{Types Expressing Invariants.}

The balancing invariant for AVL trees is simple: any internal node in the tree
has children whose heights differ by no more than one.  In this section, 
we define types that express this invariant.
Here is our core data structure for AVL trees (indexed by tree height).
{\small
\begin{verbatim}
data Avl :: Nat ~> *0 where
  Leaf :: Avl Z
  Node :: Balance hL hR hMax -> Avl hL -> Int -> Avl hR -> Avl (S hMax)
\end{verbatim}
}
\noindent
A binary tree has two constructors -- one for (empty) leaves 
and one for internal nodes carrying data.
An auxiliary type captures the balancing constraints.
{\small
\begin{verbatim}
data Balance:: Nat ~> Nat ~> Nat ~> *0 where
  Less :: Balance h     (S h) (S h)
  Same :: Balance h     h     h
  More :: Balance (S h) h     (S h)
\end{verbatim}
}
\noindent
Think of the type \verb|Balance hL hR hMax| as a predicate stating
(1) that \verb|hL| and \verb|hR| differ by at most one, and
(2) that \verb|hMax| is the maximum of \verb|hL| and \verb|hR|.
For any given internal node, there are only three possibilities for 
the relative heights of its subtrees:
\[
1 + \verb|hL| = \verb|hR| \quad \textrm{or} \quad
    \verb|hL| = \verb|hR| \quad \textrm{or} \quad
    \verb|hL| = \verb|hR| + 1 
\]

These three possibilities correspond to the three constructors of the
datatype \verb|Balance|.
Under this interpretation of \verb|Balance|, we see that the \verb|h| in 
\verb|(Avl h)| really does capture the height of the tree (leaves have height 
zero and the height of an internal node is the successor of the maximum of 
the heights of its children).

Finally, we would like to protect users of our library from having to 
deal with height indices in their own code.  To this end, we define a 
wrapper type that hides away the height index.

{\small
\begin{verbatim}
data AVL:: *0 where
  AVL:: (Avl h) -> AVL
\end{verbatim}
}
\noindent
In this type the {\tt h} is existentially quantified.
This is the type that users will see.

The {\tt data} declarations are all the code we ever need write to guarantee that every AVL tree 
in our library is well-balanced.  Because these type declarations express the
balancing invariants, the problem of deciding whether our implementation 
respects those invariants reduces to the problem of deciding type-correctness,
which the \om{} type-checker does for us automatically.

%%%%%%%%%%%%%%%%%%%%%%%%%%%%%%%%%%%%%%%%%%%%%%%%%%%%%%%%%%%%%%%%%%%%%%%%%%%%%%%%
\subsubsection{Basic operations.}

The two most basic operations are constructing an empty tree and testing 
an element for membership in the tree.
{\small
\begin{verbatim}
empty :: AVL
empty = AVL Leaf
\end{verbatim}

\begin{verbatim}
element :: Int -> AVL -> Bool
element x (AVL t) = elem x t

elem :: Int -> Avl h -> Bool
elem x Leaf = False
elem x (Node _ l y r)
  | x == y  = True
  | x < y   = elem x l
  | x > y   = elem x r
\end{verbatim}
}
\noindent
The remaining operations of insertion and deletion are much more interesting.

%%%%%%%%%%%%%%%%%%%%%%%%%%%%%%%%%%%%%%%%%%%%%%%%%%%%%%%%%%%%%%%%%%%%%%%%%%%%%%%%
\subsubsection{Balancing Constructors.}

The algorithms for insertion and deletion each follow the same basic pattern: 
First do the insertion (or deletion) as you would for any other binary 
search tree.  Then re-balance any subtree that became unbalanced in the 
process.  The tool used for re-balancing is tree rotation, which is best
described visually.
\begin{center}
\begin{tabular}{m{80pt}m{2em}m{80pt}}
\avl{0} & 
\vbox{\hbox{$\Longrightarrow$}\hbox{$\Longleftarrow$}} & 
\avl{1} 
\end{tabular}
\end{center}
The transformation of the tree on the left to the tree on the 
right is \emph{right rotation} and the opposite transformation 
is called \emph{left rotation}.  This operation preserves the BST
invariant.   However, they do \emph{not} preserve the balancing invariant, 
which is precisely why they are useful for rebalancing.

It turns out that we can package up all necessary rotations into a
couple of \emph{smart constructors}, \verb|rotr| and \verb|rotl|.
Think of \verb|rotr lc x rc| as a smart version of \verb|Node ? lc x rc| where
\begin{enumerate}
\item We don't have to say (or know) how the resulting tree is balanced, and
\item The subtrees, \verb|lc| and \verb|rc|, don't quite balance out because
        \verb|height(lc)| $=$ \verb|height(rc) + 2|
     and therefore we must do some rightward rebalancing rotation(s).
\end{enumerate}


\begin{figure*}
{\tt \small
\begin{tabular}{ccc}
rotr (Node Same a x b) y c & = & R(Node Less a x (Node More b y c)) \\ 
\avl{2} & & \avl{3} \\ \\
rotr (Node More a x b) y c & = & L(Node Same a x (Node Same b y c)) \\ 
\avl{4} & & \lift{\avl{5}} \\ \\
{
\parbox{5.5cm}{
rotr (Node Less a x \\
\phantom{rotr (Node)}(Node Same b y c)) z d} }
 & = &
{
\parbox{5.5cm}{
         L(Node Same  (Node Same a x b) y \\
\phantom{L(Node Same)}(Node Same c z d) )} } \\ \\
\avl{6} & & \lift{\avl{7}} \\ 
%rotr (Node Less a x (Node Less b y c)) z d
{
\parbox{5.5cm}{
rotr (Node Less a x \\
\phantom{rotr (Node)}(Node Less b y c)) z d} }
 & = &
%L(Node Same (Node More a x b) y (Node Same c z d)) \\ \\
{
\parbox{5.5cm}{
         L(Node Same  (Node More a x b) y \\
\phantom{L(Node Same)}(Node Same c z d) )} } \\ \\
\avl{8} & & \lift{\avl{9}} \\ \\
%rotr (Node Less a x (Node More b y c)) z d
{
\parbox{5.5cm}{
rotr (Node Less a x \\
\phantom{rotr (Node)}(Node More b y c)) z d} }
 & = &
%L(Node Same (Node Same a x b) y (Node Less c z d)) \\ \\
{
\parbox{5.5cm}{
         L(Node Same  (Node Same a x b) y \\
\phantom{L(Node Same)}(Node Less c z d) )} } \\ \\
\avl{10} & & \lift{\avl{11}} \\
\end{tabular}
}
\caption{Each substantive case in the definition of {\tt rotr}.}
\label{fig:rotr}
\end{figure*}


The only wrinkle in the ``smart constructor'' story is that the height of the
resulting tree depends on what rotations were performed.  However, the result
height ranges over merely two values, so we just return a value of a sum 
type\footnote{In \om{} the value constructors {\tt L ::~a -> (a+b)} 
and {\tt R ::~b -> (a+b)} are used to construct sums.}.
Here is the code:

{\small
\begin{verbatim}
rotr :: Avl (2+n)t -> Int -> Avl n -> ( Avl(2+n)t + Avl (3+n)t )
rotr Leaf x a = unreachable
rotr (Node Less a x Leaf) y b = unreachable
-- single rotations
rotr (Node Same a x b) y c = R(Node Less a x (Node More b y c))
rotr (Node More a x b) y c = L(Node Same a x (Node Same b y c))
-- double rotations
rotr (Node Less a x (Node Same b y c)) z d = 
   L(Node Same (Node Same a x b) y (Node Same c z d))
rotr (Node Less a x (Node Less b y c)) z d = 
   L(Node Same (Node More a x b) y (Node Same c z d))
rotr (Node Less a x (Node More b y c)) z d = 
   L(Node Same (Node Same a x b) y (Node Less c z d))
\end{verbatim}
}
\noindent
Figure~\ref{fig:rotr} depicts the rotation for each substantive case in the
        definition of \verb|rotr|.
The algorithm for \verb|rotl| is perfectly symmetric to that for \verb|rotr|.

{\small
\begin{verbatim}
rotl :: Avl n -> Int -> Avl (2+n)t -> ( Avl (2+n)t + Avl (3+n)t )
rotl a x Leaf = unreachable
rotl a x (Node More Leaf y b) = unreachable
-- single rotations
rotl a u (Node Same b v c) = R(Node More (Node Less a u b) v c)
rotl a u (Node Less b v c) = L(Node Same (Node Same a u b) v c)
-- double rotations
rotl a u (Node More (Node Same x m y) v c) = 
   L(Node Same (Node Same a u x) m (Node Same y v c))
rotl a u (Node More (Node Less x m y) v c) = 
   L(Node Same (Node More a u x) m (Node Same y v c))
rotl a u (Node More (Node More x m y) v c) = 
   L(Node Same (Node Same a u x) m (Node Less y v c))
\end{verbatim}
}
\noindent
As these functions are both self-contained and non-recursive, we see that
they operate in constant time.

%%%%%%%%%%%%%%%%%%%%%%%%%%%%%%%%%%%%%%%%%%%%%%%%%%%%%%%%%%%%%%%%%%%%%%%%%%%%%%%%
\subsubsection{Insertion.}

When we insert an element into an AVL tree, the height of the
tree either remains the same or increases by at most one.
We therefore arrive at the following type for insertion:

{\small
\begin{verbatim}
ins :: Int -> Avl n -> (Avl n + Avl (S n))
\end{verbatim}
}
The code for \verb|ins| is an elaborate case analysis.
The first decision to make is whether we're at the right spot for 
insertion.  If so, then do the insertion (or not, depending on whether 
the value already exists in the tree), and then return.  If not, make the 
appropriate recursive call and then rebalance.
Most of the work goes into determining how to rebuild a balanced tree 
by choosing the correct \verb|Balance| value or rebalancing constructor.

{\small
\begin{verbatim}
ins :: Int -> Avl n -> (Avl n + Avl (S n))
ins x Leaf = R(Node Same Leaf x Leaf)
ins x (Node bal a y b)
  | x == y = L(Node bal a y b)
  | x < y  = case ins x a of
               L a -> L(Node bal a y b)
               R a ->
                 case bal of
                   Less -> L(Node Same a y b)
                   Same -> R(Node More a y b)
                   More -> rotr a y b -- rebalance!
  | x > y  = case ins x b of
               L b -> L(Node bal a y b)
               R b -> 
                 case bal of
                   Less -> rotl a y b -- rebalance!
                   Same -> R(Node Less a y b)
                   More -> L(Node Same a y b)
\end{verbatim}
}
\noindent
Figure~\ref{fig:ins} depicts each case in the \verb|x < y| branch.
Now we wrap this function up to work on user-level AVL trees.

{\small
\begin{verbatim}
insert :: Int -> AVL -> AVL
insert x (AVL t) = case ins x t of L t -> AVL t; R t -> AVL t
\end{verbatim}
}

\begin{figure*}
\begin{center}
\begin{tabular}{ccc}
INPUT & & OUTPUT \\ \\
\avl{12} & \lift{\lift{\parbox{2.4in}{\begin{center} Post-insertion height is the same. \\ Keep the same {\tt Balance}\end{center}}}} & \avl{13}
\\ \\
\lift{\avl{14}} & \lift{\lift{\parbox{2.4in}{\begin{center} Height increases. \\ Change {\tt Balance} from {\tt Same} to {\tt More} \end{center}}}} & \avl{15}
\\ \\
\avl{16} & \lift{\lift{\parbox{2.4in}{\begin{center} Height increases. \\ Change {\tt Balance} from {\tt Less} to {\tt Same} \end{center}}}} & \avl{17}
\\ \\
\lift{\avl{18}} & \lift{\lift{\lift{\parbox{2.4in}{\begin{center} Height increases. \\ Rebalance with {\tt rotr} \end{center}}}}} & \avl{19}
\\ \\
\end{tabular}
\caption{Rebalancing after insertion into the left child.}
\label{fig:ins}
\end{center}
\end{figure*}

\hide{
Consider the case marked by the comment above. Here the variables in scope
have the following types.
{\small
\begin{verbatim}
tree :: Avl (2+_b)t
a :: Avl _b
rc :: Avl (1+_b)t
x :: Avl _b
\end{verbatim}
}
\noindent
The resulting tree must have either the type \verb$(Avl (2+_b)t)$
or \verb$(Avl (3+_b)t)$. One way to construct such a result is
{\small
\begin{verbatim}
Less x y rc :: Avl (2+_b)t
\end{verbatim}
}
\noindent
Thus we use the left injection (\verb+L+) of the sum type to return
a value with the correct type. Almost all of the other cases are similar.
There are two interesting cases that are not so simple, labeled 
{\tt Rotate Right} and {\tt Rotate Left} in comments. Consider the
first lableled {\tt Rotate Left}. Here the variables in scope
have the types.
{\small
\begin{verbatim}
tree :: Avl (2+_b)t
a :: Avl _b
rc :: Avl (1+_b)t
tSn :: Avl (2+_b)t
\end{verbatim}
}
}% END \hide

%%%%%%%%%%%%%%%%%%%%%%%%%%%%%%%%%%%%%%%%%%%%%%%%%%%%%%%%%%%%%%%%%%%%%%%%%%%%%%%%
\subsubsection{Deletion.}

Whereas insertion always places an element in the fringe of a 
tree, deletion may find the targeted element somewhere in the interior.
For this reason, deletion is a more complex operation.  The strategy for 
deleting the value $x$ at an interior node is to first replace its value with 
that of the minimum value $z$ of its right child (or the maximum value of its
left child, depending on the policy).  Then delete $z$ (which is always at a leaf) 
from the right child.

We will calculate the minimum value in a tree and delete it in a single pass.
The operation only works on trees of height $\geq 1$ (which are therefore
non-empty).  The returned tree might have decreased in size by one.

{\small
\begin{verbatim}
delMin :: Avl (S n) -> (Int, (Avl n + Avl (S n)))
delMin Leaf = unreachable
delMin (Node Less Leaf x r) = (x,L r)
delMin (Node Same Leaf x r) = (x,L r)
delMin (Node More Leaf x r) = unreachable
delMin (Node bal (l@(Node _ _ _ _)) x r) =
      case delMin l of
        (y,R l) -> (y,R(Node bal l x r))
        (y,L l) ->
          case bal of
            Same -> (y,R(Node Less l x r))
            More -> (y,L(Node Same l x r))
            Less -> (y,rotl l x r) -- rebalance!
\end{verbatim}
}
\noindent
Deletion of the minimum element requires rebalancing operations on the way 
up, just as in insertion.

When we delete an element from an AVL tree, the height of the
tree either remains the same or decreases by at most one.
We therefore arrive at the following type for deletion:

{\small
\begin{verbatim}
del :: Int -> Avl (S n) -> (Avl n + Avl (S n))
\end{verbatim}
}
The code for \verb|del| is an elaborate case analysis.
The first decision to make is whether we're at the right spot for 
deletion.  If so, then do the deletion (or not, depending on whether 
the value exists in the tree) and return.  If not, make the appropriate 
recursive call and then rebalance.
Most of the work goes into determining how to rebuild a balanced tree 
by choosing the correct \verb|Balance| value or rebalancing constructor.

{\small
\begin{verbatim}
del :: Int -> Avl n -> (Avl n + exists m .(Equal (S m) n,Avl m))
del y Leaf = L Leaf
del y (Node bal l x r)
  | y == x = case r of
               Leaf ->  
                 case bal of
                   Same -> R(Ex(Eq,l))
                   More -> R(Ex(Eq,l))
                   Less -> unreachable
               Node _ _ _ _ ->
                 case (bal,delMin r) of
                   (_,z,R r) -> L(Node bal l z r)
                   (Same,z,L r) -> L(Node More l z r)
                   (Less,z,L r) -> R(Ex(Eq,Node Same l z r))
                   (More,z,L r) -> 
                      case rotr l z r of -- rebalance!
                        L t -> R(Ex(Eq,t))
                        R t -> L t
  | y < x = case del y l of
              L l -> L(Node bal l x r)
              R(Ex(Eq,l)) -> 
                case bal of
                  Same -> L(Node Less l x r)
                  More -> R(Ex(Eq,Node Same l x r))
                  Less -> 
                     case rotl l x r of -- rebalance!
                       L t -> R(Ex(Eq,t))
                       R t -> L t
  | y > x = case del y r of
              L r -> L(Node bal l x r)
              R(Ex(Eq,r)) -> 
                case bal of
                  Same -> L(Node More l x r)
                  Less -> R(Ex(Eq,Node Same l x r))
                  More -> 
                     case rotr l x r of -- rebalance!
                       L t -> R(Ex(Eq,t))
                       R t -> L t
\end{verbatim}
}

\hide{
AN ALTERNATIVE \ldots
{\small
\begin{verbatim}
data DelAns:: Nat ~> *0 where
 SameHeight :: Avl n -> DelAns n
 ShrunkBy1 :: Avl n -> DelAns (S n)

del :: Int -> Avl n -> DelAns n
del y Leaf = SameHeight Leaf
del y (Node bal l x r)
  | y==x = case r of
             Leaf ->  
               case bal of
                 Same -> ShrunkBy1 l
                 More -> ShrunkBy1 l
                 Less -> unreachable
             Node _ _ _ _ ->
               case (bal,delMin r) of
                 (_,z,R r) -> SameHeight(Node bal l z r)
                 (Same,z,L r) -> SameHeight(Node More l z r)
                 (Less,z,L r) -> ShrunkBy1(Node Same l z r)
                 (More,z,L r) -> 
                    case rotr l z r of -- rebalance!
                      L t -> ShrunkBy1 t
                      R t -> SameHeight t
\end{verbatim}
}
}
\noindent
Now we wrap this function up to work on user-level AVL trees.

{\small
\begin{verbatim}
delete :: Int -> AVL -> AVL
delete x (AVL t) = case del x t of L t -> AVL t; R t -> AVL t
\end{verbatim}
}


\begin{exercise}

{\bf Red Black Trees.}
A red-black tree is a binary search tree with the following additional invariants:

\begin{enumerate}
\item Each node is colored either red or black
\item The root is black
\item The leaves are black
\item Each Red node has Black children \label{RedKidsAreBlack}
\item For all internal nodes, each path from that node to a descendant 
        leaf contains the same number of black nodes.
\end{enumerate}

We can encode these invariants by thinking of each internal node as having two
attributes: a color and a black-height. We will use a GADT, we call {\tt SubTree},
with two indexes, one of them a {\tt Nat} (for the black-height) and the other a
{\tt Color}.

{\small
\begin{verbatim}
data Color:: *1 where 
  Red:: Color
  Black:: Color

data SubTree:: Color ~> Nat ~> *0 where 
 Leaf:: SubTree Black Z
 RNode:: SubTree Black n -> Int -> SubTree Black n -> SubTree Red n
 BNode:: SubTree cL m    -> Int -> SubTree cR m    -> SubTree Black (S m)

data RBTree:: *0 where
 Root:: SubTree Black n -> RBTree
\end{verbatim}}
\noindent
Note how the black height increases only on black nodes. The type
{\tt RBTree} encodes a ``full" Red-Black tree, forcing the root
to be black, but placing no restriction on the black-height.
Write an insertion function for Red-Black trees.
A solution to this exercise is found in Appendix \ref{redblack}.
\end{exercise}

\section{\om\ as a Meta Language}

It has become common practice when designing a new language to study the
relationship between a static semantics (a type system) and a dynamic
semantics (a meaning function). This process is often exploratory. The
designer has an idea, the approach is analyzed, and hopefully the
consequences of the approach are quickly discovered. Automated aid in this
process would be a great boon.

The ultimate goal of this exploratory process is a type system, a semantics,
and a proof. The proof witnesses the fact that {\em well-typed programs do not
go wrong}\cite{Milner78} for the language under consideration. The most common
way to perform such a proof is by a subject reduction proof in the style of
Wright and Felleisen\cite{Wright:94} on a small step semantics, though there
other approaches as well\cite{Milner78,DamasThesis}. Such proofs require an amazing amount of
detail and are most often carried out by hand, and are thus subject to all the
foils of human endeavors. 

\om\ is our attempt at developing a generic meta-language that could be used
for exploring the static and dynamic semantics for new
object-languages\cite{SheardPasalic2002,Sheard01} that
could aid in the generation of such proofs. This section describes how 
\om\ can be used as a meta-language. We show that:


\begin{itemize}

\item Much of the work of exploring the nuances of
a type system for a new language can be assisted by using mechanized tools 
-- a generic meta-language.

\item Such tools need not be much more complicated than your favorite
functional language (Haskell), and are thus within the reach of most language researchers.

\item The automation helps language designers visualize the consequences
of their design choices quickly, and thus helps speed the design process.

\item The artifacts created by this exploration are machine checked proofs,
and are hence less subject to error than proofs constructed by the more
traditional approach.

\end{itemize}

\subsection{Object Languages}

In meta-programming systems meta-programs manipulate object-programs. 
Meta-programs may construct object-programs, combine object-program 
fragments into larger object-programs, observe the structure and 
other properties of object-programs, and execute object-programs to 
obtain their values.

There are several important kinds of meta-programming scenarios: program generators,
and program analyses. Each of these scenarios has a number of distinguishing
characteristics.

A program generator (a meta-program) solves a particular problem by constructing
another program (an object-program) that solves the problem at hand. Usually the
generated (object) program is ``specialized" for the particular problem and uses
less resources than a general purpose, non-generator solution.

A program analysis (a meta-program) observes the structure and environment of an
object-program and computes some value as a result. Results can be data- or
control-flow graphs, or even another object-program with properties based on the
properties of the source object-program. Examples of these kinds of meta-systems
are: program transformers, optimizers, and partial evaluation systems.

A language model (a meta-program) gives meaning to, and points out properties
of an object-language. Examples of these include type systems, type judgments,
denotational and operational semantics, and small-step semantics.

\subsection{Representing Object Programs}

Meta-programs must represent object-programs as data. Object program
representations usually fall into one of three categories. (1) Strings, (2) Algebraic
datatypes, or (3) Quasi-quote systems. Other representations
(as graphs for example) are possible, but not widespread.

With the string encoding, we represent the code fragment {\tt f(x,y)} simply as
\verb+"f(x,y)"+. While constructing and combining fragments represented by strings can
be done concisely, deconstructing them is quite verbose, and in essence
degenerates into a parsing problem. More seriously, there is
no automatically verifiable guarantee that programs thusly constructed are
syntactically correct. For example, \verb+"f(,y)"+ can have the static type {\tt string}, but
this clearly does not imply that this string represents a syntactically correct
program.

\subsection{Object-programs as Algebraic Datatypes}
With the Algebraic datatype encoding, we can address the syntactic correctness problem. A
datatype encoding is essentially the same as what is called abstract syntax or
parse trees. The encoding of the fragment \verb+plus(x,y)+ in an \om\ datatype might be:\\
{\small \verb+Apply Plus (Tuple [Variable "x" ,Variable "y"])+}\\ using a datatype declared
as follows:

{\small 
\begin{verbatim}
data Exp:: *0 where
  Variable:: String  -> Exp    -- x
  Constant:: Int -> Exp        -- 5
  Plus:: Exp                   -- plus
  Less:: Exp                   -- less
  Apply:: Exp -> Exp -> Exp    -- Apply Plus (x,y)
  Tuple:: [Exp] -> Exp         -- (x,y)
\end{verbatim}}
Using a datatype encoding has an immediate benefit: correct typing for the
meta-program ensures correct syntax for all object-programs. Because \om\
(like most functional languages) supports
pattern matching over datatypes, deconstructing programs becomes easier than with
the string representation. However, constructing programs is now more verbose
because we must use the cumbersome constructors like Variable, Apply, and Tuple.

\subsection{Representing Programs using Quasi-quotes}
Quasi-quotation is an attempt to represent object-programs
without cumbersome constructor functions. Here
the actual representation of object-code is hidden from the user by the means of
a quotation mechanism. Object code is constructed by placing ``quotation"
annotations around normal code fragments. 
The quasi-quotation approach is the approach used in MetaML,
Template Haskell, and the staged fragment of \om.

In the staged fragment of \om\ (Section \ref{stage}), quasi-quotations are called staging annotations, 
and include Brackets \verb+[| |]+ and Escape \verb+$+.
An expression \verb+[| e |]+ is a quotation, and it builds the code
representation of \verb+e+ (a data structure); \verb+$(e)+ is an anti-quotation,
and splices the code obtained by evaluating
\verb+e+ into the body of a surrounding bracketed expression (embedding one data structure into
another). The quotation and anti-quotation mechanism abstracts
the actual data-type representing code.

In a quasi-quoted system, the meta-language may now enforce the type-correctness
of the object language as well as the meta-language,
and avoid the problems associated with a constructor based approach.
The major disadvantages of quasi-quoted systems are

\begin{itemize}

\item There is usually only a single object-language, and it must
be built into the meta-language.

\item The quasi-quote mechanism is great for constructing code,
but less useful for taking code apart, especially code
with binding constructs.

\item The type system of the meta-language must be aware of the type system
of the object language. Usually this is accomplished by making
the meta-language and the object-language the same language. Heterogeneous
quasi-quote systems are rare because of this.

\end{itemize}

In the remainder of this section, we eschew the quasi-quote mechanism
in favor of using GADTs in an effort to address these disadvantages.

\subsection{Interpreters in a Typed Meta-language}

Often one would like to build an interpreter or evaluation function
for an object-language. In a typed meta-language, it is
necessary to define a {\tt Value} domain, that is a labeled sum of
all the possible result types of evaluating an expression. In the {\tt Exp}
type above this would include both integers and booleans (as these are the 
types of the ranges of the functions {\tt Plus} and {\tt Less}), as
well as functions and tuples.

{\small 
\begin{verbatim}
data Value :: *0 where
  IntV:: Int -> Value
  BoolV:: Bool -> Value
  FunV:: (Value -> Value) -> Value
  TupleV :: [Value] -> Value
\end{verbatim}}

The evaluation function is then a case analysis over the
structure of terms, recursively evaluating sub-terms into
values, and then combining the sub-values into 
answer values.

{\small 
\begin{verbatim}
eval:: (String -> Value) -> Exp -> Value
eval env (Variable s) = env s
eval env (Constant n) = IntV n
eval env Plus = FunV plus
  where plus (TupleV[IntV n ,IntV m]) = IntV(n+m)
eval env Less = FunV plus
  where plus (TupleV[IntV n ,IntV m]) = BoolV(n < m) 
eval env (Apply f x) = 
  case eval env f of
    FunV g -> g (eval env x)
eval env (Tuple xs) = TupleV(map (eval env) xs)    
\end{verbatim}}
\noindent
The key observation is -- there is considerable overhead in such a function. It must first
interpret the structure of the expressions, and it must perform quite a bit of tagging
and un-tagging by applying the {\tt Value} constructors ({\tt IntV}, {\tt BoolV}, {\tt
FunV}, and {\tt TupleV}), and deconstructing them when appropriate.

\subsection{Staging an Interpreter}

We may remove the interpretive overhead by using staging.
Like the evaluation function, the staged evaluation function is a case analysis over the
structure of terms, recursively evaluating sub-terms into
code values, and then splicing the smaller code values into
larger code values. 

{\small 
\begin{verbatim}
stagedEval:: (String -> Code Value) -> Exp -> Code Value
stagedEval env (Variable s) = env s
stagedEval env (Constant n) = lift(IntV n)
stagedEval env Plus = [| FunV plus |]
  where plus (TupleV[IntV n ,IntV m]) = IntV(n+m)
stagedEval env Less = [| FunV less |]
  where less (TupleV[IntV n ,IntV m]) = BoolV(n < m) 
stagedEval env (Apply f x) = 
      [| apply $(stagedEval env f) $(stagedEval env x) |]
  where apply (FunV g) x = g x
stagedEval env (Tuple xs) = [| TupleV $(mapLift (stagedEval env) xs) |]
  where mapLift f [] = lift []
        mapLift f (x:xs) = [| $(f x) : $(mapLift f xs) |]
\end{verbatim}}
We may observe the result of staging by applying {\tt stagedEval} to
an actual {\tt Exp}.

{\small 
\begin{verbatim}
exp1 = Apply Plus (Tuple [Variable "x" ,Variable "y"]) -- (+)(x,y)

ans = stagedEval f exp1 
  where f "x" = lift(IntV 3)
        f "y" = lift(IntV 4)
        

ans = [| %apply (%FunV %plus) (%TupleV [IntV 3,IntV 4]) |] : Code Value  
\end{verbatim}}

We have removed the interpretive overhead, but the tagging and untagging
overhead remains. This overhead is caused by using a disjoint sum as
the range of the evaluator, which is necessary in a typed meta-language.
This not the only problem when using algebraic datatypes to encode
object-languages in a strongly typed meta-language like Haskell. The
algebraic datatype approach to encoding object-languages does not track
the type correctness of the object-program. We will fix both these
problems by representing object-programs using GADTs rather than
Algebraic datatypes.

\subsection{Typed Object-languages using GADTs} \label{tagged}

GADTs allow
us to build datatypes indexed by another type. We can use the
GADT to represent object programs (just as we use algebraic
datatype to represent object programs), but we may also
use the type index to represent the type of the object-language
program being represented. A simple typed object-language example is:

{\small
\begin{verbatim}
data Term:: *0 ~> *0 where
  Const :: Int -> Term Int               -- 5
  Add:: Term ((Int,Int) -> Int)          -- (+)
  LT:: Term ((Int,Int) -> Bool)          -- (<)
  Ap:: Term(a -> b) -> Term a -> Term b  -- (+) (x,y)
  Pair:: Term a -> Term b -> Term(a,b)   -- (x,y)
\end{verbatim}}

Above we introduced the new type constructor {\tt Term}, which is a
representation of a simple object-language of constants, pairs, and numeric
operators. 
{\tt Term}s are a typed object-language representation, i.e. a data
structure that represents terms in some object-language. The meta-level
type of the representation, i.e. the {\tt a} in ({\tt Term a}), indicates the type 
of the object-level term. This is made possible by the flexibility of
the GADT mechanism. Using typed object-level terms,
it is impossible to construct ill-typed term representations, because
the meta-language type system enforces this constraint.

{\small
\begin{verbatim}
ex1 :: Term Int  
ex1 = Ap Add (Pair (Const 3) (Const 5))

ex2 :: Term (Int,Int)
ex2 = Pair ex1 (Const 1)
\end{verbatim}}
Attempting to construct an ill-typed object term, like {\tt (Ap (Const 3) (Const 5))},
causes a meta-level (\om) type error. Another advantage of using
GADTs rather than ADTs is that it is now possible to construct a 
tagless\cite{SheardPasalic2002,Taha:2001:TEJ,TahaTag2000}
interpreter directly:

{\small
\begin{verbatim}
evalTerm :: Term a -> a
evalTerm (Const x) = x
evalTerm Add = \ (x,y) -> x+y
evalTerm LT = \ (x,y) -> x<y
evalTerm (Ap f x) = evalTerm f (evalTerm x)
evalTerm (Pair x y) = (evalTerm x,evalTerm y)
\end{verbatim}}
In a language without GADTs, as we illustrated
in Section \ref{tagged}, we would need to employ universal value domain like
{\tt Value}. See \cite{PasalicLingerGpce} 
for a detailed discussion of this phenomena. Such a tagless interpreter
has the structure of a large step (or operational) semantics. If the {\tt eval}
function is total and well-typed at the meta-level, it implies
that the object-level semantics (defined by {\tt eval}) is also well-typed.
Every well-typed object level term evaluates to a well-formed value.

\begin{exercise} \label{objectVar}

In the object-languages we have seen so far, there are no variables. One
way to add variables to a typed object language is to add a variable
constructor tagged by a name and a type. A singleton type representing
all the possible types of a program term is necessary. For example
we may add a {\tt Var} constructor as follows (where the {\tt Rep} 
is similar to the {\tt Rep} type from Exercise \ref{reptype}).

{\small
\begin{verbatim}
data Term:: *0 ~> *0 where
  Var:: String -> Rep t -> Term t        -- x
  Const :: Int -> Term Int               -- 5
  . . .
\end{verbatim}}
Write a GADT for {\tt Rep}.
Now the evaluation function for {\tt Term}
needs an environment that can store
many different types. One possibility is
use existentially quantified types
in the environment as we did in
Exercise \ref{existVar}. Something like:

{\small
\begin{verbatim}
type Env = [exists t . (String,Rep t,t)]  

eval:: Term t -> Env -> t
\end{verbatim}}
Write the evaluation function for the {\tt Term}
type extended with variables. You will need a function
akin to {\tt sameNat} from Exercise \ref{samenat},
except it will have prototype: 
{\tt sameRep:: Rep a -> Rep b -> Maybe(Equal a b)}
\end{exercise}

\begin{exercise} \label{countingvars}
Another way to add variables to a typed object language is
to reflect the name 
and type of variables in the meta-level types of the terms in which they occur. Consider the GADTs:

{\small
\begin{verbatim}
data VNum:: Tag ~> *0 ~> Row Tag *0 ~> *0 where
  Zv:: VNum l t (RCons l t row)
  Sv:: VNum l t (RCons a b row) -> VNum l t (RCons x y (RCons a b row))
 deriving Nat(u)
 
data Exp2:: Row Tag *0 ~> *0 ~> *0 where
  Var:: Label v -> VNum v t e -> Exp2 e t
  Less:: Exp2 e Int -> Exp2 e Int -> Exp2 e Bool
  Add:: Exp2 e Int -> Exp2 e Int -> Exp2 e Int
  If:: Exp2 e Bool -> Exp2 e t -> Exp2 e t -> Exp2 e t
\end{verbatim}}
What are the types of the terms {\tt (Var `x 0u)}, {\tt (Var `x 1u)}, and
{\tt (Var `x 2u)}. Now the evaluation function for {\tt Exp2}
needs an environment that stores both integers and booleans.
Write a datatype declaration for the environment, and then
write the evaluation function. One way to approach this
is to use existentially quantified types
in the environment as we did in
Exercises \ref{existVar} and \ref{objectVar}. Better mechanisms exist. Can you think
of one?
\end{exercise}

\subsection{Tagless Staged Interpreters}

By staging an object-level type indexed GADT we can remove both
the interpretive and tagging overhead.

{\small
\begin{verbatim}
stagedEvalTerm :: Term a -> Code a
stagedEvalTerm (Const x) = lift x
stagedEvalTerm Add = [| add |]
  where add (x,y) = x+y
stagedEvalTerm LT = [| less |]
  where less (x,y) = x < y
stagedEvalTerm (Ap f x) = [| $(stagedEvalTerm f) $(stagedEvalTerm x) |]
stagedEvalTerm (Pair x y) = [|($(stagedEvalTerm x),$(stagedEvalTerm y))|]

ex2 = (Pair (Ap Add (Pair (Const 3) (Const 5))) (Const 1))
\end{verbatim}}
We can stage a program like {\tt ex2} by applying {\tt stagedEvalTerm}
to produce some code. For {\tt ex2} we get: {\tt [| (add (3, 5), 1) |]}.
Note that both the interpretive overhead, and the tagging overhead, have
been completely removed.

\begin{exercise}
A staged evaluator is a simple compiler. Many compilers
have an optimization phase. Consider the term
language with variables from Exercise \ref{objectVar}.

{\small
\begin{verbatim}
data Term:: *0 ~> *0 where
  Var:: String -> Rep t -> Term t
  Const :: Int -> Term Int               -- 5
  Add:: Term ((Int,Int) -> Int)          -- (+)
  LT:: Term ((Int,Int) -> Bool)          -- (<)
  Ap:: Term(a -> b) -> Term a -> Term b  -- (+) (x,y)
  Pair:: Term a -> Term b -> Term(a,b)   -- (x,y)
\end{verbatim}}
Can you write a well-typed staged
evaluator the performs optimizations like
constant folding, and applies laws like $(x+0) = x$
before generating code?
\end{exercise}

\subsection{A Typed-Object Language with Binding}\label{binding}

Object languages with variables and binding structures are harder
to represent in a way that reflects the type of the object-language
term in the type of its meta-language representation. 

This is because if we change the type of the object-level variables,
the type of the whole object-level term may also change. The key
to this dilemma is to represent the type of the free variables
in a term, as well as the type of the term, in the type of its
meta-level representation. We do this by indexing terms by two
indexes: first, the terms object-level type, and second, a type
level structure encoding the environment (i.e. a mapping from
variables to their types) in which the term has that type.

If we represent variables by labels (see Section \ref{tag}), we can
represent the environment by a row. A {\tt Row} is nothing
more than a list-like structure (storing pairs of elements at each ``cons" node) at the
type level (see Exercise \ref{ExRow}).

{\small
\begin{verbatim}
data Row :: a ~> b ~> *1 where
   RNil :: Row x y
   RCons :: x ~> y ~> Row x y ~> Row x y
 deriving Record(r)
\end{verbatim} }
For example the type: \verb+(RCons 3t Int RNil)+ is classified by {\tt (Row Nat *0)}.
Note that we have defined a syntactic extension for rows tagged by {\tt r}. Thus
\verb+(RCons 3t Int RNil)+ will display as \verb+{3t=Int}r+. An environment is just a
type classified by {\tt (Row Tag *0)}. We define a new value level type, {\tt Lam},
indexed by environments (represented by  {\tt (Row Tag *0)}) and types (represented by {\ *0}).

{\small
\begin{verbatim}
data Lam:: Row Tag *0 ~> *0 ~> *0  where
  Var   :: Label s -> Lam (RCons s t env) t
  Shift :: Lam env t -> Lam (RCons s q env) t
  Abs   :: Label a -> Lam (RCons a s env) t -> Lam env (s -> t)
  App   :: Lam env (s -> t) -> Lam env s -> Lam env t
\end{verbatim}}

The first index to {\tt Lam}, is a Row tracking its variables,
and the second index, tracks the object-level type of the term.
For example a term with variables {\tt x} and {\tt y} might have type
\verb+Lam {`x:Int, `y:Bool; u}r Int+.

The key to this approach is the typing of the constructor functions for variables ({\tt
Var}) and lambda expressions ({\tt Abs}). Consider the {\tt Var} constructor
function. To construct a variable we simply apply {\tt Var} to a label, and its type
reflects this. For example here is the output from a short interactive session with the
\om~ interpreter.

{\small
\begin{verbatim}
prompt> Var `name
(Var `name) : forall a (b:Row Tag *0).Lam {`name=a; b}r a
   
prompt> Var `age
(Var `age) : forall a (b:Row Tag *0).Lam {`age=a; b}r a
\end{verbatim}}

Variables are really De Bruijn-like in their behavior. Variables created
with {\tt Var} all have index level 0. The two examples have different
names in the same index position, and they would clash if they were both
used in the same lambda term. To shift the position of variable to a
different index, we use the constructor \verb+Shift:: Lam a b -> Lam {c=d; a}r b+
(see Exercise \ref{objectVar} for an alternative mechanism to distinguish
variables). To define two variables {\tt x} and {\tt y} for use in the
same environment we shift one of them into a different index. We type a
few examples at the \om\ top-level loop to illustrate the this.

{\small
\begin{verbatim}
prompt> Var `x
(Var `x) : Lam {`x=a; b}r a 

prompt> Shift(Var `y)
(Shift (Var `y)) : Lam {a=b,`y=c; d}r c

prompt> Shift (Shift (Var `z))
(Shift (Shift (Var `z))) : Lam {a=b,c=d,`z=e; f}r e
\end{verbatim}} 
\noindent
A {\tt Lam} term represented by {\tt (Var `x)} has the tag {\tt `x} appearing
as the first element in the environment row. By applying {\tt Shift}
once, the tag {\tt `x} is pushed into the second element of the row,
a second {\tt Shift} pushes it into the third element, etc.

The {\tt Abs} constructor binds the first tag in the
first element of the row, removing the tag and its associated type
from the environment, and shifting the others towards the
front of the environment.

{\small
\begin{verbatim}
prompt> App (Var `a) (Shift (Var `b))
(App (Var `a) (Shift (Var `b))) : Lam {`a=a -> b,`b=a; c}r b

prompt> Abs `f (Abs `x (App (Shift (Var `f)) (Var `x)))
(Abs `f (Abs `x (App (Shift (Var `f)) (Var `x)))) 
    : Lam a ((b -> c) -> b -> c)
\end{verbatim}} 
\noindent

Note how terms with free variables have non-trivial environment
indexes which mention their free variables. 
For example the first term's type is indexed
by the {\tt Row}: {\small {\tt \{`a=a -> b,`b=a; c\}r}} indicating
that both {\tt `a} and {\tt `b} are free variables in the term.
To build an evaluator
for an object-level typed term (Section \ref{tagglessvars}), we
will need a data structure, pairing variables with their values,
for each free variable in the term. We can package up a set of
these values using a record.

A {\tt Record} structure is a labeled tuple. We use
the labels to name the variables. A Record is a level 0
value. Its type is indexed by the level 1 type {\tt Row}.
We can define this data structures as follows.

{\small
\begin{verbatim}
data Record :: Row Tag *0 ~> *0 where
    RecNil :: Record RNil
    RecCons :: Label a -> b -> Record r -> Record (RCons a b r)
  deriving Record()
\end{verbatim}}
Note that we have defined a syntactic extension for records tagged by
the empty tag. Thus we may use the record syntax (with no tag) to build
records.

{\small
\begin{verbatim}
prompt> {`a=34,`b="abc"}
{`a=34,`b="abc"} : Record {`a=Int,`b=[Char]}r
\end{verbatim}}

\subsection{A Tagless Interpreter for a Language with Variables}\label{tagglessvars}

The typed-object language {\tt Lam} can be supplied with a typed
evaluation function. The key is to supply a record that supplies exactly
the values necessary for the free variables in the term being evaluated.
The type system ensures that the record and the free variables coincide.

{\small
\begin{verbatim}
evalLam:: Record r -> Lam r t -> t
evalLam (RecCons _ v r) (Var _)   = v
evalLam RecNil          (Var _)   = unreachable
evalLam (RecCons _ _ r) (Shift e) = evalLam r e
evalLam RecNil          (Shift _) = unreachable
evalLam env     (Abs lab body) = \ x -> evalLam (RecCons lab x env) body
evalLam env     (App f x)      = (evalLam env f) (evalLam env x)
\end{verbatim}}

\begin{exercise}
Instead of using {\tt Var} and {\tt Shift}, fold the ideas from Exercise
\ref{countingvars} into the {\tt Lam} datatype, and then write
the evaluation function for this GADT.
\end{exercise}

\subsection{A Staged Interpreter for a Language with Variables}
It is even possible to stage such an interpreter. One complication
is that the record encoding the environment will not
pair variables with values, but instead it will pair variables with code. To enable this
we define the staged record.

{\small
\begin{verbatim}
data StaticRecord:: Row Tag *0 ~> *0 where
  StNil :: StaticRecord RNil
  StCons:: Label t -> Code x -> StaticRecord r -> StaticRecord (RCons t x r)

stageLam:: StaticRecord r -> Lam r t -> Code t
stageLam (StCons _ code r) (Var _)        = code
stageLam StNil             (Var _)        = unreachable
stageLam (StCons _ _ r)    (Shift e)      = stageLam r e
stageLam StNil             (Shift _)      = unreachable
stageLam env               (App f x)      = 
   [| $(stageLam env f) $(stageLam env x) |]
stageLam env               (Abs lab body) = 
   [| \ x -> $(stageLam (StCons lab [|x|] env) body) |]
\end{verbatim}}

%%%%%%%%%%%%%%%%%%%%%%%%%%%%%%%%%%%%%%%%
\subsection{Small step semantics}

The datatype declarations for representing well-typed terms in the
previous sections bear a striking similarity to the typing
judgments for those languages. For example consider:
\newcommand{\jgmt}[3]{{#1}\vdash{#2}:{#3}}
\begin{mathpar}
\inferrule
        {~}
        {\jgmt{\Gamma,x\colon\tau}{x}{\tau}}\textsc{Var}

\inferrule
        {\jgmt{\Gamma}{e}{\tau}}
        {\jgmt{\Gamma,x\colon\sigma}{e}{\tau}}\textsc{Shift}

\inferrule
        {\jgmt{\Gamma,x\colon\tau}{e}{\sigma}}
        {\jgmt{\Gamma}{\lambda x.e}{\tau\to\sigma}}\textsc{Abs}

\inferrule
        {\jgmt{\Gamma}{e_1}{\tau\to\sigma} \\ \jgmt{\Gamma}{e_2}{\tau}}
        {\jgmt{\Gamma}{e_1\;e_2}{\sigma}}\textsc{Abs}
\end{mathpar}
The similarity justifies a slight change in perspective.  
We have been thinking of \verb|Lam| as representing a piece of abstract syntax, 
but we may also think of it as representing a typing derivation.

The latter perspective supports an interesting approach to studying the 
meta-theory of object languages.  A typing derivation is a \emph{proof}
that a given term has a given type in a given context.  So total functions 
that transform proofs into other proofs can be considered as constructive 
proofs of results in the meta-theory of our object language.  This is 
the approach taken in Twelf, where the meta-language is a Prolog-style 
logic language.  In \om{}, we can write our meta-programs in a functional
programming style.

In the remainder of this section, we build, in several steps, a 
proof of type soundness for our little language. 
Our proof has the following basic structure (due to Wright and Felleisen)\cite{Wright:94}.
\begin{enumerate}
\item All terms are categorized syntactically as either values or non-values.
\item A reduction relation $e \rightarrow e'$ comprising 
        a small-step operational semantics is given.
\item Any non-value term that cannot be reduced any further is considered to 
        exhibit a run-time error.
\item \textbf{Progress.} Any well-typed term $e$ is either a value or can
        step to another well-typed term $e'$ (that is, $e \rightarrow e'$).
\item \textbf{Preservation.} The reduction relation preserves types: 
      If $e$ has type $\tau$ and $e \rightarrow e'$, then $e'$ has type $\tau$.
\item Therefore if a term is well-typed, and we reduce it until no more
        reduction steps are possible, then the resulting term must be a value
        (rather than a term exhibiting a run-time error).
\end{enumerate}

To begin, we slightly modify our {\tt Lam} datatype
from Section \ref{binding}. We call the datatype {\tt E}
(for {\it E}xpression), and change the constructor names,
to avoid confusion between the two. The substantive changes
include the addition of a new type index ({\tt Mode} explained
in greater detail below),
and a shift from usings types of kind {\tt *0}, as indexes
indicating the type of a term, to types of kind {\tt ObjType}
(also explained in greater detail below). 
To highlight
these changes, we have included the classification
of the old type {\tt Lam} for comparison.

We emphasize, as explained earlier, that
the datatype {\tt E} can
be thought of as both abstract syntax or a typing derivation.

{\small
\begin{verbatim}
--   Lam::         Row Tag *0      ~> *0      ~> *0

data E  :: Mode ~> Row Tag ObjType ~> ObjType ~> *0  where
  Const:: Rel a b -> b -> E Val env a
  Var  :: Label s -> E Val (RCons s t env) t
  Shift:: E m env t -> E m (RCons s q env) t
  Lam  :: Label a -> E m (RCons a s env) t -> E Val env (ArrT s t)
  App  :: E m1 env (ArrT s t) -> E m2 env s -> E Exp env t
\end{verbatim}}

\subsubsection{Values versus computations.}

The first step to proving type-soundness in \om{} by this method is to 
distinguish between values and non-values.  We accomplish this by
the introduction of the new index {\tt Mode}. 

{\small
\begin{verbatim}
data Mode:: *1 where
 Exp:: Mode
 Val:: Mode
\end{verbatim}}

\noindent
Go back and study how the {\tt Mode} index is used
in the types of the constructor functions of {\tt E}.
Note how terms in normal form have types where {\tt Val}
is the first index, and ones with redexes have types where {\tt Exp} is
the first index. Consider the short \om\ session:

{\small
\begin{verbatim}
prompt> Const IntR 3
(Const IntR 3) : E Val a IntT 

prompt> Lam `x (Var `x)
(Lam `x (Var `x)) : E Val a (ArrT b b)

prompt> App (Lam `x (Var `x)) (Const IntR 3)
(App (Lam `x (Var `x)) (Const IntR 3)) : E Exp a IntT
\end{verbatim}}

\subsubsection{Object-types versus meta-types.}

In {\tt E}, we no longer use types of kind {\tt *0} as object-level
types. We do this because we wish to lift some, but not all,
meta-level values into constants in the object-language. In this
example we wish to lift integer constants, and n-ary functions
over integers (the so-called $\delta$-reductions). To accomplish this we
define a new kind to represent object-level types.

{\small
\begin{verbatim}
data ObjType:: *1 where
 ArrT:: ObjType ~> ObjType ~> ObjType
 IntT:: ObjType
\end{verbatim}}
This new kind appears as the third index of {\tt E}, and also as an
index to the {\tt Row} comprising the environment.
The constructor {\tt Const} lifts only those values classified
by types that are related to some {\tt ObjType} by the witness
relation {\tt Rel}.

{\small
\begin{verbatim}
data Rel:: ObjType ~> *0 ~> *0 where
 IntR:: Rel IntT Int
 IntTo:: Rel b s -> Rel (ArrT IntT b) (Int -> s)
   -- First order functions only as constants
\end{verbatim}}
The structure of {\tt Rel} relates only integers and first-order, n-ary
functions over integers to the type {\tt ObjType}. Consider the short \om\ session:

{\small
\begin{verbatim}
prompt> IntR
IntR : Rel IntT Int

prompt> IntTo IntR
(IntTo IntR) : Rel (ArrT IntT IntT) (Int -> Int)

prompt> IntTo (IntTo IntR)
(IntTo (IntTo IntR)) : Rel (ArrT IntT (ArrT IntT IntT)) (Int -> Int -> Int)
\end{verbatim}}

\subsubsection{Static versus Dynamic test for Mode.}

Finally, on occasion we will need to observe the structure of
an object-level term, and compute whether it is a value in normal
form, or a term with a redex. We do this by defining a singleton type
reflecting the kind {\tt Mode} into the value world,
and by writing a total function that computes 
a safe approximation of the mode of any expression. By safe, we mean that no
term is ever indexed by {\tt Exp} if it is a value, though some terms
might be indexed by {\tt Exp} even though they do not contain a redex.
Such terms generally have the form {\tt (App (Var `x) \_)}, i.e. an
application with a variable in the function part).

{\small
\begin{verbatim}
data Mode':: Mode ~> *0 where
  Exp':: Mode' Exp
  Val':: Mode' Val

mode :: E m e t -> Mode' m
mode (Lam v body) = Val'
mode (Var v) = Val'
mode (Const r v) = Val'
mode (Shift e) = mode e
mode (App _ _) = Exp'
\end{verbatim}}

\subsubsection{Summary of changes.}
Thus
a well-typed term of type {\tt (E m env t)} is 
(1) a data structure representing an object-level
term, (2) a derivation that the term is well typed with type {\tt t}
in environment {\tt env}, 
and (3) a derivation that the term has mode {\tt m}. 
Lets review the roles of the 3 kinds of indexes to {\tt E}.

\begin{itemize}
\item{\tt Mode.} The mode of the term. Either a {\tt Val}, a term
in normal form, or an {\tt Exp}, a term with redex.

\item{\tt Row Tag ObjType.} The environment which indicates
the position and type of the free variables in the term.

\item{\tt ObjType.} The object-level type of the term. Because of the
relation {\tt Rel}, we know only first order functions
can be lifted from the meta-language to the object language.
\end{itemize}

There are two kinds of redexes in a term. $\beta$-redexes 
(explicit $\lambda$ - expressions in the function position
of an application) and $\delta$-redexes (higher-order
constants in the function position
of an application). We give meaning
to $\beta$-redexes by the use of substitution. Thus we
need a well-typed version of substitution over object-level terms
represented by {\tt E}.


\subsubsection{Substitution lemma}

The key lemma behind the preservation part of the type-soundness proof is called
the \emph{substitution lemma}. The lemma says that if a term $e$ has type
$\sigma$ under the assumption that some variable $x$ has type $\tau$, then
substituting any term $e'$ of type $\tau$ for $x$ in $e$ yields $e[e'/x]$
of type $\sigma$. In our version of the preservation proof, the lemma
exhibits itself as a total well-typed function that performs substitution.

We choose to represent substitutions as data structures. This provides
another example of object language syntax because our syntax is similar to
explicit substitutions~\cite{BenaissaBLR96}. In this approach a
substitution of type {\tt (Sub e1 e2)} is a mapping from one environment
(of kind {\tt e1}) to another (of kind {\tt e2}).

{\small
\begin{verbatim}
data Sub:: Row Tag ObjType ~> Row Tag ObjType ~> *0 where
  Id:: Sub r r
  Bind:: Label t -> E m r2 x -> Sub r r2 -> Sub (RCons t x r) r2
  Push:: Sub r1 r2 -> Sub (RCons a b r1) (RCons a b r2)

subst:: E m1 r t -> Sub r s -> exists m2 . E m2 s t
subst t           Id           = Ex t
subst (Const r c) sub          = Ex (Const r c)
subst (Var v)     (Bind u e r) = Ex e
subst (Var v)     (Push sub)   = Ex (Var v)
subst (Shift e)   (Bind _ _ r) = subst e r
subst (Shift e)   (Push sub)   = case subst e sub of {Ex a -> Ex(Shift a)}
subst (App f x)   sub          = case (subst f sub,subst x sub) of
                                  (Ex g,Ex y) -> Ex(App g y)
subst (Lam v x)   sub          = case subst x (Push sub) of
                                  (Ex body) -> Ex(Lam v body)
\end{verbatim}}

\subsubsection{Preservation.}

In our proof, we perform steps 2 (define the one-step evaluation
relation), 4 (prove progress), and 5 (prove type preservation)
at once by defining a total single-step operation that operates on
well-typed non-value closed terms. Its type is given by\\ {\tt onestep
:: E m Closed t -> (E Exp Closed t + E Val Closed t)}.\\ Read logically
this type says that every closed term (regardless of whether it is a
value or an expression with a redex) can be transformed into another
closed term with the same type, or is already a value.

{\small
\begin{verbatim}
type Closed = RNil

onestep :: E m Closed t -> (E Exp Closed t + E Val Closed t)
onestep (Var v)      = unreachable
onestep (Shift e)    = unreachable
onestep (Lam v body) = R (Lam v body)
onestep (Const r v)  = R(Const r v)
onestep (App e1 e2)  =
  case (mode e1,mode e2) of
    (Exp',_) ->
      case onestep e1 of
        L e -> L(App e e2)
        R v -> L(App v e2)
    (Val',Exp') ->
      case onestep e2 of
        L e -> L(App e1 e)
        R v -> L(App e1 v)
    (Val',Val') -> rule e1 e2
\end{verbatim}}

This function is a non-recursive case analysis. The {\tt Var} and {\tt
Shift} cases are unreachable (they cannot be closed terms). The {\tt
Lam} and {\tt Const} cases are already values. Observing the mode of
the two parts of an application we have three choices. If the function
is an expression with a possible redex, we take one step in the
function part, and then rebuild the term. If the function part is a value,
we must apply one of the $\beta$- or $\delta$-rules. Note that the
function part is always a closed term with an {\tt (ArrT \_ \_)}
object-level type.

{\small
\begin{verbatim}
rule::  E Val Closed (ArrT a b) ->
        E Val Closed a ->
        (E Exp Closed b + E Val Closed b)
rule (Var _)   _ = unreachable
rule (Shift _) _ = unreachable
rule (App _ _) _ = unreachable
-- The beta-rule
rule (Lam x body) v =
  let (Ex term) = subst body (Bind x v Id)
  in case mode term of
       Exp' -> L term
       Val' -> R term
rule (Const IntR _)      _                   = unreachable
rule (Const (IntTo b) _) (Var _)             = unreachable
rule (Const (IntTo b) _) (Shift _)           = unreachable
rule (Const (IntTo b) _) (App _ _)           = unreachable
rule (Const (IntTo b) f) (Lam x body)        = unreachable
rule (Const (IntTo b) f) (Const (IntTo _) x) = unreachable
-- The delta-rule
rule (Const (IntTo b) f) (Const IntR x)      = R(Const b (f x))
\end{verbatim}}
There are eleven cases. Nine of which are unreachable from type
considerations (i.e. the inputs are not values, are not closed, or the
first argument does not have an arrow type). We have structured our
function body to make it explicit that we have covered every case. This
allows us to prove (by a meta-level argument) that {\tt rule} is total. In
other systems (i.e. Twelf, Coq, etc.) this argument can be enforced by the
type-system of the meta-language. In these systems all functions are total
(or they are not accepted). In \om, we aspire to this level of automated
assistance, but as we think of \om\ as a programming language (not a proof
system) we must support both total and partial functions. We hope to
separate total and partial functions by using the type system sometime in
the near future. 

The function {\tt onestep} makes progress. By inspecting the code we see
all values are immediately returned, and all non-values actually take one
step forward.
 
 
 \subsection{Example: Constructing Typing Derivations at Runtime}
 
 At first glance, using GADTs to represent object-languages
 solves many problems. But, further introspection reveals a subtle
 problem. We can build typed object-level terms by typing
 constructed terms into our program using the constructors of the GADT, but how do we build
 such terms algorithmically? I.e. how do we write a parser,
 for example, that builds a well-typed object-level term? What
 would the type of the parser be?
 The type {\tt (parse:: String -> E m e t)} is clearly not
 sufficient. Not every string can be parsed.
 But the type  {\tt (parse:: String -> Maybe(E m e t))} is also
 not sufficient. What mode, environment, and object level type
 should constrain the meta-level type variables {\tt m}, {\tt e}, and
 {\tt t}? The type
 {\tt (parse:: String -> exists m e t . Maybe(E m e t))} is closer
 to the mark, but this is also too unconstrained.  We expect
 some properties to be true of these type variables. One solution
 is to build runtime representations that represent the constraints
 we envision, and runtime tests for these constraints, that we can execute at runtime.
 
 We do this by building a singleton type to reflect the
 object-level types as meta-level runtime values
 (Section \ref{singleton} and Exercise \ref{reptype}), and a runtime test for equality
 of these object-level type indexes (Exercises \ref{samenat} and \ref{objectVar}).
 
{\small
\begin{verbatim}
data Rep:: ObjType ~> *0 where
 I:: Rep IntT
 Ar:: Rep a -> Rep b -> Rep (ArrT a b)
\end{verbatim}}

In the function
{\tt compare}, because we want our runtime tests to report interesting error
messages, the comparison returns a sum type, were the left injection (a
failure) is an error message, and the right injection (a success) is an
equality proof. Because the partial application of the type constuctor
{\tt (+)} to {\tt String} is monadic
\footnote{{\tt return:: a -> (String + a)\\
return x = R x\\
bind:: (String + a) -> (a -> (String + b)) -> (String + b)\\
bind (L message) f = Left message\\
bind (R x) f = f x\\}}, we use the {\tt do} notation
to specify what happens on success. On failure
(of either {\tt (comapre x s)} or {\tt (compare y t)}) the error message
in the left injection will be propogated.

{\small
\begin{verbatim}
compare:: Rep a -> Rep b -> (String + Equal a b)
compare I I = R Eq
compare (Ar x y) (Ar s t) =
  do { Eq <- compare x s
     ; Eq <- compare y t
     ; R Eq}
compare I (Ar x y) = L "I /= (Ar _ _)"
compare (Ar x y) I = L "(Ar _ _) /= I"
\end{verbatim}}


We will break our parsing problem into two parts. First, parsing a string
into an untyped object-language representation (not shown in this paper,
as this is the ordinary parsing problem). Second, transforming
this untyped representation into a well-typed GADT representing
a typed object-language term (or typing derivation, depending upon
your perspective). In this report, we assume that the untyped
representation suggests a type for every variable, and that
our algorithm checks that this suggestion is correct. The inference
problem is much harder, and not shown here. Our untyped
representation follows:

{\small
\begin{verbatim}
data Term:: *0 where
  C:: Int -> Term
  Ab:: String -> Rep a -> Term -> Term
  Ap:: Term -> Term -> Term
  V:: String -> Term
\end{verbatim}}

We will check each term with respect to a given environment which maps
every variable to an object-level type. It will also store the string used
to name the variable in the untyped representation, and the label used to
represent the variable in the typed-representation. Such an environment is
indexed by {\tt (Row Tag ObjType)} in the same manner as terms {\tt E} and
substitutions {\tt Sub}.

{\small
\begin{verbatim}
data Env:: Row Tag ObjType ~> *0 where
    Enil:: Env RNil
    Econs:: Label t -> (String,Rep x) -> Env e -> Env (RCons t x e)
  deriving Record(e)
\end{verbatim}}
A key component of our algorithm, to produce a well-typed
representation from an untyped representation, is to
look up the type of a variable. 

{\small
\begin{verbatim}
fail:: String -> (String + a)
fail s = L s

lookup:: String -> Env e -> (String + exists t m .(E m e t,Rep t))
lookup name Enil = fail ("Name not found: "++name)
lookup name {l=(s,t);rs}e | eqStr name s = R(Ex(Var l,t))
lookup name {l=(s,t);rs}e =
  do { Ex(v,t) <- lookup name rs
     ; R(Ex(Shift v,t)) } 
\end{verbatim}}

If successful, both a representation of a type, and a term with that type
are returned. Now we need put all this machinery together. The type
checker is a program with the following prototype:\\ {\tt tc:: Term -> Env e ->
(String + exists t m . (E m e t,Rep t))}\\ Read logically, for every
untyped term, and every environment with types for variables reflected in
the row {\tt e}, we can either report a type-checking error, or 
return a representation of a
typed term. In this representation (consisting of a pair
of a term and a singleton), its actual type and its mode are
existentially quantified, but the actual object-level type
is reflected in the ``shape" of the runtime singleton object.

{\small
\begin{verbatim}
tc:: Term -> Env e -> (String + exists t m . (E m e t,Rep t))
tc (V s) env = lookup s env
tc (Ap f x) env =
  do { Ex(f',ft) <- tc f env
     ; Ex(x',xt) <- tc x env
     ; case ft of
        (Ar a b) ->
           do { Eq <- compare a xt
              ; R(Ex(App f' x',b)) }
        _ -> fail "Non fun in Ap" }
tc (Ab s t body) env =
  do { let (Hidden l) = newLabel s
     ; Ex(body',et) <- tc body {l=(s,t); env}e
     ; R(Ex(Lam l body',Ar t et)) }
tc (C n) env = R(Ex(Const IntR n,I))
\end{verbatim}}

The application case is the most interesting. First,
recursively type-check the function and argument, returning
typed terms {\tt f'} and {\tt x'}, and reflected types {\tt ft}
and {\tt xt}. If either of these fails, the monad syntax causes the whole function
to fail. Test that the function argument is really a function, and then
compare the domain with the type of the argument. Only if this succeeds,
and we have a proof that the two types are equal, can the whole case succeed.



 
 {\small
\begin{verbatim}

\end{verbatim}} 

 
%%%%%%%%%%%%%%%%%%%%%%%%%%%%%%%%%%%%%%%%
\subsection{The bottom line}

The ability to define type-indexed GADTs, and the ability to define new
kinds, creates a rich playground for those wishing to explore the
design of new languages. These features, along with the use of rank-N
polymorphism (which is beyond the scope of this paper) make \om~ a
better meta-language than Haskell. In order to explore the design of a
new language one can proceed as follows:

\begin{itemize}

\item First, represent the object-language as a type-indexed GADT. The indexes
correspond to static properties of the program.

\item The indexes can have arbitrary structure, because they are introduced as
the type constructors of new kinds.

\item The typed constructor functions of the object-language GADT define a
static semantics for the object language.

\item Meta-programs written in \om manipulate object-language
represented as data, and check and maintain the properties captured in
the type indexes by using the meta-language type system. This lets us
build and test type systems interactively.

\item A dynamic semantics for the language can be defined by (1) writing either
a large step semantics in the form of an interpreter or evaluation
function, or by (2) writing a small step semantics in terms of substitution
over the term language. In either case, the type system of the meta-language
guarantees that these meta-level programs maintain object level type-safety.

\item Normal operations such as pretty-printing and parsing functions can also be constructed,
albeit with a little more cleverness than is ordinarily required.

\end{itemize}




\section{Using Terms as Theorems}\label{theorem}

We can use a value of type \verb+(Nat' n)+ as a proof
that \verb+n+ is a natural number. In \om, ordinary
datatypes can be used as constraints over types. A constraint
can be discharged by exhibiting a non-divergent term with that type.
The classic datatype used in this fashion is the equality type
from Section \ref{equal}. Recall:
%%%%%%%%%%%%%%%%%%
\begin{verbatim}
data Equal :: a ~> a ~> *0 where
  Eq:: Equal x x
\end{verbatim}
The \verb+Equal+ constraint can be applied to all types of the same kind
because it is level polymorphic (see section \ref{level}).
Thus \verb+(Equal 2t 3t)+ and \verb+(Equal Int Bool)+ are both
well formed, but neither is inhabited (i.e. there are no non-divergent values with these types
since (\verb+Int+ $\neq$ \verb+Bool+) and (\verb+(S(S Z))+ $\neq$ \verb+(S(S(S Z)))+)).
The normal mode of use is to construct terms with types
like {\tt (Equal} {\it x} {\it y}{\tt )} 
where {\it x} and {\tt y} are type
level function applications. 
For example consider the type of the function \verb+plusZ+ below.
Its type: ({\tt Nat' n -> Equal {plus n Z} n}) when read logically means
{\it for all natural numbers {\tt n}, n+0 = n}. One way
to prove this is with a proof by induction over {\tt n}. The following
recursive definition of {\tt plusZ} is a term witnessing
this property. The {\tt theorem} clause, inside the
defintion of {\tt plusZ} is a mechanism that helps organize this
proof, and is explained in detail in the sequel.

%%%%%%%%%%%%%%%%%%
{\small
\begin{verbatim}
plusZ :: Nat' n -> Equal {plus n Z} n
plusZ Z = Eq
plusZ (S m) = Eq
  where theorem indHyp = plusZ m
\end{verbatim}}

This function is a proof by induction
that for all natural numbers {\tt n} : \verb+{plus n 0t}+ =  \verb+n+. The
definition exhibits a well-typed, total function with this type.
The declaration, \verb+where theorem indHyp = plusZ m+, instructs the
type checker to use the type of the term \verb+(plusZ m)+ as a reasoning rule.
Thus we may assume its type: \verb+(Equal {plus b 0t} b)+
while discharging \verb+(Equal (S{plus b Z}) (S b))+.

To see that {\tt plusZ} is well typed, the type checker does the following.
The expected type is the type given in the function prototype. We
compute the type of both the left- and right-hand-side of the equation
defining a clause. We compare the expected type with the computed type
for both the left- and right-hand-sides. This comparison generates
some necessary equalities (for each side) to make the expected and computed
types equal. We assume the left-hand-side
equalities to prove the right-hand-side equalities. To see this in
action, consider the two clauses of the definition of \verb+plusZ+.

\begin{enumerate}

\item \deffbox{Nat' n}{Equal \plus{n}{Z} n}{plusZ Z}{Eq}{Nat' Z}{Equal a a}{n = Z}{(a = n, a= \plus{n}{Z})}

\vspace*{.1in}
In the first case, the left-hand-side equalities let us
assume \verb+n+ = \verb+Z+. The right-hand-side equalities require us
to establish that \verb+a+ = \verb+{plus n Z}+ and \verb+a+ = \verb+n+.  This can be established
{\it iff} \verb+ n+ = \verb+{plus n Z}+. Using the assumption that 
\verb+n+ = \verb+Z+, we are left with the requirement that  \verb+Z+ = \verb+{plus Z Z}+,
which is easy to prove using the definition of \verb+plus+.
\vspace*{.1in} 

\item \deffbox{Nat' n}{Equal \plus{n}{Z} n}{plusZ (S m)}{Eq}{Nat' (S b)}{Equal a a}{n = (S b)}{(a = n, a= \plus{n}{Z})}

\vspace*{.2in}
In the second case, the left-hand-side assumptions are \verb+n+ = 
\verb+(S b)+ (where the pattern introduced variable {\tt m} has type {\tt (Nat' b)}).
The right-hand-side equalities require us to establish that \verb+a+ =
\verb+{plus n Z}+ and \verb+a+ = \verb+n+.  Again, this can only be
established if  \verb+ n+ = \verb+{plus n Z}+. Using the assumption
that \verb+n+ = \verb+(S b)+, we are left with the requirement that
\verb+(S b)+ = \verb+{plus (S b) Z}+. Using the definition
of {\tt plus}, this reduces to \verb+(S b)+ = \verb+(S{plus b Z})+.
To establish this fact, we use the inductive hypothesis.
Since the argument {\tt (S m)} is finitely
constructed, and the function \verb+plusZ+ is total, the term, 
\verb+(plusZ m)+ exhibits a proof that \verb+(Equal {plus b Z} b)+. 

\end{enumerate}

Other interesting facts, that are established in the same
way, but omitted for brevity, include:

 
{\small
\begin{verbatim}
plusS :: Nat' n -> Equal {plus n (S m)} (S{plus n m})  
plusCommutes :: Nat' n -> Nat' m -> Equal {plus n m} {plus m n}
plusAssoc :: Nat' n -> Equal {plus {plus n b} c} {plus n {plus b c}}
plusNorm :: Nat' x -> Equal {plus x {plus y z}} {plus y {plus x z}}
\end{verbatim}}

\begin{exercise}\label{convlemma}
Write an \om\ function body for each of the prototypes above. The function bodies
for {\tt plusS} and {\tt plusAssoc} are very similar to {\tt plusZ}.
The other two require appealing to theorems in addition to an induction
hypotheses. In fact, {\tt plusCommutes} requires both {\tt plusZ} and
{\tt plusS} in addition to an induction hypothesis. We leave it to you
to figure out what theorem is required for {\tt plusNorm}.
\end{exercise}

\subsection{Self Describing Combinatorial Circuits}

Our next example is the description of combinatorial circuits. We will use types
to ensure that our descriptions describe what they implement. We first describe the
{\tt Bit} type.

%%%%%%%%%%%%%%%%%%
{\small
\begin{verbatim}
data Bit:: Nat ~> *0 where
  One :: Bit (S Z)
  Zero :: Bit Z
\end{verbatim}}

Like \verb+Nat'+, {\tt Bit} is a singleton type (see Section \ref{singleton}), there is only
one value for each type. Note how the type of a bit carries the 
value of the bit as a natural number as its type index. 
I.e.  \verb+(One :: Bit 1t)+  and \verb+(Zero :: Bit 0t)+.
We exploit this
to define a data structure representing a base-2 number
as a sequence of bits. The idea is for a value
of type \verb+(Binary Bit w v)+ to represent a binary number
built from a sequence of {\tt Bits}, with width \verb+w+ and value \verb+v+.

%%%%%%%%%%%%%%%%%%
{\small
\begin{verbatim}
data Binary:: (Nat ~> *0) ~> Nat ~> Nat ~> *0 where
  Nil :: Binary bit Z Z
  Cons:: bit i -> Binary bit w n -> Binary bit (S w) {plus {plus n n} i}
\end{verbatim}}

Note that the type of the elements in the sequence has been abstracted
to be any type constructor classified by the kind \verb+(Nat ~> *0)+.
In our first few examples, we will construct lists of {\tt ({\it Bit}
i)}, so we will have values with type {\tt (Binary {\it Bit} len
value)} as a result. Later in the text, we will build binary numbers from
other representations of bits.

A value with type {\tt (Binary Bit 2t 3t)} is a sequence of {\tt (Bit j)}
values. The individual {\tt j}'s are combined to represent a
 binary number with value {\tt 3t}.
Binary numbers are stored least significant bit first. Prefixing a new
bit shifts the previous bits into the next significant position, so the
value of the new number is the value of the new bit plus twice
the value of the old bits. Thus the type expression \verb+{plus {plus n n} i}+ in
the type of \verb+Cons+ which prefixes a new bit. For example consider the term: 
\verb+(Cons Zero (Cons One (Cons Zero (Cons Zero Nil))))+ that has type \verb+(Binary Bit 4t 2t)+.
I.e. ``0100" (where the least significant bit is left-most) has value 2 and width 4.



If we add three one-bit numbers, we always get a two bit result. We
can write this function as follows.

%%%%%%%%%%%%%%%%%%
{\small
\begin{verbatim}
add3Bits:: (Bit i) -> (Bit j) -> (Bit k) -> 
           Binary Bit 2t {plus {plus j k} i}
add3Bits Zero Zero Zero = Cons Zero (Cons Zero Nil)
add3Bits Zero Zero One  = Cons One  (Cons Zero Nil)
add3Bits Zero One  Zero = Cons One  (Cons Zero Nil)
add3Bits Zero One  One  = Cons Zero (Cons One  Nil)
add3Bits One  Zero Zero = Cons One  (Cons Zero Nil)
add3Bits One  Zero One  = Cons Zero (Cons One  Nil)
add3Bits One  One  Zero = Cons Zero (Cons One  Nil)
add3Bits One  One  One  = Cons One  (Cons One  Nil) 
\end{verbatim}}

This function is an exhaustive case analysis of all 8 possible
combination of bits. It is exhaustive and total. Consider
type checking one case.

\vspace*{+.1in}
\hspace*{-.3in}
{\tiny
\deffboxShort{Bit i -> Bit j -> Bit k}{Binary Bit \two \hspace*{.1in}\plus{\plus{j}{k}}{i}}
{add3Bits Zero One One}{Cons Zero (Cons One  Nil)}
{Bit \zero \ -> Bit \one \ -> Bit \one}
{\begin{tabular}[t]{lll}
Binary Bit \two  \\
\plusH{\plusH{\plus{\plus{\zero}{\zero}}{\one}}
                   {\plus{\plus{\zero}{\zero}}
             {\one}}}
      {\zero}
\end{tabular}}               
{(i = \zero,j = \one,k = \one)}
{\begin{tabular}[t]{l}
  \plus{\plus{j}{k}}{i} = \\
\plusH{\plusH{\plus{\plus{\zero}{\zero}}{\one}}
                   {\plus{\plus{\zero}{\zero}}
             {\one}}}
      {\zero}  
\end{tabular}} } 
\vspace*{+.1in}
                      
Under the assumptions, both parts of the equality in the requirements
for the right-hand-side reduce to
\verb+(Binary Bit t2 2t)+, so the clause is well typed.
Iterating \verb+add3Bits+, we can construct a ripple carry adder, whose type
states that it is really an addition function!

%%%%%%%%%%%%%%%%%%
{\small
\begin{verbatim}
add :: Bit c -> 
       Binary Bit n i -> 
       Binary Bit n j -> Binary Bit (S n) {plus {plus i j} c}
add c Nil Nil = Cons c Nil
add c (Cons x xs) (Cons y ys) = 
  case add3Bits c x y of
    (Cons bit (Cons c2 Nil)) -> Cons bit (add c2 xs ys)
       where theorem plusCommutes, plusAssoc, plusNorm
\end{verbatim}}

The function \verb+add+ is type checked in the same manner as we illustrated
with \verb+plusZ+ and \verb+add3Bits+. In  \verb+add+, the type
checker relies on the three theorems \verb+plusCommutes+, \verb+plusAssoc+, 
\verb+plusNorm+ that are the focus of Exercise \ref{convlemma} from the end of Section \ref{theorem}.
We repeat their types here for convenience.
%%%%%%%%%%%%%%

{\small
\begin{verbatim}
plusCommutes :: Equal {plus n m}          {plus m n}
plusAssoc    :: Equal {plus {plus n b} c} {plus n {plus b c}}
plusNorm     :: Equal {plus x {plus y z}} {plus y {plus x z}}
\end{verbatim}}

When used in conjunction, these theorems act as a set of left-to-right rewriting rules,
and have a very strong normalizing effect. This effect occurs because the theorems {\tt
plusCommutes} and {\tt plusNorm} are only applied if the rewritten term is
lexigraphically smaller than the original term. For example, while type checking {\tt
add} the type checker uses them to repeatedly rewrite the term:\\ 
\verb+{plus {plus {plus {plus x3 x3} x2} {plus {plus x5 x5} x4}} x1}+ \hspace*{.5in} to the term:\\ 
\verb+{plus x1 {plus x2 {plus x3 {plus x3 {plus x4 {plus x5 x5}}}}}}+


\begin{exercise}
Repeat the progression of defining the GADT {\tt Binary} through
defining the function {\tt add}, but this time make {\tt Binary}
store most-significant bits on the left.
\end{exercise}

 \subsection{Symbolically Combining Bits}
 
 While we have shown how to use types to describe properties of programs, our adder
 is not a very effective hardware description.  We need a data structure that can
 represent not only the constant bits, \verb+One+ and \verb+Zero+, but also
 operations on bits. This motivates \verb+BitX+ (for eXtended bit).
 
%%%%%%%%%%%%%%%%%%
{\small
\begin{verbatim}
data BitX:: Nat ~> *0 where
  OneX :: BitX (S Z)
  ZeroX :: BitX Z
  And:: BitX i -> BitX j -> BitX {and i j}
  Or:: BitX i -> BitX j -> BitX {or i j}
  Xor:: BitX i -> BitX j -> BitX {xor i j}
\end{verbatim}}

In order to track the result of {\it and}ing ({\it or}ing, {\it xor}ing) two bits,
we need the {\tt and} ({\tt or}, {\tt xor}) functions at the type level. These
functions take any two natural numbers as input, but always return \verb+0t+ or \verb+1t+
as a result.

\vspace*{.15in}
\hspace*{-0.25in}
\begin{tabular}{l|l}
\begin{minipage}[t]{2.0in}
{\small
\begin{verbatim}
and :: Nat ~> Nat ~> Nat
{and Z Z} = Z
{and Z (S n)} = Z
{and (S n) Z} = Z
{and (S n) (S n)} = S Z
\end{verbatim}}
\end{minipage}
&
\begin{minipage}[t]{2.0in}
{\small
\begin{verbatim}
 or :: Nat ~> Nat ~> Nat
 {or Z Z} = Z
 {or Z (S n)} = S Z
 {or (S n) Z} = S Z
 {or (S n) (S n)} = S Z
\end{verbatim}}
\end{minipage}
\end{tabular}

\begin{exercise}
Write the \om\ type-level function:\\
\verb+xor :: Nat ~> Nat ~> Nat+\\ that implements the exclusive-or function.
\end{exercise}
 

We can prove a number of interesting theorems about these functions
by exhibiting terms with logical types. As we did with \verb+add3Bits+,
these functions are basically an exhaustive analysis of the cases.
Here we prove that {\tt and} is associative.
%%%%%%%%%%%%%%%%%%

{\small
\begin{verbatim}
andAs :: Bit a -> Bit b -> Bit c -> 
         Equal {and {and a b} c} {and a {and b c}}
andAs Zero Zero Zero = Eq
andAs Zero Zero One  = Eq
andAs Zero One  Zero = Eq
andAs Zero One  One  = Eq
andAs One  Zero Zero = Eq
andAs One  Zero One  = Eq
andAs One  One  Zero = Eq
andAs One  One  One  = Eq
\end{verbatim}}
Note, that this is a theorem about \verb+Bit a+, \verb+Bit b+, and \verb+Bit c+,
not about natural numbers {\tt a}, {\tt b}, and {\tt c}. I.e.\\
{\small \verb+(Bit a -> Bit b -> Bit c -> Equal {and {and a b} c} {and a {and b c}})+}\\
is a theorem but\\
{\small \verb+(Nat' a -> Nat' b -> Nat' c -> Equal {and {and a b} c} {and a {and b c}}}+}\\
is not.   A number of other useful theorems are proved in a similar manner.

{\small
\begin{verbatim}
andZ1:: Bit a -> Equal {and a Z} Z
andZ2:: Bit a -> Equal {and Z a} Z
andOne2:: Bit a -> Equal {and a (S Z)} a
andOne1:: Bit a -> Equal {and (S Z) a} a
\end{verbatim}}

\begin{exercise}
Following the pattern of {\tt AndAs}, write function definitions for the
above prototypes.
\end{exercise}

Every \verb+(BitX i)+ can be evaluated into a \verb+(Bit i)+ by applying
the definitions of the operations {\tt and}, {\tt or} and {\tt xor}.
This is the purpose of the function \verb+fromX+.
Since the operations are functions at the type level, and we need 
operations on bits (which live at the value level) we define
the functions {\tt and'}, {\tt or'} and {\tt xor'}.

\vspace*{.15in}

{\small
\begin{verbatim}
fromX :: BitX n -> Bit n
fromX OneX = One
fromX ZeroX = Zero
fromX (Or x y) = or' (fromX x) (fromX y)
fromX (And x y) = and' (fromX x) (fromX y)
fromX (Xor x y) = xor' (fromX x) (fromX y)
fromX (And3 x y z) = 
      and' (fromX x) (and' (fromX y) (fromX z))

\end{verbatim}}

{\small
\begin{verbatim}
and' :: Bit i -> Bit j -> Bit {and i j}
and' Zero Zero = Zero
and' Zero One = Zero
and' One Zero = Zero
and' One One = One

or' :: Bit i -> Bit j -> Bit {or i j}
xor' :: Bit i -> Bit j -> Bit {xor i j}
\end{verbatim}}
\vspace*{.15in}

\begin{exercise}
Write \om\ function bodies for the
omitted functions {\tt or'} and {\tt xor'}.
\end{exercise}


Because every \verb+(BitX i)+ can be evaluated into a \verb+(Bit i)+,
we can lift theorems about \verb+Bit+ to theorems about \verb+BitX+.
For example, consider the theorem:
%%%%%%%%%%%%%%%%%%
{\small
\begin{verbatim}
andAs:: Bit a -> Bit b -> Bit c -> Equal {and {and a b} c} {and a {and b c}}
\end{verbatim}}
If {\tt a}, {\tt b} and {\tt c} are {\tt Bit}s, then  {\tt a}, {\tt b} and {\tt c}
associate under {\tt and}.  This is not the case for arbitrary  {\tt a}, {\tt b} and {\tt c}.
Recall that the natural number indexes to {\tt Bit} can only be 0 or 1. A similar
theorem holds if  {\tt a}, {\tt b} and {\tt c} are {\tt BitX}, and this theorem
can be computed from the theorem involving {\tt Bit}.

{\small
\begin{verbatim}
andAssoc:: BitX a -> BitX b -> BitX c -> 
           Equal {and {and a b} c} {and a {and b c}}
andAssoc a b c = andAs (fromX a) (fromX b) (fromX c)
\end{verbatim}}

So unlike {\tt andAs}, where we could not lift a theorem about {\tt Bit} to
a theorem about {\tt Nat}, every theorem about {\tt Nat} can be lifted to a theorem about {\tt Bit}.
With these tools,  we can build a ripple carry adder that performs
addition by applying the bit operations. For example, to add
three one-bit numbers to obtain a two-bit result, we need to
construct a logical formula that captures the following table.

%%%%%%%%%%%%%%%%%%
{\small
\begin{verbatim}
inputs     sum

i j k      high bit  low bit
------
0 0 0      0         0
0 0 1      0         1
0 1 0      0         1          low bit  = (Xor i (Xor j k))
0 1 1      1         0          high bit = (Or (And i j) 
1 0 0      0         1                         (Or (And i k) 
1 0 1      1         0                             (And j k)))
1 1 0      1         0
1 1 1      1         1
\end{verbatim}}

To implement this is \om, we introduce a 2-bit number \verb+Pair+ (more significant bit on the left),
and the function \verb+addthree+.


%%%%%%%%%%%%%%%%%%
{\small
\begin{verbatim}
data Pair:: Nat ~> *0 where
 Pair:: BitX hi -> BitX lo -> Pair {plus {plus hi hi} lo}

addthree :: BitX i -> BitX j -> BitX k -> Pair {plus j {plus k i}}
addthree i j k = Pair (Or (And i j) (Or (And i k) (And j k)))
                                    (Xor i (Xor j k))
  where theorem lemma = logic3 (fromX i) (fromX j) (fromX k) 
\end{verbatim}}
Unlike the function \verb+add3Bits+, we cannot type check \verb+addthree+
by exhaustively enumerating all possible inputs because there
are an infinite number of possible terms of type \verb+(BitX i)+ for each
natural number \verb+i+. But we can prove a lemma about \verb+Bit+
(which we can prove by exhaustive analysis) and then lift it to a theorem
about \verb+BitX+. This is the role of the term \verb+(logic3 (fromX i) (fromX j) (fromX k))+
in the \verb+theorem+ clause in \verb+addthree+.

%%%%%%%%%%%%%%%%%%
{\small
\begin{verbatim}
logic3 :: Bit i -> Bit j -> Bit k -> 
          (Equal {plus {plus {or {and i j} 
                                 {or {and i k} {and j k}}} 
                             {or {and i j} 
                                 {or {and i k} {and j k}}}} 
                       {xor i {xor j k}}} 
                 {plus j {plus k i}})                                            
logic3 Zero Zero Zero = Eq
logic3 Zero Zero One  = Eq
logic3 Zero One  Zero = Eq
logic3 Zero One  One  = Eq
logic3 One  Zero Zero = Eq
logic3 One  Zero One  = Eq
logic3 One  One  Zero = Eq
logic3 One  One  One  = Eq
\end{verbatim}}
We can now re-implement our ripple carry adder, but this time by
symbolically combining the input bits, to compute
the output bits as a logical function of the inputs. This function
has a similar type, the same structure, and uses the same theorems as the function \verb+add+.
%%%%%%%%%%%%%%%%%%

{\small
\begin{verbatim}
addBits :: BitX c -> Binary BitX n i -> Binary BitX n j -> 
           Binary BitX (S n) {plus {plus i j} c}
addBits c Nil Nil = Cons c Nil
addBits c (Cons x xs) (Cons y ys) = 
  case addthree c x y of
    (Pair c2 bit) -> Cons bit (addBits c2 xs ys)
       where theorem plusCommutes, plusAssoc, plusNorm
\end{verbatim}}

To actually compute a circuit we need to have some symbolic inputs. We
do this by extending the type \verb+BitX+ with a constructor to represent variables.
We can then construct some inputs, and compute the description
of an adder. Our function works on inputs of any size.

%%%%%%%%%%%%%%%%%%
{\small
\begin{verbatim}
data BitX:: Nat ~> *0 where
  . . .
  X:: Int -> BitX a
  
xs :: Binary BitX 2t {plus {plus a a} b}
xs = Cons (X 1) (Cons (X 2) Nil)  

ys :: Binary BitX 2t {plus {plus a a} b}
ys = Cons (X 3) (Cons (X 4) Nil)
carry = (X 5)

ans = addBits carry xs ys  
\end{verbatim}}
Here \verb+xs+ and \verb+ys+ are two-bit symbolic inputs, and \verb+carry+ is a symbolic
input carry. Calling \verb+addBits+ we construct an output which is a
\verb+(Binary Bit)+ list with three elements, each of which is a combinatorial
function of the input bits, whose value is guaranteed by the
types to be the sum of the inputs! Below, we display the output with a pretty
printer that displays \verb+(X n)+ as ``xn", and indents the display
to emphasize its structure.

%%%%%%%%%%%%%%%%%%
{\small
\begin{verbatim}
(Cons (Xor x5
          (Xor x1 x3))
(Cons (Xor (Or (And x5 x1)
               (Or (And x5 x3)
                   (And x1 x3)))
           (Xor x2 x4))
(Cons (Or (And (Or (And x5 x1)
                   (Or (And x5 x3)
                       (And x1 x3)))
               x2)
          (Or (And (Or (And x5 x1)
                       (Or (And x5 x3)
                           (And x1 x3)))
                   x4)
             (And x2 x4)))
Nil)))             
\end{verbatim}}

The key property here is that the type of this structure guarantees
that it implements an addition function.

\begin{exercise}
There are many equivalencies between boolean expressions. Any
function with the type: {\tt  (BitX n -> Maybe (BitX n))} can be thought of
as a meaning preserving transformation. Given a value typed {\tt v:: BitX n},
a meaning preserving transformation returns {\tt (Just u)} or {\tt Nothing}. If it returns {\tt (Just u)}
then {\tt u} is semantically equivalent to {\tt v}. If it returns {\tt Nothing}
we interpret this to mean the transformation did not apply to {\tt v}.
Choose a few boolean laws and implement them as meaning preserving transformations
as discussed above.
\end{exercise}

\begin{exercise}
Transformations can be combined by placing them in a list, and applying them
using transformation combinators. Consider functions with the types below:

{\small
\begin{verbatim}
first:: [BitX n -> Maybe(BitX n)] -> BitX n -> Maybe(BitX n) 
all:: [BitX n -> Maybe(BitX n)] -> BitX n -> [BitX n]
\end{verbatim}}
\vspace*{-.05in}

The combinator {\tt first} lifts
a list of transformations to a single transformation, applying the first
applicable transformation in the list.
The combinator {\tt all} finds all applicable transformations
and returns a list of all possible results, including the
untransformed term as well. Define these two functions in \om.

The combinator {\tt retry} continually re-applies a meaning preserving
transformation until the term reaches a fixed-point. What is the type of {\tt
retry}? Write an \om\ function body for {\tt retry}.
What other combinators can you think of?
\end{exercise}

\subsection{A Caveat} \label{caveat}

The addition of the variable \verb+BitX+ constructor \verb+X+ was necessary if we
want to use our functions to build hardware descriptions. Without it, we can only
build constant combinatorial circuits!  Unfortunately, it breaks the soundness of
our descriptions. The lack of soundness flows from the fact that our function
\verb+fromX+ is no longer total. How do we turn a variable into a {\tt Bit}? Thus,
we can no longer lift facts about the functions {\tt and}, {\tt or}, and {\tt xor} and
the type {\tt Bit} to facts about the type {\tt BitX}. To overcome this limitation
we would need to track the variables in the type of {\tt BitX} objects. For example
we may write
\verb+(BitX Bit env width value)+ as the type of a binary number whose 
free variables are described by {\tt env}. Now, we must recast our theorems in 
terms of \verb+(BitX Bit env width value)+ and well formed environments {\tt env}.
This is sufficient, because a well formed environment means every variable
will eventually be replaced by a bit, and in this new formulation the lifting
of theorems hold. 

\begin{exercise}
Using the patterns discussed in Section \ref{binding} for 
languages with binding structures, re-do the progression
from the GADT {\tt BitX} to the function {\tt addBits}, but this time track
the variables in the types of {\tt BitX}. Recast the theorems
about {\tt BitX} so that they hold for all environments.
\end{exercise}

\section{Conclusion}

We hope that the programs and exercises described in this paper
give you, the reader, an appreciation for the power of types
in describing the properties of programs. Additional
resources and papers can be found on the authors web page
\verb+http://cs.pdx.edu/~sheard+ where you can also
obtain the \om\ system for download.

\subsection{Relation to other systems}

In order to make \om\ accessible to as broad an audience as possible,
it is built around a framework which appears to the user to be a
pure but strict version of Haskell. \om\ was designed, first and
foremost, to be a programming language. Our goal was to design a
language where program specifications, program properties, program
analyses, proofs about programs, and programs themselves, are all
represented using a single unifying notion of term. Thus programmers
communicate many different things using the same language.

Our second goal was to make \om\ a logic, in which our reasoning would
be sound. This is the basis of our decision to make \om\ strict. We
made this design decision because the use of GADTs as proof objects
requires that bottom not be an inhabitant of certain types. Strictness
is part of our eventual strategy to accomplish that goal. This goal is
not yet achieved.

There are many systems where soundness was the principal goal, and has been
achieved. All of the examples, except for the staged examples, could be done in
these languages as well. Such systems were principally designed to be logical
frameworks or theorem provers. These include Inductive
Families~\cite{Coquand:1994:IDT,Dybjer:1999:FAI}, theorem provers (Coq~\cite{COQ74},
Isabelle~\cite{Paulson90lacs}), logical frameworks (Twelf~\cite{CADE99*202},
LEGO~\cite{LuoPollack92}), and proof assistants (ALF~\cite{oai:CiteSeerPSU:38734},
Agda~\cite{Agda}). Recently, there has been much interest in systems that use
dependent types to build ``practical" systems that are part language, part
reasoning system. These systems include Augustsson's Cayenne
language~\cite{oai:CiteSeerPSU:339817,Augustsson:1999:CLD}, McBride's
Epigram~\cite{epigram}, Stump's Rogue-Sigma-Pi~\cite{Rogue,RSP1}, Xi and Pfenning's
Dependent ML~\cite{Xi:1998:EAB,Davies97}, and Xi's Applied Type
Systems~\cite{ATS2004,ATS2005}. In fact, we owe a large debt to all these
systems for inspiration.
 
We realize that just {\em a little} loss in soundness makes all
our reasoning claims vacuous, but we are working to fill
these gaps.  Our goal is to do this in a different manner
than the systems listed above, which require all functions to be total
in order to ensure soundness. We wish to use types to separate
terminating functions from non-terminating functions, and make
logical claims only about the terminating fragment of the language.
This seems almost a necessary condition for a system that
claims to be a programming language. In any case, these issues
have little effect on our use of \om\ to program generic programs,
since logical soundness is not an issue in this domain.


\section{Acknowledgments}

This paper was originally prepared for the 
Central-European Functional Programming School held in
Cluj,  Romania, between 25-30 June, 2007.
The text, code, and examples come from many sources. Here is quick
(though, not by any means exhaustive) list.

\begin{enumerate} 

\item Section \ref{simple} comes from the paper {\em  Generic Programming in \om}\cite{GenProgOmega}.
This paper has many additional examples, not covered in this report.

\item Section \ref{features} also comes from the same paper, as well as
the {\em \om\ Users' Guide}\cite{OmegaGuide}, both of which have their roots in the paper
{\em Putting Curry-Howard to Work}\cite{CurryHoward}.

\item The shaped tree example with paths (Section \ref{shape}) comes from
discussions with James L. Caldwell from the University of Wyoming.

\item The exercise about Red-Black trees comes from some examples
posted by Hong-Wei Xi for the ATS system.\\\verb+http://www.cs.bu.edu/~hwxi/ATS/ATS.html+

\item The Leibniz equality GADT has its roots in two 
papers\cite{HinzeHaskellWorkshop02,BaSw02}
which introduce similar types with slight variants.

\item The material on using \om\ as a meta-language comes
from an unpublished paper called {\em Playing with Types}\cite{Playing},
though many of the examples used here are original to this
report.

\item The combinatorial circuit examples come from a paper
{\em Types and Hardware Description Languages}\cite{HFL07}
prepared for the {\em Hardware design and Functional Languages}
workshop held March 24-25 2007, in Braga, Portugal.

\item The answers to selected exercises was created by Ki-Yung Ahn.

\item This material is based upon work supported by the National Science
Foundation under Grant No. 0613969.
\end{enumerate}


\bibliographystyle{plain}
\bibliography{final}

\appendix

\section{Red-Black Tree Insertion} \label{redblack}

{\small
\begin{verbatim}
-------------------------------------------------------------------------
-- Introduce a new kind to represent colors

kind Color  = Red | Black

-------------------------------------------------------------------------
-- Top-level type that hides both 
-- color of the node and tree height

data RBTree:: *0 where
 Root:: SubTree Black n -> RBTree

-------------------------------------------------------------------------
-- GADT that captures invariants

data SubTree:: Color ~> Nat ~> *0 where
 Leaf:: SubTree Black Z
 RNode:: SubTree Black n ->
         Int ->
         SubTree Black n ->
         SubTree Red n
 BNode:: SubTree cL m ->
         Int ->
         SubTree cR m ->
         SubTree Black (S m)

-------------------------------------------------------------------------
-- A Ctxt records where we've been as we descend 
-- down into a tree as we search for a value

data Dir = LeftD | RightD

data Ctxt:: Color ~> Nat ~> *0 where
  Nil:: Ctxt Black n
  RCons:: Int -> Dir ->
          SubTree Black n ->
          Ctxt Red n ->
          Ctxt Black n
  BCons:: Int -> Dir ->
          SubTree c1 n ->
          Ctxt Black (S n) ->
          Ctxt c n

-------------------------------------------------------------------------
-- Turn a Red tree into a black tree. Always
-- possible, since Black nodes do not restrict
-- the color of their sub-trees.

blacken :: SubTree Red n -> SubTree Black (S n)
blacken (RNode l e r) = (BNode l e r)


-------------------------------------------------------------------------
-- A singleton type representing Color at
-- the value level.

data CRep :: Color ~> *0 where
  Red   :: CRep Red
  Black :: CRep Black

color :: SubTree c n -> CRep c
color Leaf = Black
color (RNode _ _ _) = Red
color (BNode _ _ _) = Black

-------------------------------------------------------------------------
-- fill a context with a subtree to regain the original 
-- RBTree, works if the colors and black depth match up

fill :: Ctxt c n -> SubTree c n -> RBTree
fill Nil t = Root t
fill (RCons e LeftD  uncle c) tree = fill c (RNode uncle e tree)
fill (RCons e RightD uncle c) tree = fill c (RNode tree  e uncle)
fill (BCons e LeftD  uncle c) tree = fill c (BNode uncle e tree)
fill (BCons e RightD uncle c) tree = fill c (BNode tree  e uncle)

insert :: Int -> RBTree -> RBTree
insert e (Root t) = insert_ e t Nil

-------------------------------------------------------------------------
-- as we walk down the tree, keep track of everywhere 
-- we've been in the Ctxt input.

insert_ :: Int -> SubTree c n -> Ctxt c n -> RBTree
insert_ e (RNode l e' r) ctxt
        | e < e'        = insert_ e l (RCons e' RightD r ctxt)
        | True          = insert_ e r (RCons e' LeftD  l ctxt)
insert_ e (BNode l e' r) ctxt
        | e < e'        = insert_ e l (BCons e' RightD r ctxt)
        | True          = insert_ e r (BCons e' LeftD  l ctxt)
-- once we get to the bottom we "insert" the node as a Red node.
-- since this may break invariant, we may need do some patch work
insert_ e Leaf ctxt = repair (RNode Leaf e Leaf) ctxt

-------------------------------------------------------------------------
-- Repair a tree if its out of balance. The Ctxt holds
-- crucial information about colors of parent and 
-- grand-parent nodes.

repair :: SubTree Red n -> Ctxt c n -> RBTree
repair t (Nil)                 = Root (blacken t)
repair t (BCons e LeftD  sib c) = fill c (BNode sib e t)
repair t (BCons e RightD sib c) = fill c (BNode t e sib)
-- these are the tricky cases
repair t (RCons e dir sib (BCons e' dir' uncle ctxt)) =
  case color uncle of
    Red   -> repair (recolor dir e sib dir' e' (blacken uncle) t) ctxt
    Black -> fill ctxt (rotate dir e sib dir' e' uncle t)
repair t (RCons e dir sib (RCons e' dir' uncle ctxt)) = unreachable

recolor :: Dir -> Int -> SubTree Black n ->
           Dir -> Int -> SubTree Black (S n) ->
           SubTree Red n -> SubTree Red (S n)
recolor LeftD  pE sib RightD gE uncle t = RNode (BNode sib pE t) gE uncle
recolor RightD pE sib RightD gE uncle t = RNode (BNode t pE sib) gE uncle
recolor LeftD  pE sib LeftD  gE uncle t = RNode uncle gE (BNode sib pE t)
recolor RightD pE sib LeftD  gE uncle t = RNode uncle gE (BNode t pE sib)

rotate :: Dir -> Int -> SubTree Black n ->
          Dir -> Int -> SubTree Black n ->
          SubTree Red n -> SubTree Black (S n)
rotate RightD pE sib RightD gE uncle (RNode x e y) = 
   BNode (RNode x e y) pE (RNode sib gE uncle)
rotate LeftD  pE sib RightD gE uncle (RNode x e y) = 
   BNode (RNode sib pE x) e (RNode y gE uncle)
rotate LeftD  pE sib LeftD  gE uncle (RNode x e y) = 
   BNode (RNode uncle gE sib) pE (RNode x e y)
rotate RightD pE sib LeftD  gE uncle (RNode x e y) = 
   BNode (RNode uncle gE x) e (RNode y pE sib)
\end{verbatim}}

\section{Inductively Sequential Functions} \label{IndSeq}
We restrict the form of function definitions at the type level and higher to be inductively
sequential~\cite{conf/alp/Antoy92}. If a type function is not
inductively sequential then the type checker rejects that type function.

Inductively sequential type functions ensures a sound and complete narrowing strategy for answering 
type-checking time questions. The class of inductively sequential functions is a large
one, in fact every Haskell function has an inductively sequential definition. The
inductively sequential restriction affects the form of the equations, and not the
functions that can be expressed. Informally, a function definition is
inductively sequential if all its clauses are non-overlapping. For example
the definition of {\tt zip1} is not inductively sequential, but the equivalent
program {\tt zip2} is.

{\small
\begin{verbatim}
zip1 (x:xs) (y:ys) = (x,y): (zip1 xs ys)
zip1 xs ys = []

zip2 (x:xs) (y:ys) = (x,y): (zip2 xs ys)
zip2 (x:xs) []     = []
zip2 []     ys     = []
\end{verbatim}}

The definition for {\tt zip1} is not inductively sequential, since its two clauses overlap. In general
any non-inductively sequential definition can be turned into an inductively
sequential definition by duplicating some of its clauses, instantiating variable patterns
with constructor based patterns. This will make the new clauses non-overlapping.
We do not think this burden is too much of a burden to pay, since
it is applied only to functions at the type level, and it supports
sound and complete narrowing strategies. In addition to the 
inductively sequential form required for type functions, \om\ assumes that
each type function is a total terminating function. This assumption
is not currently enforced, and it is up to the programmer
to ensure that this is the case.

\section{Answers to Selected Exercises} \label{answers}

{\small
\begin{verbatim}

--------------
-- Exercise  1
--------------
data Seq :: *0 ~> Nat ~> *0 where
  Snil  :: Seq a Z
  Scons :: a -> Seq a n -> Seq a (S n)

length :: Seq a n -> Int
length Snil         = 0
length (Scons _ xs) = 1 + length xs

-- we can can also use (Nat' n) (see 3.7) 
-- to ensure that the size of the result is n

length' :: Seq a n -> Nat' n
legnth' Snil         = Z
length' (Scons _ xs) = S (length' xs)

--------------
-- Exercise  3
--------------
data Color :: *1 where
  Red   :: Color
  Black :: Color

data RBT :: Color ~> *0 where
  LeafB :: RBT Black
  NodeR :: RBT Black -> RBT Black -> RBT Red
  NodeB :: RBT cL    -> RBT cR    -> RBT Black


--------------
-- Exercise  4
--------------
plus :: Nat ~> Nat ~> Nat
{plus Z m} = m
{plus (S n) m} = S {plus n m}

mult :: Nat ~> Nat ~> Nat
{mult Z m} = Z
{mult (S n) m} = {plus {mult n m} m}


--------------
-- Exercise  5
--------------
data Boolean :: *1 where
  T :: Boolean
  F :: Boolean

odd :: Nat ~> Boolean
{odd Z}     = F
{odd (S Z)} = T
{odd (S (S n))} = {odd n}

--------------
-- Exercise  6
--------------

or :: Boolean ~> Boolean ~> Boolean
{or T b} = T
{or F b} = b

-- The function (not :: Bool -> Bool) is predefined
-- so we use different name
not' :: Boolean ~> Boolean
{not' T} = F
{not' F} = T


--------------
-- Exercise  7
--------------
data Shape :: *1 where
  Tp :: Shape
  Nd :: Shape
  Fk :: Shape ~> Shape ~> Shape

data Path :: Shape ~> *0 ~> *0 where
  None  :: Path Tp a
  Here  :: b -> Path Nd b
  Left  :: Path x a -> Path (Fk x y) a
  Right :: Path y a -> Path (Fk x y) a

data Tree :: Shape ~> *0 ~> *0 where
  Tip :: Tree Tp a
  Node :: a -> Tree Nd a
  Fork :: Tree x a -> Tree y a -> Tree (Fk x y) a

extract :: Path sh a -> Tree sh a -> a
extract None      Tip          = error "(extract None Tip) has nothing"
extract (Here _)  (Node v)     = v
extract (Left p)  (Fork lt rt) = extract p lt
extract (Right p) (Fork lt rt) = extract p rt


--------------
-- Exercise  8
--------------
data ListShape :: *1 where
  LSnil  :: ListShape
  LScons :: ListShape ~> ListShape

data List :: ListShape ~> *0 ~> *0 where
  Lnil  :: List LSnil a
  Lcons :: a -> List sh a -> List (LScons sh) a

data ListPath :: ListShape ~> *0 ~> *0 where
  ListNone :: ListPath LSnil a
  ListHere :: b -> ListPath (LScons sh) b
  ListNext :: ListPath sh a -> ListPath (LScons sh) a

find :: (a -> a -> Bool) -> a -> List sh a -> Maybe(ListPath sh a)
find eq n Lnil         
   = Nothing
find eq n (Lcons x xs) 
   = if eq n x 
        then Just (ListHere n)
        else case find eq n xs of
               Nothing -> Nothing
               Just p  -> Just (ListNext p)


--------------
-- Exercise  9
--------------
data Rep :: *0 ~> *0 where
  Int  :: Rep Int
  Bool :: Rep Bool
  Prod :: Rep a -> Rep b -> Rep (a,b)
  List :: Rep a -> Rep [a]

showR :: Rep a -> a -> String
showR Int        n  = show n
showR Bool       True  = "True"
showR Bool       False  = "False"
showR (Prod x y) (a,b) = "("++showR x a++","++showR y b++")"
showR (List t)   xs = "["++ help xs ++ "]"
  where help [x] = showR t x
        help []  = ""
        help (x:xs) = showR t x++","++help xs

--------------
-- Exercise 10
--------------
data Plus :: Nat ~> Nat ~> Nat ~> *0 where
  PlusZ :: Plus Z m m
  PlusS :: Plus n m z -> Plus (S n) m (S z)

plus2v3v5v :: Plus 2t 3t 5t
plus2v3v5v = PlusS (PlusS PlusZ)

plus2v1v3v :: Plus 2t 1t 3t
plus2v1v3v = PlusS (PlusS PlusZ)

plus2v6v8v :: Plus 2t 6t 8t
plus2v6v8v = PlusS (PlusS PlusZ)


--------------
-- Exercise 11
--------------
data LE :: Nat ~> Nat ~> *0 where
  LeZ :: LE Z n
  LeS :: LE n m -> LE (S n) (S m)

sumandLessThanOrEqualToSum :: Plus a b c -> LE a c
sumandLessThanOrEqualToSum PlusZ     = LeZ
sumandLessThanOrEqualToSum (PlusS p) = LeS (sumandLessThanOrEqualToSum p)

-- Can we define a function with type (Plus a b c -> LE b c)?
-- not exactly, but we can write one with a similar type.

sumandLTorEQ2sum' :: Nat' c -> Plus a b c -> LE b c
sumandLTorEQ2sum' n     PlusZ     = same n
sumandLTorEQ2sum' Z     (PlusS _) = unreachable
sumandLTorEQ2sum' (S n) (PlusS p) = predLE (sumandLTorEQ2sum' n p)

-- see Exercise 13 for the definitions of same and predLE.

--------------
-- Exercise 12
--------------
even :: Nat ~> Boolean
{even Z}     = T
{even (S Z)} = F
{even (S (S n))} = {even n}

data EvenRel :: Nat ~> Boolean ~> *0 where
  Er0  :: EvenRel 0t T
  Er1  :: EvenRel 1t F
  ErSS :: EvenRel n b -> EvenRel (S (S n)) b


--------------
-- Exercise 13
--------------
same :: Nat' n -> LE n n
same Z     = LeZ
same (S n) = LeS (same n)

predLE :: LE m n -> LE m (S n)
predLE LeZ     = LeZ
predLE (LeS p) = LeS (predLE p)


--------------
-- Exercise 14
--------------
trans :: LE a b -> LE b c -> LE a c
trans LeZ      _        = LeZ
trans (LeS _)  LeZ      = unreachable
trans (LeS p1) (LeS p2) = LeS (trans p1 p2)


--------------
-- Exercise 15
--------------
f15 :: Nat' b -> Plus a b c -> LE b c
f15 n     PlusZ     = same n
f15 Z     (PlusS _) = LeZ
f15 (S n) (PlusS p) = predLE (f15 (S n) p)

--------------
-- Exercise 16
--------------
sameNat' :: Nat' a -> Nat' b -> Maybe (Equal a b)
sameNat' Z     Z     = Just Eq
sameNat' Z     (S _) = Nothing
sameNat' (S _) Z     = Nothing
sameNat' (S n) (S m) = case sameNat' n m of
                         Nothing -> Nothing
                         Just Eq -> Just Eq 
\end{verbatim}}

{\small 
\begin{verbatim}
--------------
-- Exercise 17
--------------
filter :: (a->Bool) -> Seq a n -> exists m . (Nat' m, Seq a m)
filter p Snil         = Ex (Z, Snil)
filter p (Scons x xs) =
  case filter p xs of
    Ex (n, xs') -> if p x then Ex (S n, Scons x xs')
                          else Ex (n, xs')

filter' :: (a->Bool) -> Seq a n -> exists m . (LE m n, Nat' m, Seq a m)
filter' p Snil         = Ex (LeZ, Z, Snil)
filter' p (Scons x xs) =
  case filter' p xs of
    Ex (le, n, xs') -> if p x then Ex (LeS le, S n, Scons x xs')
                              else Ex (predLE le, n, xs')

--------------
-- Exercise 18
--------------
pow :: Int -> Code Int -> Code Int
pow 0 _ = [| 1 |]
pow n x = [| $(x) * $(pow (n - 1) x) |]

--------------
-- Exercise 19
--------------
-- Row is already defined so we use MyRow

data MyRow :: a ~> c ~> *1 where
  Rnil :: MyRow e f
  Rcons :: e ~> f ~> MyRow e f ~> MyRow e f
 deriving Record(mr)

-- We derive syntax 'mr' because the
-- predefined Row uses syntax 'r' already.

--------------
-- Exercise 20
--------------
data Nsum :: *0 ~> *0 where
  SumZ :: Nsum Int
  SumS :: Nsum x -> Nsum (Int -> x)
 deriving Nat(i)

-- 0i : Nsum Int
-- 1i : Nsum (Int -> Int)
-- 2i : Nsum (Int -> Int -> Int)

add :: Nsum i -> i
add = add' 0

add' :: Int -> Nsum i -> i
add' x 0i     = x
add' x (1+n)i = \k -> add' (x+k) n


--------------
-- Exercise 21
--------------
data Expr :: *0 where
  VarExpr  :: Label t -> Expr
  PlusExpr :: Expr -> Expr -> Expr

valueOf :: Expr -> [exists t .(Label t,Int)] -> Int
valueOf (VarExpr v)    env = lookup v env
valueOf (PlusExpr x y) env = valueOf x env + valueOf y env

lookup :: Label v -> [exists t .(Label t,Int)] -> Int
lookup v ((Ex(u,n)):xs) = 
 case labelEq v u of 
   Just Eq -> n
   Nothing -> lookup v xs

pair1:: exists t .(Label t,Int)
pair1 = Ex(`a,5)

pair2:: exists t .(Label t,Int)
pair2 = Ex(`x,22)

pair3:: exists t .(Label t,Int)
pair3 = Ex(`z,2)

table :: [exists t .(Label t,Int)]
table = [pair1,pair2,pair3]

xValue = valueOf (VarExpr `x) table


--------------
-- Exercise 22
--------------

-- see Appendix A


--------------
-- Exercise 23
--------------

{-  already defined in Exercise 9 
data Rep :: *0 ~> *0 where
  Int  :: Rep Int
  Bool :: Rep Bool
  Prod :: Rep a -> Rep b -> Rep (a,b)
  List :: Rep a -> Rep [a]
-} 
equalRep :: Rep a -> Rep b -> Maybe (Equal a b)
equalRep Int        Int        = Just Eq
equalRep Bool       Bool       = Just Eq
equalRep (Prod a b) (Prod c d) = 
  case equalRep a c of
    Nothing -> Nothing
    Just Eq -> case equalRep b d of
                 Nothing -> Nothing
                 Just Eq -> Just Eq
-- alternatively we could use Monad syntax 
equalRep (Prod a b) (Prod c d) = 
  do { Eq <- equalRep a c
     ; Eq <- equalRep b d
     ; return Eq}
 where monad maybeM     
equalRep _ _ = Nothing

maybeM = Monad (Just) bind fail
  where bind (Just x) f = f x
        fail s = Nothing
        
data Term :: *0 ~> *0 where
  Var   :: String -> Rep t -> Term t         -- x
  Const :: Int -> Term Int                   -- 5
  Add   :: Term ((Int,Int) -> Int)           -- (+)
  LT    :: Term ((Int,Int) -> Bool)          -- (<)
  Ap    :: Term(a -> b) -> Term a -> Term b  -- (+) (x,y)
  Pair  :: Term a -> Term b -> Term(a,b)     -- (x,y)        
  
type Env = [ exists t . (String, Rep t, t) ]  

lookupWithRepr :: Env -> Rep t -> String -> t
lookupWithRepr [] r1 x1 = error "variable not found"
lookupWithRepr (Ex(x,r,v):ts) r1 x1 
  = if eqStr x x1
       then case equalRep r r1 of
              Just Eq -> v
              Nothing -> lookupWithRepr ts r1 x1
       else lookupWithRepr ts r1 x1
       
uncurry f (x,y) = f x y

eval :: Term t -> Env -> t
eval (Var x r)  env = lookupWithRepr env r x
eval (Const i)  _   = i
eval Add        _   = uncurry (+)
eval LT         _   = uncurry (<)
eval (Ap f p)   env = (eval f env) (eval p env)
eval (Pair a b) env = (eval a env, eval b env)       

--------------
-- Exercise 25
--------------

opt:: Term a -> Term a
opt (Ap Add (Pair (Const n) (Const m))) 
  -- constant folding
  = Const(n+m)
opt (Ap Add (Pair (Const 0) x)) 
  -- law: (0 + x)=x
  = x
opt (Ap Add (Pair x (Const 0))) 
  -- law: (x + 0)=x
  = x
opt (Ap x y) = Ap (opt x) (opt y)
opt (Pair x y) = Pair (opt x) (opt y)
opt x = x

-- can you make opt work for (x + (3 + -3)) or (1 + (2 + 4))

stagedEvalTerm :: Term a -> Code a
stagedEvalTerm (Const x) = lift x
stagedEvalTerm Add = [| add |]
  where add (x,y) = x+y
stagedEvalTerm LT = [| less |]
  where less (x,y) = x < y
stagedEvalTerm (Ap f x) = [| $(stagedEvalTerm f) $(stagedEvalTerm x) |]
stagedEvalTerm (Pair x y) = [|($(stagedEvalTerm x),$(stagedEvalTerm y))|]


optStagedEvalTerm x = stagedEvalTerm(opt x)

\end{verbatim}}
 
\end{document}
