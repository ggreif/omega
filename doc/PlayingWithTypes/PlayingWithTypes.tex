\documentclass{sigplanconf}
\usepackage{hyperref}   
\usepackage{makeidx,multicol,amssymb}
% Psfig/TeX Release 1.3
% dvips version
%
% All software, documentation, and related files in this distribution of
% psfig/tex are Copyright 1987, 1988 Trevor J. Darrell
%
% Permission is granted for use and non-profit distribution of psfig/tex 
% providing that this notice be clearly maintained, but the right to
% distribute any portion of psfig/tex for profit or as part of any commercial
% product is specifically reserved for the author.
%
% $Header: /u/sheard/cvs//GPCE05MetaMLTypes/psfig.tex,v 1.1.1.1 2005/04/07 18:31:53 sheard Exp $
% $Source: /u/sheard/cvs//GPCE05MetaMLTypes/psfig.tex,v $
%
% Thanks to Greg Hager (GDH) and Ned Batchelder for their contributions
% to this project.
%
% Modified by J. Daniel Smith on 9 October 1990 to accept the
% %%BoundingBox: comment with or without a space after the colon.  Stole
% file reading code from Tom Rokicki's EPSF.TEX file (see below).

%
% from a suggestion by eijkhout@csrd.uiuc.edu to allow
% loading as a style file:
\edef\psfigRestoreAt{\catcode`@=\number\catcode`@\relax}
\catcode`\@=11\relax
\newwrite\@unused
\def\typeout#1{{\let\protect\string\immediate\write\@unused{#1}}}
\typeout{psfig/tex 1.4-dvips}


%% Here's how you define your figure path.  Should be set up with null
%% default and a user useable definition.

\def\figurepath{./}
\def\psfigurepath#1{\edef\figurepath{#1}}

%
% @psdo control structure -- similar to Latex @for.
% I redefined these with different names so that psfig can
% be used with TeX as well as LaTeX, and so that it will not 
% be vunerable to future changes in LaTeX's internal
% control structure,
%
\def\@nnil{\@nil}
\def\@empty{}
\def\@psdonoop#1\@@#2#3{}
\def\@psdo#1:=#2\do#3{\edef\@psdotmp{#2}\ifx\@psdotmp\@empty \else
    \expandafter\@psdoloop#2,\@nil,\@nil\@@#1{#3}\fi}
\def\@psdoloop#1,#2,#3\@@#4#5{\def#4{#1}\ifx #4\@nnil \else
       #5\def#4{#2}\ifx #4\@nnil \else#5\@ipsdoloop #3\@@#4{#5}\fi\fi}
\def\@ipsdoloop#1,#2\@@#3#4{\def#3{#1}\ifx #3\@nnil 
       \let\@nextwhile=\@psdonoop \else
      #4\relax\let\@nextwhile=\@ipsdoloop\fi\@nextwhile#2\@@#3{#4}}
\def\@tpsdo#1:=#2\do#3{\xdef\@psdotmp{#2}\ifx\@psdotmp\@empty \else
    \@tpsdoloop#2\@nil\@nil\@@#1{#3}\fi}
\def\@tpsdoloop#1#2\@@#3#4{\def#3{#1}\ifx #3\@nnil 
       \let\@nextwhile=\@psdonoop \else
      #4\relax\let\@nextwhile=\@tpsdoloop\fi\@nextwhile#2\@@#3{#4}}
% 
%
%%%%%%%%%%%%%%%%%%%%%%%%%%%%%%%%%%%%%%%%%%%%%%%%%%%%%%%%%%%%%%%%%%%
% file reading stuff from epsf.tex
%   EPSF.TEX macro file:
%   Written by Tomas Rokicki of Radical Eye Software, 29 Mar 1989.
%   Revised by Don Knuth, 3 Jan 1990.
%   Revised by Tomas Rokicki to accept bounding boxes with no
%      space after the colon, 18 Jul 1990.
%   Portions modified/removed for use in PSFIG package by
%      J. Daniel Smith, 9 October 1990.
%
\newread\ps@stream
\newif\ifnot@eof       % continue looking for the bounding box?
\newif\if@noisy        % report what you're making?
\newif\if@atend        % %%BoundingBox: has (at end) specification
\newif\if@psfile       % does this look like a PostScript file?
%
% PostScript files should start with `%!'
%
{\catcode`\%=12\global\gdef\epsf@start{%!}}
\def\epsf@PS{PS}
%
\def\epsf@getbb#1{%
%
%   The first thing we need to do is to open the
%   PostScript file, if possible.
%
\openin\ps@stream=#1
\ifeof\ps@stream\typeout{Error, File #1 not found}\else
%
%   Okay, we got it. Now we'll scan lines until we find one that doesn't
%   start with %. We're looking for the bounding box comment.
%
   {\not@eoftrue \chardef\other=12
    \def\do##1{\catcode`##1=\other}\dospecials \catcode`\ =10
    \loop
       \if@psfile
	  \read\ps@stream to \epsf@fileline
       \else{
	  \obeyspaces
          \read\ps@stream to \epsf@tmp\global\let\epsf@fileline\epsf@tmp}
       \fi
       \ifeof\ps@stream\not@eoffalse\else
%
%   Check the first line for `%!'.  Issue a warning message if its not
%   there, since the file might not be a PostScript file.
%
       \if@psfile\else
       \expandafter\epsf@test\epsf@fileline:. \\%
       \fi
%
%   We check to see if the first character is a % sign;
%   if so, we look further and stop only if the line begins with
%   `%%BoundingBox:' and the `(atend)' specification was not found.
%   That is, the only way to stop is when the end of file is reached,
%   or a `%%BoundingBox: llx lly urx ury' line is found.
%
          \expandafter\epsf@aux\epsf@fileline:. \\%
       \fi
   \ifnot@eof\repeat
   }\closein\ps@stream\fi}%
%
% This tests if the file we are reading looks like a PostScript file.
%
\long\def\epsf@test#1#2#3:#4\\{\def\epsf@testit{#1#2}
			\ifx\epsf@testit\epsf@start\else
\typeout{Warning! File does not start with `\epsf@start'.  It may not be a PostScript file.}
			\fi
			\@psfiletrue} % don't test after 1st line
%
%   We still need to define the tricky \epsf@aux macro. This requires
%   a couple of magic constants for comparison purposes.
%
{\catcode`\%=12\global\let\epsf@percent=%\global\def\epsf@bblit{%BoundingBox}}
%
%
%   So we're ready to check for `%BoundingBox:' and to grab the
%   values if they are found.  We continue searching if `(at end)'
%   was found after the `%BoundingBox:'.
%
\long\def\epsf@aux#1#2:#3\\{\ifx#1\epsf@percent
   \def\epsf@testit{#2}\ifx\epsf@testit\epsf@bblit
	\@atendfalse
        \epsf@atend #3 . \\%
	\if@atend	
	   \if@verbose{
		\typeout{psfig: found `(atend)'; continuing search}
	   }\fi
        \else
        \epsf@grab #3 . . . \\%
        \not@eoffalse
        \global\no@bbfalse
        \fi
   \fi\fi}%
%
%   Here we grab the values and stuff them in the appropriate definitions.
%
\def\epsf@grab #1 #2 #3 #4 #5\\{%
   \global\def\epsf@llx{#1}\ifx\epsf@llx\empty
      \epsf@grab #2 #3 #4 #5 .\\\else
   \global\def\epsf@lly{#2}%
   \global\def\epsf@urx{#3}\global\def\epsf@ury{#4}\fi}%
%
% Determine if the stuff following the %%BoundingBox is `(atend)'
% J. Daniel Smith.  Copied from \epsf@grab above.
%
\def\epsf@atendlit{(atend)} 
\def\epsf@atend #1 #2 #3\\{%
   \def\epsf@tmp{#1}\ifx\epsf@tmp\empty
      \epsf@atend #2 #3 .\\\else
   \ifx\epsf@tmp\epsf@atendlit\@atendtrue\fi\fi}


% End of file reading stuff from epsf.tex
%%%%%%%%%%%%%%%%%%%%%%%%%%%%%%%%%%%%%%%%%%%%%%%%%%%%%%%%%%%%%%%%%%%

\def\psdraft{
	\def\@psdraft{0}
	%\typeout{draft level now is \@psdraft \space . }
}
\def\psfull{
	\def\@psdraft{100}
	%\typeout{draft level now is \@psdraft \space . }
}
\psfull
\newif\if@prologfile
\newif\if@postlogfile
\def\pssilent{
	\@noisyfalse
}
\def\psnoisy{
	\@noisytrue
}
\psnoisy
%%% These are for the option list.
%%% A specification of the form a = b maps to calling \@p@@sa{b}
\newif\if@bbllx
\newif\if@bblly
\newif\if@bburx
\newif\if@bbury
\newif\if@height
\newif\if@width
\newif\if@rheight
\newif\if@rwidth
\newif\if@clip
\newif\if@verbose
\def\@p@@sclip#1{\@cliptrue}

%%% GDH 7/26/87 -- changed so that it first looks in the local directory,
%%% then in a specified global directory for the ps file.

\def\@p@@sfile#1{\def\@p@sfile{null}%
	        \openin1=#1
		\ifeof1\closein1%
		       \openin1=\figurepath#1
			\ifeof1\typeout{Error, File #1 not found}
			\else\closein1
			    \edef\@p@sfile{\figurepath#1}%
                        \fi%
		 \else\closein1%
		       \def\@p@sfile{#1}%
		 \fi}
\def\@p@@sfigure#1{\def\@p@sfile{null}%
	        \openin1=#1
		\ifeof1\closein1%
		       \openin1=\figurepath#1
			\ifeof1\typeout{Error, File #1 not found}
			\else\closein1
			    \def\@p@sfile{\figurepath#1}%
                        \fi%
		 \else\closein1%
		       \def\@p@sfile{#1}%
		 \fi}

\def\@p@@sbbllx#1{
		%\typeout{bbllx is #1}
		\@bbllxtrue
		\dimen100=#1
		\edef\@p@sbbllx{\number\dimen100}
}
\def\@p@@sbblly#1{
		%\typeout{bblly is #1}
		\@bbllytrue
		\dimen100=#1
		\edef\@p@sbblly{\number\dimen100}
}
\def\@p@@sbburx#1{
		%\typeout{bburx is #1}
		\@bburxtrue
		\dimen100=#1
		\edef\@p@sbburx{\number\dimen100}
}
\def\@p@@sbbury#1{
		%\typeout{bbury is #1}
		\@bburytrue
		\dimen100=#1
		\edef\@p@sbbury{\number\dimen100}
}
\def\@p@@sheight#1{
		\@heighttrue
		\dimen100=#1
   		\edef\@p@sheight{\number\dimen100}
		%\typeout{Height is \@p@sheight}
}
\def\@p@@swidth#1{
		%\typeout{Width is #1}
		\@widthtrue
		\dimen100=#1
		\edef\@p@swidth{\number\dimen100}
}
\def\@p@@srheight#1{
		%\typeout{Reserved height is #1}
		\@rheighttrue
		\dimen100=#1
		\edef\@p@srheight{\number\dimen100}
}
\def\@p@@srwidth#1{
		%\typeout{Reserved width is #1}
		\@rwidthtrue
		\dimen100=#1
		\edef\@p@srwidth{\number\dimen100}
}
\def\@p@@ssilent#1{ 
		\@verbosefalse
}
\def\@p@@sprolog#1{\@prologfiletrue\def\@prologfileval{#1}}
\def\@p@@spostlog#1{\@postlogfiletrue\def\@postlogfileval{#1}}
\def\@cs@name#1{\csname #1\endcsname}
\def\@setparms#1=#2,{\@cs@name{@p@@s#1}{#2}}
%
% initialize the defaults (size the size of the figure)
%
\def\ps@init@parms{
		\@bbllxfalse \@bbllyfalse
		\@bburxfalse \@bburyfalse
		\@heightfalse \@widthfalse
		\@rheightfalse \@rwidthfalse
		\def\@p@sbbllx{}\def\@p@sbblly{}
		\def\@p@sbburx{}\def\@p@sbbury{}
		\def\@p@sheight{}\def\@p@swidth{}
		\def\@p@srheight{}\def\@p@srwidth{}
		\def\@p@sfile{}
		\def\@p@scost{10}
		\def\@sc{}
		\@prologfilefalse
		\@postlogfilefalse
		\@clipfalse
		\if@noisy
			\@verbosetrue
		\else
			\@verbosefalse
		\fi
}
%
% Go through the options setting things up.
%
\def\parse@ps@parms#1{
	 	\@psdo\@psfiga:=#1\do
		   {\expandafter\@setparms\@psfiga,}}
%
% Compute bb height and width
%
\newif\ifno@bb
\def\bb@missing{
	\if@verbose{
		\typeout{psfig: searching \@p@sfile \space  for bounding box}
	}\fi
	\no@bbtrue
	\epsf@getbb{\@p@sfile}
        \ifno@bb \else \bb@cull\epsf@llx\epsf@lly\epsf@urx\epsf@ury\fi
}	
\def\bb@cull#1#2#3#4{
	\dimen100=#1 bp\edef\@p@sbbllx{\number\dimen100}
	\dimen100=#2 bp\edef\@p@sbblly{\number\dimen100}
	\dimen100=#3 bp\edef\@p@sbburx{\number\dimen100}
	\dimen100=#4 bp\edef\@p@sbbury{\number\dimen100}
	\no@bbfalse
}
\def\compute@bb{
		\no@bbfalse
		\if@bbllx \else \no@bbtrue \fi
		\if@bblly \else \no@bbtrue \fi
		\if@bburx \else \no@bbtrue \fi
		\if@bbury \else \no@bbtrue \fi
		\ifno@bb \bb@missing \fi
		\ifno@bb \typeout{FATAL ERROR: no bb supplied or found}
			\no-bb-error
		\fi
		%
		\count203=\@p@sbburx
		\count204=\@p@sbbury
		\advance\count203 by -\@p@sbbllx
		\advance\count204 by -\@p@sbblly
		\edef\@bbw{\number\count203}
		\edef\@bbh{\number\count204}
		%\typeout{ bbh = \@bbh, bbw = \@bbw }
}
%
% \in@hundreds performs #1 * (#2 / #3) correct to the hundreds,
%	then leaves the result in @result
%
\def\in@hundreds#1#2#3{\count240=#2 \count241=#3
		     \count100=\count240	% 100 is first digit #2/#3
		     \divide\count100 by \count241
		     \count101=\count100
		     \multiply\count101 by \count241
		     \advance\count240 by -\count101
		     \multiply\count240 by 10
		     \count101=\count240	%101 is second digit of #2/#3
		     \divide\count101 by \count241
		     \count102=\count101
		     \multiply\count102 by \count241
		     \advance\count240 by -\count102
		     \multiply\count240 by 10
		     \count102=\count240	% 102 is the third digit
		     \divide\count102 by \count241
		     \count200=#1\count205=0
		     \count201=\count200
			\multiply\count201 by \count100
		 	\advance\count205 by \count201
		     \count201=\count200
			\divide\count201 by 10
			\multiply\count201 by \count101
			\advance\count205 by \count201
			%
		     \count201=\count200
			\divide\count201 by 100
			\multiply\count201 by \count102
			\advance\count205 by \count201
			%
		     \edef\@result{\number\count205}
}
\def\compute@wfromh{
		% computing : width = height * (bbw / bbh)
		\in@hundreds{\@p@sheight}{\@bbw}{\@bbh}
		%\typeout{ \@p@sheight * \@bbw / \@bbh, = \@result }
		\edef\@p@swidth{\@result}
		%\typeout{w from h: width is \@p@swidth}
}
\def\compute@hfromw{
		% computing : height = width * (bbh / bbw)
		\in@hundreds{\@p@swidth}{\@bbh}{\@bbw}
		%\typeout{ \@p@swidth * \@bbh / \@bbw = \@result }
		\edef\@p@sheight{\@result}
		%\typeout{h from w : height is \@p@sheight}
}
\def\compute@handw{
		\if@height 
			\if@width
			\else
				\compute@wfromh
			\fi
		\else 
			\if@width
				\compute@hfromw
			\else
				\edef\@p@sheight{\@bbh}
				\edef\@p@swidth{\@bbw}
			\fi
		\fi
}
\def\compute@resv{
		\if@rheight \else \edef\@p@srheight{\@p@sheight} \fi
		\if@rwidth \else \edef\@p@srwidth{\@p@swidth} \fi
}
%		
% Compute any missing values
\def\compute@sizes{
	\compute@bb
	\compute@handw
	\compute@resv
}
%
% \psfig
% usage : \psfig{file=, height=, width=, bbllx=, bblly=, bburx=, bbury=,
%			rheight=, rwidth=, clip=}
%
% "clip=" is a switch and takes no value, but the `=' must be present.
\def\psfig#1{\vbox {
	% do a zero width hard space so that a single
	% \psfig in a centering enviornment will behave nicely
	%{\setbox0=\hbox{\ }\ \hskip-\wd0}
	%
	\ps@init@parms
	\parse@ps@parms{#1}
	\compute@sizes
	%
	\ifnum\@p@scost<\@psdraft{
		\if@verbose{
			\typeout{psfig: including \@p@sfile \space }
		}\fi
		%
		\special{ps::[begin] 	\@p@swidth \space \@p@sheight \space
				\@p@sbbllx \space \@p@sbblly \space
				\@p@sbburx \space \@p@sbbury \space
				startTexFig \space }
		\if@clip{
			\if@verbose{
				\typeout{(clip)}
			}\fi
			\special{ps:: doclip \space }
		}\fi
		\if@prologfile
		    \special{ps: plotfile \@prologfileval \space } \fi
		\special{ps: plotfile \@p@sfile \space }
		\if@postlogfile
		    \special{ps: plotfile \@postlogfileval \space } \fi
		\special{ps::[end] endTexFig \space }
		% Create the vbox to reserve the space for the figure
		\vbox to \@p@srheight true sp{
			\hbox to \@p@srwidth true sp{
				\hss
			}
		\vss
		}
	}\else{
		% draft figure, just reserve the space and print the
		% path name.
		\vbox to \@p@srheight true sp{
		\vss
			\hbox to \@p@srwidth true sp{
				\hss
				\if@verbose{
					\@p@sfile
				}\fi
				\hss
			}
		\vss
		}
	}\fi
}}
\def\psglobal{\typeout{psfig: PSGLOBAL is OBSOLETE; use psprint -m instead}}
\psfigRestoreAt






				% <---- load psfig macros
\newcommand{\psinput}[1]{
  \mbox{\psfig{figure=#1}}}
\newcommand{\hgap}{\hspace{12pt}}
\newcommand{\vgap}{\\[12pt]}
\newcommand{\pscap}[2]{
  \parbox{2.01in}{
   \begin{center}
    \psinput{#1} \\ #2
   \end{center}
  }
}

% Psfig/TeX Release 1.3
% dvips version
%
% All software, documentation, and related files in this distribution of
% psfig/tex are Copyright 1987, 1988 Trevor J. Darrell
%
% Permission is granted for use and non-profit distribution of psfig/tex 
% providing that this notice be clearly maintained, but the right to
% distribute any portion of psfig/tex for profit or as part of any commercial
% product is specifically reserved for the author.
%
% $Header: /u/sheard/cvs//GPCE05MetaMLTypes/psfig.tex,v 1.1.1.1 2005/04/07 18:31:53 sheard Exp $
% $Source: /u/sheard/cvs//GPCE05MetaMLTypes/psfig.tex,v $
%
% Thanks to Greg Hager (GDH) and Ned Batchelder for their contributions
% to this project.
%
% Modified by J. Daniel Smith on 9 October 1990 to accept the
% %%BoundingBox: comment with or without a space after the colon.  Stole
% file reading code from Tom Rokicki's EPSF.TEX file (see below).

%
% from a suggestion by eijkhout@csrd.uiuc.edu to allow
% loading as a style file:
\edef\psfigRestoreAt{\catcode`@=\number\catcode`@\relax}
\catcode`\@=11\relax
\newwrite\@unused
\def\typeout#1{{\let\protect\string\immediate\write\@unused{#1}}}
\typeout{psfig/tex 1.4-dvips}


%% Here's how you define your figure path.  Should be set up with null
%% default and a user useable definition.

\def\figurepath{./}
\def\psfigurepath#1{\edef\figurepath{#1}}

%
% @psdo control structure -- similar to Latex @for.
% I redefined these with different names so that psfig can
% be used with TeX as well as LaTeX, and so that it will not 
% be vunerable to future changes in LaTeX's internal
% control structure,
%
\def\@nnil{\@nil}
\def\@empty{}
\def\@psdonoop#1\@@#2#3{}
\def\@psdo#1:=#2\do#3{\edef\@psdotmp{#2}\ifx\@psdotmp\@empty \else
    \expandafter\@psdoloop#2,\@nil,\@nil\@@#1{#3}\fi}
\def\@psdoloop#1,#2,#3\@@#4#5{\def#4{#1}\ifx #4\@nnil \else
       #5\def#4{#2}\ifx #4\@nnil \else#5\@ipsdoloop #3\@@#4{#5}\fi\fi}
\def\@ipsdoloop#1,#2\@@#3#4{\def#3{#1}\ifx #3\@nnil 
       \let\@nextwhile=\@psdonoop \else
      #4\relax\let\@nextwhile=\@ipsdoloop\fi\@nextwhile#2\@@#3{#4}}
\def\@tpsdo#1:=#2\do#3{\xdef\@psdotmp{#2}\ifx\@psdotmp\@empty \else
    \@tpsdoloop#2\@nil\@nil\@@#1{#3}\fi}
\def\@tpsdoloop#1#2\@@#3#4{\def#3{#1}\ifx #3\@nnil 
       \let\@nextwhile=\@psdonoop \else
      #4\relax\let\@nextwhile=\@tpsdoloop\fi\@nextwhile#2\@@#3{#4}}
% 
%
%%%%%%%%%%%%%%%%%%%%%%%%%%%%%%%%%%%%%%%%%%%%%%%%%%%%%%%%%%%%%%%%%%%
% file reading stuff from epsf.tex
%   EPSF.TEX macro file:
%   Written by Tomas Rokicki of Radical Eye Software, 29 Mar 1989.
%   Revised by Don Knuth, 3 Jan 1990.
%   Revised by Tomas Rokicki to accept bounding boxes with no
%      space after the colon, 18 Jul 1990.
%   Portions modified/removed for use in PSFIG package by
%      J. Daniel Smith, 9 October 1990.
%
\newread\ps@stream
\newif\ifnot@eof       % continue looking for the bounding box?
\newif\if@noisy        % report what you're making?
\newif\if@atend        % %%BoundingBox: has (at end) specification
\newif\if@psfile       % does this look like a PostScript file?
%
% PostScript files should start with `%!'
%
{\catcode`\%=12\global\gdef\epsf@start{%!}}
\def\epsf@PS{PS}
%
\def\epsf@getbb#1{%
%
%   The first thing we need to do is to open the
%   PostScript file, if possible.
%
\openin\ps@stream=#1
\ifeof\ps@stream\typeout{Error, File #1 not found}\else
%
%   Okay, we got it. Now we'll scan lines until we find one that doesn't
%   start with %. We're looking for the bounding box comment.
%
   {\not@eoftrue \chardef\other=12
    \def\do##1{\catcode`##1=\other}\dospecials \catcode`\ =10
    \loop
       \if@psfile
	  \read\ps@stream to \epsf@fileline
       \else{
	  \obeyspaces
          \read\ps@stream to \epsf@tmp\global\let\epsf@fileline\epsf@tmp}
       \fi
       \ifeof\ps@stream\not@eoffalse\else
%
%   Check the first line for `%!'.  Issue a warning message if its not
%   there, since the file might not be a PostScript file.
%
       \if@psfile\else
       \expandafter\epsf@test\epsf@fileline:. \\%
       \fi
%
%   We check to see if the first character is a % sign;
%   if so, we look further and stop only if the line begins with
%   `%%BoundingBox:' and the `(atend)' specification was not found.
%   That is, the only way to stop is when the end of file is reached,
%   or a `%%BoundingBox: llx lly urx ury' line is found.
%
          \expandafter\epsf@aux\epsf@fileline:. \\%
       \fi
   \ifnot@eof\repeat
   }\closein\ps@stream\fi}%
%
% This tests if the file we are reading looks like a PostScript file.
%
\long\def\epsf@test#1#2#3:#4\\{\def\epsf@testit{#1#2}
			\ifx\epsf@testit\epsf@start\else
\typeout{Warning! File does not start with `\epsf@start'.  It may not be a PostScript file.}
			\fi
			\@psfiletrue} % don't test after 1st line
%
%   We still need to define the tricky \epsf@aux macro. This requires
%   a couple of magic constants for comparison purposes.
%
{\catcode`\%=12\global\let\epsf@percent=%\global\def\epsf@bblit{%BoundingBox}}
%
%
%   So we're ready to check for `%BoundingBox:' and to grab the
%   values if they are found.  We continue searching if `(at end)'
%   was found after the `%BoundingBox:'.
%
\long\def\epsf@aux#1#2:#3\\{\ifx#1\epsf@percent
   \def\epsf@testit{#2}\ifx\epsf@testit\epsf@bblit
	\@atendfalse
        \epsf@atend #3 . \\%
	\if@atend	
	   \if@verbose{
		\typeout{psfig: found `(atend)'; continuing search}
	   }\fi
        \else
        \epsf@grab #3 . . . \\%
        \not@eoffalse
        \global\no@bbfalse
        \fi
   \fi\fi}%
%
%   Here we grab the values and stuff them in the appropriate definitions.
%
\def\epsf@grab #1 #2 #3 #4 #5\\{%
   \global\def\epsf@llx{#1}\ifx\epsf@llx\empty
      \epsf@grab #2 #3 #4 #5 .\\\else
   \global\def\epsf@lly{#2}%
   \global\def\epsf@urx{#3}\global\def\epsf@ury{#4}\fi}%
%
% Determine if the stuff following the %%BoundingBox is `(atend)'
% J. Daniel Smith.  Copied from \epsf@grab above.
%
\def\epsf@atendlit{(atend)} 
\def\epsf@atend #1 #2 #3\\{%
   \def\epsf@tmp{#1}\ifx\epsf@tmp\empty
      \epsf@atend #2 #3 .\\\else
   \ifx\epsf@tmp\epsf@atendlit\@atendtrue\fi\fi}


% End of file reading stuff from epsf.tex
%%%%%%%%%%%%%%%%%%%%%%%%%%%%%%%%%%%%%%%%%%%%%%%%%%%%%%%%%%%%%%%%%%%

\def\psdraft{
	\def\@psdraft{0}
	%\typeout{draft level now is \@psdraft \space . }
}
\def\psfull{
	\def\@psdraft{100}
	%\typeout{draft level now is \@psdraft \space . }
}
\psfull
\newif\if@prologfile
\newif\if@postlogfile
\def\pssilent{
	\@noisyfalse
}
\def\psnoisy{
	\@noisytrue
}
\psnoisy
%%% These are for the option list.
%%% A specification of the form a = b maps to calling \@p@@sa{b}
\newif\if@bbllx
\newif\if@bblly
\newif\if@bburx
\newif\if@bbury
\newif\if@height
\newif\if@width
\newif\if@rheight
\newif\if@rwidth
\newif\if@clip
\newif\if@verbose
\def\@p@@sclip#1{\@cliptrue}

%%% GDH 7/26/87 -- changed so that it first looks in the local directory,
%%% then in a specified global directory for the ps file.

\def\@p@@sfile#1{\def\@p@sfile{null}%
	        \openin1=#1
		\ifeof1\closein1%
		       \openin1=\figurepath#1
			\ifeof1\typeout{Error, File #1 not found}
			\else\closein1
			    \edef\@p@sfile{\figurepath#1}%
                        \fi%
		 \else\closein1%
		       \def\@p@sfile{#1}%
		 \fi}
\def\@p@@sfigure#1{\def\@p@sfile{null}%
	        \openin1=#1
		\ifeof1\closein1%
		       \openin1=\figurepath#1
			\ifeof1\typeout{Error, File #1 not found}
			\else\closein1
			    \def\@p@sfile{\figurepath#1}%
                        \fi%
		 \else\closein1%
		       \def\@p@sfile{#1}%
		 \fi}

\def\@p@@sbbllx#1{
		%\typeout{bbllx is #1}
		\@bbllxtrue
		\dimen100=#1
		\edef\@p@sbbllx{\number\dimen100}
}
\def\@p@@sbblly#1{
		%\typeout{bblly is #1}
		\@bbllytrue
		\dimen100=#1
		\edef\@p@sbblly{\number\dimen100}
}
\def\@p@@sbburx#1{
		%\typeout{bburx is #1}
		\@bburxtrue
		\dimen100=#1
		\edef\@p@sbburx{\number\dimen100}
}
\def\@p@@sbbury#1{
		%\typeout{bbury is #1}
		\@bburytrue
		\dimen100=#1
		\edef\@p@sbbury{\number\dimen100}
}
\def\@p@@sheight#1{
		\@heighttrue
		\dimen100=#1
   		\edef\@p@sheight{\number\dimen100}
		%\typeout{Height is \@p@sheight}
}
\def\@p@@swidth#1{
		%\typeout{Width is #1}
		\@widthtrue
		\dimen100=#1
		\edef\@p@swidth{\number\dimen100}
}
\def\@p@@srheight#1{
		%\typeout{Reserved height is #1}
		\@rheighttrue
		\dimen100=#1
		\edef\@p@srheight{\number\dimen100}
}
\def\@p@@srwidth#1{
		%\typeout{Reserved width is #1}
		\@rwidthtrue
		\dimen100=#1
		\edef\@p@srwidth{\number\dimen100}
}
\def\@p@@ssilent#1{ 
		\@verbosefalse
}
\def\@p@@sprolog#1{\@prologfiletrue\def\@prologfileval{#1}}
\def\@p@@spostlog#1{\@postlogfiletrue\def\@postlogfileval{#1}}
\def\@cs@name#1{\csname #1\endcsname}
\def\@setparms#1=#2,{\@cs@name{@p@@s#1}{#2}}
%
% initialize the defaults (size the size of the figure)
%
\def\ps@init@parms{
		\@bbllxfalse \@bbllyfalse
		\@bburxfalse \@bburyfalse
		\@heightfalse \@widthfalse
		\@rheightfalse \@rwidthfalse
		\def\@p@sbbllx{}\def\@p@sbblly{}
		\def\@p@sbburx{}\def\@p@sbbury{}
		\def\@p@sheight{}\def\@p@swidth{}
		\def\@p@srheight{}\def\@p@srwidth{}
		\def\@p@sfile{}
		\def\@p@scost{10}
		\def\@sc{}
		\@prologfilefalse
		\@postlogfilefalse
		\@clipfalse
		\if@noisy
			\@verbosetrue
		\else
			\@verbosefalse
		\fi
}
%
% Go through the options setting things up.
%
\def\parse@ps@parms#1{
	 	\@psdo\@psfiga:=#1\do
		   {\expandafter\@setparms\@psfiga,}}
%
% Compute bb height and width
%
\newif\ifno@bb
\def\bb@missing{
	\if@verbose{
		\typeout{psfig: searching \@p@sfile \space  for bounding box}
	}\fi
	\no@bbtrue
	\epsf@getbb{\@p@sfile}
        \ifno@bb \else \bb@cull\epsf@llx\epsf@lly\epsf@urx\epsf@ury\fi
}	
\def\bb@cull#1#2#3#4{
	\dimen100=#1 bp\edef\@p@sbbllx{\number\dimen100}
	\dimen100=#2 bp\edef\@p@sbblly{\number\dimen100}
	\dimen100=#3 bp\edef\@p@sbburx{\number\dimen100}
	\dimen100=#4 bp\edef\@p@sbbury{\number\dimen100}
	\no@bbfalse
}
\def\compute@bb{
		\no@bbfalse
		\if@bbllx \else \no@bbtrue \fi
		\if@bblly \else \no@bbtrue \fi
		\if@bburx \else \no@bbtrue \fi
		\if@bbury \else \no@bbtrue \fi
		\ifno@bb \bb@missing \fi
		\ifno@bb \typeout{FATAL ERROR: no bb supplied or found}
			\no-bb-error
		\fi
		%
		\count203=\@p@sbburx
		\count204=\@p@sbbury
		\advance\count203 by -\@p@sbbllx
		\advance\count204 by -\@p@sbblly
		\edef\@bbw{\number\count203}
		\edef\@bbh{\number\count204}
		%\typeout{ bbh = \@bbh, bbw = \@bbw }
}
%
% \in@hundreds performs #1 * (#2 / #3) correct to the hundreds,
%	then leaves the result in @result
%
\def\in@hundreds#1#2#3{\count240=#2 \count241=#3
		     \count100=\count240	% 100 is first digit #2/#3
		     \divide\count100 by \count241
		     \count101=\count100
		     \multiply\count101 by \count241
		     \advance\count240 by -\count101
		     \multiply\count240 by 10
		     \count101=\count240	%101 is second digit of #2/#3
		     \divide\count101 by \count241
		     \count102=\count101
		     \multiply\count102 by \count241
		     \advance\count240 by -\count102
		     \multiply\count240 by 10
		     \count102=\count240	% 102 is the third digit
		     \divide\count102 by \count241
		     \count200=#1\count205=0
		     \count201=\count200
			\multiply\count201 by \count100
		 	\advance\count205 by \count201
		     \count201=\count200
			\divide\count201 by 10
			\multiply\count201 by \count101
			\advance\count205 by \count201
			%
		     \count201=\count200
			\divide\count201 by 100
			\multiply\count201 by \count102
			\advance\count205 by \count201
			%
		     \edef\@result{\number\count205}
}
\def\compute@wfromh{
		% computing : width = height * (bbw / bbh)
		\in@hundreds{\@p@sheight}{\@bbw}{\@bbh}
		%\typeout{ \@p@sheight * \@bbw / \@bbh, = \@result }
		\edef\@p@swidth{\@result}
		%\typeout{w from h: width is \@p@swidth}
}
\def\compute@hfromw{
		% computing : height = width * (bbh / bbw)
		\in@hundreds{\@p@swidth}{\@bbh}{\@bbw}
		%\typeout{ \@p@swidth * \@bbh / \@bbw = \@result }
		\edef\@p@sheight{\@result}
		%\typeout{h from w : height is \@p@sheight}
}
\def\compute@handw{
		\if@height 
			\if@width
			\else
				\compute@wfromh
			\fi
		\else 
			\if@width
				\compute@hfromw
			\else
				\edef\@p@sheight{\@bbh}
				\edef\@p@swidth{\@bbw}
			\fi
		\fi
}
\def\compute@resv{
		\if@rheight \else \edef\@p@srheight{\@p@sheight} \fi
		\if@rwidth \else \edef\@p@srwidth{\@p@swidth} \fi
}
%		
% Compute any missing values
\def\compute@sizes{
	\compute@bb
	\compute@handw
	\compute@resv
}
%
% \psfig
% usage : \psfig{file=, height=, width=, bbllx=, bblly=, bburx=, bbury=,
%			rheight=, rwidth=, clip=}
%
% "clip=" is a switch and takes no value, but the `=' must be present.
\def\psfig#1{\vbox {
	% do a zero width hard space so that a single
	% \psfig in a centering enviornment will behave nicely
	%{\setbox0=\hbox{\ }\ \hskip-\wd0}
	%
	\ps@init@parms
	\parse@ps@parms{#1}
	\compute@sizes
	%
	\ifnum\@p@scost<\@psdraft{
		\if@verbose{
			\typeout{psfig: including \@p@sfile \space }
		}\fi
		%
		\special{ps::[begin] 	\@p@swidth \space \@p@sheight \space
				\@p@sbbllx \space \@p@sbblly \space
				\@p@sbburx \space \@p@sbbury \space
				startTexFig \space }
		\if@clip{
			\if@verbose{
				\typeout{(clip)}
			}\fi
			\special{ps:: doclip \space }
		}\fi
		\if@prologfile
		    \special{ps: plotfile \@prologfileval \space } \fi
		\special{ps: plotfile \@p@sfile \space }
		\if@postlogfile
		    \special{ps: plotfile \@postlogfileval \space } \fi
		\special{ps::[end] endTexFig \space }
		% Create the vbox to reserve the space for the figure
		\vbox to \@p@srheight true sp{
			\hbox to \@p@srwidth true sp{
				\hss
			}
		\vss
		}
	}\else{
		% draft figure, just reserve the space and print the
		% path name.
		\vbox to \@p@srheight true sp{
		\vss
			\hbox to \@p@srwidth true sp{
				\hss
				\if@verbose{
					\@p@sfile
				}\fi
				\hss
			}
		\vss
		}
	}\fi
}}
\def\psglobal{\typeout{psfig: PSGLOBAL is OBSOLETE; use psprint -m instead}}
\psfigRestoreAt







\usepackage{epsf}
\usepackage{epsfig}
\usepackage{graphics,color}

\newcommand{\om}{$\Omega$mega}


\begin{document}



\title{Playing with Type Systems:\\
\large{Automated assistance in the design of programming languages}\\
\small{Revision 2}}

\authorinfo{Tim Sheard}
           {Portland State University}
           {sheard@cs.pdx.edu}
\maketitle

\begin{abstract} 

We introduce the programming language \om, and illustrate how it can be used
as a general purpose meta-language. \om\ has a sophisticated type system built
on top of Generalized Algebraic Datatypes and an extensible kind system. It
allows users to encode object-level types in the meta-level data structures
that represent object-level terms. We show \om\ can be used to explore
language designs interactively, by constructing both static and dynamic
semantics as \om\ programs. These programs can be seen as partially checked
proofs of soundness of the system under exploration. As a large example we
present a new type system for MetaML, that is simpler than previous type systems.


\end{abstract}


\section{Introduction.}\label{intro}
It has become common practice when designing a new language to create
both a static semantics (type system) and a dynamic semantics and to prove
the soundness of the type system with respect to the dynamic semantics.
This process is often exploratory. The designer has an idea,  the
approach is analyzed, and hopefully the consequences of the approach
are quickly discovered. Automated aid in this process would be a great boon.

The ultimate goal of this exploratory process is a type system, a semantics,
and a proof. The proof witnesses the fact that {\em well-typed programs do not
go wrong}~\cite{Milner78} for the language under consideration. 
Approaches include a subject reduction proof in the style of
Wright and Felleisen\cite{Wright:94} on a small step semantics, or
by using denotational approaches. In either case, proofs require an amazing amount of
detail and are most often carried out by hand, and are thus subject to all the
foils of human endeavors. It has long been a personal goal of the author to
develop a generic meta-language that could be used for exploring the static
and dynamic semantics for new object-languages\cite{SheardPasalic2002,Sheard01} 
that could aid in the generation of such proofs. Many others have
had similar desires and the development of systems like
Twelf\cite{CADE99*202} and Coq\cite{COQ74} attest to the broad appeal of 
these ideas. But, while the author owes much to these systems for inspiration,
he has always desired a system whose use was closer in style
to the use of a programming language than existing systems. Hence the development of the language \om.
This paper reports on our experience using the
language \om\cite{Sheard:2004:LF,omegamanual,SheardLinger,SheardLogFrWks04} as
just such a meta-language. In this paper we show that:

\begin{itemize}

\item Much of the work of exploring the nuances of
a type system for a new language can be assisted by using mechanized tools 
-- a generic meta-language.
We call this exploration playing with type systems.

\item Such tools need not be much more complicated than your favorite
functional language (Haskell), and are thus within the reach most language researchers.
After all playing should be child's work.

\item The automation helps language designers visualize the consequences
of their design choices quickly, and thus helps speed the design process.

\item The artifacts created by this exploration, while
not quite proofs in the full sense, are checked by machine, 
and are hence less subject to error than pencil and paper proofs constructed
by hand.

\end{itemize}

In addition to the broad goals about meta-language design, the author 
has also devoted much energy to searching for a simple elegant type system
for a multi-stage language such as MetaML. Such a
system should not throw away too many good
programs\cite{Moggi-Taha-Benaissa-Sheard-ESOP99,TS00,Taha99} (thus losing
much of its usefulness) or be overly
complex\cite{Calgano-Moggi-Sheard-JFP03,T00,TahNie03}. 
After many attempts, the search has been fruitful. Thus, this paper
has two purposes. First, it demonstrates the efficacy of the language
\om{} as a first step towards the design of a general purpose meta-language.
Second, it illustrates that a simple elegant type system for a multi-stage
language is possible. The approach is so general, that we believe it can
be applied to any multistage language, and as such is of wide spread
interest to the program generation community.

The paper is divided into three parts. First, in Section \ref{omega} we
introduce the language \om. Second, in Section \ref{metaml} we discuss the
multi-staged language MetaML, and explain why it is so hard to develop a sound
type system for a staged language. In this section we provide several small
programs that illustrate the multiple causes of soundness errors in a staged
language. Third, in Section \ref{lang} we play with type systems. In this
section we introduce our static and dynamic semantics for a multi-stage language as
an \om\ program. We demonstrate that the ability to explore issues is greatly
enhanced by the machine assistance supplied by \om, and that the artifacts
produced consist of machine checked proofs.

\section{The \om~ meta-programming language.} \label{omega}


Haskell makes a fair meta-language. Its support for abstraction and first
class functions make it an excellent tool for defining object-language
programs as Haskell data, and for defining meta-language manipulations as
Haskell functions. Unfortunately, its type system is too weak to be a true
generic meta-programming system. In particular Haskell lacks the facilities to
define and enforce type-systems for object-language programs represented as
meta-language data. To overcome this deficiency we have defined the language
\om.

\begin{itemize}

\item \om~ is based on Haskell so
that it is easy for Haskell programmers to learn.

\item We started by adding some features and removing others. To make
our first attempt simple we have removed the class system of Haskell,
and made \om~ strict but pure. We have tried hard to keep all the other
features of Haskell not affected by these choices intact.

\item We added small (backward compatible) features that support 
programs that use a form of refinement or dependent types. We tried hard
not to disturb the functional programming style -- in particular the phase
distinction between values and types. The features we added are Generalized
Algebraic Datatypes (GADTs) and an extensible kind system.

\item We built a non-trivial implementation and have programmed up a
wide array of examples from the literature.

\end{itemize}

\subsection{Generalized Algebraic Datatypes}

The key to this style of programming is the use of Generalized
Algebraic Datatypes (GADTs), a generalization of the normal
Algebraic Datatypes (ADT) available in functional programming
languages such as Haskell, ML, and O'Caml. Implementing GADTs
in a functional language requires only a small, backward
compatible, change to the ADT notion and is easily understood
by functional programmers. A language with GADTs can
support several closely related concepts such Refinement
Types\cite{113468,Xi:1999:DTP,Davies97}, Guarded Recursive Datatype
Constructors\cite{XiCheChe03}, Inductive
Families\cite{Coquand:1994:IDT,Dybjer:1999:FAI}, First-class phantom
types\cite{Hinze:03:Phantom}, Silly
Type Families\cite{sillyTF}, and Equality Qualified
Types\cite{Sheard:2004:LF,SheardLogFrWks04}. 
% Wobbly types\cite{wobbly}, 
There are many
examples of the usefulness of such concepts in the
recent literature\cite{Baars:2002:TDT,XiChen2003,HinzeHaskellWorkshop02,PasalicLingerGpce,pottier2,pottier1,Xi:1998:EAB}.



ADTs generalize other forms of structuring data such as enumerations,
records, and tagged variants. For example,
in \om~ (and in Haskell) we write:

\begin{verbatim}
 -- An enumeration type
data Major = CompScience | English | Chemistry    

type Number = Int
type Title = String
 -- A record structure
data Class = MakeClass Number Title  

-- A tagged Variant
data Person = Teacher [Class]  | Student Major    
\end{verbatim}

We assume the reader has a certain familiarity with ADTs. In
particular that values of ADTs are constructed by {\em constructors}
and that they are taken apart by the use of {\em pattern matching}.
Two valuable features of ADTS are their ability to be {\em
parameterized by a type} and to be {\em recursive}. A simple example
that employs both these features is:
\begin{verbatim}
data Tree a 
   = Fork (Tree a) (Tree a) 
   | Node a 
   | Tip
\end{verbatim}
This declaration defines the polymorphic {\tt Tree} type constructor. Example tree
types include {\tt (Tree Int)} and {\tt (Tree Bool)}. In fact the
type constructor {\tt Tree} can be applied to any type whatsoever.
Note how the constructor functions ({\tt Fork},\ {\tt Node}) and constants ({\tt Tip})
are given polymorphic types.

\vspace*{0.1in}
\noindent 
{\tt Fork :: forall a . Tree a -> Tree a -> \underline{Tree a}}\\
{\tt Node :: forall a . a -> \underline{Tree a}}\\
{\tt Tip :: forall a . \underline{Tree a}}
 \vspace*{0.1in}
 
When we define a parameterized algebraic datatype, the syntactic formation
rules of the {\tt data} declaration enforce the following restriction: The
range of every constructor function, and the type of every constructor
constant must be a polymorphic instance of the new type constructor being
defined. Notice how the constructors for {\tt Tree} all have range ({\tt Tree
{\it a}}) with a polymorphic type variable {\it a}.  We can remove this
restriction {\it semantically} by strengthening the type system of the
language. The restriction is {\it syntactically} removed by the following
mechanism. In a {\tt data} declaration, 
Rather than supplying the type arguments to the type being defined,
the user supplys an explicit kind declaration; and
rather than leaving the range of the
constructor functions implicit, the user replaces the enumeration of the
constructors and the type of their domains with a full explicit typing of each
constructor. The only restriction being that the range must be some instance
of the type being defined. For example we could define the {\tt Tree}
type as follows using the new syntax.
\begin{verbatim}
data Tree::  *0 ~> *0 where
  Fork:: Tree a -> Tree a -> Tree a
  Node:: a -> Tree a
  Tip:: Tree a
\end{verbatim}
\noindent
It is not necessary to use the new syntax, since {\tt Tree} meets
the restriction. As we will see below, there exist many useful types
that do not. Removing the restriction requires new type checking rules that are
beyond the scope of this paper, but which have been well
studied\cite{wobbly,XiCheChe03,Hinze:03:Phantom}. 


\subsection{Representing Object-Programs with Types as Data}
This simple extension allows
us to build datatypes representing object-programs, whose
meta-level types encode the object-level types of the programs
represented. A very simple object-language example with types is:

\begin{verbatim}
data Term :: *0 ~> *0 where
  Const :: a -> Term a
  Pair :: Term a -> Term b -> Term (a,b)
  App :: Term(a -> b) -> Term a -> Term b
\end{verbatim}

Above we introduced the new type constructor {\tt Term}, which is a
representation of a simple object-language of constants and pairs. {\tt Term}
is a type constructor. It must be applied to another type to create a valid
new type of kind {\tt *0} (star-zero). We say {\tt Term} is classified by the kind
\verb+*0 ~> *0+. The kind \verb+*0+ classifies all types that classify
computable values. For more on kinds, see Section \ref{kinds}. No restriction
is placed on the types of the constructors except that the range of each
constructor must be a fully applied instance of the type being defined, and
that the type of the constructor as a whole must be classified by \verb+*0+.
Note how the range of {\tt Pair} is a non-trivial instance of {\tt Term}.

As promised, {\tt Term}s are a typed object-language representation, i.e. a data
structure that represents terms in some object-language. The meta-level
type of the representation ({\tt Term a}), indicates the type 
of the object-level term ({\tt a}). This is made possible only by the flexibility of
the GADT mechanism. Using typed object-level terms,
it is impossible to construct ill-typed term representations, because
the meta-language type system enforces this constraint.
\begin{verbatim}
ex1 :: Term Int  
ex1 = App (App (Const (+)) (Const 3)) (Const 5)

ex2 :: Term (Int,String)
ex2 = Pair ex1 (Const "z")
\end{verbatim}
Attempting to construct an ill-typed object term, like {\tt (App (Const 3) (Const 5))},
causes a meta-level (\om) type error. 

Another advantage of using GADTs rather than ADTs is that it is now possible to
construct a tagless\cite{SheardPasalic2002,Taha:2001:TEJ,TahaTag2000}
interpreter directly. An interpreter is a function that takes a term to the
value which is the meaning of that term. In a language without GADTs terms are
untyped, and values are some universal value domain like: {\tt data V = IntV
Int | PairV V V | FunV (V -> V)}, and the interpeter is said to be tagged,
since each value must be tagged with one of the constructors of the universal
domain (such as {\tt IntV}, {\tt PairV}, or {\tt FunV}). See
\cite{PasalicLingerGpce} for a detailed discussion of this phenomena.
But using GADTs none of this is necessary. Consider:

\begin{verbatim}
eval :: Term a -> a
eval (Const x) = x
eval (App f x) = eval f (eval x)
eval (Pair x y) = (eval x,eval y)
\end{verbatim}

\noindent
Here the universal domain is not necessary, and the tagless interpreter has the structure of a
denotational semantics. Because the {\tt eval} function is total and
well-typed at the meta-level, it also implies that the object-level
semantics (defined by {\tt eval}) is also well-typed. As long as {\tt
eval} is total, {\em every} well-typed object level term evaluates to
a well-formed value.

While we worked hard to make this look like Haskell programming, there
are some key differences. First, the prototype declaration ({\tt eval :: Term
a -> a}) is required, not optional. Functions which pattern match over
GADTs can be type checked, but type inference is much harder (see
\cite{pottier3} for work on how this might be done). Functions that
don't pattern match over GADTs can have Hindley-Milner types inferred
for them (see \cite{wobbly} for how this mixture of type-checking and
type-inference is done). Requiring prototypes for only some functions
should be familiar to Haskell programmers because polymorphic-recursive
functions already require prototypes\cite{oai:CiteSeerPSU:312481}. 

\subsection{Values, Types, Kinds, and Sorts}\label{kinds}

In Haskell, values are classified by types. In a similar fashion, types can be
classified by kinds, and kinds can be classified by sorts. We indicate
clasification as a relation using the infix symbol({\tt ::}). We say ({\tt ::})
is overloaded because the same notation can be used in all three contexts: {\em
value}{\tt ::}{\em type}, and {\em type}{\tt ::}{\em kind}, and {\em kind}{\tt
::}{\em sort}.

Some concrete examples at the value level include:
{\tt 5::Int}, and {\tt [True]::[Bool]}.  We say {\tt 5} is classified by {\tt Int}.
At the type level:
{\tt Int::*0 }, \verb+Term:: *0 ~> *0+. We say {\tt Int}
is classified by star-zero, and {\tt Tree} is classified
by star-zero to star-zero. {\tt *0}, and \verb+*0 ~> *0+ are kinds.
At the kind level both: \verb+*0:: *1+ and \verb+(*O ~> *0):: *1+. Here, \verb+*1+
is a sort.

The kind {\tt *0} is interesting because it classifies all
types that classify values (things we actually can compute). For example,
{\tt Int:: *0}, and {\tt [Int]:: *0}, but not {\tt Tree:: *0}, because
there are no values of type {\tt Tree} (but there are values of type {\tt Tree Int}).

{\em New kinds} are introduced by the {\em kind} declaration that also
introduces the type constructors that produce types classified by that kind,
just as {\em new types} are introduced by the {\em data} declaration along with
the value constructors that produce values classified by that type. The 
{\tt data} and {\tt kind} declarations introduce similar structures, but at
different levels in the type hierarchy. Two interesting kind declarations,
of interest in this paper follow:

\begin{verbatim}
kind Nat = Z | S Nat
\end{verbatim}

The {\tt Nat} declaration introduces the kind {\tt Nat} and two new
type constructors {\tt Z} and {\tt S} which encode the natural
numbers at the type level.
The type {\tt Z} has kind {\tt Nat}, and {\tt S} has kind {\tt Nat \verb+~+>
Nat}. The type {\tt S} is a type constructor, so it has a higher-order kind. We
indicate this using the classifies relation as follows:
\begin{verbatim}
Z :: Nat
S :: Nat ~> Nat
Nat :: *1
\end{verbatim}

The classification {\tt Nat::*1} indicates that {\tt Nat} is a kind classified
by the sort {\tt *1}. Both {\tt Nat} and {\tt *0} are kinds at the same
``level"  --- they are both classified by {\tt *1}.

\begin{verbatim}
kind Row (x :: *1) = RNil | RCons x (Row x)
\end{verbatim}

\noindent
{\tt Row}s are list like structures at the type level,
which are constructed by the type constructors {\tt RNil} and {\tt RCons}.
For example {\tt (RCons Int (RCons Bool RNil))} and {\tt (RCons Z RNil)}.
Such types are kinded by {\tt Row}.
The kind {\tt Row} is a higher order kind, and is classified by
\verb+(*1 ~> *1)+.  Thus {\tt Row} must be applied to a kind to be well formed
(the argument indicates the kind of the types stored in the row).
For example if we classify the examples above we get:
\begin{verbatim}
(RCons Int (RCons Bool RNil)):: Row *0
(RCons Z RNil):: Row Nat
\end{verbatim}
\noindent
Both \verb+*1+ and \verb+(*1 ~> *1)+ are
kinds classified by the sort \verb+*2+.  

\begin{figure}[t]
\center{
{\scriptsize
\begin{tabular}{|ccccccc|} \hline
{\tiny value name space} & $\mid$ & {\tiny type name space} & & & & \\ \hline
value & $\mid$ & type & $\mid$ & kind &$\mid$ & sort \\ \hline
5     &::& Int  &::& *0  &::& *1  \\ \hline
[2]   &::& [Int]&::& *0  &::& *1  \\ \hline
      &  & []   &::& *0 $\leadsto$ *0 &::& *1\\ \hline
      &  &      &  & Nat &::& *1\\ \hline  
      &  &  Z   &::& Nat &::& *1\\ \hline   
      &  &  S   &::& Nat $\leadsto$ Nat &::& *1\\ \hline 
      &  & Nat'   &::& Nat $\leadsto$ *0 &::& *1 \\ \hline  
  Z   &::& Nat' Z &::& * 0               &::& *1 \\ \hline   
  S   &::& Nat' m $\rightarrow$ Nat' (S m)  &::& *0 &::& *1\\ \hline        
      &  & Term  &::&  *0 $\leadsto$ * 0 &::& *1\\ \hline
Const &::& a $\rightarrow$ Term a &::& *0 &::& *1\\ \hline 
%     &  & Has  &::& Tag $\leadsto$ *1 $\leadsto$ HasType &::& *1\\ \hline
      &  &      &  & Row  &::& *1 $\leadsto$ *1\\ \hline
%     &  & RCons&::& k $\leadsto$ Row k $\leadsto$ Row k &::& *1\\ \hline
      &  & RNil &::& Row k &::& *1\\ \hline      
\end{tabular}}}

\caption{Values, types, kinds, and sorts, are the first few levels of 
         the infinite classification hierarchy. As is done in Haskell, there are two name spaces. One name space, names values,
and the other name space, names types, kinds, sorts etc. Two objects can have
the same name if they live in different name spaces. This form of
overloading is often exploited in \om~(as well as in Haskell) programs.
Note that {\tt Z} and {\tt S} are defined in both the value name space and 
the type name space.}\label{table}
\hrule
\end{figure}

We illustrate the relationship between
values, types, kinds, and sorts by example in a table found in Figure
\ref{table}.
Note that the table has empty slots. Not all types classify values.
For example, there are no values of type \verb+[]+ (list) or \verb+Z+, but
there are values of type \verb+[Int]+. The same holds at the kind level.
Not all kinds classify types. For example, there are no types classified
by \verb+Row+, but {\tt RNil} could be classified by \verb+(Row Int)+. 
Note the different kinds of arrows ($\rightarrow$ and $\leadsto$). The first
is used to classify functions at the value level. The second is used to classify
type constructors. In \om~ programs we write \verb+->+ for $\rightarrow$, and
\verb+~>+ for $\leadsto$.

\subsection{Singleton Types}
It is sometimes useful to build representations of types at the value level.
Such representations are called {\em singleton types} if they are encoded
by a type constructor whose argument indicates the type being represented.
For example consider:
\begin{verbatim}
data Nat':: Nat ~> *0 where
   Z:: Nat' Z 
   S::  (Nat' x) -> Nat' (S x)
\end{verbatim}
\noindent
Values classified by the type {\tt (Nat' a)} are reflections of the {\em types}
classified by the {\em kind} {\tt Nat}. The value constructors of the {\tt data}
declaration for {\tt Nat'} mirror the type constructors in the {\tt kind}
declaration of {\tt Nat}, and the type index of {\tt Nat'} is equal to the
type reflected. For example, the value \verb+S(S(S Z))+ is classified by the
type \verb+Nat'(S(S(S Z)))+. We say that {\tt Nat'} is a {\em singleton type}
because there is only one element of any singleton type. For example, only {\tt
S (S Z)} inhabits the type {\tt Nat' (S (S Z))}. As discussed in Figure
\ref{table}, we exploit the separate name spaces for value and types by using
the same names for the type constructors of kind {\tt Nat} ({\tt S} and {\tt
Z}) and the constructor functions of {\tt data} type {\tt Nat'} ({\tt S} and
{\tt Z}) to emphasize the close relationship between {\tt Nat} and {\tt Nat'}.



We will find natural numbers at the type level (and their {\tt Nat'}
reflections at the value level) to be so useful we introduce some syntactic
sugar for constructing such types (or values). For example,
{\tt \#0 = Z}, and {\tt \#1 = (S Z)}, and {\tt \#2 = (S (S Z))} etc.
We also support the syntax {\tt \#(n + x) = (S$_1$(S$_2$ ... (S$_n$ x)))}.
This syntactic sugar is analogous to
the use of square brackets to describe lists in addition to the use
of the constructor \verb+(:)+.

\subsection{Tags and Labels}

Many object languages have a notion of name. To make representing names in the
type system easy we introduce the notion of Tags and Labels. As a {\em first
approximation}, consider the finite kind {\tt Tag} and its singleton type {\tt
Label}:

\begin{verbatim}
kind Tag = A | B | C

data Label:: Tag ~> *0 where
  A:: Label A
  B:: Label B
  C:: Label C
\end{verbatim}

Here, we again deliberately use the value/type name space
overloading first discussed in Figure \ref{table}. The names {\tt A}, {\tt B}, and {\tt C} 
are defined in both the value and type name spaces. They 
name different,
but related objects in each space.
At the value level every {\tt Label} has a type index
that reflects its value. I.e. {\tt A::Label A}, and {\tt B::Label B}, and {\tt
C::Label C}. Now consider a countably infinite set of tags and labels. We can't
define this explicitly, but we can build such a type as a primitive inside of
\om. At the type level, every legal identifier whose name is preceded by a
back-tick ({\tt `}) is a type classified by the kind {\tt Tag}. For example, the type {\tt
`abc} is classified by {\tt Tag}. At the value level, every such symbol {\tt `x} is classified
by the type {\tt (Label `x)}.

\subsection{Rows, Records, and HasType.}

The kind {\tt Row} classifies list-like data structures at the type level.
The kind {\tt HasType} classifies pairs at the type level. 
\begin{verbatim}
kind HasType = Has Tag *0
\end{verbatim}
It aggregates
a {\tt Tag} and any type classified by {\tt *0}. For example, \verb+(Has `a Int)::HasType+.
We can construct lists (at the type level) of such pairs using the type constructors of {\tt Row}.
For example,
\begin{verbatim}
(RCons (Has `a Int) (RCons (Has `b Bool) RNil))
  ::Row HasType
\end{verbatim}
Note, \verb+(RCons (Has `a Int) (RCons (Has `b Bool) RNil))+ is a type,
and that \verb+(Row HasType)+ is a kind.
Such a type can be thought of as classifying records at the value
level. We can define such records within \om~ as follows:

\begin{verbatim}
data Rec:: Row HasType ~> *0 where
  RecNil :: Rec RNil
  RecCons:: Label s -> t -> 
            Rec r -> 
            Rec (RCons (Has s t) r)
\end{verbatim}

We can construct records by using the constructor functions
{\tt RecNil} and {\tt RecCons}. Such values have types
{\tt (Rec {\it r})} where {\tt r} is classified by {\tt (Row HasType)}. For example,
consider

\begin{verbatim}
r1 :: Rec(RCons (Has `x Int) 
                (RCons (Has `a Bool) RNil))
r1 = RecCons `x 5 (RecCons `a True RecNil)

r2:: Rec u -> Rec(RCons (Has `x Int) 
                        (RCons (Has `a Bool) u))
r2 x = RecCons `x 5 (RecCons `a True x)
\end{verbatim}

It is interesting to note that we have managed to express
a simple form of Wand's (or Remy's) row-polymorphism\cite{IC::Wand1991,94:2} in \om\ just by
using kinds.  We have found {\tt Row} and {\tt HasType} so useful we have built
special syntactic sugar for printing them. For example,~ ~
\verb+Rec(RCons (Has `x Int) (RCons (Has `a Bool) RNil))+ ~ prints
as~  \verb+Rec {`x:Int,`a:Bool}+.
The syntactic sugar for {\tt Row} and {\tt HasType} 
replaces {\tt RCons} and {\tt RNil} with squiggly brackets, and
replaces {\tt Has} with colon. A type classified by {\tt Row} whose
(ultimate) tail is not {\tt RNil} (i.e. a type variable) prints with a trailing semi-colon.
For example,\\ \verb+Rec(RCons (Has `x Int) (RCons (Has `a Bool) w))+ ~ prints
as~  \verb+Rec {`x:Int,`a:Bool; w}+.

\subsection{An Object Language with Binding.}\label{binding}

Rows and records allow us to define object-languages with binding structures
that track their free variables in their meta-level types. The object-language
{\tt (Lam env t)} represents the simply typed lambda calculus.

\begin{verbatim}
data Lam:: Row HasType ~> *0 ~> *0  where
  Var      :: Label s -> Lam (RCons (Has s t) env) t
  Shift    :: Lam env t -> 
              Lam (RCons (Has s q) env) t
  Abstract :: Label a -> 
              Lam (RCons (Has a s) env) t -> 
              Lam env (s -> t)
  Apply    :: Lam env (s -> t) -> 
              Lam env s -> Lam env t
\end{verbatim}

The first index to {\tt Lam}, {\tt env} is a Row tracking its variables,
and the second index, {\tt t} tracks the object-level type of the term.
For example, a term with variables {\tt x} and {\tt y} might have type
\verb+Lam {`x:Int, `y:Bool; u} Int+. This is made possible by the use
of {\tt Row} and {\tt HasType} in the GADT representing lambda terms.

The key to this approach is the typing of the object-language constructor
functions for variables and lambda expressions. Consider the {\tt Var}
constructor function.
To construct a variable we simply apply {\tt Var} to a label, and its type
reflects this. For example, here is the output from a short interactive
session with the \om~ interpreter.
\begin{verbatim}
prompt> Var `name
(Var `name):: 
forall a (u:Row HasType) . Lam {`name:a; u} a
   
prompt> Var `age
(Var `age):: 
forall a (u:Row HasType) . Lam {`age:a; u} a
\end{verbatim}

Variables behave like Bruijn indices. Variables created with
{\tt Var} are like the natural number 0. A variable
can be lifted to the next natural number by the successor operator
{\tt Shift}. To understand why this is useful
consider that the two examples have different names in the
same index position. The two variables would clash if they were both used in the same
lambda term. To shift the position of variable to a different index, we use the
\verb+Shift:: Lam u a -> Lam {v:b; u} a+ constructor. Rather
than counting with natural numbers (as is done with de Bruijn indices)
we ``count" with rows, recording both its symbolic name and its type. Here is how we could
define two variables {\tt x} and {\tt y} for use in the same environment.

\begin{verbatim}
x :: Lam {`x:a; u} a 
x = Var `x 

y :: Lam {u:a,`y:b; v} b
y = (Shift (Var `y)) 
\end{verbatim}


The type system now tracks the variables in an expressions.
\begin{verbatim}
z :: Lam {`x:a -> b,`y:a; u} b
z = (Apply (Var `x) (Shift (Var `y))) 
\end{verbatim}

Finally, and of great interest, we can build a well-typed
evaluator for the GADT {\tt Lam}.
\begin{verbatim}
evalLam :: Lam env t -> Rec env -> t
evalLam (Var _) (RecCons _ y _) = y
evalLam (Shift x) (RecCons _ _ rs) = evalLam x rs
evalLam (Abstract l _ body) rs = 
   \ x -> evalLam body (RecCons l x rs)
evalLam (Apply f x) rs = 
   (evalLam f rs) (evalLam x rs)
\end{verbatim}
We declare that the type of our evaluation function
is as follows: (\verb+Lam env t -> Rec env -> t+). We can interpret
this to mean that every well typed {\tt Lam} term with type {\tt t} under an 
environment with shape {\tt env}, can be given meaning as a function
from a record with shape {\tt env} to {\tt t}. The function {\tt evalLam}
is a denotational semantics. It provides a meaning for every well formed
lambda term.

In essence, the well-typing of the evaluation function is one of three parts
that comprise a proof of soundness of the type system with respect to the
semantics. The other two parts are proofs of totality and
compositionality.

\begin{itemize} 

\item{\bf Totality.} To ensure that every term is mapped to a well-typed value,
we must ensure that {\tt evalLam} is total. That is, it terminates for every
well-typed lambda term. Every well-typed {\tt Lam} term matches one of the
clauses of {\tt evalLam}, and every recursive call of {\tt evalLam} is called
on a smaller subterm of the original argument, so every call will terminate
with a value if the input term is finite and the meta-language (in this case
\om) is strongly normalizing. Note that the input to {\tt evalLam} is
a meta-language term, so if the meta-language is strongly normalizing
no infinite inputs are possible. The key is a strongly normalizing
meta-language.

\item {\bf Compositionality.} The meaning of every term is composed
only from the meaning of its subterms.

\end{itemize}

While neither of these two are currently enforced by \om, we have plans
to provide such support in the future.

Other kinds of proofs are also
possible. Proofs in style of Wright and
Felleisen\cite{Wright:94} work on a small step semantics defined in terms of
substitution. \om\ can also be used to define languages in this way. See our
paper {\em Meta-programming with Built-in Type Equality}\cite{SheardLogFrWks04}
which was presented at the Logical Frameworks and Meta-Languages Workshop to
see how this can be done.

\subsection{The bottom line.}

The ability to define GADTs, and the ability to define new kinds,
creates a rich playground for those wishing to explore the design of new
languages. These features, along with the use of rank-N polymorphism (which we
will illustrate by example later in the paper) make \om~ a better meta-language
than Haskell. In order to explore the design of a new language one 
can proceed as follows:

\begin{itemize}

\item First, represent the object-language as a type indexed GADT. The indexes
correspond to static properties of the program.

\item The indexes can have arbitrary structure, because they are introduced as
the type constructors of new kinds.

\item The typed constructor functions of the object-language GADT define a
static semantics for the object language.

\item Meta programs written in \om, manipulate object-language represented as
data, and check and maintain the properties captured in the type indexes
by using the meta-language type system. This lets us build and test type
systems interactively.

\item A dynamic semantics for the language can be defined by (1) writing either
a denotational semantics in the form of an interpreter or evaluation
function, or by (2) writing a small step semantics in terms of substitution
over the term language. In either case, the type system of the meta-language
guarantees that these meta-level programs maintain object level type-safety.

\end{itemize}

\section{The MetaML language.} \label{metaml}

MetaML is a homogeneous, manually annotated, run-time code generation system.
In MetaML we use angle brackets (\verb+< >+) as quotations, and tilde 
(\verb+~ +) as the anti-quotation. We call the object-level code inside a pair of
angle brackets, along with its anti-quoted holes a {\em template}, because its
stands for a computation that will build an object-code fragment with the
shape of the quoted code. In MetaML the angle brackets, the escapes, the {\tt
lift}s, and the {\tt run} operator are staging annotations. They indicate the
boundaries within a MetaML program where the program text moves from 
meta-program to object-program. The staging annotations in MetaML are placed
manually by the programmer and are considered part of the language. In MetaML
the staging annotations have semantic meaning, they are part of the language
definition, not just hints or directions to language preprocessors.
A simple example using templates follows:

{\small
\begin{verbatim}
-| val x = <3 + 2> ;
val x = <3 %+ 2> : <int>

-| val code = <show ~x> ;
val code = <%show (3 %+ 2)> : <string>
\end{verbatim}
}

In this example we construct the object-program fragment {\tt x} and use the
anti-quotation mechanism to splice it into the object-program fragment {\tt
code}. Note how the definition of {\tt code} uses a template with a hole (the
escaped expression \verb+~x+). MetaML also statically scopes free variable
occurrences in code templates. This is called {\em cross-stage persistence}.
Variables defined in earlier stages are available for use in later stages.

{\small
\begin{verbatim}
-|  fun f x y = x + y - 1;
val f = fn  : int -> int -> int

-| val z = <f 4 5>;
val z = <%f 4 5> : <int>

-| let fun f x y = not x andalso y 
   in run z end;
val it = 8 : int
\end{verbatim}}

\noindent

When printing code with a lexically scoped variable, the pretty printer places
the percent-sign ({\tt \verb+%+}) in front of {\tt f} in the code template 
{\tt z} to indicate that this is a statically bound object-variable (not a free
object-variable). 
Note how the free variable {\tt f} in the code template {\tt z} refers to
the function {\tt f:int -> int -> int}, which was in scope when the
template was defined, and not the function {\tt f:bool -> bool -> bool}
which was in scope when {\tt z} was run. Variables in MetaML are lexically
scoped even across code brackets. Because code may be spliced or run in a context
far removed from the scope where it was defined, this lexical scoping
becomes important. 

The {\tt run} operator in MetaML transforms a piece of code into the program it
represents. It is useful to think of {\tt run} as indicating the composition of
run-time compilation with execution.  In the example below, we first build a
generator ({\tt power\_gen}). Apply it to obtain a piece of code ({\tt power\_code}).
Run the code to obtain a function ({\tt power\_fun}). And then apply the function to
obtain an answer ({\tt 125}).
{\small
\begin{verbatim}
-| fun power_gen m = 
     let fun f n x = 
             if n = 0 then <1> 
                      else <~x * ~(f (n-1) x)>
     in <let fun power x = ~(f m <x>) in power end> end;
val power_gen = fn  : int -> <int -> int>

-| val power_code = power_gen 3;
val power_code = 
<let fun power x = x * x * x * 1 in power end> 
: <int -> int>

-| val power_fun = run power_code;
val power_fun = fn  : int -> int

-| power_fun 5;
val it = 125 : int
\end{verbatim}}

The introduction of staging annotations such as, run, bracket, and escape
considerably complicates the type system for any staged language. Most type
systems disqualify bad programs that have a small class of errors. This class
of errors includes, use of a variable in a scope where the variable is not
declared, and use of a value in a manner inconsistent with its type -- i.e.
using an integer where a float is expected, or applying something that is not
a function to an argument. Once staging annotations are added to a language
the class of errors becomes measurably larger.
New problems include tracking the stage at which a variable is declared
(as well as its scope), the execution of code with free variables, and
tracking alpha-renamed variables. Of course all the old problems still
exist. We illustrate these three new
problems with simple examples below.

\begin{itemize}

\item{\bf Correct use of variables.} The staged type system helps the user
construct well formed object-programs. One of its most useful features is that
it tracks the level at which variables are bound. Attempts to use variables at
a level lower than the level at which they are bound makes no sense. For example:

\begin{verbatim}
-| fun id x = x;
val id = Fn  : 'a  -> 'a 

-| <fn x => ~(id x) - 4>;
Error: The term: x  Variable bound in stage 1 
                    used too early in stage 0
\end{verbatim}
In the above example {\tt x} is a stage 1 variable, but because of the
anti-quotation it is used at stage 0. This is semantically meaningless
and should be reported as a type error.\vspace*{0.1in}


\item{\bf Running code with free variables.}
In MetaML we use {\tt run} to move from one stage to the next. Because
it is legal to use the anti-quotation under a lambda-binding, there
is the possibility that {\tt run} may be applied to variables that will not
be bound until some later stage. For example:
{\small
\begin{verbatim}
-| val bad = <fn x => ~(run <x>) + 4>;
\end{verbatim}}

\noindent
will cause an error because the object-variable {\tt x} will not be bound until the
whole piece of code is run, and then finally applied. \vspace*{0.1in}

\item{\bf Tracking Alpha-renaming.} The third problem is very subtle. It was
described in Walid Taha's thesis as {\em the puzzle}.

\begin{verbatim}
-| val puzzle = 
   <fn a => ~((fn x => <x>)(fn y => <a>)) 0>;
\end{verbatim}

Using the (small step) substitution semantics for a staged language given in
the thesis, the expression \verb+(run puzzle) 5+ should evaluate to 
\verb+<5>+, but the implementation of MetaML raised an error. The subtlety comes
because bound variables in object code in MetaML are generative. I.e. the
object variable {\tt a} in the puzzle will be alpha renamed to some new
variable name, say \verb+a_97+. The generative approach is essential to avoid
inadvertent name capture in generated code. This name gets stored in the
environment of the closure implementing \verb+(fn y => <a>)+, this closure is
then embedded inside the code value created by the template \verb+<x>+,
creating the object code \verb+<fn a_97 => %x 0>+ where the \verb+%x+ is a
cross-stage persistent value linked to the closure. This value will only be
executed when the larger enclosing code \verb+<fn a_97 => %x 0>+ is run. When
the entailing function \verb+(fn a_97 => %x 0)+ is applied (to say \verb+5+),
the variable \verb+a_97+ becomes bound in the environment to \verb+5+. Somehow
this binding needs to be propagated inside the cross-stage persistent function
represented by \verb+%x+. Doing this considerably complicates the
implementation, and inadversely affects its efficiency.

\end{itemize}

The kind of errors described above do not occur in normal (unstaged)
programs. They complicate the semantics, the type systems, and the
implementations meta-programming systems. There has been much thought put
into devising type-systems which will disallow such programs
\cite{Moggi-Taha-Benaissa-Sheard-ESOP99,Calgano-Moggi-Sheard-JFP03,T00,TahNie03}. 
The author finds all of these systems lacking in some degree or other, even
though he had a hand in devising some of them.

There has been much debate about the third example. Is it a good program?
If so then the implementation must be much more complicated than it might
otherwise be. If it is not a good program, this raises two questions.
First, how do we get a type system to disqualify it? And, second, how do
we know we won't be disqualifying important programs that real users want
to write? The second objection is an easier one to answer. Had the user
performed the explicit beta reduction inside the escape, i.e. written:

\begin{verbatim}
-| val puzzle = <fn a => ~(<fn y => <a>>) 0>;
\end{verbatim}
None of the problems associated with this example would occur. So the challenge
is to come up with a type system that rejects the original program
but not this one. In our opinion, this is a superior solution to complicating
the implementation to accommodate the original program (which we believe
no one would ever deliberately write). 



\section{MetaML as an \om~ program.} \label{lang}

In this section we play with types. We explore, in \om, several
different formulations for a MetaML type system.
We will discard several of our attempts, as our exploration 
points out their deficiencies. This section is meant
to illustrate how \om\ is useful
for exploring language design issues. Those interested in
the final MetaML type system may skip ahead to Section \ref{final}.

We will develop a type system for an abstraction of MetaML that
includes all of MetaML's important features. It has a standard lambda
calculus fragment, and a staging fragment. The staging fragment
includes brackets, escape, run, and cross-stage-persistence.

As we did in Section \ref{binding} with the {\tt Lam} GADT, we represent the 
lambda fragment with tagged de Bruijn style variables. The staging fragment
is more problematic. How do we deal with the multiple levels in
a staged expression? The standard solution\cite{T00,Taha99} is to use a level-indexed
family of expressions. In \om\ we would extend the {\tt Lam}
GADT as follows:
\begin{verbatim}
data Lam:: Nat ~> Row HasType ~> *0 ~> *0  where
  Var      :: Label s -> 
              Lam n (RCons (Has s t) env) t
  Shift    :: Lam n env t -> 
              Lam n (RCons (Has s q) env) t
  Abstract :: Label a -> 
              Lam n (RCons (Has a s) env) t -> 
              Lam n env (s -> t)
  Apply    :: Lam n env (s -> t) -> 
              Lam n env s -> 
              Lam n env t
  Bracket  :: Lam n env t -> Lam (S n) env t
  Escape   :: Lam (S n) env t -> Lam n env t
\end{verbatim} 
Note, how a natural number index is added to {\tt Lam}, and that how the
constructor for bracket lifts the index of a term, and how
the constructor for escape drops the index of a term. It is at this point
 in the design exploration that the automation inherent that the meta-language
 begins to payoff. A well-typed term built with the constructors of {\tt Lam}
 is a derivation of the type of that term. 
 By entering simple expressions at the \om\ prompt we get
 the \om\ type system to check the well-formedness of the derivation, and to display the type of the term.
 For example, by typing: \verb+Bracket (Apply (Var `f) (Shift (Var `x)))+
 we get the following result:
 \begin{verbatim}
 Lam #(1+u) {`f:a -> b,`x:a; v} b
 \end{verbatim}
Right away, an error becomes obvious in our formulation. Shouldn't an
expression that is bracketed have a code type? We need typing rules some thing
like the following for escape and bracket.
\begin{verbatim}
  Bracket  :: Lam n env t -> Lam (S n) env (Code t)
  Escape   :: Lam (S n) env (Code t) -> Lam n env t
\end{verbatim} 
But, what then is {\tt Code}? Some thought leads to the following definition.
\begin{verbatim}
data Code t = exists env . Code (Lam Z env t)
\end{verbatim}
Code is just a {\tt Lam} term at level zero ({\tt Z}) which can be typed in
some environment\footnote{Which environment actually matters, but will discuss
this in more detail later.}.  Now, let's observe the type of the same term:
\begin{verbatim}
Bracket (Apply (Var `f) (Shift (Var `x))) 
:: Lam #(1+u) {`f:a -> b,`x:a; v} (Code b)
\end{verbatim}
Almost, but the environment index, \verb+{`f:a -> b,`x:a; v}+,
does not indicate the stage at which a variable is bound. This can be
fixed by making the environment be a row of triples (rather than pairs).
We introduce the kind {\tt HasT} to encode triples, and adjust the definition of {\tt Var},
{\tt Shift}, and {\tt Abstract}. 
\begin{verbatim}
kind HasT = H3 Tag Nat *0

data Lam:: Nat ~> Row HasT ~> *0 ~> *0  where
  Var      :: Label s -> Nat' n -> 
              Lam n (RCons (H3 s n t) env) t
  Abstract :: Label a -> 
              Lam n (RCons (H3 a n s) env) t -> 
              Lam n env (s -> t)
  Shift    :: Lam n env t -> 
              Lam n (RCons (H3 s m q) env) t
\end{verbatim}
When a variable is used, it must be applied to a singleton {\tt (Nat' n)}
to indicate at what level it was defined. Consider again the type of
the example term (adjusted to include level information on the variables).
\begin{verbatim}
Bracket (Apply (Var `f #0) (Shift (Var `x #0)))
:: Lam #1 {`f^#0:(a -> b), `x^#0:a; u} (Code b)
\end{verbatim}

\noindent
We have (again) introduced some syntactic sugar. When displaying a type
of kind {\tt HasT}, we display \verb+(H3 `tag #n typ)+ as \verb+(`tag^#n:typ)+.
A new problem arises if the two variables come from two different stages. 
Assume {\tt `f} comes from stage 0, and {\tt `x} from
stage 1, as it might in the MetaML term \verb+(fn f => <fn x => f x>)+. This makes {\tt `f} a cross stage persistent
value. We replace {\tt (Var `x \#0)}
with {\tt (Var `x \#1)}. Observe the types of {\tt f} and {\tt x}, and the
constructor {\tt Apply}:
\begin{verbatim}
Var `f #0 :: Lam #0 {`f^#0:a; u} a
Var `x #1 :: Lam #1 {`x^#1:a; u} a
Apply :: Lam u v (a -> b) -> Lam u v a -> Lam u v b
\end{verbatim}
As we intended, the levels of the two terms now differ. The term {\tt f} is at level 0,
and the term {\tt x} is at level 1. The application of {\tt f} to {\tt x}
\verb+(Apply (Var `f #0) (Shift (Var `x #1)))+
will be ill-typed, because {\tt Apply} requires that the level of the two
terms be the same. The problem here is similar to the problem of constructing a term
with two different variables with different names but with the same de Bruijn
indices. We solved that problem be introducing the {\tt Shift} operator.
We can solve the level problem by similar means. Introduce a new constructor
of {\tt Lam} terms that raises the level of a term. We call this constructor
{\tt Cross}, because it used when we want cross-stage persistence.
\begin{verbatim}
 Cross    :: Lam n env t -> Lam (S n) env t
\end{verbatim} 
Using {\tt Cross} on {\tt f}, the term is now well typed:
\begin{verbatim}
Apply (Cross (Shift (Var `f #0))) (Var `x #1)
:: Lam #1 {`x^#1:a, `f^#0:(a -> b); u} (Code b)
\end{verbatim}
The term is a second level term. This is indicated in the level index
(\verb+#1+). It mentions two variables {\tt f} defined at level {\tt \#0} with
type {\tt (a -> b)}, and {\tt x} defined at level {\tt \#1} with type {\tt b}.
The type of the term is {\tt (Code b)}. We have succeeded in describing
well-formed object-level terms using the type system of the meta-language!

\begin{figure}[t]
\hrule
\vspace{1.5ex}
\begin{center}
 \scalebox{0.40}{\psinput{stack.eps}}
\end{center}

\caption{A typing judgment contains a ``sliding band" of type contexts. The band
is ``implemented" using a single type context, and two stacks of type contexts.
The depth of the stacks is unbounded, since a multistage program can
have an arbitrary number of stages.} \label{band}

\vspace{1ex}
\hrule
\end{figure}

Unfortunately, while an interesting exercise, this particular path is hard to
extend. The problem comes from having variables from all levels in a
single environment. Rhetorically, should it be necessary for the variables
{\tt f} and {\tt x} to have different deBruijn indices? While executing in the
second stage, the environment including the binding for {\tt f} (as well as
the need for {\tt f}) will be long gone. The single environment approach also
causes many problems when doing proofs. It is necessary to construct lemmas
which talk about projections over environments. A projection projects only those
variables defined a single stage. The key to a simple and elegant type system
for a staged language is to break from tradition. Do not use an level-indexed
term, and do not use a single environment.

\section{Sliding Bands of Contexts} \label{final}

The key to typing staged terms is to think about
the environment as a ``sliding band" of type contexts. The lambda
fragment binds and looks up variables in the ``present" stage. Staging operations
slide the pointer. The structure of the environment is illustrated in
Figure \ref{band}. 

Capturing this structure in the object-level type system as a meta-level
GADT is an interesting, though not too difficult exercise. We start
by introducing a new GADT for terms {\tt Exp}. Like {\tt Lam}, it has
multiple indexes. We write
{\tt (Exp past now future t)} where {\tt past} and {\tt future} are
stacks of contexts, {\tt now} is the current environment, and {\tt t}
is the type of the term.

For mnemonic purposes we would like the past stack to ``grow" on the right, and
the future stack to ``grow" on the left. To do this we use meta-level pairs
(i.e. {\tt (a,b)}). Unfortunately, the type constructor for pairs only allows
types of kind {\tt *0} as arguments, and environments (as we have defined
them) have kind {\tt (Row HasType)}. Fortunately we also defined the type
constructor {\tt Rec} (for constructing records) which when applied to a {\tt
(Row HasType)} has kind {\tt *0}. To push a context {\tt (r::Row HasType)} on
to a right growing stack {\tt (past::*0)} we write {\tt (past,Rec r)}, and to push a
context {\tt r} on to a left growing stack {\tt future} we write {\tt (Rec r,future)}.
Keeping these explanations in mind study the GADT for expressions below:
\begin{verbatim}
data Cd n f t = Cd (forall p . Exp p n f t)

data Exp:: *0 ~> Row HasType ~> *0 ~> *0 ~> *0 where 
  -- The lambda fragment
  V:: Label s -> Exp p (RCons (Has s t) env) f t
  Sh::  Exp p env f t -> 
        Exp p (RCons (Has s q) env) f t
  Abs:: Label x -> 
        Exp p (RCons (Has x s) n) f t -> 
        Exp p n f (s -> t)
  App:: Exp p n f (t1->t) -> 
        Exp p n f t1 -> 
        Exp p n f t
  
  -- the staging fragment
  Br::  Exp (p,Rec n) c f t -> 
        Exp p n (Rec c,f) (Cd c f t)
  Esc:: Exp p b (Rec n,f) (Cd n f t) -> 
        Exp (p,Rec b) n f t
  Csp:: Exp p b (Rec n,f) t -> 
        Exp (p,Rec b) n f t
  Run:: (forall x . Exp p n (Rec x,f) (Cd x f t)) -> 
        Exp p n f t
  
  -- pairs and constants
  Pair:: Exp p n f s -> 
         Exp p n f t -> 
         Exp p n f (s,t)
  Const:: t -> Exp past now future t
\end{verbatim}
The type {\tt Cd} is similar to the type {\tt Code} from
Section \ref{lang}. It is nothing more than a restricted form
of {\tt Exp}, and is the meta-level type used to implement
object level terms with code type. In the examples below
we often contrast {\tt Lam} terms from Section \ref{lang}
with {\tt Exp} terms from this section and the distiction
between {\tt Code} (used in {\tt Lam} terms) and {\tt Cd} (used in {\tt Exp}
terms) is important.

The {\tt Exp} type is how we represent object-level MetaML terms in \om.
The lambda fragment of the language is completely standard. It basically
ignores (by keeping constant) the first and third indexes which are the
stacks of contexts. Its approach to variables is the same as in the
{\tt Lam} object-language, using {\tt Sh} to shift variables into increasing
de Bruijn indices.
The staging fragment is reminiscent of the level indexed {\tt Lam}
terms. But rather than changing the level index, these constructors
manipulate the two context stacks. Let's compare
the bracket constructor from the two approaches.
\begin{verbatim}
Bracket :: Lam u v a -> 
           Lam #(1+u) v (Code a)

Br      :: Exp (a,Rec u) v b c -> 
           Exp a u (Rec v,b) (Cd v b c)
\end{verbatim}
While the {\tt Lam} bracket constructor increments the level index,
the {\tt Exp} bracket constructor shifts the top context on the past
stack ({\tt Rec u}) into the now context, and shifts the current context ({\tt v}) onto the top
of the future stack. 
\begin{verbatim}
Escape :: Lam #(1+u) v (Code a) -> 
          Lam u v a
          
Esc    :: Exp a u (Rec v,b) (Cd v b c) -> 
          Exp (a,Rec u) v b c

Cross  :: Lam u v a -> 
          Lam #(1+u) v a
          
Csp    :: Exp a u (Rec v,b) c -> 
          Exp (a,Rec u) v b c
\end{verbatim}
A similar (but inverse) relationship holds for the escape operator. The
cross-stage persistence operator, rather than changing the level of a term,
cause its argument to reach into the top most
past context to compute its type. Lets observe the type of our
example from the previous section in both systems.
\begin{verbatim}
Apply (Cross (Shift (Var `f #0))) (Var `x #1)
:: Lam #1 {`x^#1:a, `f^#0:(a -> b); u} b

App (Csp(Sh(V `f))) (V `x)
:: Exp (a,Rec {u:b,`f:c -> d; v}) {`x:c; w} e d

App (Csp (V `f)) (V `x)
:: Exp (a,Rec {`f:c -> d; v}) {`x:c; w} e d
\end{verbatim}
Notice how the environment of the two stages is now completely separate. Both
{\tt f} and {\tt x} could live at the same de Bruijn index since
the two environments are disjoint. The third term illustrates
what happens when we de Bruijn shift neither variable.
If we level shift the last expression
by wrapping the whole thing inside a bracket we obtain.

\begin{verbatim}
Br(App (Csp (V `f)) (V `x))
:: Exp a {`f:c -> d; v} 
         (Rec {`x:c; w},e) 
         (Cd {`x:c; w} e d)
\end{verbatim}
Notice how the {\tt Br} shifts the top context in the past stack into 
the current environment, and makes the type of the resulting
expression have code type.

The last staging fragment constructor is {\tt Run}. The typing
rule for run is intimately tied with the declaration for code.
As in the level indexed example, code is should be nothing
more than an expression at level 0 typed in some environment.
\begin{verbatim}
data Cd n f t = Cd (forall p . Exp p n f t)
\end{verbatim}
A term is at level 0 if it has no embedded escapes or cross-stage
persistent sub-terms. We capture this by requiring that the past
environment is universally quantifiable. To better understand
this requirement, consider a term
with an embedded escape versus one without.
\begin{verbatim}
App (Esc (V `f)) (Const 3)
:: Exp (a,Rec {`f:Cd u b (Int -> c); v}) u b c

App (V `f) (Const 3)
:: Exp a {`f:Int -> b; u} c b
\end{verbatim}
Notice the rich structure of the past stack
\begin{verbatim}
(a,Rec {`f:Cd u b (Int -> c); v})
\end{verbatim}
when the escape is present, and that it is simply a universally quantified 
variable ({\tt a}) when there are no escapes. It seems polymorphism
is as useful as counting levels! Finally we can explain the type for the
run construct.
\begin{verbatim}
data Exp:: *0 ~> Row HasType ~> *0 ~> *0 ~> *0 where
  ...
  Run:: (forall x . Exp p n (Rec x,f) (Cd x f t)) -> 
        Exp p n f t
\end{verbatim}

A piece of code can be run if it has no free variables. We use the same trick
for testing for the lack of free variables, as we did for testing for the lack of
escapes: polymorphism. A term can be run only if has code type {\tt (Cd x f
t)}, and the current environment ({\tt x}) is universally quantifiable.


\begin{figure*}[t]
\vspace{1.5ex}
\begin{center}
 \scalebox{0.85}{\psinput{rules.eps}}
\end{center}

\vspace*{-.4in}
\caption{Compare the type inference rules with the type
of the constructors {\tt Br} and {\tt Esc}. }\label{typerules}
\hrule
\end{figure*}

Reflect for a minute on what we have just done. The meta-level types of the
constructors for the object-level terms {\tt Exp} specifies a complete type
system for object-level terms for a multi-stage language with all the features
of bracket, escape, run, and cross-stage persistence. The correspondence
between typing rules and the types of the object-level constructors is one-to-
one. This is illustrated in Figure \ref{typerules}.

\subsection{An Evaluation Function.}

The last step is to show that the evaluation of a well typed term
does not go wrong. There are several ways to describe what evaluation
means. In this section we define an
interpreter just as we did for {\tt Lam} in Section
\ref{binding}, we introduce an evaluation function with type:
\verb+Exp past now future t ->+ \verb+Rec now -> t+. Every well
typed {\tt Exp} term with type {\tt t} under an environment with shape {\tt now} is given
meaning as a function from a record with shape {\tt now} to {\tt t}.


\begin{verbatim}
eval :: Exp past now future t -> Rec now -> t
eval (Const n) env = n
eval (V a)  (RecCons b y x) = y
eval (Sh e) (RecCons b x y) = eval e y
eval (App f x) env = (eval f env) (eval x env)
eval (Abs a e) env = \ v -> eval e (RecCons a v env)
eval (Pair x y) env = (eval x env, eval y env)
eval (Br e) env = Cd (bd (EnvZ env) e)
eval (Run e) env = case eval e env of 
                     Cd x -> eval x RecNil
\end{verbatim}

As always, once the type system is right, the eval function is
simplicity itself. Lets consider the cases one at a time.

\begin{itemize}
\item{\bf {\small (Const n)}.} Constants evaluate to themselves.

\item{\bf {\small (V a)}.} The value of every (unshifted) variable is stored as the
first element of the record storing the environment.  There is no need to check
that the label {\tt a} and {\tt b} match, because the type system ensures that they do.

\item{\bf {\small (Sh e)}.} A shifted expression ignores the first element of the
record. Simply evaluate the unshifted expression in the rest of the record.

\item{\bf {\small (App f x)}.} Evaluate the function part of the application. This must be a function. Apply
it to the result of evaluating the argument part of the application.

\item{\bf {\small (Abs a e)}.} The result of an abstraction must be a
function, so we return a meta-level lambda abstraction. The body of the
object-level term is well typed in an environment which maps its formal
parameter to the first element of the record. So simply evaluate the body in a
new record with the meta-level lambda bound value as an additional element at the front.

\item{\bf {\small (Pair x y)}.} Evaluate each part of a pair, and build a pair
with the result.

\item{\bf {\small (Br e)}.} Build an expression that is polymorphic in the
past with the function {\tt bd}, then make code out of it with the
constructor for code {\tt Cd}. We discuss building code with {\tt bd} in the
next section.

\item{\bf {\small (Run e)}.} The expression {\tt e} must evaluate
to code whose underlying {\tt Exp} is typable in any environment.
Thus, evaluate {\tt e}, extract the code {\tt x}, and evaluate that 
in the empty record. Any record will do, {\tt RecNil} is a convenient one.
\end{itemize}

\subsection{Building code.} 

To evaluate a code template, that is an {\tt Exp} whose top-level constructor
is {\tt Br}, we must walk over the term removing embedded escapes and 
cross-stage persistent terms. Inside a bracket these two kinds of subterms 
are the only ones that refer to values in the current (or past) environments. We must embed these
values in the code produced now, because when the resulting code is evaluated,
the values in the current environment will no longer be accessible.
To understand this, recall how run evaluates code in the empty environment.
If we don't capture the values of variables from lower levels when the
code is built, it will be too late when the code is run.

Escapes are removed by splicing in the code obtained
by evaluating the term inside the escape. This is how template
holes are spliced in. Cross-stage persistent
term are removed by evaluating them in the current environment,
and lifting the value obtained up to the code level. If all the escapes
and all the cross-stage persistent terms are removed, the resulting
expression will be polymorphic in the past stack. This is captured by the 
type of the
build function: \verb+(bd :: Env a z -> Exp a n f t -> Exp z n f t)+.
Note how the past index ({\tt z}) of the {\tt Exp} in the range of {\tt bd}
is unrelated to the type of the past stack ({\tt a}) of the input {\tt Exp}.

The {\tt bd} function walks over an expression, copying it except
when it reaches an escape or cross-stage persistent sub-term. Notice
how only the cases for {\tt Esc} and {\tt Csp} do anything interesting.

\begin{verbatim}
data Env:: *0 ~> *0 ~> *0 where
  EnvZ:: a -> Env (b,a) c
  EnvS::  Env a b -> Env (a,c) (b,c)   
  
bd :: Env a z -> Exp a n f t -> Exp z n f t
bd env (Const n) = Const n 
bd env (V z) = V z
bd env (App x y) = App (bd env x) (bd env y)
bd env (Abs a e) = Abs a (bd env e)
bd env (Pair x y) = Pair (bd env x) (bd env y)
bd env (Br e) = Br(bd (EnvS env) e)
bd env (Run e) = Run(bd env e)
bd (EnvZ env) (Esc e) = 
  case eval e env of 
    Cd x -> x
bd (EnvS r) (Esc e) = 
  case bd r e of 
    Br x -> x
    y -> Esc y 
bd (EnvZ env) (Csp e) = Const(eval e env)
bd (EnvS r) (Csp e) = Csp(bd r e)  
\end{verbatim}

Not all escapes or cross-stage persistent can be removed. Those embedded
inside more than the original surrounding brackets refer to values available
when the code being built will be run. For example, when evaluating a term
like \verb+<f ~x <g ~y>>+ we rebuild the term  \verb+(f ~x <g ~y>)+. Only the
first level escape \verb+~x+ should be evaluated and then spliced in the rebuilding process. In
order to know which escapes (and cross-stage persistent terms) to process,
brackets must be counted as {\tt bd} crawls over a term. Counting brackets is the role of the
{\tt Env} parameter to {\tt bd}. The number of brackets inside the original
pair (that was removed in the {\tt Br} case of {\tt eval}) can be determined by the
number of {\tt EnvS} constructors wrapped around the current environment. Thus
\verb+(EnvS (EnvS (EnvZ env)))+ means the {\tt bd} is processing a subterm
inside of two additional sets of brackets., while \verb+(EnvZ env)+ means zero
additional sets of brackets.

The only interesting cases inside {\tt bd} are the {\tt (Br e)}, {\tt (Esc e)},
and {\tt (Csp e)} cases. For the {\tt (Br e)} case simply rebuild the
subterm, but wrap an extra {\tt EnvS} term around the environment to
record the fact. For the {\tt (Esc e)} case, if the environment
is {\tt (EnvZ env)} then the subterm {\tt e} should be evaluated,
and the resulting code term spliced in. If the environment
is {\tt (EnvS env)} then at least one extra pair of brackets surrounds
this term. Simply rebuild it (remembering to count down by one by removing
the {\tt EnvS}) and wrap an {\tt Esc} around the result. An optimization
is possible here. Consider a subterm inside a bracketed term
like \verb+~<e>+, this could be replaced by \verb+e+. We recognize
this by comparing the rebuilt subterm. If it is itself a bracketed term,
then we can apply the rule, and don't need to wrap the extra escape around
the returned result. The cases for {\tt Csp} are almost identical
except we use {\tt Const} to lift a value into a piece of code when
the environment count is zero.

\subsection{Well-typed MetaML programs do not go wrong}

In Section \ref{binding} we showed that a denotational semantics for
the lambda calculus was sound by showing that it was well-typed,
total, and compositional. 

Showing that well-typed programs do not go wrong for MetaML is
more complicated. First, our evaluator is not a denotational
semantics, because it is not compositional. The problematic case is
the evaluation of a {\tt Run} term, where the {\tt eval} function
is called twice.
\begin{verbatim}
eval (Run e) env = case eval e env of 
                     Cd x -> eval x RecNil
\end{verbatim}
The second call to {\tt eval} is on a term that is not a subterm of the original term. Note if
evaluating {\tt e} from a term {\tt (Run e)} causes an infinite sequence of
{\tt Run} subterms, the evaluator would not be total. 

Can we argue the non-compositional evaluator is sound? I.e. that every
well-typed term is mapped to a meta-level value with that type. To show
this we must answer the question: {\em What's a well-typed term in
MetaML?} In MetaML only level 0 terms can be evaluated i.e. As discussed
in Section \ref{lang}, level 0 terms have no escapes at level 0, and a
term is at level 0, if and only if, its is polymorphic in its past. Thus
we need argue soundness for terms polymorphic in their past.

Showing {\em totality} is problematic for two reasons. The first is the
possibility of evaluation leading to an infinite sequence of {\tt Run}s
inside {\tt Run}s. We are currently working on  a solution in which the
nesting level of {\tt Run} is encoded as an additional type index to {\tt
Exp}. We hope to report on this
in the final paper.

The second cause of non-totality is non-exhaustiveness. The function
{\tt eval} is not defined on terms built with the {\tt Esc} or {\tt Csp}
constructors. Fortunately, code polymorphic in its past cannot contain
either {\tt Esc} or {\tt Csp} at level 0. This leads us to the following
strategy, first define a version of {\tt eval}, called {\tt eval0}, that
can only be applied to terms polymorphic in their past.
 
\begin{verbatim}
eval0 :: (forall p . Exp p now future t) -> 
         Rec now -> t
eval0 exp env = eval exp env
  where eval::Exp past now future t -> Rec now -> t
        eval (Const n) env = n
        ...
        eval (Run e) env = 
           case eval e env of 
             Cd x -> eval0 x RecNil 
             
        bd::Env a z -> Exp a n f t -> Exp z n f t
        bd env (Const n) = Const n 
        ...
\end{verbatim}
By defining {\tt eval} and {\tt bd} as local functions of {\tt eval0}
It is impossible to apply {\tt eval} to terms not polymorphic in
their past, since all access to {\tt eval} is through {\tt eval0}
which requires polymorphic terms. All recursive calls to {\tt eval} and
{\tt Bd}
(except the second one in the {\tt Run} case in {\tt eval}) are applied to strict
subterms of the original term, so if the original terms was
polymorphic, so must all these sub-terms be polymorphic. The
problematic second call to {\tt eval} in the {\tt Run} case
which is applied to the return result of applying {\tt eval} to a 
subterm, can be replaced with {\tt eval0}, because expressions
inside the {\tt Cd} constructor must also be level 0 terms.
Thus both {\tt eval} and {\tt eval0}
are total, in that they will never be called on a term for
which they are not defined. 
While not a complete proof, we have made subtantial progress.
In the next section we demonstrate that the type system distinguishes
the subtle staging errors discussed in Section \ref{metaml}.

\section{Evaluating the type system.}
Section~\ref{metaml} introduced three programs that highlight
subtle typing issues for multi-stage programs.  We now evaluate the
novel multi-stage type system discussed in the previous
section based on how it performs on these issues.

\begin{itemize}
\item{\bf Correct use of variables.}
The following program exhibits a level-mismatch error, and is incorrect.
\begin{verbatim}
<fn x => ~(id x)>;
\end{verbatim}
This program makes use of the variable x at a stage 
prior to the stage in which it is bound. What about the \om{} encoding of this program?
\begin{verbatim}
Br (Abs `x (Esc (App (Abs `y (V `y)) (V `x))))
\end{verbatim}
The interpreter gives this term the following type:
\begin{verbatim}
Exp a {`x:Cd {`x:b; u} c d; v} 
      (Rec u,c) 
      (Cd u c (b -> d)))
\end{verbatim}

The variable \verb|`x| shows up in two different environments at two different
stages. Each environment in the sliding band of type contexts is a separate
name space, so the two occurrences of \verb|`x| in the \om\ encoding aren't
the same variable, as they are in the MetaML program. So while this is a ``type correct"
program in the \om\ encoding, it isn't an encoding of \verb|<fn x => ~(id x)>|.

This brings up an important point. The cross-stage persistence
construct does not appear in MetaML, so there must be some syntactic
sugar in the parsing stage which translates MetaML syntax into the {\tt Exp} data-structures.
This preprocessing step counts brackets, and assigns to every
variable a static level at its point of definition. At each point
of use, a number of cross-stage persistent annotations ({\tt Csp}) are
inserted. The correct number is the difference between the level
at which a variable is defined, and the level at which it is used.
If this number is negative, the program is ill-typed. So this kind
of error is actually caught before type checking in the syntactic
sugar pre-processing phase.

\vspace*{.1in}
\item{\bf Running code with free variables.} The following program should be
rejected because the {\tt run} will force the evaluation of the not-yet-bound
variable {\tt x}.
\begin{verbatim}
val bad = <fn x => ~(run <x>) + 4>;
\end{verbatim}

How does this program hold up in our \om{} encoding?
\begin{verbatim}
|- Br (Abs `x (Esc (Run (Br (V `x)))))
Error: Br (V `x) isn't polymorphic enough
Expected type: forall v. 
               Exp a u (Rec v,b) 
                       (Cd v b c) 
   Found type: Exp a u (Rec {`x:c; w},b) 
                       (Cd {`x:c; w} b c))
\end{verbatim}

The program is rejected because the term \verb+(Br (V `x))+
isn't polymorphic enough. The term has the type
\begin{verbatim}
Exp a u (Rec {`x:b; v},c) (Cd {`x:b; v} c b)
\end{verbatim}
See how the variable \verb+x+ shows up in the current context of
the code type. Run can be applied only to code whose
current context is completely polymorphic.

\vspace*{.1in}
\item{\bf Tracking Alpha-renaming.} 
Recall the test programs highlighting the subtle issues at work here.
\begin{verbatim}
-| val puzzle1 = 
       <fn a => ~((fn x => <x>)(fn y => <a>)) 0>;
-| val puzzle2 = 
       <fn a => ~(<fn y => <a>>) 0>;
\end{verbatim}
We argued that \verb|(run puzzle1)| should cause a type error, but \verb|(run puzzle2)| 
shouldn't.

The \om{} encoding behaves in exactly this way. Consider
the type of the \om\ translation of {\tt puzzle1}.
\begin{verbatim}
Br (Abs `a 
        (App (Esc (App (Abs `x (Br (Csp (V `x)))) 
                       (Abs `y (Br (V `a)))))
             (Const 0) ))
:: Exp a u (Rec v,b) 
           (Cd v b (c -> Cd {`a:c; v} b c))
\end{verbatim}
It appears polymorphic enough. Note the current
context is typed by the type variable {\tt v} inside the {\tt Cd} type. But further
inspection shows that this type variable {\tt v} also appears
in the type of the code returned \verb+(c -> Cd {`a:c; v} b c)+.
Thus it isn't really polymorphic at all.
\begin{verbatim}
-| run puzzle1
Error: puzzle1 isn't polymorphic enough
Expected Type: 
   (forall v. Exp a u (Rec v,b) 
                      (Cd v b c))
Found Type:   Exp a u (Rec v,b) 
                      (Cd v b (c -> Cd {`a:c; v} 
                                       b c)))
\end{verbatim}
The type given to \verb|puzzle1| illustrates of the problem.  The 
environment \verb|v| is needed even after the function is applied.
But if the user performs the beta-reduction
manually, then the problem does not occur.  
{\small
\begin{verbatim}
-| puzzle2
   Br (Abs `a (App (Esc (Br (Abs `y (Br (V `a))))) 
                   (Const 0)))
 :: Exp a u (Rec v,(Rec {`a:b; w},c))
            (Cd v (Rec {`a:b; w},c) 
                  (d -> Cd {`a:b; w} c b))
-| Run puzzle2
   Run (Br (Abs `a (App (Esc (Br (Abs `y (Br (V `a))))) 
                   (Const 0))))
 :: Exp a u (Rec {`a:b; v},c) 
            (d -> Cd {`a:b; v} c b)
\end{verbatim}}
\end{itemize}

The \om\ function {\tt eval} shows that if we disallow {\tt puzzle1}
we can use a very simple implementation, yet still remain safe.

\section{Related work.}

There are two areas of related work that should be discussed.
First, type systems for multi-stage languages, and second, the
design of meta-languages for specifying and reasoning about
object-languages. 

{\bf Type Systems.} Of the many papers on type systems for multi-stage languages
only two achieve the scope of the system presented in this paper
(in terms of the features supported), and one should be
discussed because it uses a similar, but weaker approach.
They are:
{\it Environment Classifiers}\cite{TahNie03} by Taha and Nielsen, 
{\it Meta-Programming through Typeful Code Representation}\cite{XiChen2003}
by Chiyan Chen and Hongwei Xi, 
and {\it A Modal Analysis of Staged Computation} by Davies and Pfenning.

Neither of the first two has the simplicity of the 
system presented in this paper, and the third
covers a much simpler language.
The first two employ a single type context
that records the type of all variables, rather than a sliding band
of type contexts. The third uses a single stack, rather than
a sliding band (two stacks).

The paper {\it Environment Classifiers} employs a level-stratified syntax and
a complicated syntactic notion called {\it demotion}. Demotion is a term to
term transformation (performed under a set of variables) that essentially
places additional cross-stage persistent annotations around the variables
mentioned in the set. The notion of demotion is necessary to show that
reduction preserves typing. This considerably complicates the type
system.

The paper {\it Meta-Programming through Typeful Code Representation}
employs a de Bruijn notation similar to the notation used in
this paper, but relies an a complicated notion of projection. 
Projection, computes a new type context with only those
variables from specified levels from the ``master" context with
information about all levels. This considerably complicates
proofs about program properties, like subject reduction. Neither system
approaches the simplicity or the compactness of the system presented here.

The paper {\it A Modal Analysis of Staged Computation} 
\cite{Davies-Pfenning-JACM} provides a type system for several staged languages
based on intuitionist modal logic S4. While similar is some
respects to the type system for MetaML presented here it has some
basic differences. First, the languages presented can only build code
for closed terms. For example: \verb+<fn x => ~(f <x>)>+ could not be
typed because the argument to {\tt f} is a code term with a free
variable bound in the present level (from the perspective of inside
the brackets). This severely limits the code generators that can be
expressed. Second, the type system uses a single stack of
environments one for each level, as opposed to a sliding band (a pair
of stacks of environments), this simplification is sufficient because
the language lacks the escape annotation. Instead it uses a weaker
{\em unbox} construct that cannot be placed inside of brackets, also
limiting the expressiveness of the language. Finally, none of there
languages provides the explicit cross operator that lifts values from
previous stages into code.

{\bf Meta-languages.} There are many systems that could be used as 
meta-languages for manipulating object-languages. They include Inductive
Families\cite{Coquand:1994:IDT,Dybjer:1999:FAI}, theorem provers
(Coq\cite{COQ74}, Isabelle\cite{Paulson90lacs}), logical frameworks
(Twelf\cite{CADE99*202}, LEGO\cite{LuoPollack92}), proof assistants
(ALF\cite{oai:CiteSeerPSU:38734}, Agda\cite{Agda}) dependently typed languages
(Epigram\cite{epigram}, RSP\cite{Rogue}).

These systems all choose a point in the design space where types and values
are indistinguishable. In \om\ we choose a point in the design space with a
strict phase distinction between types and values. We also have a strong
desire to minimize type annotations and other administrative work.

Several systems choose a point in the design space closer to ours, where the
distinction between types and values is preserved. We owe much to these works
for inspiration, examples, and implementation techniques. These include
Guarded Recursive Datatype Constructors\cite{XiCheChe03}, First-class phantom
types\cite{Hinze:03:Phantom}, Wobbly types\cite{wobbly}, and Silly Type
Families\cite{sillyTF}. There are only minor syntactic differences between
these systems and our own, and we imagine they could be used in a similar
manner, except for one very important thing. In these systems, type indexes
are restricted to types classified by {\tt *0}, because the systems have no
way of introducing new kinds. We consider the introduction of new kinds as an
important contribution of our work. Without new kinds, rows, natural numbers,
and other important meta-level entities at the type level are not possible.

\section{Conclusion.} 

The goals of this paper were three fold. First, to introduce
the meta-language \om. Second, to illustrate that important explorations
in language design can be assisted with the correct tools. Third,
to demonstrate that a sound type system for a multi-stage
language can also be simple, concise, and easy to understand.
We believe the system using a sliding band of type contexts can generalize
to any multi-stage language and is of general interest to all who
study program generation. As illustrated the type system can disqualify
programs with subtle properties, and can thus simplify the semantics
of multi-stage languages because they no-longer need to address programs
with these subtle characteristics.

\bibliographystyle{plain}
\bibliography{proposal}

\end{document}

