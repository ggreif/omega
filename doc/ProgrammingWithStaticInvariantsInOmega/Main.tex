\documentclass[10pt]{article}
%% ODER: format ==         = "\mathrel{==}"
%% ODER: format /=         = "\neq "
\makeatletter

\usepackage{amstext}
\usepackage{amssymb}
\usepackage{stmaryrd}
\DeclareFontFamily{OT1}{cmtex}{}
\DeclareFontShape{OT1}{cmtex}{m}{n}
  {<5><6><7><8>cmtex8
   <9>cmtex9
   <10><10.95><12><14.4><17.28><20.74><24.88>cmtex10}{}
\DeclareFontShape{OT1}{cmtex}{m}{it}
  {<-> ssub * cmtt/m/it}{}
\newcommand{\texfamily}{\fontfamily{cmtex}\selectfont}
\DeclareFontShape{OT1}{cmtt}{bx}{n}
  {<5><6><7><8>cmtt8
   <9>cmbtt9
   <10><10.95><12><14.4><17.28><20.74><24.88>cmbtt10}{}
\DeclareFontShape{OT1}{cmtex}{bx}{n}
  {<-> ssub * cmtt/bx/n}{}
\newcommand{\tex}[1]{\text{\texfamily#1}}	% NEU

\newcommand{\Sp}{\hskip.33334em\relax}


\newcommand{\Conid}[1]{\mathit{#1}}
\newcommand{\Varid}[1]{\mathit{#1}}
\newcommand{\anonymous}{\kern0.06em \vbox{\hrule\@width.5em}}
\newcommand{\plus}{\mathbin{+\!\!\!+}}
\newcommand{\bind}{\mathbin{>\!\!\!>\mkern-6.7mu=}}
\newcommand{\sequ}{\mathbin{>\!\!\!>}}
\renewcommand{\leq}{\leqslant}
\renewcommand{\geq}{\geqslant}
\usepackage{polytable}

%mathindent has to be defined
\@ifundefined{mathindent}%
  {\newdimen\mathindent\mathindent\leftmargini}%
  {}%

\def\resethooks{%
  \global\let\SaveRestoreHook\empty
  \global\let\ColumnHook\empty}
\newcommand*{\savecolumns}[1][default]%
  {\g@addto@macro\SaveRestoreHook{\savecolumns[#1]}}
\newcommand*{\restorecolumns}[1][default]%
  {\g@addto@macro\SaveRestoreHook{\restorecolumns[#1]}}
\newcommand*{\aligncolumn}[2]%
  {\g@addto@macro\ColumnHook{\column{#1}{#2}}}

\resethooks

\newcommand{\onelinecommentchars}{\quad-{}- }
\newcommand{\commentbeginchars}{\enskip\{-}
\newcommand{\commentendchars}{-\}\enskip}

\newcommand{\visiblecomments}{%
  \let\onelinecomment=\onelinecommentchars
  \let\commentbegin=\commentbeginchars
  \let\commentend=\commentendchars}

\newcommand{\invisiblecomments}{%
  \let\onelinecomment=\empty
  \let\commentbegin=\empty
  \let\commentend=\empty}

\visiblecomments

\newlength{\blanklineskip}
\setlength{\blanklineskip}{1mm}

\newcommand{\hsindent}[1]{\quad}% default is fixed indentation
\newcommand{\NB}{\textbf{NB}}
\newcommand{\Todo}[1]{$\langle$\textbf{To do:}~#1$\rangle$}

\makeatother

\usepackage{latexsym}
\usepackage{fancyvrb}
\usepackage{graphics}
\usepackage{multicol}

\newcommand{\Wmega}{\ensuremath{\Omega}mega}
\def\NN{\mathbb{N}}

% author commands
\newcommand{\hide}[1]{}
\newcommand{\edit}[1]{\marginpar{\raggedright\footnotesize{#1}}}
%\newcommand{\edit}[1]{} % when we're done

\newcommand{\fig}[2]{\parbox[c]{#1}{\includegraphics{#2}}}



\title{Programming with Static Invariants in \Wmega{}}
\author{Nathan Linger \& Tim Sheard}

\begin{document}
\maketitle 

\begin{abstract}
We present an implementation of binomial heaps that 
enforces the balancing invariants at compile time.
We present an implementation of a \emph{typed} CPS transform 
that enforces the typing invariants of both the source and target terms.
\end{abstract}

%%%%%%%%%%%%%%%%%%%%%%%%%%%%%%%%%%%%%%%%%%%%%%%%%%%%%%%%%%%%%%%%%%%%%%%%%%%%%%%%
\section{Introduction}

There has been a lot of interest recently in generalizing algebraic
datatypes as found in most functional programming languages. These
generalizations have been called {\it First Class Phantom Types}
~\cite{FirstClassPhantomTypes} {\it Guarded ADTs}~\cite{xi-gadt,
simonet-pottier-hmg} {\it Wobbly types}~\cite{peytonjones-wobbly},
and in our own work {\it Equality Qualified Types}~\cite{GPCE,ONWARD,sheardLFM04}.
Such types allow users to parameterize types with descriptions of 
static properties of their values.
A simple example is the type constructor \ensuremath{\Conid{Seq}} which
describes sequences of elements whose lengths are statically known. For
example \ensuremath{[\mskip1.5mu \mathrm{1},\mathrm{45},\mathrm{2}\mskip1.5mu]\mathbin{::}\Conid{Seq}\;\mathrm{3}\;\Conid{Int}}. This says that \ensuremath{[\mskip1.5mu \mathrm{1},\mathrm{45},\mathrm{2}\mskip1.5mu]} is a
sequence of \ensuremath{\Conid{Int}} whose length is statically known to be \ensuremath{\mathrm{3}}.
A type system employing such a generalization can enforce strong
static invariants.
This leads to programs that are less likely to contain errors, and are
easier to write, since the types often guide the construction of the
function definitions. 

In order to explore these possibilites we implemented
a new language \Wmega. \Wmega\  is descended from Haskell -- 
Its syntax and type system
are similar to Haskell's, but it introduces a new kind of type
constraint (similar to Haskell's class constraints) which constrains
two types to be equal. A function with an equality constrained type, can
only be applied to values whose types force the constrained types to be
the same. This has been described elsewhere~\cite{GPCE,ONWARD,sheardLFM04}, and is
introduced again for the convenience of the reader in 
Section~\ref{IntroOmega}.


This paper introduces several other features of \Wmega{} that work in
combination with equality qualified types (and could also work in combination
with any of the other variants of generalized datatypes) to allow even more
expressive static invariants. These features are user-defined kinds and 
support for functions on types.  The contributions of this paper are:

\begin{itemize} 

\item Demonstrating the usefulness of an extensible kind system. Introducing a new
kind, defines a kind and new types that are classified by that kind.
The new kind is different from the kind that classifies types that
classify values (often called {\it star} or {\it type}). Types introduced
by the kind mechanism make ideal {\em parameters} to equality qualified
data types, and can be used to express a wide variety of static constraints.

\item The ability to define type functions, and the ability to use such
type functions in the types of functions manipulating values with
equality constrained types. Type functions (as opposed to type
constructors such as \ensuremath{\Conid{List}}, and \ensuremath{\Conid{Tree}}) describe computations
on types.  There use is reminiscent of the use of dependent types 
but we claim that \Wmega{}'s type system
implements a simple subset of the dependent types that is simple to
use, understand, and implement. Type functions require a new approach
to unification when integrated into a system with type inference. The
system of equality constrained types provides just the hooks necessary
to do this in a safe and sound manner.\edit{what is meant by "safe and sound"?}


\item We also present two examples in \Wmega{} which
its power to soundly define functions with complex static invariants. 
The two examples are (1) an implementation of binomial heaps and (2) a
typed continuation-passing-style transform.

\end{itemize}

\section{Equality Qualified Types and Kinds in \Wmega{}}\label{IntroOmega}

We introduce equality qualified types by defining the type constructor \ensuremath{\Conid{Seq}}
from above. In order to do this we must also be able to describe natural 
numbers at the type level. To do this we use \Wmega{}'s kind system. The code 
below introduces a new kind \ensuremath{\Conid{Nat}}, new types \ensuremath{\Conid{Seq}}, \ensuremath{\Conid{Zero}}, and \ensuremath{\Conid{Succ}}, 
and a new function \ensuremath{\Varid{map}}.
\begingroup\par\noindent\advance\leftskip\mathindent\(
\begin{pboxed}\SaveRestoreHook
\column{B}{@{}l@{}}
\column{4}{@{}l@{}}
\column{14}{@{}l@{}}
\column{21}{@{}l@{}}
\column{32}{@{}l@{}}
\column{E}{@{}l@{}}
\fromto{B}{E}{{}{\bf kind}\;\Conid{Nat}\mathrel{=}\Conid{Zero}\mid \Conid{Succ}\;\Conid{Nat}{}}
\nextline[\blanklineskip]
\fromto{B}{E}{{}\mathbf{data}\;\Conid{Seq}\;\Varid{a}\;\Varid{n}{}}
\nextline
\fromto{B}{4}{{}\hsindent{4}{}}
\fromto{4}{14}{{}\mathrel{=}{}}
\fromto{14}{32}{{}\Conid{Nil}\;{}}
\fromto{32}{E}{{}\mathbf{where}\;\Varid{n}\mathrel{=}\Conid{Zero}{}}
\nextline
\fromto{B}{4}{{}\hsindent{4}{}}
\fromto{4}{14}{{}\mid \exists\;\Varid{m}. {}}
\fromto{14}{32}{{}\Conid{Cons}\;\Varid{a}\;(\Conid{Seq}\;\Varid{a}\;\Varid{m})\;{}}
\fromto{32}{E}{{}\mathbf{where}\;\Varid{n}\mathrel{=}\Conid{Succ}\;\Varid{m}{}}
\nextline[\blanklineskip]
\fromto{B}{E}{{}\Varid{map}\mathbin{::}(\Varid{a}\to \Varid{b})\to \Conid{Seq}\;\Varid{a}\;\Varid{n}\to \Conid{Seq}\;\Varid{b}\;\Varid{n}{}}
\nextline
\fromto{B}{21}{{}\Varid{map}\;\Varid{f}\;(\Conid{Nil}){}}
\fromto{21}{E}{{}\mathrel{=}\Conid{Nil}{}}
\nextline
\fromto{B}{21}{{}\Varid{map}\;\Varid{f}\;(\Conid{Cons}\;\Varid{x}\;\Varid{xs}){}}
\fromto{21}{E}{{}\mathrel{=}\Conid{Cons}\;(\Varid{f}\;\Varid{x})\;(\Varid{map}\;\Varid{f}\;\Varid{xs}){}}
\ColumnHook
\end{pboxed}
\)\par\noindent\endgroup\resethooks
\begin{figure}[t]
%\hspace*{1.4in}
\begin{center}
\begin{tabular}{ccccc}\hline
value & & type & & kind \\ \hline
5     &::& Int  &::& *    \\ %\hline
      &  & Zero &::& Nat \\ %\hline
      &  & Succ &::& Nat $\leadsto$ Nat \\ %\hline      
      &  & Seq &::& * $\leadsto$ Nat $\leadsto$ *  \\ %\hline
Nil   &::& Seq $\,\alpha\;$ Zero &::& * \\ %\hline
Cons  &::& $\alpha \rightarrow$ Seq $\alpha\; n \rightarrow$ Seq $\alpha\; ($Succ$\,n)$ &::& * \\ \hline
\end{tabular}
\end{center}
\caption{Classification of values (Nil,Cons), and types (Zero,Succ, and Seq).}\label{kindtable}
%\hrule
\end{figure}   

The kind declaration for \ensuremath{\Conid{Nat}} introduces two new
type constructors \ensuremath{\Conid{Zero}} and \ensuremath{\Conid{Succ}} which encode the natural
numbers at the type level.
Kinds are similar to types in that, while types classify values, kinds
classify types. We indicate this by the {\em classifies} relation \ensuremath{(\mathbin{::})}. For
example: \ensuremath{\mathrm{5}\mathbin{::}\Conid{Int}\mathbin{::}\mathbin{*}}. We say \ensuremath{\mathrm{5}} is classified by \ensuremath{\Conid{Int}},
and \ensuremath{\Conid{Int}} is classified by \ensuremath{\mathbin{*}} (star-zero). 
\hide{
The kind \ensuremath{\mathbin{*}} classifies all types that classify values (things we actually can
compute). \ensuremath{\mathbin{*}} is classified by \ensuremath{\mathbin{*}\mathrm{1}}, etc. There is an infinite hierarchy of
classifications.  We call this hierarchy the {\em strata}. In fact this
infinite hierarchy is why we chose the name \Wmega. The first few strata are:
values and expressions that are classified by types, types that are
classified by kinds, and kinds that are classified by sorts, etc.}
Figure~\ref{kindtable} illustrates the relationship between the 
values, types, and kinds introduced above.

Constructor functions \ensuremath{\Conid{Nil}} and \ensuremath{\Conid{Cons}}
construct elements of the \ensuremath{\Conid{Seq}} datatype. The type of a constructor function
is described in the datatype declaration. For example, the 
following clause in the
\ensuremath{\Conid{Seq}} declaration:\begingroup\par\noindent\advance\leftskip\mathindent\(
\begin{pboxed}\SaveRestoreHook
\column{B}{@{}l@{}}
\column{E}{@{}l@{}}
\fromto{B}{E}{{}\exists\;\Varid{m}. \Conid{Cons}\;\Varid{a}\;(\Conid{Seq}\;\Varid{a}\;\Varid{m})\;\mathbf{where}\;\Varid{n}\mathrel{=}\Conid{Succ}\;\Varid{m}{}}
\ColumnHook
\end{pboxed}
\)\par\noindent\endgroup\resethooks
introduces the \ensuremath{\Conid{Cons}} constructor function. Without the \ensuremath{\mathbf{where}}
qualification, the constructor function \ensuremath{\Conid{Cons}} would have type 
\ensuremath{\Conid{Cons}\mathbin{::}\Varid{a}\to \Conid{Seq}\;\Varid{a}\;\Varid{m}\to \Conid{Seq}\;\Varid{a}\;\Varid{n}}. The \ensuremath{\mathbf{where}} {\em  qualifies} \ensuremath{\Conid{Cons}}'s type, in effect
saying \ensuremath{(\Conid{Cons}\mathbin{::}\Varid{a}\to \Conid{Seq}\;\Varid{a}\;\Varid{m}\to \Conid{Seq}\;\Varid{a}\;\Varid{n})} provided \ensuremath{\Varid{n}\mathrel{=}\Conid{Succ}\;\Varid{m}}. We
capture this formally by writing the equality qualified type \ensuremath{\Conid{Cons}\mathbin{::}(\forall\;\Varid{a}\;\Varid{n}\;\Varid{m}. (\Varid{n}\mathrel{=}\Conid{Succ}\;\Varid{m})\Rightarrow \Varid{a}\to \Conid{Seq}\;\Varid{a}\;\Varid{m}\to \Conid{Seq}\;\Varid{a}\;\Varid{n})}. The equations behind the {\em fat arrow} (\ensuremath{\Rightarrow }) are
equality qualifications. Since \ensuremath{\Varid{n}} is a universally quantified type
variable, there is only one way to {\em solve} the qualification \ensuremath{\Varid{n}\mathrel{=}\Conid{Succ}\;\Varid{m}}
(by making \ensuremath{\Varid{n}} equal to \ensuremath{\Conid{Succ}\;\Varid{m}}). Because of this unique solution,
\ensuremath{\Conid{Cons}} also has the (less general) type \ensuremath{(\forall\;\Varid{a}\;\Varid{m}. \Varid{a}\to \Conid{Seq}\;\Varid{a}\;\Varid{m}\to \Conid{Seq}\;\Varid{a}\;(\Conid{S}\;\Varid{m}))}. This type guarantees that \ensuremath{\Conid{Cons}} can only be applied in contexts
where \ensuremath{\Varid{n}\mathrel{=}\Conid{Succ}\;\Varid{m}}. \hide{Existential quantification of the type variable \ensuremath{\Varid{m}}
names the intermediate length of the sublist of \ensuremath{\Conid{Cons}}, which if not
introduced in this way would appear as an unbound type variable.
{\em Equality Qualification} (indicated by the \ensuremath{\mathbf{where}} in the clauses for \ensuremath{\Conid{Nil}}
and \ensuremath{\Conid{Cons}}), and {\em existential quantification} (indicated by \ensuremath{\exists}
in the clauses for \ensuremath{\Conid{Cons}}) are the features of \Wmega{} that
are used to encode semantic properties.}


Each data-constructor declaration in \Wmega{} may contain a \ensuremath{\mathbf{where}}
clause containing a list of type equations.
The mechanism \Wmega{} uses to keep track of type equalities 
is similar to type class system of Haskell.
A special qualified~\cite{J92}
type is used to assert equality between types, and a
constraint solving system is used to track, propogate, simplify and discharge these
assertions. Equality constraints are tracked in much the same manner as
class constraints in Haskell. For example, 
when assigning a type to a type constructor, the equations
specified in the \ensuremath{\mathbf{where}} clause just become predicates in a
qualified type. 
Thus, the constructor \ensuremath{\Conid{Cons}} is given the
type \ensuremath{(\forall\;\Varid{a}\;\Varid{m}\;\Varid{n}. (\Varid{n}\mathrel{=}\Conid{Succ}\;\Varid{m})\Rightarrow \Varid{a}\to \Conid{Seq}\;\Varid{a}\;\Varid{m}\to \Conid{Seq}\;\Varid{a}\;\Varid{n})}
The equation \ensuremath{\Varid{n}\mathrel{=}\Conid{Succ}\;\Varid{m}} is just another form of predicate, similar
to the class membership predicate \ensuremath{(\Conid{Ord}\;\Varid{a})} in the Haskell type \ensuremath{(\equiv )\mathbin{::}\Conid{Ord}\;\Varid{a}\Rightarrow \Varid{a}\to \Varid{a}\to \Conid{Bool}}).

{\bf Tracking equality constraints.} When type-checking an expression,
the \Wmega{} type checker keeps two sets of equality constraints:
{\em obligations} and {\em assumptions}.

{\it Obligations.} The first set of constraints is a set of
{\em obligations}.  Obligations are generated by the type-checker
when (a) the program constructs data-values with constructors
that contain equality constraints; or (b) an explicit type signature in a
definition is encountered; or (c) when type functions are used (see Section
\ref{typefun}).

For example, consider type-checking the expression \ensuremath{\Conid{Nil}}. The constructor
\ensuremath{\Conid{Nil}} is assigned the type \ensuremath{\forall\;\Varid{a}\;\Varid{n}. \Varid{n}\mathrel{=}\Conid{Zero}\Rightarrow \Conid{Seq}\;\Varid{a}\;\Varid{n}},
thus \ensuremath{\Conid{Nil}\mathbin{::}\Conid{Seq}\;\Varid{a}\;\Varid{n}} but also generates the obligation \ensuremath{\Varid{n}\mathrel{=}\Conid{Zero}}. Since \ensuremath{\Conid{Nil}} is
polymorphic in \ensuremath{\Varid{a}} and \ensuremath{\Varid{n}}, the type variable \ensuremath{\Varid{n}} can be instantiated to \ensuremath{\Conid{Zero}}.
Instantiating \ensuremath{\Varid{n}} to \ensuremath{\Conid{Zero}} makes the equality constraint obligation \ensuremath{\Conid{Zero}\mathrel{=}\Conid{Zero}},
which can be trivially discharged by the type checker. Other
constraints may require more work to discharge.

{\it Assumptions.} The second set of constraints is a set of
{\em assumptions} or {\em facts}. Whenever, a constructor 
based pattern appears in a binding position, and the constructor
was defined using a \ensuremath{\mathbf{where}} clause, the type equalities in
the \ensuremath{\mathbf{where}} clause are added to the current set of assumptions in the
scope of the pattern. These assumptions can be used to discharge
obligations. For example, consider type checking the definition of \ensuremath{\Varid{map}}.
Recall \ensuremath{\Varid{map}\mathbin{::}(\Varid{a}\to \Varid{b})\to \Conid{Seq}\;\Varid{a}\;\Varid{n}\to \Conid{Seq}\;\Varid{b}\;\Varid{n}}.

In the first equation, the pattern \ensuremath{\Conid{Nil}} of type \ensuremath{\Conid{Seq}\;\Varid{a}\;\Varid{n}} introduces the
assumption \ensuremath{\Varid{n}\mathrel{=}\Conid{Zero}} because of \ensuremath{\Conid{Nil}}'s qualified type \ensuremath{\forall\;\Varid{a}\;\Varid{n}. \Varid{n}\mathrel{=}\Conid{Zero}\Rightarrow \Conid{Seq}\;\Varid{a}\;\Varid{n}} This assumption can be used to discharge obligations incurred on the right
hand side of the equation.  Now, consider the second 
equation \ensuremath{\Varid{map}\;\Varid{f}\;(\Conid{Cons}\;\Varid{x}\;\Varid{s})\mathrel{=}\Conid{Cons}\;(\Varid{f}\;\Varid{x})\;(\Varid{map}\;\Varid{f}\;\Varid{xs})}. We
know \ensuremath{\Varid{map}}'s type, so the second pattern \ensuremath{(\Conid{Cons}\;\Varid{x}\;\Varid{xs})} must have type \ensuremath{(\Conid{Seq}\;\Varid{a}\;\Varid{n})}. This implies \ensuremath{\Varid{x}\mathbin{::}\Varid{a}} and \ensuremath{\Varid{xs}\mathbin{::}\Conid{Seq}\;\Varid{a}\;\Varid{m}} provided \ensuremath{\Varid{n}\mathrel{=}\Conid{Succ}\;\Varid{m}}. This
equation is added to the list of assumptions when typing the
right-hand-side of the second equation: \ensuremath{\Conid{Cons}\;(\Varid{f}\;\Varid{x})\;(\Varid{map}\;\Varid{f}\;\Varid{xs})}. This right-hand-side should have type \ensuremath{(\Conid{Seq}\;\Varid{b}\;\Varid{n})} (the range of \ensuremath{\Varid{map}}'s type). Since \ensuremath{\Varid{xs}\mathbin{::}\Conid{Seq}\;\Varid{a}\;\Varid{m}} we can compute \ensuremath{(\Varid{map}\;\Varid{f}\;\Varid{xs})\mathbin{::}\Conid{Seq}\;\Varid{b}\;\Varid{m}}, Since \ensuremath{\Varid{x}\mathbin{::}\Varid{a}} we can
compute \ensuremath{(\Varid{f}\;\Varid{x})\mathbin{::}\Varid{b}}, thus it appears \ensuremath{(\Conid{Cons}\;(\Varid{f}\;\Varid{x})\;(\Varid{map}\;\Varid{f}\;\Varid{xs}))\mathbin{::}\Conid{Seq}\;\Varid{b}\;(\Conid{Succ}\;\Varid{m})}. But given the equality constraint \ensuremath{\Varid{n}\mathrel{=}\Conid{Succ}\;\Varid{m}} the right hand side has
type \ensuremath{\Conid{Seq}\;\Varid{b}\;\Varid{n}} as expected.
Equality qualified types can encode powerful static invariants about
data. For example the type of \ensuremath{\Varid{map}} provides a proof that \ensuremath{\Varid{map}}
returns a list of the same length as its input list.

\section{Type functions in \Wmega}\label{typefun}

We now introduce type-functions. To see why this is necessary
we ask the rhetorical question: {\em What is the type of the append
function on static-length lists}? The append function should have type
\ensuremath{\Conid{Seq}\;\Varid{a}\;\Varid{n}\to \Conid{Seq}\;\Varid{a}\;\Varid{m}\to \Conid{Seq}\;\Varid{a}\;(\Varid{n}\mathbin{+}\Varid{m})}. In
order to type such functions it is necessary to do arithmetic at
the type level. This means defining functions over types. In \Wmega{}
we do this by writing equations over types. We surround type-function
application by braces (e.g. \ensuremath{\{\mskip1.5mu \Varid{sum}\;\Varid{x}\;\Varid{y}\mskip1.5mu\}}) to distinguish
it from type-constructor application (e.g. \ensuremath{\Conid{List}\;\Conid{Int}}). We also define
the type-function \ensuremath{\Varid{sum}} and the value-function \ensuremath{\Varid{append}} of static sequences.\\

\begingroup\par\noindent\advance\leftskip\mathindent\(
\begin{pboxed}\SaveRestoreHook
\column{B}{@{}l@{}}
\column{17}{@{}l@{}}
\column{21}{@{}l@{}}
\column{22}{@{}l@{}}
\column{E}{@{}l@{}}
\fromto{B}{E}{{}\Varid{sum}\mathbin{::}\Conid{Nat}\leadsto\Conid{Nat}\leadsto\Conid{Nat}{}}
\nextline
\fromto{B}{17}{{}\{\mskip1.5mu \Varid{sum}\;\Conid{Zero}\;{}}
\fromto{17}{21}{{}\Varid{m}\mskip1.5mu\}{}}
\fromto{21}{E}{{}\mathrel{=}\Varid{m}{}}
\nextline
\fromto{B}{17}{{}\{\mskip1.5mu \Varid{sum}\;(\Conid{Succ}\;\Varid{n})\;{}}
\fromto{17}{21}{{}\Varid{m}\mskip1.5mu\}{}}
\fromto{21}{E}{{}\mathrel{=}\Conid{Succ}\;\{\mskip1.5mu \Varid{sum}\;\Varid{n}\;\Varid{m}\mskip1.5mu\}{}}
\nextline[\blanklineskip]
\fromto{B}{E}{{}\Varid{append}\mathbin{::}\Conid{Seq}\;\Varid{a}\;\Varid{n}\to \Conid{Seq}\;\Varid{a}\;\Varid{m}\to \Conid{Seq}\;\Varid{a}\;\{\mskip1.5mu \Varid{sum}\;\Varid{n}\;\Varid{m}\mskip1.5mu\}{}}
\nextline
\fromto{B}{22}{{}\Varid{append}\;(\Conid{Nil})\;{}}
\fromto{22}{E}{{}\Varid{ys}\mathrel{=}\Varid{ys}{}}
\nextline
\fromto{B}{22}{{}\Varid{append}\;(\Conid{Cons}\;\Varid{x}\;\Varid{xs})\;{}}
\fromto{22}{E}{{}\Varid{ys}\mathrel{=}\Conid{Cons}\;\Varid{x}\;(\Varid{append}\;\Varid{xs}\;\Varid{ys}){}}
\ColumnHook
\end{pboxed}
\)\par\noindent\endgroup\resethooks
How is a function, like append, whose type includes a type-function 
application, type checked? And, what
role do the type function equations play in the process? Consider the second
equation: \ensuremath{\Varid{append}\;(\Conid{Cons}\;\Varid{x}\;\Varid{xs})\;\Varid{ys}\mathrel{=}\Conid{Cons}\;\Varid{x}\;(\Varid{append}\;\Varid{xs}\;\Varid{ys})}. From the type
signature of \ensuremath{\Varid{append}}, we know \ensuremath{(\Conid{Cons}\;\Varid{x}\;\Varid{xs})\mathbin{::}\Conid{Seq}\;\Varid{a}\;\Varid{n}}, and  \ensuremath{\Varid{ys}\mathbin{::}\Conid{Seq}\;\Varid{a}\;\Varid{m}}. We
need to check that the right-hand-side \ensuremath{\Conid{Cons}\;\Varid{x}\;(\Varid{append}\;\Varid{xs}\;\Varid{ys})} has type 
\ensuremath{\Conid{Seq}\;\Varid{a}\;\{\mskip1.5mu \Varid{sum}\;\Varid{n}\;\Varid{m}\mskip1.5mu\}}. Because \ensuremath{\Conid{Cons}} has an equality qualified type, from
the pattern \ensuremath{(\Conid{Cons}\;\Varid{x}\;\Varid{xs})} we know \ensuremath{(\Varid{x}\mathbin{::}\Varid{a})} and \ensuremath{(\Varid{xs}\mathbin{::}\Conid{Seq}\;\Varid{a}\;\Varid{k})} plus we
get to add the equality \ensuremath{(\Varid{n}\mathrel{=}\Conid{Succ}\;\Varid{k})} to the set of assumptions. From this we
can compute that \ensuremath{(\Varid{append}\;\Varid{xs}\;\Varid{ys}\mathbin{::}\Conid{Seq}\;\Varid{a}\;\{\mskip1.5mu \Varid{sum}\;\Varid{k}\;\Varid{m}\mskip1.5mu\})}. Thus it appears that the
right- hand-side \ensuremath{\Conid{Cons}\;\Varid{x}\;(\Varid{append}\;\Varid{xs}\;\Varid{ys})} actually has type \ensuremath{\Conid{Seq}\;\Varid{a}\;(\Conid{Succ}\;\{\mskip1.5mu \Varid{sum}\;\Varid{k}\;\Varid{m}\mskip1.5mu\})}. To reconcile this apparent difference, the type checking algorithm
unifies the type \ensuremath{\Conid{Seq}\;\Varid{a}\;\{\mskip1.5mu \Varid{sum}\;\Varid{n}\;\Varid{m}\mskip1.5mu\}} with the type \ensuremath{\Conid{Seq}\;\Varid{a}\;(\Conid{Succ}\;\{\mskip1.5mu \Varid{sum}\;\Varid{k}\;\Varid{m}\mskip1.5mu\})}. This
unification succeeds only if the equation \ensuremath{\{\mskip1.5mu \Varid{sum}\;\Varid{n}\;\Varid{m}\mskip1.5mu\}\mathrel{=}\Conid{Succ}\;\{\mskip1.5mu \Varid{sum}\;\Varid{k}\;\Varid{m}\mskip1.5mu\}} holds. Since
this cannot always be immediately determined, the unification algorithm
delays unification, and generates and propagates this equation as a new obligation.
After checking
the rest of the right-hand-side (which propagates no additional obligations)
we must discharge the obligation \ensuremath{\{\mskip1.5mu \Varid{sum}\;\Varid{n}\;\Varid{m}\mskip1.5mu\}\mathrel{=}\Conid{Succ}\;\{\mskip1.5mu \Varid{sum}\;\Varid{k}\;\Varid{m}\mskip1.5mu\}} under the assumption \ensuremath{(\Varid{n}\mathrel{=}\Conid{Succ}\;\Varid{k})}.
We use the assumption by substituting \ensuremath{\Conid{Succ}\;\Varid{k}} for \ensuremath{\Varid{n}}, and then discharge \ensuremath{\{\mskip1.5mu \Varid{sum}\;(\Conid{Succ}\;\Varid{k})\;\Varid{m}\mskip1.5mu\}\mathrel{=}\Conid{Succ}\;\{\mskip1.5mu \Varid{sum}\;\Varid{k}\;\Varid{m}\mskip1.5mu\}}. By applying the second clause in the definition of
\ensuremath{\Varid{sum}} we get \ensuremath{\Conid{Succ}\;\{\mskip1.5mu \Varid{sum}\;\Varid{k}\;\Varid{m}\mskip1.5mu\}\mathrel{=}\Conid{Succ}\;\{\mskip1.5mu \Varid{sum}\;\Varid{k}\;\Varid{m}\mskip1.5mu\}} which is finally discharged.

By delaying the use of both assumptions and type-function equations until
after all other unification is performed, we avoid the problem that sometimes
occurs when type variables are not instantiated in time. For example
without knowing that \ensuremath{\Varid{n}\mathrel{=}\Conid{Succ}\;\Varid{k}} we could not apply the type-function
rule reducing \ensuremath{\Varid{sum}\;\Varid{n}\;\Varid{m}} to \ensuremath{\Conid{Succ}\;\{\mskip1.5mu \Varid{sum}\;\Varid{k}\;\Varid{m}\mskip1.5mu\}}. 

%%%%%%%%%%%%%%%%%%%%%%%%%%%%%%%%%%%%%%%%%%%%%%%%%%%%%%%%%%%%%%%%%%%%%%%%%%%%%%%%

%%%%%%%%%%%%%%%%%%%%%%%%%%%%%%%%%%%%%%%%%%%%%%%%%%%%%%%%%%%%%%%%%%%%%%%%%%%%%%%%
\section{Binomial Heaps}


Binomial Heaps are a collection
type that support the following operations, 
each listed with its time complexity.
\begingroup\par\noindent\advance\leftskip\mathindent\(
\begin{pboxed}\SaveRestoreHook
\column{B}{@{}l@{}}
\column{14}{@{}l@{}}
\column{39}{@{}l@{}}
\column{E}{@{}l@{}}
\fromto{B}{14}{{}\Varid{empty}{}}
\fromto{14}{39}{{}\mathbin{::}\Conid{Heap}{}}
\fromto{39}{E}{{}\mbox{\onelinecomment   $O(1)$}{}}
\nextline
\fromto{B}{14}{{}\Varid{singleton}{}}
\fromto{14}{39}{{}\mathbin{::}\Conid{Elem}\to \Conid{Heap}{}}
\fromto{39}{E}{{}\mbox{\onelinecomment   $O(1)$}{}}
\nextline
\fromto{B}{14}{{}\Varid{insert}{}}
\fromto{14}{39}{{}\mathbin{::}\Conid{Elem}\to \Conid{Heap}\to \Conid{Heap}{}}
\fromto{39}{E}{{}\mbox{\onelinecomment   $O(lg\ n)$}{}}
\nextline
\fromto{B}{14}{{}\Varid{merge}{}}
\fromto{14}{39}{{}\mathbin{::}\Conid{Heap}\to \Conid{Heap}\to \Conid{Heap}{}}
\fromto{39}{E}{{}\mbox{\onelinecomment   $O(lg\ n)$}{}}
\nextline
\fromto{B}{14}{{}\Varid{minElem}{}}
\fromto{14}{39}{{}\mathbin{::}\Conid{Heap}\to \Conid{Maybe}\;\Conid{Elem}{}}
\fromto{39}{E}{{}\mbox{\onelinecomment   $O(lg\ n)$}{}}
\nextline
\fromto{B}{14}{{}\Varid{extractMin}{}}
\fromto{14}{39}{{}\mathbin{::}\Conid{Heap}\to \Conid{Maybe}\;\Conid{Heap}{}}
\fromto{39}{E}{{}\mbox{\onelinecomment   $O(lg\ n)$}{}}
\ColumnHook
\end{pboxed}
\)\par\noindent\endgroup\resethooks
That every operation takes logortihmic (or better) time
is due to a sophisticated set of invariants that structure
the binomial heap. Correctly implementing heap operations 
that maintain these invariants is a non-trivial task.
Any help in this endeavor is greatly appreciated. 
In \Wmega{} we can express almost all the invariants 
and let the type checker enforce them.

The \ensuremath{\Conid{Heap}} type should really be parameterized by the type of its 
elements.  For simplicity of exposition, we just treat \ensuremath{\Conid{Elem}} as 
an abstract type that supports comparison operations.  The 
desired parameterization poses no problems but clutters
the presentation of the static properties.

\hide{\begingroup\par\noindent\advance\leftskip\mathindent\(
\begin{pboxed}\SaveRestoreHook
\column{B}{@{}l@{}}
\column{9}{@{}l@{}}
\column{10}{@{}l@{}}
\column{24}{@{}l@{}}
\column{27}{@{}l@{}}
\column{E}{@{}l@{}}
\fromto{B}{E}{{}\Varid{otherwise}\mathrel{=}\Conid{True}{}}
\nextline[\blanklineskip]
\fromto{B}{E}{{}\Varid{foldr}\;\Varid{c}\;\Varid{n}\;[\mskip1.5mu \mskip1.5mu]\mathrel{=}\Varid{n}{}}
\nextline
\fromto{B}{E}{{}\Varid{foldr}\;\Varid{c}\;\Varid{n}\;(\Varid{x}\mathbin{:}\Varid{xs})\mathrel{=}\Varid{c}\;\Varid{x}\;(\Varid{foldr}\;\Varid{c}\;\Varid{n}\;\Varid{xs}){}}
\nextline[\blanklineskip]
\fromto{B}{24}{{}\Varid{min}\;\Varid{a}\;\Varid{b}\mid \Varid{b}\mathbin{<}\Varid{a}{}}
\fromto{24}{E}{{}\mathrel{=}\Varid{b}{}}
\nextline
\fromto{B}{10}{{}\hsindent{10}{}}
\fromto{10}{24}{{}\mid \Varid{otherwise}{}}
\fromto{24}{E}{{}\mathrel{=}\Varid{a}{}}
\nextline[\blanklineskip]
\fromto{B}{E}{{}\Varid{minBy}\mathbin{::}(\Varid{a}\to \Conid{Elem})\to (\Varid{a}\to \Varid{a}\to \Varid{a}){}}
\nextline
\fromto{B}{E}{{}\Varid{minBy}\;\Varid{key}\;\Varid{a}\;\Varid{b}{}}
\nextline
\fromto{B}{9}{{}\hsindent{9}{}}
\fromto{9}{27}{{}\mid \Varid{key}\;\Varid{b}\mathbin{<}\Varid{key}\;\Varid{a}{}}
\fromto{27}{E}{{}\mathrel{=}\Varid{b}{}}
\nextline
\fromto{B}{9}{{}\hsindent{9}{}}
\fromto{9}{27}{{}\mid \Varid{otherwise}{}}
\fromto{27}{E}{{}\mathrel{=}\Varid{a}{}}
\ColumnHook
\end{pboxed}
\)\par\noindent\endgroup\resethooks
}

%%%%%%%%%%%%%%%%%%%%%%%%%%%%%%%%%%%%%%%%
\subsection{Natural Numbers as Types}

The datatypes we use to encode binomial heaps are indexed by natural
numbers. These parameters encode structural invariants for each
type. Since we do not use dependent types, we must
develop \emph{types} whose structure mirrors that of the naturals.  For
our purposes here, two representations of numbers are useful. The
natural numbers (\ensuremath{\Conid{N1}}) in unary, and the positive natural numbers (\ensuremath{\Conid{N2}})
in binary.

\begingroup\par\noindent\advance\leftskip\mathindent\(
\begin{pboxed}\SaveRestoreHook
\column{B}{@{}l@{}}
\column{36}{@{}l@{}}
\column{E}{@{}l@{}}
\fromto{B}{36}{{}{\bf kind}\;\NN_1\mathrel{=}\Conid{Zero}\mid \Conid{Succ}\;\NN_1{}}
\fromto{36}{E}{{}\mbox{\onelinecomment  $0$ or $n+1$}{}}
\nextline
\fromto{B}{36}{{}{\bf kind}\;\NN_2\mathrel{=}\Conid{One}\mid \Conid{Even}\;\NN_2\mid \Conid{Odd}\;\NN_2{}}
\fromto{36}{E}{{}\mbox{\onelinecomment  $1$ or $2n$ or $2n+1$}{}}
\ColumnHook
\end{pboxed}
\)\par\noindent\endgroup\resethooks
The binary representation is least-significant-bit first.
For example, the type \ensuremath{\Conid{Odd}\;(\Conid{Even}\;(\Conid{Odd}\;\Conid{One}))} stands for the 
number $1101_{2} = 13_{10}$.

We encode only positive naturals in the binary representation.
Instead we could include zero by defining
\ensuremath{{\bf kind}\;\NN_2\mathrel{=}\Conid{Zero2}\mid \Conid{Even}\;\NN_2\mid \Conid{Odd}\;\NN_2}.
The problem with this definition is that it does not rule out 
trailing zeroes.  For example \ensuremath{\Conid{Even}\;\Conid{Zero2}} and \ensuremath{\Conid{Zero2}} both 
stand for the number \ensuremath{\mathrm{0}}.

We define increment and addition type-functions on the binary representation 
of the naturals.
\begingroup\par\noindent\advance\leftskip\mathindent\(
\begin{pboxed}\SaveRestoreHook
\column{B}{@{}l@{}}
\column{18}{@{}l@{}}
\column{E}{@{}l@{}}
\fromto{B}{E}{{}\Varid{inc}\mathbin{::}\NN_2\leadsto\NN_2{}}
\nextline
\fromto{B}{18}{{}\{\mskip1.5mu \Varid{inc}\;\Conid{One}\mskip1.5mu\}{}}
\fromto{18}{E}{{}\mathrel{=}\Conid{Even}\;\Conid{One}{}}
\nextline
\fromto{B}{18}{{}\{\mskip1.5mu \Varid{inc}\;(\Conid{Even}\;\Varid{n})\mskip1.5mu\}{}}
\fromto{18}{E}{{}\mathrel{=}\Conid{Odd}\;\Varid{n}{}}
\nextline
\fromto{B}{18}{{}\{\mskip1.5mu \Varid{inc}\;(\Conid{Odd}\;\Varid{n})\mskip1.5mu\}{}}
\fromto{18}{E}{{}\mathrel{=}\Conid{Even}\;\{\mskip1.5mu \Varid{inc}\;\Varid{n}\mskip1.5mu\}{}}
\ColumnHook
\end{pboxed}
\)\par\noindent\endgroup\resethooks
Addition is defined as a pair of mutually recursive functions: 
\ensuremath{\Varid{add}} (without carry) and \ensuremath{\Varid{addc}} (with carry).
\begingroup\par\noindent\advance\leftskip\mathindent\(
\begin{pboxed}\SaveRestoreHook
\column{B}{@{}l@{}}
\column{7}{@{}l@{}}
\column{14}{@{}l@{}}
\column{15}{@{}l@{}}
\column{18}{@{}l@{}}
\column{19}{@{}l@{}}
\column{25}{@{}l@{}}
\column{26}{@{}l@{}}
\column{30}{@{}l@{}}
\column{31}{@{}l@{}}
\column{38}{@{}l@{}}
\column{39}{@{}l@{}}
\column{45}{@{}l@{}}
\column{46}{@{}l@{}}
\column{E}{@{}l@{}}
\fromto{B}{7}{{}\Varid{add}{}}
\fromto{7}{E}{{}\mathbin{::}\NN_2\leadsto\NN_2\leadsto\NN_2{}}
\nextline
\fromto{B}{18}{{}\{\mskip1.5mu \Varid{add}\;\Conid{One}\;{}}
\fromto{18}{30}{{}\Varid{m}\mskip1.5mu\}{}}
\fromto{30}{E}{{}\mathrel{=}\{\mskip1.5mu \Varid{inc}\;\Varid{m}\mskip1.5mu\}{}}
\nextline
\fromto{B}{18}{{}\{\mskip1.5mu \Varid{add}\;\Varid{n}\;{}}
\fromto{18}{30}{{}\Conid{One}\mskip1.5mu\}{}}
\fromto{30}{E}{{}\mathrel{=}\{\mskip1.5mu \Varid{inc}\;\Varid{n}\mskip1.5mu\}{}}
\nextline
\fromto{B}{14}{{}\{\mskip1.5mu \Varid{add}\;(\Conid{Even}\;{}}
\fromto{14}{18}{{}\Varid{n})\;{}}
\fromto{18}{25}{{}(\Conid{Even}\;{}}
\fromto{25}{30}{{}\Varid{m})\mskip1.5mu\}{}}
\fromto{30}{38}{{}\mathrel{=}\Conid{Even}\;{}}
\fromto{38}{45}{{}\{\mskip1.5mu \Varid{add}\;{}}
\fromto{45}{E}{{}\Varid{n}\;\Varid{m}\mskip1.5mu\}{}}
\nextline
\fromto{B}{14}{{}\{\mskip1.5mu \Varid{add}\;(\Conid{Even}\;{}}
\fromto{14}{18}{{}\Varid{n})\;{}}
\fromto{18}{25}{{}(\Conid{Odd}\;{}}
\fromto{25}{30}{{}\Varid{m})\mskip1.5mu\}{}}
\fromto{30}{38}{{}\mathrel{=}\Conid{Odd}\;{}}
\fromto{38}{45}{{}\{\mskip1.5mu \Varid{add}\;{}}
\fromto{45}{E}{{}\Varid{n}\;\Varid{m}\mskip1.5mu\}{}}
\nextline
\fromto{B}{14}{{}\{\mskip1.5mu \Varid{add}\;(\Conid{Odd}\;{}}
\fromto{14}{18}{{}\Varid{n})\;{}}
\fromto{18}{25}{{}(\Conid{Even}\;{}}
\fromto{25}{30}{{}\Varid{m})\mskip1.5mu\}{}}
\fromto{30}{38}{{}\mathrel{=}\Conid{Odd}\;{}}
\fromto{38}{45}{{}\{\mskip1.5mu \Varid{add}\;{}}
\fromto{45}{E}{{}\Varid{n}\;\Varid{m}\mskip1.5mu\}{}}
\nextline
\fromto{B}{14}{{}\{\mskip1.5mu \Varid{add}\;(\Conid{Odd}\;{}}
\fromto{14}{18}{{}\Varid{n})\;{}}
\fromto{18}{25}{{}(\Conid{Odd}\;{}}
\fromto{25}{30}{{}\Varid{m})\mskip1.5mu\}{}}
\fromto{30}{38}{{}\mathrel{=}\Conid{Even}\;{}}
\fromto{38}{45}{{}\{\mskip1.5mu \Varid{addc}\;{}}
\fromto{45}{E}{{}\Varid{n}\;\Varid{m}\mskip1.5mu\}{}}
\nextline[\blanklineskip]
\fromto{B}{E}{{}\Varid{addc}\mathbin{::}\NN_2\leadsto\NN_2\leadsto\NN_2{}}
\nextline
\fromto{B}{19}{{}\{\mskip1.5mu \Varid{addc}\;\Conid{One}\;{}}
\fromto{19}{31}{{}\Varid{m}\mskip1.5mu\}{}}
\fromto{31}{E}{{}\mathrel{=}\{\mskip1.5mu \Varid{inc}\;\{\mskip1.5mu \Varid{inc}\;\Varid{m}\mskip1.5mu\}\mskip1.5mu\}{}}
\nextline
\fromto{B}{19}{{}\{\mskip1.5mu \Varid{addc}\;\Varid{n}\;{}}
\fromto{19}{31}{{}\Conid{One}\mskip1.5mu\}{}}
\fromto{31}{E}{{}\mathrel{=}\{\mskip1.5mu \Varid{inc}\;\{\mskip1.5mu \Varid{inc}\;\Varid{n}\mskip1.5mu\}\mskip1.5mu\}{}}
\nextline
\fromto{B}{15}{{}\{\mskip1.5mu \Varid{addc}\;(\Conid{Even}\;{}}
\fromto{15}{19}{{}\Varid{n})\;{}}
\fromto{19}{26}{{}(\Conid{Even}\;{}}
\fromto{26}{31}{{}\Varid{m})\mskip1.5mu\}{}}
\fromto{31}{39}{{}\mathrel{=}\Conid{Odd}\;{}}
\fromto{39}{46}{{}\{\mskip1.5mu \Varid{add}\;{}}
\fromto{46}{E}{{}\Varid{n}\;\Varid{m}\mskip1.5mu\}{}}
\nextline
\fromto{B}{15}{{}\{\mskip1.5mu \Varid{addc}\;(\Conid{Even}\;{}}
\fromto{15}{19}{{}\Varid{n})\;{}}
\fromto{19}{26}{{}(\Conid{Odd}\;{}}
\fromto{26}{31}{{}\Varid{m})\mskip1.5mu\}{}}
\fromto{31}{39}{{}\mathrel{=}\Conid{Even}\;{}}
\fromto{39}{46}{{}\{\mskip1.5mu \Varid{addc}\;{}}
\fromto{46}{E}{{}\Varid{n}\;\Varid{m}\mskip1.5mu\}{}}
\nextline
\fromto{B}{15}{{}\{\mskip1.5mu \Varid{addc}\;(\Conid{Odd}\;{}}
\fromto{15}{19}{{}\Varid{n})\;{}}
\fromto{19}{26}{{}(\Conid{Even}\;{}}
\fromto{26}{31}{{}\Varid{m})\mskip1.5mu\}{}}
\fromto{31}{39}{{}\mathrel{=}\Conid{Even}\;{}}
\fromto{39}{46}{{}\{\mskip1.5mu \Varid{addc}\;{}}
\fromto{46}{E}{{}\Varid{n}\;\Varid{m}\mskip1.5mu\}{}}
\nextline
\fromto{B}{15}{{}\{\mskip1.5mu \Varid{addc}\;(\Conid{Odd}\;{}}
\fromto{15}{19}{{}\Varid{n})\;{}}
\fromto{19}{26}{{}(\Conid{Odd}\;{}}
\fromto{26}{31}{{}\Varid{m})\mskip1.5mu\}{}}
\fromto{31}{39}{{}\mathrel{=}\Conid{Odd}\;{}}
\fromto{39}{46}{{}\{\mskip1.5mu \Varid{addc}\;{}}
\fromto{46}{E}{{}\Varid{n}\;\Varid{m}\mskip1.5mu\}{}}
\ColumnHook
\end{pboxed}
\)\par\noindent\endgroup\resethooks
\hide{
For clarity, in the remainder of the paper we will use the usual 
mathematical notation for the naturals.  However, the reader should 
remember that all types of kind \ensuremath{\NN_1} or \ensuremath{\NN_2} are built up from the 
constructors presented here.

%%format Zero = "0"
%%format One = "1"
%%format Succ k = "{" k "+1}"
%%format Even n = "2 \cdot " n
%%format Odd n = "2 \cdot " n "+ 1"
}

%%%%%%%%%%%%%%%%%%%%%%%%%%%%%%%%%%%%%%%%
\subsection{Binomial Trees}


Binomial heaps are composed of simpler structures 
called binomial trees, which we explain first. 
The datatype \ensuremath{T_{\Varid{k}}} of binomial trees is based on the arithmetic identity
\[2^k \quad = \quad 1 + (2^{k-1} + 2^{k-2} + \cdots + 2^0)
  \quad \textrm{where } k \geq 0 \]
We also make use of an auxillary 
datatype \ensuremath{F_{\Varid{k}}} of forests (of tress) stored in descending order: $T_{k-1}, T_{k-2},\ldots, T_0$.
\begingroup\par\noindent\advance\leftskip\mathindent\(
\begin{pboxed}\SaveRestoreHook
\column{B}{@{}l@{}}
\column{17}{@{}l@{}}
\column{26}{@{}l@{}}
\column{45}{@{}l@{}}
\column{E}{@{}l@{}}
\fromto{B}{17}{{}\mathbf{data}\;T_{\Varid{k}}{}}
\fromto{17}{26}{{}\mathrel{=}{}}
\fromto{26}{E}{{}\Conid{Tree}\;\Conid{Elem}\;F_{\Varid{k}}{}}
\nextline
\fromto{B}{17}{{}\mathbf{data}\;F_{\Varid{k}}{}}
\fromto{17}{26}{{}\mathrel{=}{}}
\fromto{26}{45}{{}\Conid{FNil}\;{}}
\fromto{45}{E}{{}\mathbf{where}\;\Varid{k}\mathrel{=}\Conid{Zero}{}}
\nextline
\fromto{17}{26}{{}\mid \exists\;\Varid{j}. {}}
\fromto{26}{45}{{}\Conid{FCons}\;T_{\Varid{j}}\;F_{\Varid{j}}\;{}}
\fromto{45}{E}{{}\mathbf{where}\;\Varid{k}\mathrel{=}\Conid{Succ}\;\Varid{j}{}}
\ColumnHook
\end{pboxed}
\)\par\noindent\endgroup\resethooks
This definition defines type constructors \ensuremath{T_{}} and \ensuremath{F_{}} of kind \ensuremath{\NN_1\leadsto\mathbin{*}}.
The \ensuremath{\mathbf{where}} clauses in this definition effectively give the data 
constructors the following types:\begingroup\par\noindent\advance\leftskip\mathindent\(
\begin{pboxed}\SaveRestoreHook
\column{B}{@{}l@{}}
\column{9}{@{}l@{}}
\column{E}{@{}l@{}}
\fromto{B}{9}{{}\Conid{Tree}{}}
\fromto{9}{E}{{}\mathbin{::}\Conid{Elem}\to F_{\Varid{k}}\to T_{\Varid{k}}{}}
\nextline
\fromto{B}{9}{{}\Conid{FNil}{}}
\fromto{9}{E}{{}\mathbin{::}F_{\Conid{Zero}}{}}
\nextline
\fromto{B}{9}{{}\Conid{FCons}{}}
\fromto{9}{E}{{}\mathbin{::}T_{\Varid{k}}\to F_{\Varid{k}}\to F_{\Conid{Succ}\;\Varid{k}}{}}
\ColumnHook
\end{pboxed}
\)\par\noindent\endgroup\resethooks

Trees of type \ensuremath{T_{\Varid{k}}} contain exactly $2^k$ elements.
Forests of type \ensuremath{F_{\Varid{k}}} contain exactly $2^k-1$ elements.
Convince yourself of this by an inductive argument 
based on the types of the data constructors.
Note that the boundary condition $2^0=1$ holds.
\begingroup\par\noindent\advance\leftskip\mathindent\(
\begin{pboxed}\SaveRestoreHook
\column{B}{@{}l@{}}
\column{E}{@{}l@{}}
\fromto{B}{E}{{}\Varid{singletonT}\mathbin{::}\Conid{Elem}\to T_{\Conid{Zero}}{}}
\nextline
\fromto{B}{E}{{}\Varid{singletonT}\;\Varid{a}\mathrel{=}\Conid{Tree}\;\Varid{a}\;\Conid{FNil}{}}
\ColumnHook
\end{pboxed}
\)\par\noindent\endgroup\resethooks
Before we move to the inductive step we need to discuss
the heap invariant.
Binomial heaps satisfy two kinds of invariants.  The first kind of
invariant is static and structural, having to do with the
relative shapes and sizes of the trees in the heap, and is captured in
the type indexs of the data structures. This is what
the inductive proof is about. The second is the \emph{heap
invariant}. Binomial heaps are composed of Binomial trees. The heap
invariant states the every node in a binomial tree in a binomial heap has
a value no greater than that of any its descendents. The heap invariant
is captured by making the construction of binomial trees abstract.
Singleton trees vacuously meet the heap invariant (they have only one
element) and are made by using the abstract constructor (\ensuremath{\Varid{singletonT}}).
Larger trees are constructed by an abstract constant-time merge function
(\ensuremath{\Varid{mergeT}})
corresponding to the identity $2^k + 2^k = 2^{k+1}$. We construct
this function in two steps. First we define \ensuremath{\Varid{mergeT'}}:

\begingroup\par\noindent\advance\leftskip\mathindent\(
\begin{pboxed}\SaveRestoreHook
\column{B}{@{}l@{}}
\column{E}{@{}l@{}}
\fromto{B}{E}{{}\Varid{mergeT'}\mathbin{::}T_{\Varid{k}}\to T_{\Varid{k}}\to T_{\Conid{Succ}\;\Varid{k}}{}}
\nextline
\fromto{B}{E}{{}\Varid{mergeT'}\;(\Conid{Tree}\;\Varid{e}\;\Varid{f})\;\Varid{t}\mathrel{=}\Conid{Tree}\;\Varid{e}\;(\Conid{FCons}\;\Varid{t}\;\Varid{f}){}}
\ColumnHook
\end{pboxed}
\)\par\noindent\endgroup\resethooks
The top element of \ensuremath{\Varid{mergeT'}\;t_1\;t_2} is the top element of \ensuremath{t_1}. 
A simple wrapper function ensures that merging preserves the heap property
by swapping \ensuremath{t_1} with \ensuremath{t_2} when necessary.
\begingroup\par\noindent\advance\leftskip\mathindent\(
\begin{pboxed}\SaveRestoreHook
\column{B}{@{}l@{}}
\column{16}{@{}l@{}}
\column{36}{@{}l@{}}
\column{E}{@{}l@{}}
\fromto{B}{E}{{}\Varid{top}\mathbin{::}T_{\Varid{k}}\to \Conid{Elem}{}}
\nextline
\fromto{B}{E}{{}\Varid{top}\;(\Conid{Tree}\;\Varid{i}\;\anonymous )\mathrel{=}\Varid{i}{}}
\nextline[\blanklineskip]
\fromto{B}{E}{{}\Varid{mergeT}\mathbin{::}T_{\Varid{k}}\to T_{\Varid{k}}\to T_{\Conid{Succ}\;\Varid{k}}{}}
\nextline
\fromto{B}{16}{{}\Varid{mergeT}\;t_1\;t_2{}}
\fromto{16}{36}{{}\mid \Varid{top}\;t_1\leq \Varid{top}\;t_2{}}
\fromto{36}{E}{{}\mathrel{=}\Varid{mergeT'}\;t_1\;t_2{}}
\nextline
\fromto{16}{36}{{}\mid \Varid{otherwise}{}}
\fromto{36}{E}{{}\mathrel{=}\Varid{mergeT'}\;t_2\;t_1{}}
\ColumnHook
\end{pboxed}
\)\par\noindent\endgroup\resethooks
So long as we only use \ensuremath{\Varid{singletonT}} and \ensuremath{\Varid{mergeT}} to build trees, we will
preserve the heap invariant. The function \ensuremath{\Varid{mergeT}} also preserves the
structural invariant, which is manifest in its type. This completes the
inductive step of the structural proof. You can't
construct well typed binomial trees, that don't
meet the structural invariant. There is one other function that
needs to break the binomial tree abstraction (used to implement
\ensuremath{\Varid{extractMin}}) but it also preserves the heap invariant.


%%%%%%%%%%%%%%%%%%%%%%%%%%%%%%%%%%%%%%%%
\subsection{Binomial Heaps}

As we develop the arithmetic analogy, the binary representation of 
natural numbers is the basis for \emph{binomial heaps}.  Since there 
is a unique binary representation for any natural number $n$, there 
is a unique set of (distinct) binomial tree sizes that together contain 
$n$ elements.  For example, we can contain 13 elements in a collection 
of binomial heaps of types \ensuremath{T_{\mathrm{0}}}, \ensuremath{T_{\mathrm{2}}}, and \ensuremath{T_{\mathrm{3}}} because 
$13 = 2^0 + 2^2 + 2^3$.  In this way, the structure of binomial heaps 
mirrors that of binary numbers.  

Here is a first attempt at defining the \ensuremath{\Conid{Heap}} datatype.
\begingroup\par\noindent\advance\leftskip\mathindent\(
\begin{pboxed}\SaveRestoreHook
\column{B}{@{}l@{}}
\column{5}{@{}l@{}}
\column{14}{@{}l@{}}
\column{20}{@{}l@{}}
\column{29}{@{}l@{}}
\column{39}{@{}l@{}}
\column{E}{@{}l@{}}
\fromto{B}{E}{{}\mathbf{data}\;\Conid{Heap}\;\Varid{n}{}}
\nextline
\fromto{B}{5}{{}\hsindent{5}{}}
\fromto{5}{14}{{}\mathrel{=}{}}
\fromto{14}{20}{{}\Conid{Last}\;{}}
\fromto{20}{39}{{}T_{\mathbin{?}}\;{}}
\fromto{39}{E}{{}\mathbf{where}\;\Varid{n}\mathrel{=}\Conid{One}{}}
\nextline
\fromto{B}{5}{{}\hsindent{5}{}}
\fromto{5}{14}{{}\mid \exists\;\Varid{m}. {}}
\fromto{14}{29}{{}\Conid{Skip}\;{}}
\fromto{29}{39}{{}(\Conid{Heap}\;\Varid{n})\;{}}
\fromto{39}{E}{{}\mathbf{where}\;\Varid{n}\mathrel{=}\Conid{Even}\;\Varid{m}{}}
\nextline
\fromto{B}{5}{{}\hsindent{5}{}}
\fromto{5}{14}{{}\mid \exists\;\Varid{m}. {}}
\fromto{14}{20}{{}\Conid{Cons}\;{}}
\fromto{20}{29}{{}T_{\mathbin{?}}\;{}}
\fromto{29}{39}{{}(\Conid{Heap}\;\Varid{n})\;{}}
\fromto{39}{E}{{}\mathbf{where}\;\Varid{n}\mathrel{=}\Conid{Odd}\;\Varid{m}{}}
\ColumnHook
\end{pboxed}
\)\par\noindent\endgroup\resethooks
What's missing is an accumulating parameter to specify the orders 
of binomial trees.  We arrive at the following definition.
\begingroup\par\noindent\advance\leftskip\mathindent\(
\begin{pboxed}\SaveRestoreHook
\column{B}{@{}l@{}}
\column{9}{@{}l@{}}
\column{18}{@{}l@{}}
\column{30}{@{}l@{}}
\column{52}{@{}l@{}}
\column{E}{@{}l@{}}
\fromto{B}{E}{{}\mathbf{data}\;\Conid{SubHeap}\;\Varid{n}\;\Varid{k}{}}
\nextline
\fromto{B}{9}{{}\hsindent{9}{}}
\fromto{9}{18}{{}\mathrel{=}{}}
\fromto{18}{52}{{}\Conid{Last}\;T_{\Varid{k}}\;{}}
\fromto{52}{E}{{}\mathbf{where}\;\Varid{n}\mathrel{=}\Conid{One}{}}
\nextline
\fromto{B}{9}{{}\hsindent{9}{}}
\fromto{9}{18}{{}\mid \exists\;\Varid{m}. {}}
\fromto{18}{30}{{}\Conid{Skip}\;{}}
\fromto{30}{52}{{}(\Conid{SubHeap}\;\Varid{m}\;(\Conid{Succ}\;\Varid{k}))\;{}}
\fromto{52}{E}{{}\mathbf{where}\;\Varid{n}\mathrel{=}\Conid{Even}\;\Varid{m}{}}
\nextline
\fromto{B}{9}{{}\hsindent{9}{}}
\fromto{9}{18}{{}\mid \exists\;\Varid{m}. {}}
\fromto{18}{30}{{}\Conid{Cons}\;T_{\Varid{k}}\;{}}
\fromto{30}{52}{{}(\Conid{SubHeap}\;\Varid{m}\;(\Conid{Succ}\;\Varid{k}))\;{}}
\fromto{52}{E}{{}\mathbf{where}\;\Varid{n}\mathrel{=}\Conid{Odd}\;\Varid{m}{}}
\nextline[\blanklineskip]
\fromto{B}{E}{{}\mathbf{data}\;\Conid{Heap}\mathrel{=}\exists\;\Varid{n}. \Conid{Heap}\;(\Conid{SubHeap}\;\Varid{n}\;\Conid{Zero})\mid \Conid{Empty}{}}
\ColumnHook
\end{pboxed}
\)\par\noindent\endgroup\resethooks
A wrapper datatype \ensuremath{\Conid{Heap}} initializes the accumulating parameter at \ensuremath{\Conid{Zero}}.  
The wrapper type also hides the size parameter by binding it 
existentially.  This shields users of the heap library from the 
details of its implementation.

The type constructor \ensuremath{\Conid{SubHeap}} has kind \ensuremath{\NN_2\leadsto\NN_1\leadsto\mathbin{*}}.  We use the 
unary representation to count off the orders of the binomial trees in a heap, 
and the binary representation to determine the shape of the subheap.
Any \ensuremath{\Conid{SubHeap}\;\Varid{n}\;\Varid{k}} value contains $n * 2^k$ elements.  
By corollary, a \ensuremath{\Conid{Heap}} value with hidden existential parameter $n$ 
contains $n$ elements.  Figure~\ref{fig:heap} shows an example heap.

\begin{figure*}
\begin{tabular}{lc}
\parbox{2.5in}{
{\small\begingroup\par\noindent\advance\leftskip\mathindent\(
\begin{pboxed}\SaveRestoreHook
\column{B}{@{}l@{}}
\column{5}{@{}l@{}}
\column{6}{@{}l@{}}
\column{11}{@{}l@{}}
\column{19}{@{}l@{}}
\column{E}{@{}l@{}}
\fromto{B}{11}{{}\Conid{SubHeap}\;{}}
\fromto{11}{E}{{}({}}
\nextline
\fromto{B}{5}{{}\hsindent{5}{}}
\fromto{5}{11}{{}\Conid{Cons}\;{}}
\fromto{11}{E}{{}(\Conid{Tree}\;\mathrm{1}\;\Conid{FNil})\;({}}
\nextline
\fromto{B}{5}{{}\hsindent{5}{}}
\fromto{5}{11}{{}\Conid{Skip}\;{}}
\fromto{11}{E}{{}({}}
\nextline
\fromto{B}{5}{{}\hsindent{5}{}}
\fromto{5}{11}{{}\Conid{Last}\;{}}
\fromto{11}{E}{{}(\Conid{Tree}\;\mathrm{2}{}}
\nextline
\fromto{11}{19}{{}(\Conid{FCons}\;{}}
\fromto{19}{E}{{}(\Conid{Tree}\;\mathrm{4}{}}
\nextline
\fromto{19}{E}{{}(\Conid{FCons}\;(\Conid{Tree}\;\mathrm{5}\;\Conid{FNil}){}}
\nextline
\fromto{19}{E}{{}\Conid{FNil})){}}
\nextline
\fromto{11}{19}{{}(\Conid{FCons}\;{}}
\fromto{19}{E}{{}(\Conid{Tree}\;\mathrm{3}\;\Conid{FNil})\;\Conid{FNil}))){}}
\nextline
\fromto{5}{6}{{}\hsindent{1}{}}
\fromto{6}{E}{{}))){}}
\ColumnHook
\end{pboxed}
\)\par\noindent\endgroup\resethooks
}}
&
\fig{1in}{fig.1}
\\
\end{tabular}
\caption{A binomial heap of size $5_{10} = 101_{2}$.
It contains several sub-heaps. The \ensuremath{\Conid{Cons}} heap has type \ensuremath{\Conid{SubHeap}\;(\Conid{Odd}\;(\Conid{Even}\;\Conid{One}))\;\Conid{Zero}},
the \ensuremath{\Conid{Skip}} heap has type \ensuremath{\Conid{SubHeap}\;(\Conid{Even}\;\Conid{One})\;(\Conid{Succ}\;\Conid{Zero})}, and the
\ensuremath{\Conid{Last}} heap has type \ensuremath{\Conid{SubHeap}\;\Conid{One}\;(\Conid{Succ}\;(\Conid{Succ}\;\Conid{Zero}))}. Translating
the binary and unary representations in the types to decimal, we
see \ensuremath{\Conid{Cons}} with type \ensuremath{\Conid{Subheap}\;\mathrm{5}\;\mathrm{0}}, \ensuremath{\Conid{Skip}} with type \ensuremath{\Conid{Subheap}\;\mathrm{2}\;\mathrm{1}},
and \ensuremath{\Conid{Last}} with type \ensuremath{\Conid{Subheap}\;\mathrm{1}\;\mathrm{2}}.  Check for yourself that a heap
with type \ensuremath{\Conid{Subheap}\;\Varid{n}\;\Varid{k}} has $n*2^k$ elements.
}

\label{fig:heap}
\end{figure*}

\subsubsection{Primitive Heaps}
Here are primitive heaps containing 0 and 1 elements.
\begingroup\par\noindent\advance\leftskip\mathindent\(
\begin{pboxed}\SaveRestoreHook
\column{B}{@{}l@{}}
\column{E}{@{}l@{}}
\fromto{B}{E}{{}\Varid{empty}\mathbin{::}\Conid{Heap}{}}
\nextline
\fromto{B}{E}{{}\Varid{empty}\mathrel{=}\Conid{Empty}{}}
\nextline[\blanklineskip]
\fromto{B}{E}{{}\Varid{singleton}\mathbin{::}\Conid{Elem}\to \Conid{Heap}{}}
\nextline
\fromto{B}{E}{{}\Varid{singleton}\;\Varid{a}\mathrel{=}\Conid{Heap}\;(\Conid{Last}\;(\Varid{singletonT}\;\Varid{a})){}}
\ColumnHook
\end{pboxed}
\)\par\noindent\endgroup\resethooks
\subsubsection{Insertion}

Inserting a binomial tree into a binomial (sub-) heap corresponds to 
incrementing a binary number.  The tree to be inserted corresponds to 
the least significant ``carry'' bit that is being added on.  This is why 
the \ensuremath{\Varid{k}} parameters of the sub-heap and the tree have to match.
We encourage the reader to compare this definition with the definition 
of \ensuremath{\Varid{inc}}.
\begingroup\par\noindent\advance\leftskip\mathindent\(
\begin{pboxed}\SaveRestoreHook
\column{B}{@{}l@{}}
\column{12}{@{}l@{}}
\column{21}{@{}l@{}}
\column{25}{@{}l@{}}
\column{E}{@{}l@{}}
\fromto{B}{12}{{}ins{}}
\fromto{12}{E}{{}\mathbin{::}T_{\Varid{k}}\to \Conid{SubHeap}\;\Varid{n}\;\Varid{k}\to \Conid{SubHeap}\;\{\mskip1.5mu \Varid{inc}\;\Varid{n}\mskip1.5mu\}\;\Varid{k}{}}
\nextline
\fromto{B}{25}{{}ins\;\Varid{t}\;(\Conid{Last}\;\Varid{x}){}}
\fromto{25}{E}{{}\mathrel{=}\Conid{Skip}\;(\Conid{Last}\;(\Varid{mergeT}\;\Varid{x}\;\Varid{t})){}}
\nextline
\fromto{B}{21}{{}ins\;\Varid{t}\;(\Conid{Skip}\;{}}
\fromto{21}{25}{{}\Varid{h}){}}
\fromto{25}{E}{{}\mathrel{=}\Conid{Cons}\;\Varid{t}\;\Varid{h}{}}
\nextline
\fromto{B}{21}{{}ins\;\Varid{t}\;(\Conid{Cons}\;\Varid{x}\;{}}
\fromto{21}{25}{{}\Varid{h}){}}
\fromto{25}{E}{{}\mathrel{=}\Conid{Skip}\;(ins\;(\Varid{mergeT}\;\Varid{x}\;\Varid{t})\;\Varid{h}){}}
\ColumnHook
\end{pboxed}
\)\par\noindent\endgroup\resethooks
At top level, we call a wrapper function that inserts a single 
element into a \ensuremath{\Conid{Heap}} as a tree of size $2^0$.
\begingroup\par\noindent\advance\leftskip\mathindent\(
\begin{pboxed}\SaveRestoreHook
\column{B}{@{}l@{}}
\column{21}{@{}l@{}}
\column{E}{@{}l@{}}
\fromto{B}{E}{{}\Varid{insert}\mathbin{::}\Conid{Elem}\to \Conid{Heap}\to \Conid{Heap}{}}
\nextline
\fromto{B}{21}{{}\Varid{insert}\;\Varid{e}\;\Conid{Empty}{}}
\fromto{21}{E}{{}\mathrel{=}\Varid{singleton}\;\Varid{e}{}}
\nextline
\fromto{B}{21}{{}\Varid{insert}\;\Varid{e}\;(\Conid{Heap}\;\Varid{h}){}}
\fromto{21}{E}{{}\mathrel{=}\Conid{Heap}\;(ins\;(\Varid{singletonT}\;\Varid{e})\;\Varid{h}){}}
\ColumnHook
\end{pboxed}
\)\par\noindent\endgroup\resethooks
We will also need an insertion function corresponding to an 
increment-by-two operation.
\begingroup\par\noindent\advance\leftskip\mathindent\(
\begin{pboxed}\SaveRestoreHook
\column{B}{@{}l@{}}
\column{12}{@{}l@{}}
\column{22}{@{}l@{}}
\column{26}{@{}l@{}}
\column{E}{@{}l@{}}
\fromto{B}{12}{{}ins2{}}
\fromto{12}{E}{{}\mathbin{::}T_{\Conid{Succ}\;\Varid{k}}\to \Conid{SubHeap}\;\Varid{n}\;\Varid{k}\to \Conid{SubHeap}\;\{\mskip1.5mu \Varid{inc}\;\{\mskip1.5mu \Varid{inc}\;\Varid{n}\mskip1.5mu\}\mskip1.5mu\}\;\Varid{k}{}}
\nextline
\fromto{B}{26}{{}ins2\;\Varid{t}\;(\Conid{Last}\;\Varid{x}){}}
\fromto{26}{E}{{}\mathrel{=}\Conid{Cons}\;\Varid{x}\;(\Conid{Last}\;\Varid{t}){}}
\nextline
\fromto{B}{22}{{}ins2\;\Varid{t}\;(\Conid{Skip}\;{}}
\fromto{22}{26}{{}\Varid{h}){}}
\fromto{26}{E}{{}\mathrel{=}\Conid{Skip}\;(ins\;\Varid{t}\;\Varid{h}){}}
\nextline
\fromto{B}{22}{{}ins2\;\Varid{t}\;(\Conid{Cons}\;\Varid{x}\;{}}
\fromto{22}{26}{{}\Varid{h}){}}
\fromto{26}{E}{{}\mathrel{=}\Conid{Cons}\;\Varid{x}\;(ins\;\Varid{t}\;\Varid{h}){}}
\ColumnHook
\end{pboxed}
\)\par\noindent\endgroup\resethooks
\subsubsection{Merging}
Adding two binary numbers corresponds to merging two binomial heaps.
Since addition over binary numbers, is defined by mutually 
recursive type functions, we define two mutually recursive 
merge functions, one with an extra binomial tree argument 
(corresponding to a carry bit) and one without.
We encourage the reader to compare these definitions with the definitions 
of \ensuremath{\Varid{add}} and \ensuremath{\Varid{addc}}.
%% shorten the name of the merge functions so they will print properly
\begingroup\par\noindent\advance\leftskip\mathindent\(
\begin{pboxed}\SaveRestoreHook
\column{B}{@{}l@{}}
\column{9}{@{}l@{}}
\column{16}{@{}l@{}}
\column{19}{@{}l@{}}
\column{23}{@{}l@{}}
\column{30}{@{}l@{}}
\column{33}{@{}l@{}}
\column{37}{@{}l@{}}
\column{47}{@{}l@{}}
\column{67}{@{}l@{}}
\column{E}{@{}l@{}}
\fromto{B}{E}{{}\Varid{mrg}\mathbin{::}\Conid{SubHeap}\;\Varid{n}\;\Varid{k}\to \Conid{SubHeap}\;\Varid{m}\;\Varid{k}\to \Conid{SubHeap}\;\{\mskip1.5mu \Varid{add}\;\Varid{n}\;\Varid{m}\mskip1.5mu\}\;\Varid{k}{}}
\nextline
\fromto{B}{9}{{}\Varid{mrg}\;{}}
\fromto{9}{16}{{}(\Conid{Last}\;{}}
\fromto{16}{23}{{}\Varid{x})\;{}}
\fromto{23}{37}{{}\Varid{m}{}}
\fromto{37}{E}{{}\mathrel{=}ins\;\Varid{x}\;\Varid{m}{}}
\nextline
\fromto{B}{9}{{}\Varid{mrg}\;{}}
\fromto{9}{23}{{}\Varid{n}\;{}}
\fromto{23}{30}{{}(\Conid{Last}\;{}}
\fromto{30}{37}{{}\Varid{y}){}}
\fromto{37}{E}{{}\mathrel{=}ins\;\Varid{y}\;\Varid{n}{}}
\nextline
\fromto{B}{9}{{}\Varid{mrg}\;{}}
\fromto{9}{19}{{}(\Conid{Skip}\;{}}
\fromto{19}{23}{{}\Varid{n})\;{}}
\fromto{23}{33}{{}(\Conid{Skip}\;{}}
\fromto{33}{37}{{}\Varid{m}){}}
\fromto{37}{47}{{}\mathrel{=}\Conid{Skip}\;{}}
\fromto{47}{67}{{}(\Varid{mrg}\;{}}
\fromto{67}{E}{{}\Varid{n}\;\Varid{m}){}}
\nextline
\fromto{B}{9}{{}\Varid{mrg}\;{}}
\fromto{9}{19}{{}(\Conid{Skip}\;{}}
\fromto{19}{23}{{}\Varid{n})\;{}}
\fromto{23}{30}{{}(\Conid{Cons}\;{}}
\fromto{30}{33}{{}\Varid{y}\;{}}
\fromto{33}{37}{{}\Varid{m}){}}
\fromto{37}{47}{{}\mathrel{=}\Conid{Cons}\;\Varid{y}\;{}}
\fromto{47}{67}{{}(\Varid{mrg}\;{}}
\fromto{67}{E}{{}\Varid{n}\;\Varid{m}){}}
\nextline
\fromto{B}{9}{{}\Varid{mrg}\;{}}
\fromto{9}{16}{{}(\Conid{Cons}\;{}}
\fromto{16}{19}{{}\Varid{x}\;{}}
\fromto{19}{23}{{}\Varid{n})\;{}}
\fromto{23}{33}{{}(\Conid{Skip}\;{}}
\fromto{33}{37}{{}\Varid{m}){}}
\fromto{37}{47}{{}\mathrel{=}\Conid{Cons}\;\Varid{x}\;{}}
\fromto{47}{67}{{}(\Varid{mrg}\;{}}
\fromto{67}{E}{{}\Varid{n}\;\Varid{m}){}}
\nextline
\fromto{B}{9}{{}\Varid{mrg}\;{}}
\fromto{9}{16}{{}(\Conid{Cons}\;{}}
\fromto{16}{19}{{}\Varid{x}\;{}}
\fromto{19}{23}{{}\Varid{n})\;{}}
\fromto{23}{30}{{}(\Conid{Cons}\;{}}
\fromto{30}{33}{{}\Varid{y}\;{}}
\fromto{33}{37}{{}\Varid{m}){}}
\fromto{37}{47}{{}\mathrel{=}\Conid{Skip}\;{}}
\fromto{47}{67}{{}(\Varid{mrgc}\;(\Varid{mergeT}\;\Varid{x}\;\Varid{y})\;{}}
\fromto{67}{E}{{}\Varid{n}\;\Varid{m}){}}
\nextline[\blanklineskip]
\fromto{B}{E}{{}\Varid{mrgc}\mathbin{::}T_{\Varid{k}}\to \Conid{SubHeap}\;\Varid{n}\;\Varid{k}\to \Conid{SubHeap}\;\Varid{m}\;\Varid{k}\to \Conid{SubHeap}\;\{\mskip1.5mu \Varid{addc}\;\Varid{n}\;\Varid{m}\mskip1.5mu\}\;\Varid{k}{}}
\nextline
\fromto{B}{16}{{}\Varid{mrgc}\;\Varid{x}\;(\Conid{Last}\;{}}
\fromto{16}{23}{{}\Varid{y})\;{}}
\fromto{23}{37}{{}\Varid{m}{}}
\fromto{37}{E}{{}\mathrel{=}ins2\;(\Varid{mergeT}\;\Varid{x}\;\Varid{y})\;\Varid{m}{}}
\nextline
\fromto{B}{23}{{}\Varid{mrgc}\;\Varid{x}\;\Varid{n}\;{}}
\fromto{23}{30}{{}(\Conid{Last}\;{}}
\fromto{30}{37}{{}\Varid{z}){}}
\fromto{37}{E}{{}\mathrel{=}ins2\;(\Varid{mergeT}\;\Varid{x}\;\Varid{z})\;\Varid{n}{}}
\nextline
\fromto{B}{19}{{}\Varid{mrgc}\;\Varid{x}\;(\Conid{Skip}\;{}}
\fromto{19}{23}{{}\Varid{n})\;{}}
\fromto{23}{33}{{}(\Conid{Skip}\;{}}
\fromto{33}{37}{{}\Varid{m}){}}
\fromto{37}{47}{{}\mathrel{=}\Conid{Cons}\;\Varid{x}\;{}}
\fromto{47}{67}{{}(\Varid{mrg}\;{}}
\fromto{67}{E}{{}\Varid{n}\;\Varid{m}){}}
\nextline
\fromto{B}{19}{{}\Varid{mrgc}\;\Varid{x}\;(\Conid{Skip}\;{}}
\fromto{19}{23}{{}\Varid{n})\;{}}
\fromto{23}{30}{{}(\Conid{Cons}\;{}}
\fromto{30}{33}{{}\Varid{z}\;{}}
\fromto{33}{37}{{}\Varid{m}){}}
\fromto{37}{47}{{}\mathrel{=}\Conid{Skip}\;{}}
\fromto{47}{67}{{}(\Varid{mrgc}\;(\Varid{mergeT}\;\Varid{x}\;\Varid{z})\;{}}
\fromto{67}{E}{{}\Varid{n}\;\Varid{m}){}}
\nextline
\fromto{B}{16}{{}\Varid{mrgc}\;\Varid{x}\;(\Conid{Cons}\;{}}
\fromto{16}{19}{{}\Varid{y}\;{}}
\fromto{19}{23}{{}\Varid{n})\;{}}
\fromto{23}{33}{{}(\Conid{Skip}\;{}}
\fromto{33}{37}{{}\Varid{m}){}}
\fromto{37}{47}{{}\mathrel{=}\Conid{Skip}\;{}}
\fromto{47}{67}{{}(\Varid{mrgc}\;(\Varid{mergeT}\;\Varid{x}\;\Varid{y})\;{}}
\fromto{67}{E}{{}\Varid{n}\;\Varid{m}){}}
\nextline
\fromto{B}{16}{{}\Varid{mrgc}\;\Varid{x}\;(\Conid{Cons}\;{}}
\fromto{16}{19}{{}\Varid{y}\;{}}
\fromto{19}{23}{{}\Varid{n})\;{}}
\fromto{23}{30}{{}(\Conid{Cons}\;{}}
\fromto{30}{33}{{}\Varid{z}\;{}}
\fromto{33}{37}{{}\Varid{m}){}}
\fromto{37}{47}{{}\mathrel{=}\Conid{Cons}\;\Varid{x}\;{}}
\fromto{47}{67}{{}(\Varid{mrgc}\;(\Varid{mergeT}\;\Varid{y}\;\Varid{z})\;{}}
\fromto{67}{E}{{}\Varid{n}\;\Varid{m}){}}
\ColumnHook
\end{pboxed}
\)\par\noindent\endgroup\resethooks
Again, we define a wrapper function for merging top-level \ensuremath{\Conid{Heap}}s.
\begingroup\par\noindent\advance\leftskip\mathindent\(
\begin{pboxed}\SaveRestoreHook
\column{B}{@{}l@{}}
\column{19}{@{}l@{}}
\column{30}{@{}l@{}}
\column{E}{@{}l@{}}
\fromto{B}{E}{{}\Varid{merge}\mathbin{::}\Conid{Heap}\to \Conid{Heap}\to \Conid{Heap}{}}
\nextline
\fromto{B}{19}{{}\Varid{merge}\;\Conid{Empty}\;{}}
\fromto{19}{30}{{}\Varid{h2}{}}
\fromto{30}{E}{{}\mathrel{=}\Varid{h2}{}}
\nextline
\fromto{B}{19}{{}\Varid{merge}\;\Varid{h1}\;{}}
\fromto{19}{30}{{}\Conid{Empty}{}}
\fromto{30}{E}{{}\mathrel{=}\Varid{h1}{}}
\nextline
\fromto{B}{19}{{}\Varid{merge}\;(\Conid{Heap}\;\Varid{h1})\;{}}
\fromto{19}{30}{{}(\Conid{Heap}\;\Varid{h2}){}}
\fromto{30}{E}{{}\mathrel{=}\Conid{Heap}\;(\Varid{mrg}\;\Varid{h1}\;\Varid{h2}){}}
\ColumnHook
\end{pboxed}
\)\par\noindent\endgroup\resethooks
\subsubsection{Retrieving and Removing the Minimum Element} 
Due to space considerations, we omit a full exposition of 
the \ensuremath{\Varid{minElem}} and \ensuremath{\Varid{extractMin}} operations.  
The most interesting part of those definitions involves a 
type-indexed ``zipper'' type~\cite{huet-zipper} that represents
\ensuremath{\Conid{SubHeap}} contexts.  This gives evidence that Huet's zipper 
technique scales well to indexed types.
The interested reader may refer to the Appendix for details.

\hide{\begingroup\par\noindent\advance\leftskip\mathindent\(
\begin{pboxed}\SaveRestoreHook
\column{B}{@{}l@{}}
\column{3}{@{}l@{}}
\column{E}{@{}l@{}}
\fromto{3}{E}{{}\mbox{\onelinecomment  FOR TESTING PURPOSES}{}}
\nextline[\blanklineskip]
\fromto{B}{E}{{}\Varid{inserts}\;\Varid{xs}\mathrel{=}\Varid{foldr}\;\Varid{insert}\;\Varid{empty}\;\Varid{xs}{}}
\nextline
\fromto{B}{E}{{}\Varid{fromJust}\;(\Conid{Just}\;\Varid{x})\mathrel{=}\Varid{x}{}}
\nextline
\fromto{B}{E}{{}\Varid{hd}\;\Varid{h}\mathrel{=}\Varid{fromJust}\;(\Varid{minElem}\;\Varid{h}){}}
\nextline
\fromto{B}{E}{{}\Varid{tl}\;\Varid{h}\mathrel{=}\Varid{fromJust}\;(\Varid{extractMin}\;\Varid{h}){}}
\nextline[\blanklineskip]
\fromto{B}{E}{{}\Varid{heap0}\mathrel{=}\Varid{inserts}\;[\mskip1.5mu \mathrm{9},\mathrm{6},\mathrm{1},\mathrm{5},\mathrm{0},\mathrm{10},\mathrm{2},\mathrm{8},\mathrm{11},\mathrm{7},\mathrm{4},\mathrm{3},\mathrm{12}\mskip1.5mu]{}}
\nextline
\fromto{B}{E}{{}\Varid{test}\mathrel{=}\Varid{tl}\;(\Varid{tl}\;(\Varid{tl}\;(\Varid{tl}\;(\Varid{tl}\;(\Varid{tl}\;(\Varid{tl}\;(\Varid{tl}\;(\Varid{tl}\;(\Varid{tl}\;(\Varid{tl}\;(\Varid{tl}\;(\Varid{tl}\;(\Varid{heap0}))))))))))))){}}
\ColumnHook
\end{pboxed}
\)\par\noindent\endgroup\resethooks
}


%%%%%%%%%%%%%%%%%%%%%%%%%%%%%%%%%%%%%%%%%%%%%%%%%%%%%%%%%%%%%%%%%%%%%%%%%%%%%%%%



\section{Typed CPS Transform}

\newcommand{\cpsC}[1]{\langle\hspace{-0.2em}\vert {#1} \vert\hspace{-0.2em}\rangle}
\newcommand{\cpsV}[1]{\langle {#1} \rangle}
\newcommand{\cpsE}[1]{\lbrace\hspace{-0.2em}\vert {#1} \vert\hspace{-0.2em}\rbrace}
\newcommand{\lam}[2]{\lambda {#1}.\; {#2}}

We now give an \Wmega{} implementation of the call-by-value CPS 
transform as presented by Hatcliff and Danvy~\cite{hatcliff-danvy-thunks}.
They use the following syntax:
\[
\begin{array}{llcl}
(comp.)& m, n & ::= & v \mid m\ n \\
(value) & v    & ::= & i \mid x \mid \lam{x}{n} \\
\end{array}
\quad
\begin{array}{llcl}
(type) & \tau & ::= & Int \mid \tau \to \tau \mid R \\
(env.) &  \Gamma&::=& \cdot \mid \Gamma, x:\tau \\
\end{array}
\]
Lambda terms are divided into values and computations.  
We commit here to a call-by-value source language by categorizing 
variables as values.  The type $R$ is a special type introduced by the 
CPS transform as the final result type of any computation.

Then they define a call-by-value CPS transform on terms.
\[
\begin{array}{lcl}
\cpsC{v}     &=& \lam{k}{k\ \cpsV{v}} \\
\cpsC{m\ n}  &=& \lam{k}{\cpsC{m}\ (\lam{y_0}{\cpsC{n}\ (\lam{y_1}{y_0\ y_1\ k})})} \\
\end{array}
\quad
\begin{array}{lcl}
\cpsV{i}     &=& i \\
\cpsV{x}     &=& x \\
\cpsV{\lam{x}{n}} &=& \lam{x}{\cpsC{n}} \\
\end{array}
\]
The CPS transform has the following effect on types (and type environments):
\[
\begin{array}{lcl}
\cpsC{\tau}    &=& (\cpsV{\tau} \to R) \to R \\
\cpsV{Int}     &=& Int \\
\cpsV{\tau \to \sigma}   &=& {\cpsV{\tau} \to \cpsC{\sigma}} \\
\end{array}
\quad
\begin{array}{lcl}
\cpsE{\cdot}   &=& \cdot \\
\cpsE{\Gamma,x:\tau} &=& \cpsE{\Gamma},x:\cpsV{\tau} \\
\end{array}
\]

Following two (impressive looking) theorems express that the
CPS translation is type preserving:

If $\Gamma \vdash n : \tau$ then $\cpsE{\Gamma} \vdash \cpsC{n} : \cpsC{\tau}$ and

if $\Gamma \vdash v : \tau$ then $\cpsE{\Gamma} \vdash \cpsV{v} : \cpsV{\tau}$.

\noindent
Our implementation of CPS in \Wmega{} will prove these theorems using
type checking alone.


%%%%%%%%%%%%%%%%%%%%%%%%%%%%%%%%%%%%%%%%%%%%%%%%%%%%%%%%%%%%%%%%%%%%%%%%%%%%%%%%
\subsection{Type Representations and Singleton Types} \label{single}

The types of the object language can be modelled by \Wmega{} types. 
Since \Wmega{} already has \ensuremath{\Conid{Int}} types and function types (\ensuremath{\anonymous \to \anonymous }), 
we only need introduce a new type \ensuremath{R}.
\begingroup\par\noindent\advance\leftskip\mathindent\(
\begin{pboxed}\SaveRestoreHook
\column{B}{@{}l@{}}
\column{E}{@{}l@{}}
\fromto{B}{E}{{}\mathbf{data}\;R\mathrel{=}\Conid{FinalAnswer}{}}
\ColumnHook
\end{pboxed}
\)\par\noindent\endgroup\resethooks
We can define a \emph{representation type} for the subset of \Wmega{}'s
type language that we will employ in the CPS example. A representation
(a.k.a. singleton) type \ensuremath{\Conid{T}} is a parameterized type. A (run-time) value
of type \ensuremath{\Conid{T}\;\Varid{a}} represents (or mirrors the shape of) the the static type
\ensuremath{\Varid{a}}. For example consider \ensuremath{\Conid{Type}\;\Varid{t}} below.
\begingroup\par\noindent\advance\leftskip\mathindent\(
\begin{pboxed}\SaveRestoreHook
\column{B}{@{}l@{}}
\column{9}{@{}l@{}}
\column{20}{@{}l@{}}
\column{43}{@{}l@{}}
\column{E}{@{}l@{}}
\fromto{B}{E}{{}\mathbf{data}\;\Conid{Type}\;\Varid{t}{}}
\nextline
\fromto{B}{9}{{}\hsindent{9}{}}
\fromto{9}{20}{{}\mathrel{=}{}}
\fromto{20}{43}{{}\Conid{Int}\;{}}
\fromto{43}{E}{{}\mathbf{where}\;\Varid{t}\mathrel{=}\Conid{Int}{}}
\nextline
\fromto{B}{9}{{}\hsindent{9}{}}
\fromto{9}{20}{{}\mid \exists\;\Varid{a}\;\Varid{b}. {}}
\fromto{20}{43}{{}\Conid{Arr}\;(\Conid{Type}\;\Varid{a})\;(\Conid{Type}\;\Varid{b})\;{}}
\fromto{43}{E}{{}\mathbf{where}\;\Varid{t}\mathrel{=}\Varid{a}\to \Varid{b}{}}
\nextline
\fromto{B}{9}{{}\hsindent{9}{}}
\fromto{9}{20}{{}\mid {}}
\fromto{20}{43}{{}R\;{}}
\fromto{43}{E}{{}\mathbf{where}\;\Varid{t}\mathrel{=}R{}}
\ColumnHook
\end{pboxed}
\)\par\noindent\endgroup\resethooks
Note how every value of type \ensuremath{\Conid{Type}\;\Varid{a}} has the same shape as \ensuremath{\Varid{a}}. For
example: \ensuremath{\Conid{Int}\mathbin{::}\Conid{Type}\;\Conid{Int}}, and \ensuremath{R\mathbin{::}\Conid{Type}\;R}, and if \ensuremath{\Varid{a}\mathbin{::}\Conid{Type}\;\Varid{x}}
and \ensuremath{\Varid{b}\mathbin{::}\Conid{Type}\;\Varid{y}}, then \ensuremath{\Conid{Arr}\;\Varid{a}\;\Varid{b}\mathbin{::}\Conid{Type}\;(\Varid{x}\to \Varid{y})}. For example, the value
\mbox{\ensuremath{\Conid{Arr}\;\Conid{Int}\;R}} has type \mbox{\ensuremath{\Conid{Type}\;(\Conid{Int}\to R)}}. The useful
property of \ensuremath{\Conid{Type}} is that there is a one-to-one correspondence between
values of type \ensuremath{\Conid{Type}\;\Varid{t}} and types \ensuremath{\Varid{t}}. Values of type \ensuremath{\Conid{Type}\;\Varid{t}} are ordinary
run-time values and can be computed over like any other values. Knowing
the shape (which constructors are used to build it) of a value of type \ensuremath{\Conid{T}\;\Varid{t}} determines its type, and knowing the type of such a value determines
its shape.This is why they are some times called singleton types.

\hide{
This allows us to substitute run-time checks for static checks
when necessary. We will make use of this important ability
several places in the sequel, and we will be sure and
point this out when we do.
}


%%%%%%%%%%%%%%%%%%%%%%%%%%%%%%%%%%%%%%%%%%%%%%%%%%%%%%%%%%%%%%%%%%%%%%%%%%%%%%%%
\subsection{Object Terms}

As we describe the CPS transform we make use of two languages.
The object-language represents the programs we are
transforming (CPS terms) and the meta-language (\Wmega{}) describes
the transformation. Object-programs are data in the meta-language.
The \Wmega{} data-structures representing CPS object-programs
are defined using equality qualified types parameterized by two type indexes:
an environment \ensuremath{\Varid{e}}, and an object-level type \ensuremath{\Varid{t}}.  This means that 
\Wmega{}'s type system guarantees that all well-typed
meta-level terms representing object-level terms, represent only well-typed
object-terms.  The technique relies on de Bruijn indices to represent variables.
\cite{Pasalic}

\begingroup\par\noindent\advance\leftskip\mathindent\(
\begin{pboxed}\SaveRestoreHook
\column{B}{@{}l@{}}
\column{17}{@{}c@{}}
\column{17E}{@{}l@{}}
\column{20}{@{}l@{}}
\column{29}{@{}l@{}}
\column{35}{@{}l@{}}
\column{59}{@{}l@{}}
\column{E}{@{}l@{}}
\fromto{B}{17}{{}\mathbf{data}\;\Conid{Cmp}\;\Varid{e}\;\Varid{t}{}}
\fromto{17}{17E}{{}\mathrel{=}{}}
\fromto{29}{35}{{}\Conid{Val}\;{}}
\fromto{35}{E}{{}(\Conid{Val}\;\Varid{e}\;\Varid{t}){}}
\nextline
\fromto{17}{17E}{{}\mid {}}
\fromto{20}{29}{{}\exists\;\Varid{s}. {}}
\fromto{29}{35}{{}\Conid{App}\;{}}
\fromto{35}{E}{{}(\Conid{Cmp}\;\Varid{e}\;(\Varid{s}\to \Varid{t}))\;(\Conid{Cmp}\;\Varid{e}\;\Varid{s}){}}
\nextline[\blanklineskip]
\fromto{B}{17}{{}\mathbf{data}\;\Conid{Val}\;\Varid{e}\;\Varid{t}{}}
\fromto{17}{17E}{{}\mathrel{=}{}}
\fromto{29}{35}{{}\Conid{Lit}\;{}}
\fromto{35}{59}{{}\Conid{Int}\;{}}
\fromto{59}{E}{{}\mathbf{where}\;\Varid{t}\mathrel{=}\Conid{Int}{}}
\nextline
\fromto{17}{17E}{{}\mid {}}
\fromto{29}{35}{{}\Conid{Var}\;{}}
\fromto{35}{E}{{}(\Conid{Var}\;\Varid{e}\;\Varid{t})\;(\Conid{Type}\;\Varid{t}){}}
\nextline
\fromto{17}{17E}{{}\mid {}}
\fromto{20}{29}{{}\exists\;\Varid{a}\;\Varid{b}. {}}
\fromto{29}{35}{{}\Conid{Lam}\;{}}
\fromto{35}{59}{{}(\Conid{Type}\;\Varid{a})\;(\Conid{Cmp}\;(\Varid{e},\Varid{a})\;\Varid{b})\;{}}
\fromto{59}{E}{{}\mathbf{where}\;\Varid{t}\mathrel{=}\Varid{a}\to \Varid{b}{}}
\nextline[\blanklineskip]
\fromto{B}{17}{{}\mathbf{data}\;\Conid{Var}\;\Varid{e}\;\Varid{t}{}}
\fromto{17}{17E}{{}\mathrel{=}{}}
\fromto{20}{29}{{}\exists\;\Varid{f}. {}}
\fromto{29}{59}{{}\Conid{Zero}\;{}}
\fromto{59}{E}{{}\mathbf{where}\;\Varid{e}\mathrel{=}(\Varid{f},\Varid{t}){}}
\nextline
\fromto{17}{17E}{{}\mid {}}
\fromto{20}{29}{{}\exists\;\Varid{f}\;\Varid{s}. {}}
\fromto{29}{35}{{}\Conid{Succ}\;{}}
\fromto{35}{59}{{}(\Conid{Var}\;\Varid{f}\;\Varid{t})\;{}}
\fromto{59}{E}{{}\mathbf{where}\;\Varid{e}\mathrel{=}(\Varid{f},\Varid{s}){}}
\ColumnHook
\end{pboxed}
\)\par\noindent\endgroup\resethooks
These datatypes correspond to typing judgments and their values
correspond to proofs of those judgments.  We are using nested pairs
(growing to the right) to model typing environments and a fragment of
the \Wmega{} types (as discussed in Section \ref{single} above) to
model object-types. Some examples are germane here.
Shorthand for some common variable indices will prove useful.\begingroup\par\noindent\advance\leftskip\mathindent\(
\begin{pboxed}\SaveRestoreHook
\column{B}{@{}l@{}}
\column{8}{@{}l@{}}
\column{E}{@{}l@{}}
\fromto{B}{8}{{}\Varid{zero}{}}
\fromto{8}{E}{{}\mathbin{::}\Conid{Type}\;\Varid{a}\to \Conid{Cmp}\;(\Varid{e},\Varid{a})\;\Varid{a}{}}
\nextline
\fromto{B}{8}{{}\Varid{zero}\;{}}
\fromto{8}{E}{{}\Varid{t}\mathrel{=}\Conid{Val}\;(\Conid{Var}\;\Conid{Zero}\;\Varid{t}){}}
\nextline[\blanklineskip]
\fromto{B}{8}{{}\Varid{one}{}}
\fromto{8}{E}{{}\mathbin{::}\Conid{Type}\;\Varid{a}\to \Conid{Cmp}\;((\Varid{e},\Varid{a}),\Varid{b})\;\Varid{a}{}}
\nextline
\fromto{B}{8}{{}\Varid{one}\;{}}
\fromto{8}{E}{{}\Varid{t}\mathrel{=}\Conid{Val}\;(\Conid{Var}\;(\Conid{Succ}\;\Conid{Zero})\;\Varid{t}){}}
\nextline[\blanklineskip]
\fromto{B}{8}{{}\Varid{two}{}}
\fromto{8}{E}{{}\mathbin{::}\Conid{Type}\;\Varid{a}\to \Conid{Cmp}\;(((\Varid{e},\Varid{a}),\Varid{b}),\Varid{c})\;\Varid{a}{}}
\nextline
\fromto{B}{8}{{}\Varid{two}\;{}}
\fromto{8}{E}{{}\Varid{t}\mathrel{=}\Conid{Val}\;(\Conid{Var}\;(\Conid{Succ}\;(\Conid{Succ}\;\Conid{Zero}))\;\Varid{t}){}}
\ColumnHook
\end{pboxed}
\)\par\noindent\endgroup\resethooks
Consider representing the object-level term
$\lam{x:Int \to Int}{\lam{y:Int}{x\ y}}$ 
The meta-level term: \ensuremath{\Varid{m}\mathrel{=}\Conid{App}\;(\Varid{one}\;(\Conid{Arr}\;\Conid{Int}\;\Conid{Int}))\;(\Varid{zero}\;\Conid{Int})}
has type \ensuremath{\Conid{Cmp}\;((\Varid{e},\Conid{Int}\to \Conid{Int}),\Conid{Int})\;\Conid{Int}}. 
Notice how the environment is a nested pair. 
By wrapping \ensuremath{\Varid{m}} in two lambdas \ensuremath{\Conid{Lam}\;(\Conid{Arr}\;\Conid{Int}\;\Conid{Int})\;(\Conid{Val}\;(\Conid{Lam}\;\Conid{Int}\;\Varid{m}))},
we obtain a term with type \ensuremath{\Conid{Val}\;\Varid{e}\;((\Conid{Int}\to \Conid{Int})\to \Conid{Int}\to \Conid{Int}))}


%%%%%%%%%%%%%%%%%%%%%%%%%%%%%%%%%%%%%%%%%%%%%%%%%%%%%%%%%%%%%%%%%%%%%%%%%%%%%%%%
\subsection{Reading Off Types}

The abstract syntax includes enough type information to recover 
at run-time the (representation of the) type of any lambda term.
\begingroup\par\noindent\advance\leftskip\mathindent\(
\begin{pboxed}\SaveRestoreHook
\column{B}{@{}l@{}}
\column{11}{@{}l@{}}
\column{22}{@{}l@{}}
\column{E}{@{}l@{}}
\fromto{B}{11}{{}\Varid{expType}{}}
\fromto{11}{E}{{}\mathbin{::}\Conid{Cmp}\;\Varid{e}\;\Varid{t}\to \Conid{Type}\;\Varid{t}{}}
\nextline
\fromto{B}{11}{{}\Varid{expType}\;{}}
\fromto{11}{22}{{}(\Conid{Val}\;\Varid{v}){}}
\fromto{22}{E}{{}\mathrel{=}\Varid{valType}\;\Varid{v}{}}
\nextline
\fromto{B}{11}{{}\Varid{expType}\;{}}
\fromto{11}{22}{{}(\Conid{App}\;\Varid{m}\;\Varid{n}){}}
\fromto{22}{E}{{}\mathrel{=}\Varid{codomain}\;(\Varid{expType}\;\Varid{m}){}}
\nextline[\blanklineskip]
\fromto{B}{11}{{}\Varid{valType}{}}
\fromto{11}{E}{{}\mathbin{::}\Conid{Val}\;\Varid{e}\;\Varid{t}\to \Conid{Type}\;\Varid{t}{}}
\nextline
\fromto{B}{11}{{}\Varid{valType}\;{}}
\fromto{11}{22}{{}(\Conid{Lit}\;\Varid{i}){}}
\fromto{22}{E}{{}\mathrel{=}\Conid{Int}{}}
\nextline
\fromto{B}{11}{{}\Varid{valType}\;{}}
\fromto{11}{22}{{}(\Conid{Var}\;\Varid{x}\;\Varid{t}){}}
\fromto{22}{E}{{}\mathrel{=}\Varid{t}{}}
\nextline
\fromto{B}{11}{{}\Varid{valType}\;{}}
\fromto{11}{22}{{}(\Conid{Lam}\;\Varid{t}\;\Varid{n}){}}
\fromto{22}{E}{{}\mathrel{=}\Conid{Arr}\;\Varid{t}\;(\Varid{expType}\;\Varid{n}){}}
\nextline[\blanklineskip]
\fromto{B}{E}{{}\Varid{codomain}\mathbin{::}\Conid{Type}\;(\Varid{a}\to \Varid{b})\to \Conid{Type}\;\Varid{b}{}}
\nextline
\fromto{B}{22}{{}\Varid{codomain}\;(\Conid{Arr}\;\Varid{a}\;\Varid{b}){}}
\fromto{22}{E}{{}\mathrel{=}\Varid{b}{}}
\ColumnHook
\end{pboxed}
\)\par\noindent\endgroup\resethooks
%%%%%%%%%%%%%%%%%%%%%%%%%%%%%%%%%%%%%%%%%%%%%%%%%%%%%%%%%%%%%%%%%%%%%%%%%%%%%%%%
\subsection{Shifting de Bruijn Indices}

The CPS transform introduces new lambda binders for additional 
continuation parameters.  This necessitates incrementing some 
variable indices to point past any newly introduced binder, 
an operation known as a \ensuremath{\Varid{shift}}. 

\begingroup\par\noindent\advance\leftskip\mathindent\(
\begin{pboxed}\SaveRestoreHook
\column{B}{@{}l@{}}
\column{12}{@{}l@{}}
\column{15}{@{}l@{}}
\column{28}{@{}l@{}}
\column{E}{@{}l@{}}
\fromto{B}{E}{{}\Varid{shift}\mathbin{::}\Conid{Cmp}\;\Varid{e}\;\Varid{t}\to \Conid{Cmp}\;(\Varid{e},\Varid{s})\;\Varid{t}{}}
\nextline
\fromto{B}{28}{{}\Varid{shift}\;\Varid{e}{}}
\fromto{28}{E}{{}\mathrel{=}\Varid{shiftE}\;\Conid{Succ}\;\Varid{e}{}}
\nextline[\blanklineskip]
\fromto{B}{E}{{}\Varid{shiftE}\mathbin{::}(\forall\;\Varid{t}. \Conid{Var}\;\Varid{e}\;\Varid{t}\to \Conid{Var}\;\Varid{f}\;\Varid{t})\to \Conid{Cmp}\;\Varid{e}\;\Varid{t}\to \Conid{Cmp}\;\Varid{f}\;\Varid{t}{}}
\nextline
\fromto{B}{12}{{}\Varid{shiftE}\;{}}
\fromto{12}{15}{{}\Varid{f}\;{}}
\fromto{15}{28}{{}(\Conid{App}\;\Varid{n}\;\Varid{m}){}}
\fromto{28}{E}{{}\mathrel{=}\Conid{App}\;(\Varid{shiftE}\;\Varid{f}\;\Varid{n})\;(\Varid{shiftE}\;\Varid{f}\;\Varid{m}){}}
\nextline
\fromto{B}{12}{{}\Varid{shiftE}\;{}}
\fromto{12}{15}{{}\Varid{f}\;{}}
\fromto{15}{28}{{}(\Conid{Val}\;\Varid{v}){}}
\fromto{28}{E}{{}\mathrel{=}\Conid{Val}\;(\Varid{shiftV}\;\Varid{f}\;\Varid{v}){}}
\nextline[\blanklineskip]
\fromto{B}{E}{{}\Varid{shiftV}\mathbin{::}(\forall\;\Varid{t}. \Conid{Var}\;\Varid{e}\;\Varid{t}\to \Conid{Var}\;\Varid{f}\;\Varid{t})\to \Conid{Val}\;\Varid{e}\;\Varid{t}\to \Conid{Val}\;\Varid{f}\;\Varid{t}{}}
\nextline
\fromto{B}{12}{{}\Varid{shiftV}\;{}}
\fromto{12}{15}{{}\Varid{f}\;{}}
\fromto{15}{28}{{}(\Conid{Lit}\;\Varid{i}){}}
\fromto{28}{E}{{}\mathrel{=}\Conid{Lit}\;\Varid{i}{}}
\nextline
\fromto{B}{12}{{}\Varid{shiftV}\;{}}
\fromto{12}{15}{{}\Varid{f}\;{}}
\fromto{15}{28}{{}(\Conid{Var}\;\Varid{x}\;\Varid{s}){}}
\fromto{28}{E}{{}\mathrel{=}\Conid{Var}\;(\Varid{f}\;\Varid{x})\;\Varid{s}{}}
\nextline
\fromto{B}{12}{{}\Varid{shiftV}\;{}}
\fromto{12}{15}{{}\Varid{f}\;{}}
\fromto{15}{28}{{}(\Conid{Lam}\;\Varid{s}\;\Varid{n}){}}
\fromto{28}{E}{{}\mathrel{=}\Conid{Lam}\;\Varid{s}\;(\Varid{shiftE}\;(lift\;\Varid{f})\;\Varid{n}){}}
\nextline[\blanklineskip]
\fromto{B}{E}{{}lift\mathbin{::}(\forall\;\Varid{t}. \Conid{Var}\;\Varid{e}\;\Varid{t}\to \Conid{Var}\;\Varid{f}\;\Varid{t})\to \Conid{Var}\;(\Varid{e},\Varid{s})\;\Varid{t}\to \Conid{Var}\;(\Varid{f},\Varid{s})\;\Varid{t}{}}
\nextline
\fromto{B}{15}{{}lift\;\Varid{s}\;{}}
\fromto{15}{28}{{}(\Conid{Zero}){}}
\fromto{28}{E}{{}\mathrel{=}\Conid{Zero}{}}
\nextline
\fromto{B}{15}{{}lift\;\Varid{s}\;{}}
\fromto{15}{28}{{}(\Conid{Succ}\;\Varid{x}){}}
\fromto{28}{E}{{}\mathrel{=}\Conid{Succ}\;(\Varid{s}\;\Varid{x}){}}
\ColumnHook
\end{pboxed}
\)\par\noindent\endgroup\resethooks
Since \ensuremath{\Varid{shift}} is really manipulating typing judgments, we can think of 
its type as a kind of weakening lemma for de Bruijn terms.  In this view, 
the \ensuremath{\Varid{shift}} function itself is a \emph{proof} of the lemma (provided it 
terminates on all inputs).

%%%%%%%%%%%%%%%%%%%%%%%%%%%%%%%%%%%%%%%%%%%%%%%%%%%%%%%%%%%%%%%%%%%%%%%%%%%%%%%%
\subsection{CPS Transform on Types}

Now we define the effect of the call-by-value CPS transform on 
the types of object terms.  This is conveniently expressed 
with \Wmega{}'s type functions.
\begingroup\par\noindent\advance\leftskip\mathindent\(
\begin{pboxed}\SaveRestoreHook
\column{B}{@{}l@{}}
\column{8}{@{}l@{}}
\column{9}{@{}l@{}}
\column{23}{@{}l@{}}
\column{E}{@{}l@{}}
\fromto{B}{8}{{}\Varid{cps}{}}
\fromto{8}{E}{{}\mathbin{::}\mathbin{*}\leadsto\mathbin{*}\leadsto\mathbin{*}{}}
\nextline
\fromto{B}{9}{{}\{\mskip1.5mu \Varid{cps}\;{}}
\fromto{9}{23}{{}\Varid{r}\;\Varid{t}\mskip1.5mu\}{}}
\fromto{23}{E}{{}\mathrel{=}(\{\mskip1.5mu \Varid{cpsv}\;\Varid{r}\;\Varid{t}\mskip1.5mu\}\to \Varid{r})\to \Varid{r}{}}
\nextline[\blanklineskip]
\fromto{B}{8}{{}\Varid{cpsv}{}}
\fromto{8}{E}{{}\mathbin{::}\mathbin{*}\leadsto\mathbin{*}\leadsto\mathbin{*}{}}
\nextline
\fromto{B}{9}{{}\{\mskip1.5mu \Varid{cpsv}\;{}}
\fromto{9}{23}{{}\Varid{r}\;\Conid{Int}\mskip1.5mu\}{}}
\fromto{23}{E}{{}\mathrel{=}\Conid{Int}{}}
\nextline
\fromto{B}{9}{{}\{\mskip1.5mu \Varid{cpsv}\;{}}
\fromto{9}{23}{{}\Varid{r}\;(\Varid{a}\to \Varid{b})\mskip1.5mu\}{}}
\fromto{23}{E}{{}\mathrel{=}\{\mskip1.5mu \Varid{cpsv}\;\Varid{r}\;\Varid{a}\mskip1.5mu\}\to \{\mskip1.5mu \Varid{cps}\;\Varid{r}\;\Varid{b}\mskip1.5mu\}{}}
\nextline[\blanklineskip]
\fromto{B}{8}{{}\Varid{cpse}{}}
\fromto{8}{E}{{}\mathbin{::}\mathbin{*}\leadsto\mathbin{*}\leadsto\mathbin{*}{}}
\nextline
\fromto{B}{9}{{}\{\mskip1.5mu \Varid{cpse}\;{}}
\fromto{9}{23}{{}\Varid{r}\;()\mskip1.5mu\}{}}
\fromto{23}{E}{{}\mathrel{=}(){}}
\nextline
\fromto{B}{9}{{}\{\mskip1.5mu \Varid{cpse}\;{}}
\fromto{9}{23}{{}\Varid{r}\;(\Varid{a},\Varid{b})\mskip1.5mu\}{}}
\fromto{23}{E}{{}\mathrel{=}(\{\mskip1.5mu \Varid{cpse}\;\Varid{r}\;\Varid{a}\mskip1.5mu\},\{\mskip1.5mu \Varid{cpsv}\;\Varid{r}\;\Varid{b}\mskip1.5mu\}){}}
\ColumnHook
\end{pboxed}
\)\par\noindent\endgroup\resethooks
We then reflect the type functions \ensuremath{\Varid{cps}} and \ensuremath{\Varid{cpsv}} at the 
value level.
\begingroup\par\noindent\advance\leftskip\mathindent\(
\begin{pboxed}\SaveRestoreHook
\column{B}{@{}l@{}}
\column{9}{@{}l@{}}
\column{10}{@{}l@{}}
\column{22}{@{}l@{}}
\column{E}{@{}l@{}}
\fromto{B}{10}{{}\Varid{cpsT}{}}
\fromto{10}{E}{{}\mathbin{::}\Conid{Type}\;\Varid{r}\to \Conid{Type}\;\Varid{t}\to \Conid{Type}\;\{\mskip1.5mu \Varid{cps}\;\Varid{r}\;\Varid{t}\mskip1.5mu\}{}}
\nextline
\fromto{B}{9}{{}\Varid{cpsT}\;{}}
\fromto{9}{22}{{}\Varid{r}\;\Varid{t}{}}
\fromto{22}{E}{{}\mathrel{=}\Conid{Arr}\;(\Conid{Arr}\;(\Varid{cpsTv}\;\Varid{r}\;\Varid{t})\;\Varid{r})\;\Varid{r}{}}
\nextline[\blanklineskip]
\fromto{B}{10}{{}\Varid{cpsTv}{}}
\fromto{10}{E}{{}\mathbin{::}\Conid{Type}\;\Varid{r}\to \Conid{Type}\;\Varid{t}\to \Conid{Type}\;\{\mskip1.5mu \Varid{cpsv}\;\Varid{r}\;\Varid{t}\mskip1.5mu\}{}}
\nextline
\fromto{B}{9}{{}\Varid{cpsTv}\;{}}
\fromto{9}{22}{{}\Varid{r}\;\Conid{Int}{}}
\fromto{22}{E}{{}\mathrel{=}\Conid{Int}{}}
\nextline
\fromto{B}{9}{{}\Varid{cpsTv}\;{}}
\fromto{9}{22}{{}\Varid{r}\;(\Conid{Arr}\;\Varid{a}\;\Varid{b}){}}
\fromto{22}{E}{{}\mathrel{=}\Conid{Arr}\;(\Varid{cpsTv}\;\Varid{r}\;\Varid{a})\;(\Varid{cpsT}\;\Varid{r}\;\Varid{b}){}}
\ColumnHook
\end{pboxed}
\)\par\noindent\endgroup\resethooks
%%%%%%%%%%%%%%%%%%%%%%%%%%%%%%%%%%%%%%%%%%%%%%%%%%%%%%%%%%%%%%%%%%%%%%%%%%%%%%%%
\subsection{CPS Transform on Terms}

Finally we define the CPS transform on terms.  The transformation 
is given by three functions, one for each syntactic category.  
\begingroup\par\noindent\advance\leftskip\mathindent\(
\begin{pboxed}\SaveRestoreHook
\column{B}{@{}l@{}}
\column{11}{@{}l@{}}
\column{22}{@{}l@{}}
\column{27}{@{}l@{}}
\column{35}{@{}l@{}}
\column{40}{@{}l@{}}
\column{52}{@{}l@{}}
\column{59}{@{}l@{}}
\column{E}{@{}l@{}}
\fromto{B}{11}{{}\Varid{cpsC}\mathbin{::}{}}
\fromto{11}{22}{{}\Conid{Type}\;\Varid{r}\to {}}
\fromto{22}{27}{{}\Conid{Cmp}\;{}}
\fromto{27}{35}{{}\Varid{e}\;\Varid{a}\to {}}
\fromto{35}{40}{{}\Conid{Cmp}\;{}}
\fromto{40}{52}{{}\{\mskip1.5mu \Varid{cpse}\;\Varid{r}\;\Varid{e}\mskip1.5mu\}\;{}}
\fromto{52}{59}{{}\{\mskip1.5mu \Varid{cps}\;{}}
\fromto{59}{E}{{}\Varid{r}\;\Varid{a}\mskip1.5mu\}{}}
\nextline
\fromto{B}{11}{{}\Varid{cpsV}\mathbin{::}{}}
\fromto{11}{22}{{}\Conid{Type}\;\Varid{r}\to {}}
\fromto{22}{27}{{}\Conid{Val}\;{}}
\fromto{27}{35}{{}\Varid{e}\;\Varid{a}\to {}}
\fromto{35}{40}{{}\Conid{Val}\;{}}
\fromto{40}{52}{{}\{\mskip1.5mu \Varid{cpse}\;\Varid{r}\;\Varid{e}\mskip1.5mu\}\;{}}
\fromto{52}{59}{{}\{\mskip1.5mu \Varid{cpsv}\;{}}
\fromto{59}{E}{{}\Varid{r}\;\Varid{a}\mskip1.5mu\}{}}
\nextline
\fromto{B}{11}{{}\Varid{cpsX}\mathbin{::}{}}
\fromto{11}{22}{{}\Conid{Type}\;\Varid{r}\to {}}
\fromto{22}{27}{{}\Conid{Var}\;{}}
\fromto{27}{35}{{}\Varid{e}\;\Varid{a}\to {}}
\fromto{35}{40}{{}\Conid{Var}\;{}}
\fromto{40}{52}{{}\{\mskip1.5mu \Varid{cpse}\;\Varid{r}\;\Varid{e}\mskip1.5mu\}\;{}}
\fromto{52}{59}{{}\{\mskip1.5mu \Varid{cpsv}\;{}}
\fromto{59}{E}{{}\Varid{r}\;\Varid{a}\mskip1.5mu\}{}}
\ColumnHook
\end{pboxed}
\)\par\noindent\endgroup\resethooks
Each function has a \ensuremath{\Conid{Type}\;\Varid{r}} parameter because
the CPS transform is parameterized by the final result type \ensuremath{\Varid{r}}.  
We are transforming object types as well as object terms: 
the essence of a type-respecting transformation.

The \ensuremath{\Conid{Var}} and \ensuremath{\Conid{Val}} cases are straightforward.  
In fact \ensuremath{\Varid{cpsX}} is operationally the identity function, 
but it has a more complicated type.  
Another curious fact about \ensuremath{\Varid{cpsX}} is that the \ensuremath{\Conid{Type}\;\Varid{r}} parameter is 
never used, but is necessary for type-checking.  
Otherwise, in the \ensuremath{\Conid{Succ}} case, we can't unify the \ensuremath{\Varid{r}} 
of the calling \ensuremath{\Varid{cpsX}} with the fresh \ensuremath{\Varid{r'}} 
resulting from the recursive call.
\begingroup\par\noindent\advance\leftskip\mathindent\(
\begin{pboxed}\SaveRestoreHook
\column{B}{@{}l@{}}
\column{11}{@{}l@{}}
\column{20}{@{}l@{}}
\column{22}{@{}l@{}}
\column{27}{@{}l@{}}
\column{35}{@{}l@{}}
\column{40}{@{}l@{}}
\column{52}{@{}l@{}}
\column{59}{@{}l@{}}
\column{E}{@{}l@{}}
\fromto{B}{11}{{}\Varid{cpsX}\mathbin{::}{}}
\fromto{11}{22}{{}\Conid{Type}\;\Varid{r}\to {}}
\fromto{22}{27}{{}\Conid{Var}\;{}}
\fromto{27}{35}{{}\Varid{e}\;\Varid{a}\to {}}
\fromto{35}{40}{{}\Conid{Var}\;{}}
\fromto{40}{52}{{}\{\mskip1.5mu \Varid{cpse}\;\Varid{r}\;\Varid{e}\mskip1.5mu\}\;{}}
\fromto{52}{59}{{}\{\mskip1.5mu \Varid{cpsv}\;{}}
\fromto{59}{E}{{}\Varid{r}\;\Varid{a}\mskip1.5mu\}{}}
\nextline
\fromto{B}{20}{{}\Varid{cpsX}\;\Varid{r}\;(\Conid{Zero}){}}
\fromto{20}{E}{{}\mathrel{=}\Conid{Zero}{}}
\nextline
\fromto{B}{20}{{}\Varid{cpsX}\;\Varid{r}\;(\Conid{Succ}\;\Varid{x}){}}
\fromto{20}{E}{{}\mathrel{=}\Conid{Succ}\;(\Varid{cpsX}\;\Varid{r}\;\Varid{x}){}}
\nextline[\blanklineskip]
\fromto{B}{11}{{}\Varid{cpsV}\mathbin{::}{}}
\fromto{11}{22}{{}\Conid{Type}\;\Varid{r}\to {}}
\fromto{22}{27}{{}\Conid{Val}\;{}}
\fromto{27}{35}{{}\Varid{e}\;\Varid{a}\to {}}
\fromto{35}{40}{{}\Conid{Val}\;{}}
\fromto{40}{52}{{}\{\mskip1.5mu \Varid{cpse}\;\Varid{r}\;\Varid{e}\mskip1.5mu\}\;{}}
\fromto{52}{59}{{}\{\mskip1.5mu \Varid{cpsv}\;{}}
\fromto{59}{E}{{}\Varid{r}\;\Varid{a}\mskip1.5mu\}{}}
\nextline
\fromto{B}{20}{{}\Varid{cpsV}\;\Varid{r}\;(\Conid{Lit}\;\Varid{i}){}}
\fromto{20}{E}{{}\mathrel{=}\Conid{Lit}\;\Varid{i}{}}
\nextline
\fromto{B}{20}{{}\Varid{cpsV}\;\Varid{r}\;(\Conid{Var}\;\Varid{x}\;\Varid{t}){}}
\fromto{20}{E}{{}\mathrel{=}\Conid{Var}\;(\Varid{cpsX}\;\Varid{r}\;\Varid{x})\;(\Varid{cpsTv}\;\Varid{r}\;\Varid{t}){}}
\nextline
\fromto{B}{20}{{}\Varid{cpsV}\;\Varid{r}\;(\Conid{Lam}\;\Varid{t}\;\Varid{n}){}}
\fromto{20}{E}{{}\mathrel{=}\Conid{Lam}\;(\Varid{cpsTv}\;\Varid{r}\;\Varid{t})\;(\Varid{cpsC}\;\Varid{r}\;\Varid{n}){}}
\ColumnHook
\end{pboxed}
\)\par\noindent\endgroup\resethooks
The \ensuremath{\Conid{Cmp}} cases are the most involved, as they introduce 
new binders for continuation parameters.  This is where 
we need the \ensuremath{\Varid{shift}} operation.

\begingroup\par\noindent\advance\leftskip\mathindent\(
\begin{pboxed}\SaveRestoreHook
\column{B}{@{}l@{}}
\column{6}{@{}l@{}}
\column{13}{@{}l@{}}
\column{20}{@{}l@{}}
\column{22}{@{}l@{}}
\column{25}{@{}l@{}}
\column{50}{@{}l@{}}
\column{E}{@{}l@{}}
\fromto{B}{E}{{}\Varid{cpsC}\mathbin{::}\Conid{Type}\;\Varid{r}\to \Conid{Cmp}\;\Varid{e}\;\Varid{a}\to \Conid{Cmp}\;\{\mskip1.5mu \Varid{cpse}\;\Varid{r}\;\Varid{e}\mskip1.5mu\}\;\{\mskip1.5mu \Varid{cps}\;\Varid{r}\;\Varid{a}\mskip1.5mu\}{}}
\nextline
\fromto{B}{20}{{}\Varid{cpsC}\;\Varid{r}\;(\Conid{Val}\;\Varid{v}){}}
\fromto{20}{E}{{}\mathrel{=}\Conid{Val}\;(\Conid{Lam}\;\Varid{kt}\;(\Conid{App}\;(\Varid{zero}\;\Varid{kt})\;(\Varid{shift}\;(\Conid{Val}\;\Varid{v'})))){}}
\nextline
\fromto{B}{6}{{}\hsindent{6}{}}
\fromto{6}{13}{{}\mathbf{where}\;{}}
\fromto{13}{20}{{}\Varid{v'}{}}
\fromto{20}{E}{{}\mathrel{=}\Varid{cpsV}\;\Varid{r}\;\Varid{v}{}}
\nextline
\fromto{13}{20}{{}\Varid{kt}{}}
\fromto{20}{E}{{}\mathrel{=}\Conid{Arr}\;(\Varid{valType}\;\Varid{v'})\;\Varid{r}{}}
\nextline[\blanklineskip]
\fromto{B}{20}{{}\Varid{cpsC}\;\Varid{r}\;(\Conid{App}\;\Varid{m}\;\Varid{n}){}}
\fromto{20}{E}{{}\mathrel{=}\mathbf{case}\;\Varid{expType}\;\Varid{m'}\;\mathbf{of}{}}
\nextline
\fromto{20}{22}{{}\hsindent{2}{}}
\fromto{22}{E}{{}\Conid{Arr}\;(\Conid{Arr}\;(yt_0\mathord{@}(\Conid{Arr}\;yt_1\;(\Conid{Arr}\;\Varid{kt}\;\anonymous )))\;\anonymous )\;\anonymous \to {}}
\nextline
\fromto{22}{25}{{}\hsindent{3}{}}
\fromto{25}{E}{{}\Conid{Val}\;(\Conid{Lam}\;\Varid{kt}{}}
\nextline
\fromto{22}{25}{{}\hsindent{3}{}}
\fromto{25}{50}{{}(\Conid{App}\;(\Varid{shift}\;\Varid{m'})\;{}}
\fromto{50}{E}{{}(\Conid{Val}\;(\Conid{Lam}\;yt_0{}}
\nextline
\fromto{22}{25}{{}\hsindent{3}{}}
\fromto{25}{50}{{}(\Conid{App}\;(\Varid{shift}\;(\Varid{shift}\;\Varid{n'}))\;{}}
\fromto{50}{E}{{}(\Conid{Val}\;(\Conid{Lam}\;yt_1{}}
\nextline
\fromto{22}{25}{{}\hsindent{3}{}}
\fromto{25}{E}{{}(\Conid{App}\;(\Conid{App}\;(\Varid{one}\;yt_0)\;(\Varid{zero}\;yt_1))\;(\Varid{two}\;\Varid{kt})){}}
\nextline
\fromto{22}{25}{{}\hsindent{3}{}}
\fromto{25}{E}{{}))))))){}}
\nextline
\fromto{B}{6}{{}\hsindent{6}{}}
\fromto{6}{13}{{}\mathbf{where}\;{}}
\fromto{13}{20}{{}\Varid{m'}{}}
\fromto{20}{E}{{}\mathrel{=}\Varid{cpsC}\;\Varid{r}\;\Varid{m}{}}
\nextline
\fromto{13}{20}{{}\Varid{n'}{}}
\fromto{20}{E}{{}\mathrel{=}\Varid{cpsC}\;\Varid{r}\;\Varid{n}{}}
\ColumnHook
\end{pboxed}
\)\par\noindent\endgroup\resethooks
%%%%%%%%%%%%%%%%%%%%%%%%%%%%%%%%%%%%%%%%%%%%%%%%%%%%%%%%%%%%%%%%%%%%%%%%%%%%%%%%
\hide{\begingroup\par\noindent\advance\leftskip\mathindent\(
\begin{pboxed}\SaveRestoreHook
\column{B}{@{}l@{}}
\column{7}{@{}l@{}}
\column{E}{@{}l@{}}
\fromto{B}{E}{{}\mbox{\onelinecomment  playing}{}}
\nextline
\fromto{B}{E}{{}\Varid{exp}\mathrel{=}\Conid{Val}\;(\Conid{Lam}\;\Conid{Int}{}}
\nextline
\fromto{B}{7}{{}\hsindent{7}{}}
\fromto{7}{E}{{}(\Conid{Val}\;(\Conid{Lam}\;(\Conid{Arr}\;\Conid{Int}\;\Conid{Int}){}}
\nextline
\fromto{B}{7}{{}\hsindent{7}{}}
\fromto{7}{E}{{}(\Conid{App}\;(\Varid{zero}\;(\Conid{Arr}\;\Conid{Int}\;\Conid{Int}))\;(\Varid{one}\;\Conid{Int}))))){}}
\nextline[\blanklineskip]
\fromto{B}{E}{{}\Varid{test}\mathrel{=}\Varid{cpsC}\;R\;\Varid{exp}{}}
\nextline
\fromto{B}{E}{{}\mbox{\onelinecomment  lam k. k (lam a. lam k. k (lam f. lam c. (lam k. k f) (lam d. (lam k. k a) (lam e. d e c))))}{}}
\ColumnHook
\end{pboxed}
\)\par\noindent\endgroup\resethooks
}


%%%%%%%%%%%%%%%%%%%%%%%%%%%%%%%%%%%%%%%%%%%%%%%%%%%%%%%%%%%%%%%%%%%%%%%%%%%%%%%%
\section{Alternatives to Type-Functions}

There are a few alternatives to the use of type functions of \Wmega{}. We
explore them here because they can be used to give a sound basis to a
semantics for type functions. But while they are sound and do not require
the additional mechanism required by type functions, they are often clumsy
and hard to use. Our addition of type functions to \Wmega was inspired by
programming these more tedious examples, and wondering how we could make our
lives easier. Throughout this section we illustrate the techniques using the
\ensuremath{\Varid{sum}}/\ensuremath{\Varid{append}} example.
\begingroup\par\noindent\advance\leftskip\mathindent\(
\begin{pboxed}\SaveRestoreHook
\column{B}{@{}l@{}}
\column{17}{@{}l@{}}
\column{21}{@{}l@{}}
\column{22}{@{}l@{}}
\column{E}{@{}l@{}}
\fromto{B}{E}{{}\Varid{sum}\mathbin{::}\Conid{Nat}\leadsto\Conid{Nat}\leadsto\Conid{Nat}{}}
\nextline
\fromto{B}{17}{{}\{\mskip1.5mu \Varid{sum}\;\Conid{Zero}\;{}}
\fromto{17}{21}{{}\Varid{m}\mskip1.5mu\}{}}
\fromto{21}{E}{{}\mathrel{=}\Varid{m}{}}
\nextline
\fromto{B}{17}{{}\{\mskip1.5mu \Varid{sum}\;(\Conid{Succ}\;\Varid{n})\;{}}
\fromto{17}{21}{{}\Varid{m}\mskip1.5mu\}{}}
\fromto{21}{E}{{}\mathrel{=}\Conid{Succ}\;\{\mskip1.5mu \Varid{sum}\;\Varid{n}\;\Varid{m}\mskip1.5mu\}{}}
\nextline[\blanklineskip]
\fromto{B}{E}{{}\Varid{append}\mathbin{::}\Conid{Seq}\;\Varid{a}\;\Varid{n}\to \Conid{Seq}\;\Varid{a}\;\Varid{m}\to \Conid{Seq}\;\Varid{a}\;\{\mskip1.5mu \Varid{sum}\;\Varid{n}\;\Varid{m}\mskip1.5mu\}{}}
\nextline
\fromto{B}{22}{{}\Varid{append}\;(\Conid{Nil})\;{}}
\fromto{22}{E}{{}\Varid{ys}\mathrel{=}\Varid{ys}{}}
\nextline
\fromto{B}{22}{{}\Varid{append}\;(\Conid{Cons}\;\Varid{x}\;\Varid{xs})\;{}}
\fromto{22}{E}{{}\Varid{ys}\mathrel{=}\Conid{Cons}\;\Varid{x}\;(\Varid{append}\;\Varid{xs}\;\Varid{ys}){}}
\ColumnHook
\end{pboxed}
\)\par\noindent\endgroup\resethooks

One alternative is to use a datatype of explicit ``proofs'' of the 
addition relation.  This datatype is the relational incarnation of the 
\ensuremath{\Varid{sum}} type function.\begingroup\par\noindent\advance\leftskip\mathindent\(
\begin{pboxed}\SaveRestoreHook
\column{B}{@{}l@{}}
\column{4}{@{}l@{}}
\column{10}{@{}l@{}}
\column{18}{@{}l@{}}
\column{22}{@{}l@{}}
\column{35}{@{}l@{}}
\column{38}{@{}l@{}}
\column{58}{@{}l@{}}
\column{E}{@{}l@{}}
\fromto{B}{E}{{}\mathbf{data}\;\Conid{Sum}\;\Varid{n}\;\Varid{m}\;\Varid{s}{}}
\nextline
\fromto{B}{4}{{}\hsindent{4}{}}
\fromto{4}{18}{{}\mathrel{=}{}}
\fromto{18}{38}{{}\Conid{Base}\;{}}
\fromto{38}{58}{{}\mathbf{where}\;\Varid{n}\mathrel{=}\Conid{Zero},{}}
\fromto{58}{E}{{}\Varid{s}\mathrel{=}\Varid{m}{}}
\nextline
\fromto{B}{4}{{}\hsindent{4}{}}
\fromto{4}{18}{{}\mid \exists\;\Varid{n'}\;\Varid{s'}. {}}
\fromto{18}{38}{{}\Conid{Step}\;(\Conid{Sum}\;\Varid{n'}\;\Varid{m}\;\Varid{s'})\;{}}
\fromto{38}{58}{{}\mathbf{where}\;\Varid{n}\mathrel{=}\Conid{Succ}\;\Varid{n'},{}}
\fromto{58}{E}{{}\Varid{s}\mathrel{=}\Conid{Succ}\;\Varid{s'}{}}
\nextline[\blanklineskip]
\fromto{B}{E}{{}\Varid{append}\mathbin{::}\Conid{Sum}\;\Varid{n}\;\Varid{m}\;\Varid{s}\to \Conid{Seq}\;\Varid{a}\;\Varid{n}\to \Conid{Seq}\;\Varid{a}\;\Varid{m}\to \Conid{Seq}\;\Varid{a}\;\Varid{s}{}}
\nextline
\fromto{B}{10}{{}\Varid{append}\;{}}
\fromto{10}{22}{{}(\Conid{Base})\;{}}
\fromto{22}{35}{{}(\Conid{Nil})\;{}}
\fromto{35}{E}{{}\Varid{ys}\mathrel{=}\Varid{ys}{}}
\nextline
\fromto{B}{10}{{}\Varid{append}\;{}}
\fromto{10}{22}{{}(\Conid{Step}\;\Varid{sum})\;{}}
\fromto{22}{35}{{}(\Conid{Cons}\;\Varid{x}\;\Varid{xs})\;{}}
\fromto{35}{E}{{}\Varid{ys}\mathrel{=}\Conid{Cons}\;\Varid{x}\;(\Varid{append}\;\Varid{sum}\;\Varid{xs}\;\Varid{ys}){}}
\ColumnHook
\end{pboxed}
\)\par\noindent\endgroup\resethooks

This alternative has the disadvantage of requiring an extra \ensuremath{\Conid{Sum}} 
input.  We can overcome this limitation by giving 
\ensuremath{\Varid{append}} the type 
\ensuremath{\Conid{Seq}\;\Varid{a}\;\Varid{n}\to \Conid{Seq}\;\Varid{a}\;\Varid{m}\to \exists\;\Varid{s}. (\Conid{Sum}\;\Varid{n}\;\Varid{m}\;\Varid{s},\Conid{Seq}\;\Varid{a}\;\Varid{s})}.
Since \Wmega{} doesn't directly support existential types in this way, 
we simulate them using a combination of rank-2 polymorphism and 
continuation-passing style.\begingroup\par\noindent\advance\leftskip\mathindent\(
\begin{pboxed}\SaveRestoreHook
\column{B}{@{}l@{}}
\column{22}{@{}l@{}}
\column{37}{@{}l@{}}
\column{E}{@{}l@{}}
\fromto{B}{E}{{}\Varid{append}\mathbin{::}\Conid{Seq}\;\Varid{a}\;\Varid{n}\to \Conid{Seq}\;\Varid{a}\;\Varid{m}\to (\forall\;\Varid{s}. \Conid{Sum}\;\Varid{n}\;\Varid{m}\;\Varid{s}\to \Conid{Seq}\;\Varid{a}\;\Varid{s}\to \Varid{r})\to \Varid{r}{}}
\nextline
\fromto{B}{22}{{}\Varid{append}\;(\Conid{Nil})\;{}}
\fromto{22}{E}{{}\Varid{ys}\;\Varid{k}\mathrel{=}\Varid{k}\;\Conid{Base}\;\Varid{ys}{}}
\nextline
\fromto{B}{22}{{}\Varid{append}\;(\Conid{Cons}\;\Varid{x}\;\Varid{xs})\;{}}
\fromto{22}{37}{{}\Varid{ys}\;\Varid{k}\mathrel{=}\Varid{append}\;{}}
\fromto{37}{E}{{}\Varid{xs}\;\Varid{ys}\;(\lambda \Varid{sum}\;\Varid{zs}\to {}}
\nextline
\fromto{37}{E}{{}\Varid{k}\;(\Conid{Step}\;\Varid{sum})\;(\Conid{Cons}\;\Varid{x}\;\Varid{zs})){}}
\ColumnHook
\end{pboxed}
\)\par\noindent\endgroup\resethooks
In a language supporting multi-parameter type classes with 
functional dependencies~\cite{FunctionalDependencies}
(as do common extensions to Haskell), 
\edit{Does this actually work?  Try it out in GHC!}
we can forego explicitly constructing \ensuremath{\Conid{Sum}} proofs.
We instead define a type class for the \ensuremath{\Conid{Sum}} relation, following
Hallgren~\cite{hallgren-fun}\begingroup\par\noindent\advance\leftskip\mathindent\(
\begin{pboxed}\SaveRestoreHook
\column{B}{@{}l@{}}
\column{25}{@{}l@{}}
\column{30}{@{}l@{}}
\column{40}{@{}l@{}}
\column{43}{@{}l@{}}
\column{53}{@{}l@{}}
\column{E}{@{}l@{}}
\fromto{B}{53}{{}\mathbf{class}\;\Conid{Sum}\;\Varid{n}\;\Varid{m}\;\Varid{s}\mid \Varid{n}\;\Varid{m}\to \Varid{s}\;{}}
\fromto{53}{E}{{}\mathbf{where}\;\{\mskip1.5mu \mskip1.5mu\}{}}
\nextline
\fromto{B}{25}{{}\mathbf{instance}\;{}}
\fromto{25}{30}{{}\Conid{Sum}\;{}}
\fromto{30}{40}{{}\Conid{Zero}\;{}}
\fromto{40}{43}{{}\Varid{m}\;{}}
\fromto{43}{53}{{}\Varid{m}\;{}}
\fromto{53}{E}{{}\mathbf{where}\;\{\mskip1.5mu \mskip1.5mu\}{}}
\nextline
\fromto{B}{25}{{}\mathbf{instance}\;\Conid{Sum}\;\Varid{n}\;\Varid{m}\;\Varid{s}\Rightarrow {}}
\fromto{25}{30}{{}\Conid{Sum}\;{}}
\fromto{30}{40}{{}(\Conid{Succ}\;\Varid{n})\;{}}
\fromto{40}{43}{{}\Varid{m}\;{}}
\fromto{43}{53}{{}(\Conid{Succ}\;\Varid{s})\;{}}
\fromto{53}{E}{{}\mathbf{where}\;\{\mskip1.5mu \mskip1.5mu\}{}}
\ColumnHook
\end{pboxed}
\)\par\noindent\endgroup\resethooks
Then we use the same CPS trick, but omitting the proof objects.\begingroup\par\noindent\advance\leftskip\mathindent\(
\begin{pboxed}\SaveRestoreHook
\column{B}{@{}l@{}}
\column{22}{@{}l@{}}
\column{E}{@{}l@{}}
\fromto{B}{E}{{}\Varid{append}\mathbin{::}\Conid{Seq}\;\Varid{a}\;\Varid{n}\to \Conid{Seq}\;\Varid{a}\;\Varid{m}\to (\forall\;\Varid{s}. \Conid{Sum}\;\Varid{n}\;\Varid{m}\;\Varid{s}\Rightarrow \Conid{Seq}\;\Varid{a}\;\Varid{s}\to \Varid{r})\to \Varid{r}{}}
\nextline
\fromto{B}{22}{{}\Varid{append}\;(\Conid{Nil})\;{}}
\fromto{22}{E}{{}\Varid{ys}\;\Varid{k}\mathrel{=}\Varid{k}\;\Varid{ys}{}}
\nextline
\fromto{B}{22}{{}\Varid{append}\;(\Conid{Cons}\;\Varid{x}\;\Varid{xs})\;{}}
\fromto{22}{E}{{}\Varid{ys}\;\Varid{k}\mathrel{=}\Varid{append}\;\Varid{xs}\;\Varid{ys}\;(\lambda \Varid{zs}\to \Varid{k}\;(\Conid{Cons}\;\Varid{x}\;\Varid{zs})){}}
\ColumnHook
\end{pboxed}
\)\par\noindent\endgroup\resethooks
Type inference introduces all the proof objects as implicit 
dictionary arguments.  In this case, the dictionaries are empty 
and should be optimized away.

We can also wrap up the guarded existential in a datatype, but the end 
result is essentially a verbose version of continuation-passing-style.  
Which one is preferable is a matter of taste.\begingroup\par\noindent\advance\leftskip\mathindent\(
\begin{pboxed}\SaveRestoreHook
\column{B}{@{}l@{}}
\column{E}{@{}l@{}}
\fromto{B}{E}{{}\mathbf{data}\;\Conid{SumSeq}\;\Varid{n}\;\Varid{m}\mathrel{=}\exists\;\Varid{s}. \Conid{Sum}\;\Varid{n}\;\Varid{m}\;\Varid{s}\Rightarrow \Conid{SumSeq}\;(\Conid{Seq}\;\Varid{a}\;\Varid{s}){}}
\ColumnHook
\end{pboxed}
\)\par\noindent\endgroup\resethooks
\begingroup\par\noindent\advance\leftskip\mathindent\(
\begin{pboxed}\SaveRestoreHook
\column{B}{@{}l@{}}
\column{22}{@{}l@{}}
\column{33}{@{}l@{}}
\column{E}{@{}l@{}}
\fromto{B}{E}{{}\Varid{append}\mathbin{::}\Conid{Seq}\;\Varid{a}\;\Varid{n}\to \Conid{Seq}\;\Varid{a}\;\Varid{m}\to \Conid{Seq}\;\Varid{a}\;\{\mskip1.5mu \Varid{sum}\;\Varid{n}\;\Varid{m}\mskip1.5mu\}{}}
\nextline
\fromto{B}{22}{{}\Varid{append}\;(\Conid{Nil})\;{}}
\fromto{22}{E}{{}\Varid{ys}\mathrel{=}\Conid{SumSeq}\;\Varid{ys}{}}
\nextline
\fromto{B}{22}{{}\Varid{append}\;(\Conid{Cons}\;\Varid{x}\;\Varid{xs})\;{}}
\fromto{22}{33}{{}\Varid{ys}\mathrel{=}\mathbf{case}\;{}}
\fromto{33}{E}{{}\Varid{append}\;\Varid{xs}\;\Varid{ys}\;\mathbf{of}{}}
\nextline
\fromto{33}{E}{{}\Conid{SumSeq}\;\Varid{zs}\to \Conid{SumSeq}\;(\Conid{Cons}\;\Varid{x}\;\Varid{zs}){}}
\ColumnHook
\end{pboxed}
\)\par\noindent\endgroup\resethooks
It remains to be seen whether most interesting uses of type functions 
fall into the set that can be implemented using type classes. 
If so, we would like to use the syntax of type functions as an 
interface to the underlying machinery of type classes with 
functional dependencies.
Neubauer et al make a similar proposal~\cite{FunctionalNotation}.


%%%%%%%%%%%%%%%%%%%%%%%%%%%%%%%%%%%%%%%%%%%%%%%%%%%%%%%%%%%%%%%%%%%%%%%%%%%%%%%%
\section{Related Work}

\subsection{Non-regular datatypes}
Non-regular datatypes can be used to enforce structural 
invariants on datatypes.

	also encode invariants. \\
	sometimes less natural. \\
	cite Okasaki~\cite{okasaki-adventure}, 
	Hinze~\cite{hinze-}. \\

\subsection{Type Functions}
	Fun with Functional Dependencies~\cite{hallgren-fun} \\
	Functional Notation for Functional
		Dependencies~\cite{FunctionalNotation}

%%%%%%%%%%%%%%%%%%%%%%%%%%%%%%%%%%%%%%%%%%%%%%%%%%%%%%%%%%%%%%%%%%%%%%%%%%%%%%%%
\section{Conclusion}

Equality qualified types (as well as several other generalizations of 
  datatypes) allow one to index datatypes with ``values'' describing 
  the structural invariants they obey.

In a non- dependently-typed setting, these ``values'' must themselves 
be types.
User-defined kinds allow us to define simple datatypes at the type level 
  and enforce that their ``values'' are well-formed.
Type functions provide an intuitive way to compute with type-level data. 

We presented two examples of structural invariants that can be expressed 
  by equality qualified types.  The first example captured the structural 
  invariants of a heap datatype that are relevant for guaranteeing the 
  time-complexities of the heap operations.  The encoding is much more 
  natural than what can be obtained using non-regular datatypes (in the 
  similar setting of red-black trees).

In the second example, the structural invariants were the typing 
  rules of a simple object language.  The first order nature of \Wmega{}'s
  type system forced us to commit to using de Bruijn indices to 
  represent variables. 

In both examples, user-defined kinds and type functions nicely support  
  programming with equality qualified types.  The resulting programs 
  are not to complicated to understand. (We hope!)

We saw how representation types are a useful alternative to dependent types. 
  They provide a flow of information from the value realm to the type realm.

Finally, we argue that while
  type functions allow for elegant code, they need a more solid foundation.
  In particular, we would like to formalize them as an interface to the 
  well-understood mechanism of type classes with functional dependencies.

%%%%%%%%%%%%%%%%%%%%%%%%%%%%%%%%%%%%%%%%%%%%%%%%%%%%%%%%%%%%%%%%%%%%%%%%%%%%%%%%
%% BIBLIOGRAPHY
%\bibliographystyle{splncs}
\bibliographystyle{plain} %{alpha}
\bibliography{references}

\section{Appendix: Code for \ensuremath{\Varid{minElem}} and \ensuremath{\Varid{extractMin}}}

\subsection{Retrieving the Minimum Element} 

\hide{
Retrieving the minimum element from a binomial heap is fast 
because of the heap invariant maintained for binomial trees.
The only binomial tree operations used in 
the heap code so far are \ensuremath{\Varid{singletonT}} and \ensuremath{\Varid{mergeT}}.
These both maintain the invariant that every binomial tree (and 
subtree) has a minimum element at the top.  So we just need to 
check the top elements of each binomial tree when looking for 
the minimum element of a heap.  This can be done in a single 
$O (log\ n)$ traversal of the list of binomial trees.

The search takes place in two stages.  In the first stage, 
we don't know if the heap is empty so we return 
a \ensuremath{\Conid{Maybe}\;\Conid{Elem}}.  Once we find a tree (and its top element), we enter 
the second stage, carrying along the least \ensuremath{\Conid{Elem}} found so far.
The entire computation has a top-level wrapper function.
}
\begingroup\par\noindent\advance\leftskip\mathindent\(
\begin{pboxed}\SaveRestoreHook
\column{B}{@{}l@{}}
\column{20}{@{}l@{}}
\column{24}{@{}l@{}}
\column{27}{@{}l@{}}
\column{E}{@{}l@{}}
\fromto{B}{E}{{}\Varid{minElem}\mathbin{::}\Conid{Heap}\to \Conid{Maybe}\;\Conid{Elem}{}}
\nextline
\fromto{B}{27}{{}\Varid{minElem}\;\Conid{Empty}{}}
\fromto{27}{E}{{}\mathrel{=}\Conid{Nothing}{}}
\nextline
\fromto{B}{27}{{}\Varid{minElem}\;(\Conid{Heap}\;\Varid{h}){}}
\fromto{27}{E}{{}\mathrel{=}\Conid{Just}\;(minElem_1\;\Varid{h}){}}
\nextline[\blanklineskip]
\fromto{B}{E}{{}minElem_1\mathbin{::}\Conid{SubHeap}\;\Varid{n}\;\Varid{k}\to \Conid{Elem}{}}
\nextline
\fromto{B}{27}{{}minElem_1\;(\Conid{Last}\;\Varid{t}){}}
\fromto{27}{E}{{}\mathrel{=}\Varid{top}\;\Varid{t}{}}
\nextline
\fromto{B}{20}{{}minElem_1\;(\Conid{Skip}\;{}}
\fromto{20}{27}{{}\Varid{h}){}}
\fromto{27}{E}{{}\mathrel{=}minElem_1\;\Varid{h}{}}
\nextline
\fromto{B}{20}{{}minElem_1\;(\Conid{Cons}\;\Varid{t}\;{}}
\fromto{20}{27}{{}\Varid{h}){}}
\fromto{27}{E}{{}\mathrel{=}minElem_2\;\Varid{h}\;(\Varid{top}\;\Varid{t}){}}
\nextline[\blanklineskip]
\fromto{B}{E}{{}minElem_2\mathbin{::}\Conid{SubHeap}\;\Varid{n}\;\Varid{k}\to \Conid{Elem}\to \Conid{Elem}{}}
\nextline
\fromto{B}{24}{{}minElem_2\;(\Conid{Last}\;\Varid{t})\;{}}
\fromto{24}{27}{{}\Varid{x}{}}
\fromto{27}{E}{{}\mathrel{=}\Varid{min}\;\Varid{x}\;(\Varid{top}\;\Varid{t}){}}
\nextline
\fromto{B}{20}{{}minElem_2\;(\Conid{Skip}\;{}}
\fromto{20}{24}{{}\Varid{h})\;{}}
\fromto{24}{27}{{}\Varid{x}{}}
\fromto{27}{E}{{}\mathrel{=}minElem_2\;\Varid{h}\;\Varid{x}{}}
\nextline
\fromto{B}{20}{{}minElem_2\;(\Conid{Cons}\;\Varid{t}\;{}}
\fromto{20}{24}{{}\Varid{h})\;{}}
\fromto{24}{27}{{}\Varid{x}{}}
\fromto{27}{E}{{}\mathrel{=}minElem_2\;\Varid{h}\;(\Varid{min}\;\Varid{x}\;(\Varid{top}\;\Varid{t})){}}
\ColumnHook
\end{pboxed}
\)\par\noindent\endgroup\resethooks
\subsection{Removing the Minimum Element}
\hide{
Removing the minimum element is slightly more difficult because 
one must do something with the remainder of the tree that loses its 
top element.
The basic idea is to make the following three passes 
\edit{"three passes"? change this wording.} over the list of trees.
\begin{enumerate}
\item Identify the tree containing the minimum element.
\item Remove that tree from the original heap.
\item Make a second heap out of the removed tree (minus its minimum element)
      and merge it with the remaining heap.
\end{enumerate}
Using a programming technique called ``The Zipper''~\cite{huet-zipper}, 
we collapse the first two passes into one.  The idea is that 
one pass returns a \ensuremath{\Conid{Focus}} value representing the heap together with 
a focal point singling out one of the binomial trees.  The 
second pass ``heapifies'' the tree in focus and does the merge.
}
\begingroup\par\noindent\advance\leftskip\mathindent\(
\begin{pboxed}\SaveRestoreHook
\column{B}{@{}l@{}}
\column{23}{@{}l@{}}
\column{E}{@{}l@{}}
\fromto{B}{E}{{}\Varid{extractMin}\mathbin{::}\Conid{Heap}\to \Conid{Maybe}\;\Conid{Heap}{}}
\nextline
\fromto{B}{23}{{}\Varid{extractMin}\;\Conid{Empty}{}}
\fromto{23}{E}{{}\mathrel{=}\Conid{Nothing}{}}
\nextline
\fromto{B}{23}{{}\Varid{extractMin}\;(\Conid{Heap}\;\Varid{h}){}}
\fromto{23}{E}{{}\mathrel{=}\Conid{Just}\;(\Varid{unfocus}\;(\Varid{minFocus1}\;\Conid{Top}\;\Varid{h})){}}
\ColumnHook
\end{pboxed}
\)\par\noindent\endgroup\resethooks
\subsubsection{``The Zipper''.}
\hide{
 The technique is to define an auxilary data 
structure representing paths from the root of some data structure to 
its sub-parts.  In our case, we define a \ensuremath{\Conid{Path}} type that complements 
the \ensuremath{\Conid{SubHeap}} type.
}
\begingroup\par\noindent\advance\leftskip\mathindent\(
\begin{pboxed}\SaveRestoreHook
\column{B}{@{}l@{}}
\column{17}{@{}l@{}}
\column{26}{@{}l@{}}
\column{39}{@{}l@{}}
\column{49}{@{}l@{}}
\column{E}{@{}l@{}}
\fromto{B}{17}{{}\mathbf{data}\;\Conid{Path}\;\Varid{k}{}}
\fromto{17}{26}{{}\mathrel{=}{}}
\fromto{26}{49}{{}\Conid{Top}\;{}}
\fromto{49}{E}{{}\mathbf{where}\;\Varid{k}\mathrel{=}\Conid{Zero}{}}
\nextline
\fromto{17}{26}{{}\mid \exists\;\Varid{j}. {}}
\fromto{26}{39}{{}\Conid{Skip'}\;{}}
\fromto{39}{49}{{}(\Conid{Path}\;\Varid{j})\;{}}
\fromto{49}{E}{{}\mathbf{where}\;\Varid{k}\mathrel{=}\Conid{Succ}\;\Varid{j}{}}
\nextline
\fromto{17}{26}{{}\mid \exists\;\Varid{j}. {}}
\fromto{26}{39}{{}\Conid{Cons'}\;(\Conid{T}\;\Varid{j})\;{}}
\fromto{39}{49}{{}(\Conid{Path}\;\Varid{j})\;{}}
\fromto{49}{E}{{}\mathbf{where}\;\Varid{k}\mathrel{=}\Conid{Succ}\;\Varid{j}{}}
\ColumnHook
\end{pboxed}
\)\par\noindent\endgroup\resethooks
\hide{
A \ensuremath{\Conid{Path}\;\Varid{k}} value contains everything we passed by on the way to 
the $k^{\mathrm{th}}$ tree slot in a \ensuremath{\Conid{Heap}}.
Given a \ensuremath{\Conid{Path}} and \ensuremath{\Conid{SubHeap}}, we can ``zip'' them together 
to recover a \ensuremath{\Conid{Heap}}.
}
\begingroup\par\noindent\advance\leftskip\mathindent\(
\begin{pboxed}\SaveRestoreHook
\column{B}{@{}l@{}}
\column{23}{@{}l@{}}
\column{E}{@{}l@{}}
\fromto{B}{E}{{}\Varid{zipHeap}\mathbin{::}\Conid{Path}\;\Varid{k}\to \Conid{SubHeap}\;\Varid{n}\;\Varid{k}\to \Conid{Heap}{}}
\nextline
\fromto{B}{23}{{}\Varid{zipHeap}\;\Conid{Top}\;{}}
\fromto{23}{E}{{}\Varid{h}\mathrel{=}\Conid{Heap}\;\Varid{h}{}}
\nextline
\fromto{B}{23}{{}\Varid{zipHeap}\;(\Conid{Skip'}\;\Varid{p})\;{}}
\fromto{23}{E}{{}\Varid{h}\mathrel{=}\Varid{zipHeap}\;\Varid{p}\;(\Conid{Skip}\;\Varid{h}){}}
\nextline
\fromto{B}{23}{{}\Varid{zipHeap}\;(\Conid{Cons'}\;\Varid{t}\;\Varid{p})\;{}}
\fromto{23}{E}{{}\Varid{h}\mathrel{=}\Varid{zipHeap}\;\Varid{p}\;(\Conid{Cons}\;\Varid{t}\;\Varid{h}){}}
\ColumnHook
\end{pboxed}
\)\par\noindent\endgroup\resethooks
\hide{}
\begingroup\par\noindent\advance\leftskip\mathindent\(
\begin{pboxed}\SaveRestoreHook
\column{B}{@{}l@{}}
\column{25}{@{}l@{}}
\column{E}{@{}l@{}}
\fromto{B}{E}{{}\Varid{close}\mathbin{::}\Conid{Path}\;\Varid{k}\to \Conid{Heap}{}}
\nextline
\fromto{B}{25}{{}\Varid{close}\;\Conid{Top}{}}
\fromto{25}{E}{{}\mathrel{=}\Conid{Empty}{}}
\nextline
\fromto{B}{25}{{}\Varid{close}\;(\Conid{Skip'}\;\Varid{p}){}}
\fromto{25}{E}{{}\mathrel{=}\Varid{close}\;\Varid{p}{}}
\nextline
\fromto{B}{25}{{}\Varid{close}\;(\Conid{Cons'}\;\Varid{t}\;\Varid{p}){}}
\fromto{25}{E}{{}\mathrel{=}\Varid{zipHeap}\;\Varid{p}\;(\Conid{Last}\;\Varid{t}){}}
\ColumnHook
\end{pboxed}
\)\par\noindent\endgroup\resethooks
\hide{}
\begingroup\par\noindent\advance\leftskip\mathindent\(
\begin{pboxed}\SaveRestoreHook
\column{B}{@{}l@{}}
\column{25}{@{}l@{}}
\column{E}{@{}l@{}}
\fromto{B}{E}{{}\Varid{fill}\mathbin{::}\Conid{Path}\;\Varid{k}\to \Conid{Maybe}\;(\Conid{SubHeap}\;\Varid{n}\;(\Conid{Succ}\;\Varid{k}))\to \Conid{Heap}{}}
\nextline
\fromto{B}{25}{{}\Varid{fill}\;\Varid{p}\;\Conid{Nothing}{}}
\fromto{25}{E}{{}\mathrel{=}\Varid{close}\;\Varid{p}{}}
\nextline
\fromto{B}{25}{{}\Varid{fill}\;\Varid{p}\;(\Conid{Just}\;\Varid{h}){}}
\fromto{25}{E}{{}\mathrel{=}\Varid{zipHeap}\;\Varid{p}\;(\Conid{Skip}\;\Varid{h}){}}
\ColumnHook
\end{pboxed}
\)\par\noindent\endgroup\resethooks
\subsubsection{The Focus type.}
\hide{
 Now we define a type for singling 
out one tree in a \ensuremath{\Conid{Heap}}.
}
\begingroup\par\noindent\advance\leftskip\mathindent\(
\begin{pboxed}\SaveRestoreHook
\column{B}{@{}l@{}}
\column{E}{@{}l@{}}
\fromto{B}{E}{{}\mathbf{data}\;\Conid{Focus}\mathrel{=}\exists\;\Varid{n}\;\Varid{k}. \Conid{Focus}\;(\Conid{Path}\;\Varid{k})\;(\Conid{T}\;\Varid{k})\;(\Conid{Maybe}\;(\Conid{SubHeap}\;\Varid{n}\;(\Conid{Succ}\;\Varid{k}))){}}
\ColumnHook
\end{pboxed}
\)\par\noindent\endgroup\resethooks
\hide{
Once we focus in on the tree containing the minimum element.  We convert it 
into a \ensuremath{\Conid{Heap}} and merge it with what remains of the original \ensuremath{\Conid{Heap}}.
}
\begingroup\par\noindent\advance\leftskip\mathindent\(
\begin{pboxed}\SaveRestoreHook
\column{B}{@{}l@{}}
\column{E}{@{}l@{}}
\fromto{B}{E}{{}\Varid{unfocus}\mathbin{::}\Conid{Focus}\to \Conid{Heap}{}}
\nextline
\fromto{B}{E}{{}\Varid{unfocus}\;(\Conid{Focus}\;\Varid{p}\;\Varid{t}\;\Varid{mh})\mathrel{=}\Varid{merge}\;(\Varid{heapify}\;\Varid{t})\;(\Varid{fill}\;\Varid{p}\;\Varid{mh}){}}
\ColumnHook
\end{pboxed}
\)\par\noindent\endgroup\resethooks
\hide{
We will need to compare trees-in-focus according to their minimum elements.
}
\begingroup\par\noindent\advance\leftskip\mathindent\(
\begin{pboxed}\SaveRestoreHook
\column{B}{@{}l@{}}
\column{E}{@{}l@{}}
\fromto{B}{E}{{}\Varid{key}\mathbin{::}\Conid{Focus}\to \Conid{Elem}{}}
\nextline
\fromto{B}{E}{{}\Varid{key}\;(\Conid{Focus}\;\anonymous \;\Varid{t}\;\anonymous )\mathrel{=}\Varid{top}\;\Varid{t}{}}
\ColumnHook
\end{pboxed}
\)\par\noindent\endgroup\resethooks
\subsubsection{Heapifying a tree.}
\hide{
The \ensuremath{\Varid{unfocus}} function above requires us to be able to turn any tree 
of type \ensuremath{\Conid{T}\;\Varid{k}} (minus its minimum element) into a \ensuremath{\Conid{Heap}}.  A binomial tree 
without its minimum element is just a forest of trees:
\ensuremath{\Conid{T}\;(\Varid{k}\mathbin{-}\mathrm{1})}, \ensuremath{\Conid{T}\;(\Varid{k}\mathbin{-}\mathrm{2})}, \ldots, \ensuremath{\Conid{T}\;\mathrm{0}}.  
To form a heap, just reverse the list of these trees.
}
\begingroup\par\noindent\advance\leftskip\mathindent\(
\begin{pboxed}\SaveRestoreHook
\column{B}{@{}l@{}}
\column{24}{@{}l@{}}
\column{27}{@{}l@{}}
\column{E}{@{}l@{}}
\fromto{B}{E}{{}\Varid{heapify}\mathbin{::}\Conid{T}\;\Varid{k}\to \Conid{Heap}{}}
\nextline
\fromto{B}{E}{{}\Varid{heapify}\;(\Conid{Tree}\;\Varid{i}\;\Varid{f})\mathrel{=}\Varid{reverse1}\;\Varid{f}{}}
\nextline[\blanklineskip]
\fromto{B}{E}{{}\Varid{reverse1}\mathbin{::}\Conid{F}\;\Varid{k}\to \Conid{Heap}{}}
\nextline
\fromto{B}{27}{{}\Varid{reverse1}\;(\Conid{FNil}){}}
\fromto{27}{E}{{}\mathrel{=}\Conid{Empty}{}}
\nextline
\fromto{B}{27}{{}\Varid{reverse1}\;(\Conid{FCons}\;\Varid{t}\;\Varid{f}){}}
\fromto{27}{E}{{}\mathrel{=}\Varid{reverse2}\;\Varid{f}\;(\Conid{Last}\;\Varid{t}){}}
\nextline[\blanklineskip]
\fromto{B}{E}{{}\Varid{reverse2}\mathbin{::}\Conid{F}\;\Varid{k}\to \Conid{SubHeap}\;\Varid{n}\;\Varid{k}\to \Conid{Heap}{}}
\nextline
\fromto{B}{24}{{}\Varid{reverse2}\;(\Conid{FNil})\;{}}
\fromto{24}{27}{{}\Varid{h}{}}
\fromto{27}{E}{{}\mathrel{=}\Conid{Heap}\;\Varid{h}{}}
\nextline
\fromto{B}{24}{{}\Varid{reverse2}\;(\Conid{FCons}\;\Varid{t}\;\Varid{f})\;{}}
\fromto{24}{27}{{}\Varid{h}{}}
\fromto{27}{E}{{}\mathrel{=}\Varid{reverse2}\;\Varid{f}\;(\Conid{Cons}\;\Varid{t}\;\Varid{h}){}}
\ColumnHook
\end{pboxed}
\)\par\noindent\endgroup\resethooks
\hide{
This operation preserves the heap invariant because each sub-tree in 
\ensuremath{\Varid{f}} was built up using \ensuremath{\Varid{singletonT}} and \ensuremath{\Varid{mergeT}}.
}

\subsubsection{Finding the minimum Focus.}

\hide{
Finding the minimum \ensuremath{\Conid{Focus}} in a tree is just like finding the 
minimum element, except we have to pass along an accumulating \ensuremath{\Conid{Path}} 
parameter in order to build up \ensuremath{\Conid{Focus}} values along the way.
}
\begingroup\par\noindent\advance\leftskip\mathindent\(
\begin{pboxed}\SaveRestoreHook
\column{B}{@{}l@{}}
\column{17}{@{}l@{}}
\column{22}{@{}l@{}}
\column{23}{@{}l@{}}
\column{27}{@{}l@{}}
\column{31}{@{}l@{}}
\column{44}{@{}l@{}}
\column{48}{@{}l@{}}
\column{53}{@{}l@{}}
\column{E}{@{}l@{}}
\fromto{B}{17}{{}minFocus_1{}}
\fromto{17}{E}{{}\mathbin{::}\Conid{Path}\;\Varid{k}\to \Conid{SubHeap}\;\Varid{n}\;\Varid{k}\to \Conid{Focus}{}}
\nextline
\fromto{B}{31}{{}minFocus_1\;\Varid{p}\;(\Conid{Last}\;\Varid{t}){}}
\fromto{31}{E}{{}\mathrel{=}\Conid{Focus}\;\Varid{p}\;\Varid{t}\;\Conid{Nothing}{}}
\nextline
\fromto{B}{22}{{}minFocus_1\;\Varid{p}\;(\Conid{Skip}\;{}}
\fromto{22}{31}{{}\Varid{h}){}}
\fromto{31}{44}{{}\mathrel{=}minFocus_1\;{}}
\fromto{44}{E}{{}(\Conid{Skip'}\;\Varid{p})\;\Varid{h}{}}
\nextline
\fromto{B}{31}{{}minFocus_1\;\Varid{p}\;(\Conid{Cons}\;\Varid{t}\;\Varid{h}){}}
\fromto{31}{44}{{}\mathrel{=}minFocus_2\;{}}
\fromto{44}{E}{{}(\Conid{Cons'}\;\Varid{t}\;\Varid{p})\;\Varid{h}\;{}}
\nextline
\fromto{44}{E}{{}(\Conid{Focus}\;\Varid{p}\;\Varid{t}\;(\Conid{Just}\;\Varid{h})){}}
\nextline[\blanklineskip]
\fromto{B}{17}{{}minFocus_2{}}
\fromto{17}{E}{{}\mathbin{::}\Conid{Path}\;\Varid{k}\to \Conid{SubHeap}\;\Varid{n}\;\Varid{k}\to \Conid{Focus}\to \Conid{Focus}{}}
\nextline
\fromto{B}{27}{{}minFocus_2\;\Varid{p}\;(\Conid{Last}\;\Varid{t})\;{}}
\fromto{27}{31}{{}\Varid{x}{}}
\fromto{31}{E}{{}\mathrel{=}\Varid{minBy}\;\Varid{key}\;\Varid{x}\;(\Conid{Focus}\;\Varid{p}\;\Varid{t}\;\Conid{Nothing}){}}
\nextline
\fromto{B}{23}{{}minFocus_2\;\Varid{p}\;(\Conid{Skip}\;{}}
\fromto{23}{27}{{}\Varid{h})\;{}}
\fromto{27}{31}{{}\Varid{x}{}}
\fromto{31}{44}{{}\mathrel{=}minFocus_2\;{}}
\fromto{44}{53}{{}(\Conid{Skip'}\;{}}
\fromto{53}{E}{{}\Varid{p})\;\Varid{h}\;\Varid{x}{}}
\nextline
\fromto{B}{23}{{}minFocus_2\;\Varid{p}\;(\Conid{Cons}\;\Varid{t}\;{}}
\fromto{23}{27}{{}\Varid{h})\;{}}
\fromto{27}{31}{{}\Varid{x}{}}
\fromto{31}{44}{{}\mathrel{=}minFocus_2\;{}}
\fromto{44}{E}{{}(\Conid{Cons'}\;\Varid{t}\;\Varid{p})\;\Varid{h}\;{}}
\nextline
\fromto{44}{E}{{}(\Varid{minBy}\;\Varid{key}\;\Varid{x}\;{}}
\nextline
\fromto{44}{48}{{}\hsindent{4}{}}
\fromto{48}{E}{{}(\Conid{Focus}\;\Varid{p}\;\Varid{t}\;(\Conid{Just}\;\Varid{h}))){}}
\ColumnHook
\end{pboxed}
\)\par\noindent\endgroup\resethooks

\end{document}

