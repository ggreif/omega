\documentclass[10pt,twoside]{article}
\usepackage{makeidx,multicol,amssymb}
% Psfig/TeX Release 1.3
% dvips version
%
% All software, documentation, and related files in this distribution of
% psfig/tex are Copyright 1987, 1988 Trevor J. Darrell
%
% Permission is granted for use and non-profit distribution of psfig/tex 
% providing that this notice be clearly maintained, but the right to
% distribute any portion of psfig/tex for profit or as part of any commercial
% product is specifically reserved for the author.
%
% $Header: /u/sheard/cvs//GPCE05MetaMLTypes/psfig.tex,v 1.1.1.1 2005/04/07 18:31:53 sheard Exp $
% $Source: /u/sheard/cvs//GPCE05MetaMLTypes/psfig.tex,v $
%
% Thanks to Greg Hager (GDH) and Ned Batchelder for their contributions
% to this project.
%
% Modified by J. Daniel Smith on 9 October 1990 to accept the
% %%BoundingBox: comment with or without a space after the colon.  Stole
% file reading code from Tom Rokicki's EPSF.TEX file (see below).

%
% from a suggestion by eijkhout@csrd.uiuc.edu to allow
% loading as a style file:
\edef\psfigRestoreAt{\catcode`@=\number\catcode`@\relax}
\catcode`\@=11\relax
\newwrite\@unused
\def\typeout#1{{\let\protect\string\immediate\write\@unused{#1}}}
\typeout{psfig/tex 1.4-dvips}


%% Here's how you define your figure path.  Should be set up with null
%% default and a user useable definition.

\def\figurepath{./}
\def\psfigurepath#1{\edef\figurepath{#1}}

%
% @psdo control structure -- similar to Latex @for.
% I redefined these with different names so that psfig can
% be used with TeX as well as LaTeX, and so that it will not 
% be vunerable to future changes in LaTeX's internal
% control structure,
%
\def\@nnil{\@nil}
\def\@empty{}
\def\@psdonoop#1\@@#2#3{}
\def\@psdo#1:=#2\do#3{\edef\@psdotmp{#2}\ifx\@psdotmp\@empty \else
    \expandafter\@psdoloop#2,\@nil,\@nil\@@#1{#3}\fi}
\def\@psdoloop#1,#2,#3\@@#4#5{\def#4{#1}\ifx #4\@nnil \else
       #5\def#4{#2}\ifx #4\@nnil \else#5\@ipsdoloop #3\@@#4{#5}\fi\fi}
\def\@ipsdoloop#1,#2\@@#3#4{\def#3{#1}\ifx #3\@nnil 
       \let\@nextwhile=\@psdonoop \else
      #4\relax\let\@nextwhile=\@ipsdoloop\fi\@nextwhile#2\@@#3{#4}}
\def\@tpsdo#1:=#2\do#3{\xdef\@psdotmp{#2}\ifx\@psdotmp\@empty \else
    \@tpsdoloop#2\@nil\@nil\@@#1{#3}\fi}
\def\@tpsdoloop#1#2\@@#3#4{\def#3{#1}\ifx #3\@nnil 
       \let\@nextwhile=\@psdonoop \else
      #4\relax\let\@nextwhile=\@tpsdoloop\fi\@nextwhile#2\@@#3{#4}}
% 
%
%%%%%%%%%%%%%%%%%%%%%%%%%%%%%%%%%%%%%%%%%%%%%%%%%%%%%%%%%%%%%%%%%%%
% file reading stuff from epsf.tex
%   EPSF.TEX macro file:
%   Written by Tomas Rokicki of Radical Eye Software, 29 Mar 1989.
%   Revised by Don Knuth, 3 Jan 1990.
%   Revised by Tomas Rokicki to accept bounding boxes with no
%      space after the colon, 18 Jul 1990.
%   Portions modified/removed for use in PSFIG package by
%      J. Daniel Smith, 9 October 1990.
%
\newread\ps@stream
\newif\ifnot@eof       % continue looking for the bounding box?
\newif\if@noisy        % report what you're making?
\newif\if@atend        % %%BoundingBox: has (at end) specification
\newif\if@psfile       % does this look like a PostScript file?
%
% PostScript files should start with `%!'
%
{\catcode`\%=12\global\gdef\epsf@start{%!}}
\def\epsf@PS{PS}
%
\def\epsf@getbb#1{%
%
%   The first thing we need to do is to open the
%   PostScript file, if possible.
%
\openin\ps@stream=#1
\ifeof\ps@stream\typeout{Error, File #1 not found}\else
%
%   Okay, we got it. Now we'll scan lines until we find one that doesn't
%   start with %. We're looking for the bounding box comment.
%
   {\not@eoftrue \chardef\other=12
    \def\do##1{\catcode`##1=\other}\dospecials \catcode`\ =10
    \loop
       \if@psfile
	  \read\ps@stream to \epsf@fileline
       \else{
	  \obeyspaces
          \read\ps@stream to \epsf@tmp\global\let\epsf@fileline\epsf@tmp}
       \fi
       \ifeof\ps@stream\not@eoffalse\else
%
%   Check the first line for `%!'.  Issue a warning message if its not
%   there, since the file might not be a PostScript file.
%
       \if@psfile\else
       \expandafter\epsf@test\epsf@fileline:. \\%
       \fi
%
%   We check to see if the first character is a % sign;
%   if so, we look further and stop only if the line begins with
%   `%%BoundingBox:' and the `(atend)' specification was not found.
%   That is, the only way to stop is when the end of file is reached,
%   or a `%%BoundingBox: llx lly urx ury' line is found.
%
          \expandafter\epsf@aux\epsf@fileline:. \\%
       \fi
   \ifnot@eof\repeat
   }\closein\ps@stream\fi}%
%
% This tests if the file we are reading looks like a PostScript file.
%
\long\def\epsf@test#1#2#3:#4\\{\def\epsf@testit{#1#2}
			\ifx\epsf@testit\epsf@start\else
\typeout{Warning! File does not start with `\epsf@start'.  It may not be a PostScript file.}
			\fi
			\@psfiletrue} % don't test after 1st line
%
%   We still need to define the tricky \epsf@aux macro. This requires
%   a couple of magic constants for comparison purposes.
%
{\catcode`\%=12\global\let\epsf@percent=%\global\def\epsf@bblit{%BoundingBox}}
%
%
%   So we're ready to check for `%BoundingBox:' and to grab the
%   values if they are found.  We continue searching if `(at end)'
%   was found after the `%BoundingBox:'.
%
\long\def\epsf@aux#1#2:#3\\{\ifx#1\epsf@percent
   \def\epsf@testit{#2}\ifx\epsf@testit\epsf@bblit
	\@atendfalse
        \epsf@atend #3 . \\%
	\if@atend	
	   \if@verbose{
		\typeout{psfig: found `(atend)'; continuing search}
	   }\fi
        \else
        \epsf@grab #3 . . . \\%
        \not@eoffalse
        \global\no@bbfalse
        \fi
   \fi\fi}%
%
%   Here we grab the values and stuff them in the appropriate definitions.
%
\def\epsf@grab #1 #2 #3 #4 #5\\{%
   \global\def\epsf@llx{#1}\ifx\epsf@llx\empty
      \epsf@grab #2 #3 #4 #5 .\\\else
   \global\def\epsf@lly{#2}%
   \global\def\epsf@urx{#3}\global\def\epsf@ury{#4}\fi}%
%
% Determine if the stuff following the %%BoundingBox is `(atend)'
% J. Daniel Smith.  Copied from \epsf@grab above.
%
\def\epsf@atendlit{(atend)} 
\def\epsf@atend #1 #2 #3\\{%
   \def\epsf@tmp{#1}\ifx\epsf@tmp\empty
      \epsf@atend #2 #3 .\\\else
   \ifx\epsf@tmp\epsf@atendlit\@atendtrue\fi\fi}


% End of file reading stuff from epsf.tex
%%%%%%%%%%%%%%%%%%%%%%%%%%%%%%%%%%%%%%%%%%%%%%%%%%%%%%%%%%%%%%%%%%%

\def\psdraft{
	\def\@psdraft{0}
	%\typeout{draft level now is \@psdraft \space . }
}
\def\psfull{
	\def\@psdraft{100}
	%\typeout{draft level now is \@psdraft \space . }
}
\psfull
\newif\if@prologfile
\newif\if@postlogfile
\def\pssilent{
	\@noisyfalse
}
\def\psnoisy{
	\@noisytrue
}
\psnoisy
%%% These are for the option list.
%%% A specification of the form a = b maps to calling \@p@@sa{b}
\newif\if@bbllx
\newif\if@bblly
\newif\if@bburx
\newif\if@bbury
\newif\if@height
\newif\if@width
\newif\if@rheight
\newif\if@rwidth
\newif\if@clip
\newif\if@verbose
\def\@p@@sclip#1{\@cliptrue}

%%% GDH 7/26/87 -- changed so that it first looks in the local directory,
%%% then in a specified global directory for the ps file.

\def\@p@@sfile#1{\def\@p@sfile{null}%
	        \openin1=#1
		\ifeof1\closein1%
		       \openin1=\figurepath#1
			\ifeof1\typeout{Error, File #1 not found}
			\else\closein1
			    \edef\@p@sfile{\figurepath#1}%
                        \fi%
		 \else\closein1%
		       \def\@p@sfile{#1}%
		 \fi}
\def\@p@@sfigure#1{\def\@p@sfile{null}%
	        \openin1=#1
		\ifeof1\closein1%
		       \openin1=\figurepath#1
			\ifeof1\typeout{Error, File #1 not found}
			\else\closein1
			    \def\@p@sfile{\figurepath#1}%
                        \fi%
		 \else\closein1%
		       \def\@p@sfile{#1}%
		 \fi}

\def\@p@@sbbllx#1{
		%\typeout{bbllx is #1}
		\@bbllxtrue
		\dimen100=#1
		\edef\@p@sbbllx{\number\dimen100}
}
\def\@p@@sbblly#1{
		%\typeout{bblly is #1}
		\@bbllytrue
		\dimen100=#1
		\edef\@p@sbblly{\number\dimen100}
}
\def\@p@@sbburx#1{
		%\typeout{bburx is #1}
		\@bburxtrue
		\dimen100=#1
		\edef\@p@sbburx{\number\dimen100}
}
\def\@p@@sbbury#1{
		%\typeout{bbury is #1}
		\@bburytrue
		\dimen100=#1
		\edef\@p@sbbury{\number\dimen100}
}
\def\@p@@sheight#1{
		\@heighttrue
		\dimen100=#1
   		\edef\@p@sheight{\number\dimen100}
		%\typeout{Height is \@p@sheight}
}
\def\@p@@swidth#1{
		%\typeout{Width is #1}
		\@widthtrue
		\dimen100=#1
		\edef\@p@swidth{\number\dimen100}
}
\def\@p@@srheight#1{
		%\typeout{Reserved height is #1}
		\@rheighttrue
		\dimen100=#1
		\edef\@p@srheight{\number\dimen100}
}
\def\@p@@srwidth#1{
		%\typeout{Reserved width is #1}
		\@rwidthtrue
		\dimen100=#1
		\edef\@p@srwidth{\number\dimen100}
}
\def\@p@@ssilent#1{ 
		\@verbosefalse
}
\def\@p@@sprolog#1{\@prologfiletrue\def\@prologfileval{#1}}
\def\@p@@spostlog#1{\@postlogfiletrue\def\@postlogfileval{#1}}
\def\@cs@name#1{\csname #1\endcsname}
\def\@setparms#1=#2,{\@cs@name{@p@@s#1}{#2}}
%
% initialize the defaults (size the size of the figure)
%
\def\ps@init@parms{
		\@bbllxfalse \@bbllyfalse
		\@bburxfalse \@bburyfalse
		\@heightfalse \@widthfalse
		\@rheightfalse \@rwidthfalse
		\def\@p@sbbllx{}\def\@p@sbblly{}
		\def\@p@sbburx{}\def\@p@sbbury{}
		\def\@p@sheight{}\def\@p@swidth{}
		\def\@p@srheight{}\def\@p@srwidth{}
		\def\@p@sfile{}
		\def\@p@scost{10}
		\def\@sc{}
		\@prologfilefalse
		\@postlogfilefalse
		\@clipfalse
		\if@noisy
			\@verbosetrue
		\else
			\@verbosefalse
		\fi
}
%
% Go through the options setting things up.
%
\def\parse@ps@parms#1{
	 	\@psdo\@psfiga:=#1\do
		   {\expandafter\@setparms\@psfiga,}}
%
% Compute bb height and width
%
\newif\ifno@bb
\def\bb@missing{
	\if@verbose{
		\typeout{psfig: searching \@p@sfile \space  for bounding box}
	}\fi
	\no@bbtrue
	\epsf@getbb{\@p@sfile}
        \ifno@bb \else \bb@cull\epsf@llx\epsf@lly\epsf@urx\epsf@ury\fi
}	
\def\bb@cull#1#2#3#4{
	\dimen100=#1 bp\edef\@p@sbbllx{\number\dimen100}
	\dimen100=#2 bp\edef\@p@sbblly{\number\dimen100}
	\dimen100=#3 bp\edef\@p@sbburx{\number\dimen100}
	\dimen100=#4 bp\edef\@p@sbbury{\number\dimen100}
	\no@bbfalse
}
\def\compute@bb{
		\no@bbfalse
		\if@bbllx \else \no@bbtrue \fi
		\if@bblly \else \no@bbtrue \fi
		\if@bburx \else \no@bbtrue \fi
		\if@bbury \else \no@bbtrue \fi
		\ifno@bb \bb@missing \fi
		\ifno@bb \typeout{FATAL ERROR: no bb supplied or found}
			\no-bb-error
		\fi
		%
		\count203=\@p@sbburx
		\count204=\@p@sbbury
		\advance\count203 by -\@p@sbbllx
		\advance\count204 by -\@p@sbblly
		\edef\@bbw{\number\count203}
		\edef\@bbh{\number\count204}
		%\typeout{ bbh = \@bbh, bbw = \@bbw }
}
%
% \in@hundreds performs #1 * (#2 / #3) correct to the hundreds,
%	then leaves the result in @result
%
\def\in@hundreds#1#2#3{\count240=#2 \count241=#3
		     \count100=\count240	% 100 is first digit #2/#3
		     \divide\count100 by \count241
		     \count101=\count100
		     \multiply\count101 by \count241
		     \advance\count240 by -\count101
		     \multiply\count240 by 10
		     \count101=\count240	%101 is second digit of #2/#3
		     \divide\count101 by \count241
		     \count102=\count101
		     \multiply\count102 by \count241
		     \advance\count240 by -\count102
		     \multiply\count240 by 10
		     \count102=\count240	% 102 is the third digit
		     \divide\count102 by \count241
		     \count200=#1\count205=0
		     \count201=\count200
			\multiply\count201 by \count100
		 	\advance\count205 by \count201
		     \count201=\count200
			\divide\count201 by 10
			\multiply\count201 by \count101
			\advance\count205 by \count201
			%
		     \count201=\count200
			\divide\count201 by 100
			\multiply\count201 by \count102
			\advance\count205 by \count201
			%
		     \edef\@result{\number\count205}
}
\def\compute@wfromh{
		% computing : width = height * (bbw / bbh)
		\in@hundreds{\@p@sheight}{\@bbw}{\@bbh}
		%\typeout{ \@p@sheight * \@bbw / \@bbh, = \@result }
		\edef\@p@swidth{\@result}
		%\typeout{w from h: width is \@p@swidth}
}
\def\compute@hfromw{
		% computing : height = width * (bbh / bbw)
		\in@hundreds{\@p@swidth}{\@bbh}{\@bbw}
		%\typeout{ \@p@swidth * \@bbh / \@bbw = \@result }
		\edef\@p@sheight{\@result}
		%\typeout{h from w : height is \@p@sheight}
}
\def\compute@handw{
		\if@height 
			\if@width
			\else
				\compute@wfromh
			\fi
		\else 
			\if@width
				\compute@hfromw
			\else
				\edef\@p@sheight{\@bbh}
				\edef\@p@swidth{\@bbw}
			\fi
		\fi
}
\def\compute@resv{
		\if@rheight \else \edef\@p@srheight{\@p@sheight} \fi
		\if@rwidth \else \edef\@p@srwidth{\@p@swidth} \fi
}
%		
% Compute any missing values
\def\compute@sizes{
	\compute@bb
	\compute@handw
	\compute@resv
}
%
% \psfig
% usage : \psfig{file=, height=, width=, bbllx=, bblly=, bburx=, bbury=,
%			rheight=, rwidth=, clip=}
%
% "clip=" is a switch and takes no value, but the `=' must be present.
\def\psfig#1{\vbox {
	% do a zero width hard space so that a single
	% \psfig in a centering enviornment will behave nicely
	%{\setbox0=\hbox{\ }\ \hskip-\wd0}
	%
	\ps@init@parms
	\parse@ps@parms{#1}
	\compute@sizes
	%
	\ifnum\@p@scost<\@psdraft{
		\if@verbose{
			\typeout{psfig: including \@p@sfile \space }
		}\fi
		%
		\special{ps::[begin] 	\@p@swidth \space \@p@sheight \space
				\@p@sbbllx \space \@p@sbblly \space
				\@p@sbburx \space \@p@sbbury \space
				startTexFig \space }
		\if@clip{
			\if@verbose{
				\typeout{(clip)}
			}\fi
			\special{ps:: doclip \space }
		}\fi
		\if@prologfile
		    \special{ps: plotfile \@prologfileval \space } \fi
		\special{ps: plotfile \@p@sfile \space }
		\if@postlogfile
		    \special{ps: plotfile \@postlogfileval \space } \fi
		\special{ps::[end] endTexFig \space }
		% Create the vbox to reserve the space for the figure
		\vbox to \@p@srheight true sp{
			\hbox to \@p@srwidth true sp{
				\hss
			}
		\vss
		}
	}\else{
		% draft figure, just reserve the space and print the
		% path name.
		\vbox to \@p@srheight true sp{
		\vss
			\hbox to \@p@srwidth true sp{
				\hss
				\if@verbose{
					\@p@sfile
				}\fi
				\hss
			}
		\vss
		}
	}\fi
}}
\def\psglobal{\typeout{psfig: PSGLOBAL is OBSOLETE; use psprint -m instead}}
\psfigRestoreAt






				% <---- load psfig macros
\newcommand{\psinput}[1]{
  \mbox{\psfig{figure=#1}}}
\newcommand{\hgap}{\hspace{12pt}}
\newcommand{\vgap}{\\[12pt]}
\newcommand{\pscap}[2]{
  \parbox{2.01in}{
   \begin{center}
    \psinput{#1} \\ #2
   \end{center}
  }
}

% Psfig/TeX Release 1.3
% dvips version
%
% All software, documentation, and related files in this distribution of
% psfig/tex are Copyright 1987, 1988 Trevor J. Darrell
%
% Permission is granted for use and non-profit distribution of psfig/tex 
% providing that this notice be clearly maintained, but the right to
% distribute any portion of psfig/tex for profit or as part of any commercial
% product is specifically reserved for the author.
%
% $Header: /u/sheard/cvs//GPCE05MetaMLTypes/psfig.tex,v 1.1.1.1 2005/04/07 18:31:53 sheard Exp $
% $Source: /u/sheard/cvs//GPCE05MetaMLTypes/psfig.tex,v $
%
% Thanks to Greg Hager (GDH) and Ned Batchelder for their contributions
% to this project.
%
% Modified by J. Daniel Smith on 9 October 1990 to accept the
% %%BoundingBox: comment with or without a space after the colon.  Stole
% file reading code from Tom Rokicki's EPSF.TEX file (see below).

%
% from a suggestion by eijkhout@csrd.uiuc.edu to allow
% loading as a style file:
\edef\psfigRestoreAt{\catcode`@=\number\catcode`@\relax}
\catcode`\@=11\relax
\newwrite\@unused
\def\typeout#1{{\let\protect\string\immediate\write\@unused{#1}}}
\typeout{psfig/tex 1.4-dvips}


%% Here's how you define your figure path.  Should be set up with null
%% default and a user useable definition.

\def\figurepath{./}
\def\psfigurepath#1{\edef\figurepath{#1}}

%
% @psdo control structure -- similar to Latex @for.
% I redefined these with different names so that psfig can
% be used with TeX as well as LaTeX, and so that it will not 
% be vunerable to future changes in LaTeX's internal
% control structure,
%
\def\@nnil{\@nil}
\def\@empty{}
\def\@psdonoop#1\@@#2#3{}
\def\@psdo#1:=#2\do#3{\edef\@psdotmp{#2}\ifx\@psdotmp\@empty \else
    \expandafter\@psdoloop#2,\@nil,\@nil\@@#1{#3}\fi}
\def\@psdoloop#1,#2,#3\@@#4#5{\def#4{#1}\ifx #4\@nnil \else
       #5\def#4{#2}\ifx #4\@nnil \else#5\@ipsdoloop #3\@@#4{#5}\fi\fi}
\def\@ipsdoloop#1,#2\@@#3#4{\def#3{#1}\ifx #3\@nnil 
       \let\@nextwhile=\@psdonoop \else
      #4\relax\let\@nextwhile=\@ipsdoloop\fi\@nextwhile#2\@@#3{#4}}
\def\@tpsdo#1:=#2\do#3{\xdef\@psdotmp{#2}\ifx\@psdotmp\@empty \else
    \@tpsdoloop#2\@nil\@nil\@@#1{#3}\fi}
\def\@tpsdoloop#1#2\@@#3#4{\def#3{#1}\ifx #3\@nnil 
       \let\@nextwhile=\@psdonoop \else
      #4\relax\let\@nextwhile=\@tpsdoloop\fi\@nextwhile#2\@@#3{#4}}
% 
%
%%%%%%%%%%%%%%%%%%%%%%%%%%%%%%%%%%%%%%%%%%%%%%%%%%%%%%%%%%%%%%%%%%%
% file reading stuff from epsf.tex
%   EPSF.TEX macro file:
%   Written by Tomas Rokicki of Radical Eye Software, 29 Mar 1989.
%   Revised by Don Knuth, 3 Jan 1990.
%   Revised by Tomas Rokicki to accept bounding boxes with no
%      space after the colon, 18 Jul 1990.
%   Portions modified/removed for use in PSFIG package by
%      J. Daniel Smith, 9 October 1990.
%
\newread\ps@stream
\newif\ifnot@eof       % continue looking for the bounding box?
\newif\if@noisy        % report what you're making?
\newif\if@atend        % %%BoundingBox: has (at end) specification
\newif\if@psfile       % does this look like a PostScript file?
%
% PostScript files should start with `%!'
%
{\catcode`\%=12\global\gdef\epsf@start{%!}}
\def\epsf@PS{PS}
%
\def\epsf@getbb#1{%
%
%   The first thing we need to do is to open the
%   PostScript file, if possible.
%
\openin\ps@stream=#1
\ifeof\ps@stream\typeout{Error, File #1 not found}\else
%
%   Okay, we got it. Now we'll scan lines until we find one that doesn't
%   start with %. We're looking for the bounding box comment.
%
   {\not@eoftrue \chardef\other=12
    \def\do##1{\catcode`##1=\other}\dospecials \catcode`\ =10
    \loop
       \if@psfile
	  \read\ps@stream to \epsf@fileline
       \else{
	  \obeyspaces
          \read\ps@stream to \epsf@tmp\global\let\epsf@fileline\epsf@tmp}
       \fi
       \ifeof\ps@stream\not@eoffalse\else
%
%   Check the first line for `%!'.  Issue a warning message if its not
%   there, since the file might not be a PostScript file.
%
       \if@psfile\else
       \expandafter\epsf@test\epsf@fileline:. \\%
       \fi
%
%   We check to see if the first character is a % sign;
%   if so, we look further and stop only if the line begins with
%   `%%BoundingBox:' and the `(atend)' specification was not found.
%   That is, the only way to stop is when the end of file is reached,
%   or a `%%BoundingBox: llx lly urx ury' line is found.
%
          \expandafter\epsf@aux\epsf@fileline:. \\%
       \fi
   \ifnot@eof\repeat
   }\closein\ps@stream\fi}%
%
% This tests if the file we are reading looks like a PostScript file.
%
\long\def\epsf@test#1#2#3:#4\\{\def\epsf@testit{#1#2}
			\ifx\epsf@testit\epsf@start\else
\typeout{Warning! File does not start with `\epsf@start'.  It may not be a PostScript file.}
			\fi
			\@psfiletrue} % don't test after 1st line
%
%   We still need to define the tricky \epsf@aux macro. This requires
%   a couple of magic constants for comparison purposes.
%
{\catcode`\%=12\global\let\epsf@percent=%\global\def\epsf@bblit{%BoundingBox}}
%
%
%   So we're ready to check for `%BoundingBox:' and to grab the
%   values if they are found.  We continue searching if `(at end)'
%   was found after the `%BoundingBox:'.
%
\long\def\epsf@aux#1#2:#3\\{\ifx#1\epsf@percent
   \def\epsf@testit{#2}\ifx\epsf@testit\epsf@bblit
	\@atendfalse
        \epsf@atend #3 . \\%
	\if@atend	
	   \if@verbose{
		\typeout{psfig: found `(atend)'; continuing search}
	   }\fi
        \else
        \epsf@grab #3 . . . \\%
        \not@eoffalse
        \global\no@bbfalse
        \fi
   \fi\fi}%
%
%   Here we grab the values and stuff them in the appropriate definitions.
%
\def\epsf@grab #1 #2 #3 #4 #5\\{%
   \global\def\epsf@llx{#1}\ifx\epsf@llx\empty
      \epsf@grab #2 #3 #4 #5 .\\\else
   \global\def\epsf@lly{#2}%
   \global\def\epsf@urx{#3}\global\def\epsf@ury{#4}\fi}%
%
% Determine if the stuff following the %%BoundingBox is `(atend)'
% J. Daniel Smith.  Copied from \epsf@grab above.
%
\def\epsf@atendlit{(atend)} 
\def\epsf@atend #1 #2 #3\\{%
   \def\epsf@tmp{#1}\ifx\epsf@tmp\empty
      \epsf@atend #2 #3 .\\\else
   \ifx\epsf@tmp\epsf@atendlit\@atendtrue\fi\fi}


% End of file reading stuff from epsf.tex
%%%%%%%%%%%%%%%%%%%%%%%%%%%%%%%%%%%%%%%%%%%%%%%%%%%%%%%%%%%%%%%%%%%

\def\psdraft{
	\def\@psdraft{0}
	%\typeout{draft level now is \@psdraft \space . }
}
\def\psfull{
	\def\@psdraft{100}
	%\typeout{draft level now is \@psdraft \space . }
}
\psfull
\newif\if@prologfile
\newif\if@postlogfile
\def\pssilent{
	\@noisyfalse
}
\def\psnoisy{
	\@noisytrue
}
\psnoisy
%%% These are for the option list.
%%% A specification of the form a = b maps to calling \@p@@sa{b}
\newif\if@bbllx
\newif\if@bblly
\newif\if@bburx
\newif\if@bbury
\newif\if@height
\newif\if@width
\newif\if@rheight
\newif\if@rwidth
\newif\if@clip
\newif\if@verbose
\def\@p@@sclip#1{\@cliptrue}

%%% GDH 7/26/87 -- changed so that it first looks in the local directory,
%%% then in a specified global directory for the ps file.

\def\@p@@sfile#1{\def\@p@sfile{null}%
	        \openin1=#1
		\ifeof1\closein1%
		       \openin1=\figurepath#1
			\ifeof1\typeout{Error, File #1 not found}
			\else\closein1
			    \edef\@p@sfile{\figurepath#1}%
                        \fi%
		 \else\closein1%
		       \def\@p@sfile{#1}%
		 \fi}
\def\@p@@sfigure#1{\def\@p@sfile{null}%
	        \openin1=#1
		\ifeof1\closein1%
		       \openin1=\figurepath#1
			\ifeof1\typeout{Error, File #1 not found}
			\else\closein1
			    \def\@p@sfile{\figurepath#1}%
                        \fi%
		 \else\closein1%
		       \def\@p@sfile{#1}%
		 \fi}

\def\@p@@sbbllx#1{
		%\typeout{bbllx is #1}
		\@bbllxtrue
		\dimen100=#1
		\edef\@p@sbbllx{\number\dimen100}
}
\def\@p@@sbblly#1{
		%\typeout{bblly is #1}
		\@bbllytrue
		\dimen100=#1
		\edef\@p@sbblly{\number\dimen100}
}
\def\@p@@sbburx#1{
		%\typeout{bburx is #1}
		\@bburxtrue
		\dimen100=#1
		\edef\@p@sbburx{\number\dimen100}
}
\def\@p@@sbbury#1{
		%\typeout{bbury is #1}
		\@bburytrue
		\dimen100=#1
		\edef\@p@sbbury{\number\dimen100}
}
\def\@p@@sheight#1{
		\@heighttrue
		\dimen100=#1
   		\edef\@p@sheight{\number\dimen100}
		%\typeout{Height is \@p@sheight}
}
\def\@p@@swidth#1{
		%\typeout{Width is #1}
		\@widthtrue
		\dimen100=#1
		\edef\@p@swidth{\number\dimen100}
}
\def\@p@@srheight#1{
		%\typeout{Reserved height is #1}
		\@rheighttrue
		\dimen100=#1
		\edef\@p@srheight{\number\dimen100}
}
\def\@p@@srwidth#1{
		%\typeout{Reserved width is #1}
		\@rwidthtrue
		\dimen100=#1
		\edef\@p@srwidth{\number\dimen100}
}
\def\@p@@ssilent#1{ 
		\@verbosefalse
}
\def\@p@@sprolog#1{\@prologfiletrue\def\@prologfileval{#1}}
\def\@p@@spostlog#1{\@postlogfiletrue\def\@postlogfileval{#1}}
\def\@cs@name#1{\csname #1\endcsname}
\def\@setparms#1=#2,{\@cs@name{@p@@s#1}{#2}}
%
% initialize the defaults (size the size of the figure)
%
\def\ps@init@parms{
		\@bbllxfalse \@bbllyfalse
		\@bburxfalse \@bburyfalse
		\@heightfalse \@widthfalse
		\@rheightfalse \@rwidthfalse
		\def\@p@sbbllx{}\def\@p@sbblly{}
		\def\@p@sbburx{}\def\@p@sbbury{}
		\def\@p@sheight{}\def\@p@swidth{}
		\def\@p@srheight{}\def\@p@srwidth{}
		\def\@p@sfile{}
		\def\@p@scost{10}
		\def\@sc{}
		\@prologfilefalse
		\@postlogfilefalse
		\@clipfalse
		\if@noisy
			\@verbosetrue
		\else
			\@verbosefalse
		\fi
}
%
% Go through the options setting things up.
%
\def\parse@ps@parms#1{
	 	\@psdo\@psfiga:=#1\do
		   {\expandafter\@setparms\@psfiga,}}
%
% Compute bb height and width
%
\newif\ifno@bb
\def\bb@missing{
	\if@verbose{
		\typeout{psfig: searching \@p@sfile \space  for bounding box}
	}\fi
	\no@bbtrue
	\epsf@getbb{\@p@sfile}
        \ifno@bb \else \bb@cull\epsf@llx\epsf@lly\epsf@urx\epsf@ury\fi
}	
\def\bb@cull#1#2#3#4{
	\dimen100=#1 bp\edef\@p@sbbllx{\number\dimen100}
	\dimen100=#2 bp\edef\@p@sbblly{\number\dimen100}
	\dimen100=#3 bp\edef\@p@sbburx{\number\dimen100}
	\dimen100=#4 bp\edef\@p@sbbury{\number\dimen100}
	\no@bbfalse
}
\def\compute@bb{
		\no@bbfalse
		\if@bbllx \else \no@bbtrue \fi
		\if@bblly \else \no@bbtrue \fi
		\if@bburx \else \no@bbtrue \fi
		\if@bbury \else \no@bbtrue \fi
		\ifno@bb \bb@missing \fi
		\ifno@bb \typeout{FATAL ERROR: no bb supplied or found}
			\no-bb-error
		\fi
		%
		\count203=\@p@sbburx
		\count204=\@p@sbbury
		\advance\count203 by -\@p@sbbllx
		\advance\count204 by -\@p@sbblly
		\edef\@bbw{\number\count203}
		\edef\@bbh{\number\count204}
		%\typeout{ bbh = \@bbh, bbw = \@bbw }
}
%
% \in@hundreds performs #1 * (#2 / #3) correct to the hundreds,
%	then leaves the result in @result
%
\def\in@hundreds#1#2#3{\count240=#2 \count241=#3
		     \count100=\count240	% 100 is first digit #2/#3
		     \divide\count100 by \count241
		     \count101=\count100
		     \multiply\count101 by \count241
		     \advance\count240 by -\count101
		     \multiply\count240 by 10
		     \count101=\count240	%101 is second digit of #2/#3
		     \divide\count101 by \count241
		     \count102=\count101
		     \multiply\count102 by \count241
		     \advance\count240 by -\count102
		     \multiply\count240 by 10
		     \count102=\count240	% 102 is the third digit
		     \divide\count102 by \count241
		     \count200=#1\count205=0
		     \count201=\count200
			\multiply\count201 by \count100
		 	\advance\count205 by \count201
		     \count201=\count200
			\divide\count201 by 10
			\multiply\count201 by \count101
			\advance\count205 by \count201
			%
		     \count201=\count200
			\divide\count201 by 100
			\multiply\count201 by \count102
			\advance\count205 by \count201
			%
		     \edef\@result{\number\count205}
}
\def\compute@wfromh{
		% computing : width = height * (bbw / bbh)
		\in@hundreds{\@p@sheight}{\@bbw}{\@bbh}
		%\typeout{ \@p@sheight * \@bbw / \@bbh, = \@result }
		\edef\@p@swidth{\@result}
		%\typeout{w from h: width is \@p@swidth}
}
\def\compute@hfromw{
		% computing : height = width * (bbh / bbw)
		\in@hundreds{\@p@swidth}{\@bbh}{\@bbw}
		%\typeout{ \@p@swidth * \@bbh / \@bbw = \@result }
		\edef\@p@sheight{\@result}
		%\typeout{h from w : height is \@p@sheight}
}
\def\compute@handw{
		\if@height 
			\if@width
			\else
				\compute@wfromh
			\fi
		\else 
			\if@width
				\compute@hfromw
			\else
				\edef\@p@sheight{\@bbh}
				\edef\@p@swidth{\@bbw}
			\fi
		\fi
}
\def\compute@resv{
		\if@rheight \else \edef\@p@srheight{\@p@sheight} \fi
		\if@rwidth \else \edef\@p@srwidth{\@p@swidth} \fi
}
%		
% Compute any missing values
\def\compute@sizes{
	\compute@bb
	\compute@handw
	\compute@resv
}
%
% \psfig
% usage : \psfig{file=, height=, width=, bbllx=, bblly=, bburx=, bbury=,
%			rheight=, rwidth=, clip=}
%
% "clip=" is a switch and takes no value, but the `=' must be present.
\def\psfig#1{\vbox {
	% do a zero width hard space so that a single
	% \psfig in a centering enviornment will behave nicely
	%{\setbox0=\hbox{\ }\ \hskip-\wd0}
	%
	\ps@init@parms
	\parse@ps@parms{#1}
	\compute@sizes
	%
	\ifnum\@p@scost<\@psdraft{
		\if@verbose{
			\typeout{psfig: including \@p@sfile \space }
		}\fi
		%
		\special{ps::[begin] 	\@p@swidth \space \@p@sheight \space
				\@p@sbbllx \space \@p@sbblly \space
				\@p@sbburx \space \@p@sbbury \space
				startTexFig \space }
		\if@clip{
			\if@verbose{
				\typeout{(clip)}
			}\fi
			\special{ps:: doclip \space }
		}\fi
		\if@prologfile
		    \special{ps: plotfile \@prologfileval \space } \fi
		\special{ps: plotfile \@p@sfile \space }
		\if@postlogfile
		    \special{ps: plotfile \@postlogfileval \space } \fi
		\special{ps::[end] endTexFig \space }
		% Create the vbox to reserve the space for the figure
		\vbox to \@p@srheight true sp{
			\hbox to \@p@srwidth true sp{
				\hss
			}
		\vss
		}
	}\else{
		% draft figure, just reserve the space and print the
		% path name.
		\vbox to \@p@srheight true sp{
		\vss
			\hbox to \@p@srwidth true sp{
				\hss
				\if@verbose{
					\@p@sfile
				}\fi
				\hss
			}
		\vss
		}
	}\fi
}}
\def\psglobal{\typeout{psfig: PSGLOBAL is OBSOLETE; use psprint -m instead}}
\psfigRestoreAt







\usepackage{epsf}
\usepackage{epsfig}
\usepackage{graphics,color}

\newcommand{\om}{$\Omega$mega}

\setlength{\textheight}{8.5in}
\setlength{\textwidth}{6.4in}
\setlength{\oddsidemargin}{-.2in}
\setlength{\evensidemargin}{-.2in}
\setlength{\topmargin}{-0.25in}


%\setcounter   {topnumber}{3}
%\renewcommand {\topfraction}{1.0}
%\setcounter   {totalnumber}{4}
%\renewcommand {\textfraction}{0.2}
%\renewcommand {\floatpagefraction}{0.99} 
%\renewcommand {\floatsep}{0.1}
 


%\setlength{\linewidth}{0.2cm}

\begin{document}



\title{Another Look at Hardware Design Languages:\\
\large{Freeing ourselves from the tyranny of the host language}\\
\ \\
%\bf{*** DRAFT ***\\do not distribute}
}
\author{
Tim Sheard\\
Computer Science Department\\
Maseeh College of Engineering and Computer Science\\
Portland State University\\
sheard@cs.pdx.edu
}

\maketitle

\section{Introduction}\label{intro}

Using a functional language (mostly Haskell) as a host language for embedding
hardware description languages is a popular theme. Examples include Ruby,
Hydra, Hawk, and Lava. The reasons\cite{koen2001} for this approach to hardware description
language design are numerous:

\begin{itemize}

\item The syntax of the host language can be reused. There is no need to
invent a new syntax or write parsers.

\item All the machinery of the host language can be reused. Tools for the
language such as compilers and debuggers, and language features such as its
module system and type system.

\item Even the programmers who already know the host language can be reused
without re-training.

\item Finally, hardware descriptions are now first class objects in the
host language and are subject to all the properties of host language objects.
\end{itemize}

Of course there are disadvantages to such an approach, and these cannot be
overlooked. They include:

\begin{itemize}

\item The syntax of the two languages may be radically different. So users of
the hardware description language feel that they are being forced into an
unnatural programing style.

\item The features of the host language may not match the features of the
embedded language. Type systems, in particular are an example of this kind of
mismatch. In this case, unnatural or awkward embeddings must be used. These
embeddings increase the complexity of the system, and make it
harder to use and maintain.

\end{itemize}

Most proponents of the embedded language approach recognize these
disadvantages, and note that a stand-alone language (as opposed to an
embedding) with its own syntax, type system, and implementation might be
more useful. But most, pay only lip service to actually building such
an implementation. We believe it is time to actually take such a leap.

\section{Advantages of a stand-alone implementation.}

A stand-alone implementation of an HDL could provide its users with
several important advantages. These advantages include a type system
(or systems, see Section \ref{static}) specific to the domain of
hardware description, and a cleaner distinction between the multiple
stages inherent in the design of hardware. An embedded approach is
stuck with the type system of the host language. A stand-alone system
is free from this constraint. A stand-alone implementation is also
free to address the phase distinction in new and novel ways. We now
discuss some of the problems inherent in choosing an embedded
approach.

\subsection{Intertwined recursion}

In many current embedded HDLs two different forms of recursion are
implemented with the same mechanism in the host language. These two
forms play very different roles, and because they are implemented
with the same mechanism, they cannot later be easily separated. The
two forms of recursion are:

\begin{itemize}

\item Constructing large circuits with repeated patterns from smaller
ones can be described by writing inductive functions that induct over
some index that describes the connection pattern between the
patterns. This index is often something as simple as an integer, but
can be much more complicated, such as a tree.

\item Feed back loops in sequential circuits can be described by
recursive equations.

\end{itemize}

There is a phase distinction between the two kinds of recursion.
The first describes a computation in the first phase when we are
constructing the design. It is a real program. It should be inductive,
in that such programs should always terminate (we don't want to wait
forever for our design to be produced). We may, or may not, want to
enforce this.

The second describes the recurring behavior described in the design.
This behavior ``runs" in the second phase when the design is
implemented. As far as the HDL is concerned, the second is a piece of
data. These are truly recursive, in that they define infinite behavior
patterns, but they should always be captured by a finite description
(after all, we cannot build infinitely large hardware devices).

Current embedded approaches mix these two types of recursion in ways
that are hard to separate. They are both implemented by recursion in
the host language. This has two unfortunate consequences. First it
forces us to use a lazy semantics. This may or may not be a good
thing, but we should have a choice. Second, hardware designs with
feedback loops always create infinite structures (why we need
laziness). This forces us to use ``tricks" to observe the sharing in
the cyclic heap objects that represent the data, so that we can recover the
finite description (of the infinite behavior) we should have built
in the first place.
 
\subsection{Type Systems that are to Weak.}

An embedded approach forces the HDL to share the host languages type
system. Even the rich type system of Haskell (the most common host
language) with its rich class mechanism and rank-N polymorphism is
too weak to capture some simple static properties of HDL programs.
Examples include:

\begin{itemize}

\item{\bf Typed HDL Object Descriptions.} HDL programs produce
hardware descriptions. Such descriptions are data in the HDL, but
they also represent programs (or hardware). Such descriptions could
have types, and we would like to ensure that the designer only
produce well-typed designs. Current languages use {\em phantom-types}
\cite{Hinze:03:Phantom,fluet-02,Leijen:2000:DSE} to help ensure that we only {\em build} well typed designs,
but phantom types are truly phantom when {\em analyzing} designs.
Type information is lost in the analysis phase and cannot be
recovered.

\item{\bf Structures with Fixed Width.}
Hardware is full of structures with fixed width: 32-bit registers,
64-bit busses, 8-bit adders. Can we specify generic algorithms that
manipulate n-bit entities, then specialize them to a fixed width, and be
sure we never mix two similar objects with different widths?

\item{\bf Powers of 2.} Many abstract designs (such as a butterfly
circuit) can be precisely described using regular patterns only
when the size of the design is a power of 2, (or in some examples
a perfect square). Such a property can be statically enforced, but
the type system of the host language is usually too weak.

\item{\bf Interface Types.} Many signals have directionality
(input v.s. output). A static type system could prevent the error
of connecting two inputs, or leaving a signal dangling.

\item{\bf Protocol Types.} Many hardware descriptions must behave
according to a certain protocol. For example, a simple
handshake protocol depends on {\em request} and {\em acknowledge}
signals strictly alternating. Some protocols look more like
complex automata than types. Can a static type system enforce
such behavior?

\item{\bf Counting the Delay in a Circuit.} Good designs follow patterns.
For example, some design patterns insist that every cyclic loop nest
includes a delay. Can we statically enforce such design patterns?

\end{itemize}


\section{The next HDL.}

A stand-alone implementation of a HDL should have a number of
desirable properties. We discuss these properties at a high level of
abstraction here, and then in the ensuing sub-sections, describe how
these properties might be implemented.


\begin{itemize} 

\item{\bf Descriptions as first class data.} Hardware descriptions
are both data and programs. But, it is important that they be first
class data structures. Their very purpose is to be manipulated as
data. Anything that makes this hard is in direct competition with
their primary role. In particular, they should not be encoded as
functions or as cyclic heap structures, because this makes it
difficult, if not impossible, to perform the analyses they were
designed for.

\item{\bf Typed representations.} Hardware descriptions should be
typed. It should be possible to describe rich abstractions such as
tuples, and busses of width {\it n}. These abstractions should be
available in all stages. In particular it should be possible to
construct both tuples of circuits,  and circuits of tuples (as well as
other richer data structures). The type system should apply in the
analysis phase as well as the construction phase of the descriptions.

\item{\bf Special syntactic sugar.} A stand-alone implementation
should support syntactic sugar to make it easy to capture  
hardware descriptions.

\item{\bf Multiple static analyses.} A stand-alone implementation can
support multiple static analyses, one for each important static
property. The specification and checking of different analyses
should be separable to minimize the conceptual effort needed to
``debug" errors when they occur.

\item{\bf Multiple interpretations.} A hardware description has
multiple interpretations. The same piece of data should be
interpretable in multiple ways.

\end{itemize}

Haskell makes an excellent host language. Its support for abstraction
and first class functions make it an excellent meta-language to
describe the manipulation of HDL object-language programs. But, its
type system is too weak, and its syntax is too restrictive to
describe the phase distinction between the meta-programs and the
hardware descriptions as data. For these reasons, we propose building
a stand-alone HDL with much the same look and feel of Haskell, but
with features that allow it to overcome these
restrictions. Imagine an Haskell-like language with a stronger type
system, a slightly altered syntax, and features designed for
describing hardware. We have, and this paper describes what we came up with.

\subsection{Generalized Algebraic Datatypes}

A hardware description is a piece of data. We will represent it as a {\em
type-indexed} algebraic datatype. Algebraic data types (ADT's) are an
abstraction available in many functional languages (such as Haskell, ML, or
O'Caml) that allow users to define inductively formed structured data. A
type-indexed algebraic datatype is an extension to ADTs we will explain in a
moment.

ADTs generalize other forms of structuring data such as enumerations,
records, and tagged variants. For example,
in Haskell we might write:

\begin{verbatim}
-- An enumeration type
data Color = Red | Blue | Green
-- A record structure
data Address = MakeAddress Number Street Town 
-- A tagged Variant
data Person = Teacher [Class]  | Student Major
\end{verbatim}

We assume the reader has a certain familiarity with ADTs. In
particular that values of ADTs are constructed by {\em constructors}
and that they are taken apart by the use of {\em pattern matching}.
Two valuable features of ADTS are their ability to be {\em
parameterized by a type} and to be {\em recursive}. A simple example
that employs both these features is:
\begin{verbatim}
data Tree a 
   = Fork (Tree a) (Tree a) 
   | Node a 
   | Tip
\end{verbatim}
This declaration defines the polymorphic {\tt Tree} type constructor. Example tree
types include {\tt (Tree Int)} and {\tt (Tree Bool)}. In fact the
type constructor {\tt Tree} can be applied to any type whatsoever.
Note how the constructor functions ({\tt Fork},\ {\tt Node}) and constants ({\tt Tip})
are given polymorphic types.

\noindent
{\tt Fork :: forall a . Tree a -> Tree a -> \underline{Tree a}}\\
{\tt Node :: forall a . a -> \underline{Tree a}}\\
{\tt Tip :: forall a . \underline{Tree a}}

When we define a parameterized algebraic datatype, the syntactic
formation rules of the {\tt data} declaration
enforce the following restriction: The range of every constructor function,
and the type of every constructor constant must be a polymorphic instance of
the new type constructor being defined. Notice how the constructors for {\tt
Tree} all have range ({\tt Tree {\it a}}) with a polymorphic type variable
{\it a}.  We can remove this restriction by strengthening the type system of the
language. The restriction is {\bf syntactically} removed by the following mechanism. In a 
{\tt data} declaration, rather than leaving the range of the constructor functions
implicit, the user replaces the enumeration of the constructors and the
type of their domains with a full explicit typing of each constructor.
The only restriction being that the range must be some instance of
the type being defined. {\bf Semantically}, this requires new type checking
rules. This simple extension allows us to build indexed-datatypes. A simple example
with relevance to hardware design can be found in Figure \ref{simple}.

\begin{figure}
\hrule
{\small
\begin{verbatim}

data Exp ::  *0 ~> *0 where
  O      :: Exp Bit                             -- Bits
  I      :: Exp Bit
  And    :: Exp Bit -> Exp Bit -> Exp Bit
  Or     :: Exp Bit -> Exp Bit -> Exp Bit
  Not    :: Exp Bit -> Exp Bit
  Xor    :: Exp Bit -> Exp Bit -> Exp Bit
  
  Lit    :: Int -> Exp Int                      -- Integers
  Equal  :: Exp Int -> Exp Int -> Exp Bit
  Plus   :: Exp Int -> Exp Int -> Exp Int
  Minus  :: Exp Int -> Exp Int -> Exp Int
  Times  :: Exp Int -> Exp Int -> Exp Int
\end{verbatim}}
\vspace*{-0.2in}
\caption{Type-indexed ADT capturing simple circuit descriptions.
Notice how the full type of every constructor is explicitly supplied, and how the range of
each constructor is not necessarily a polymorphic instance of {\tt Exp}.}\label{simple}
\hrule
\end{figure}

Circuits are nothing more than a data structure. They are built
by the constructor functions of their datatype. Simple combinational
circuits can be described as values of the {\tt Exp}
datatype which is defined in Figure \ref{simple}. The type constructor
{\tt Exp} takes a type as an argument (which we call its type index). 
The type index of {\tt Exp} tells us what kind of circuit it encodes --
{\tt (Exp Bit)} if it represents a combinational circuit of bits,
and {\tt (Exp Int)} if it represents a combinational circuit of integers.
To fully understand this we need to take a short side trip into the
world of kinds.

\subsection{Values, Types, Kinds, and Sorts}


The kind of {\tt Exp} is \verb+*0 ~> *0+.
Kinds are similar to types in that, while types classify values, kinds classify
types. We indicate this by the overloaded {\em classifies} relation ({\tt ::}). For example:
{\tt 5::Int}, and {\tt Int::*0 }. We say {\tt 5} is classified by {\tt Int}, and {\tt Int}
is classified by {\tt *0} (star-zero). The kind {\tt *0} classifies all
types that classify values (things we actually can compute).

{\bf New types} are introduced by the {\bf data} declaration along with the
value constructors that produce values of that type. {\bf New kinds}
are introduced by the {\bf kind} declaration along with the
type constructors that produce types of that kind. Several declarations,
for types and kinds of interest in this paper follow:
\begin{verbatim}
kind Nat = Z | S Nat

data Nat':: Nat ~> *0 where
   Z:: Nat' Z 
   S::  (Nat' x) -> Nat' (S x)

kind Row (x :: *0) = RNil | RCons x (Row x)

kind Ctype = Comb | Seq 
\end{verbatim}

The {\tt Nat} declaration introduces two new
type constructors {\tt Z} and {\tt S} which encode the natural
numbers at the type level.
The type {\tt Z} has kind {\tt Nat}, and {\tt S} has kind {\tt Nat \verb+~+>
Nat}. The type {\tt S} is a type constructor, so it has a higher-order kind. We
indicate this using the classifies relation as follows:
\begin{verbatim}
Z :: Nat
S :: Nat ~> Nat
Nat :: *1
\end{verbatim}

The classification {\tt Nat::*1} indicates that {\tt Nat} is at the same
``level" as {\tt *0} --- they are both classified by {\tt *1}.
The kinds {\tt Ctype} and {\tt Row} are also kinds different from {\tt *0}. We summarize the relationship
between values, types, kinds, and sorts in the table below.\\

{\small
\begin{tabular}{|ccccccc|}
{\scriptsize value name space} & $\mid$ & {\scriptsize type name space} \\ \hline
value & $\mid$ & type & $\mid$ & kind &$\mid$ & sort \\ \hline
5     &::& Int  &::& *0  &::& *1  \\ \hline
[2]   &::& [Int]&::& *0  &::& *1  \\ \hline
      &  & []   &::& *0 $\leadsto$ *0 &::& *1\\ \hline
      &  &      &  & Nat &::& *1\\ \hline  
      &  &  Z   &::& Nat &::& *1\\ \hline   
      &  &  S   &::& Nat $\leadsto$ Nat &::& *1\\ \hline 
      &  & Nat'   &::& Nat $\leadsto$ *0 &::& *1 \\ \hline  
  Z   &::& Nat' Z &::& * 0               &::& *1 \\ \hline   
  S   &::& Nat' m $\leadsto$ Nat' (S m)  &::& *0 &::& *1\\ \hline        
      &  & Exp  &::&  *0 $\leadsto$ * 0 &::& *1\\ \hline
      &  &      &  & Ctype   &::& *1\\ \hline
      &  & Comb &::& Ctype   &::& *1\\ \hline
      &  & Seq  &::& Ctype   &::& *1\\ \hline
      &  &      &  & Row  &::& *1 $\leadsto$ *1\\ \hline
      &  & RCons&::& k $\leadsto$ Row k $\leadsto$ Row k &::& *1\\ \hline
      &  & RNil &::& Row k &::& *1\\ \hline      
\end{tabular}} \\

Note that the table is somewhat incomplete. Not all types classify values.
For example there are no values of type \verb+[]+ (list) or \verb+Z+, but
there are values of type \verb+[Int]+. The same holds at the kind level.
Not all kinds classify types. For example, there are no types classified
by \verb+Row+, but {\tt RNil} could be classified by \verb+(Row Int)+. As
is done in Haskell, there are two name spaces. One name space names
values, and the other name space names types, kinds, sorts etc. Two
objects can have the same name if they live in different name spaces.

We put this principal to work in the {\tt Nat'} datatype. {\tt Nat'} is a
reflection of the {\tt Nat} type into the value world where the value
constructors of the {\tt data} declaration for {\tt Nat'} mirror the type
constructors in the {\tt kind} declaration of {\tt Nat}, and the type index
of {\tt Nat'} exactly captures the kind reflected. We are able to use the
same names for the constructor functions of {\tt Nat'} as the type
constructors of {\tt Nat} because the name space for values and types is
separate. {\tt Nat'} is a singleton type. There is only one element of any
singleton type. For example only {\tt S (S Z)} inhabits the type {\tt Nat' (S
(S Z))}. Using the same names for the type constructors of kind {\tt Nat}
({\tt S} and {\tt Z}),
and the constructor functions of {\tt data} type {\tt Nat'} ({\tt S} and {\tt Z}), is deliberate.
This is possible because of the separate name spaces for values and types.


We will find natural numbers at the type level (and their {\tt Nat'}
reflections at the value level) to be so useful we introduce some syntactic
sugar for constructing such types (or values). For example:
{\tt \#0 = Z}, and {\tt \#1 = (S Z)}, and {\tt \#2 = (S (S Z))} etc.
We also support the syntax {\tt \#(n + x) = (S$_1$(S$_2$ ... (S$_n$ x)))}.
This is one advantage of not using an embedded approach, the ability to
introduce new syntax for commonly used idioms. This is analogous to
the use of square brackets to describe lists in addition to the use
of the constructor \verb+(:)+.

\subsection{Returning to Hardware Descriptions as First Class Type-Indexed Data}

A value of type {\tt (Exp Bit)} is interpreted as a 
combinational circuit with type {\tt Bit}. A simple example of a
function that manipulates circuits is:

\begin{verbatim} 
halfAdd :: (Exp Bit) -> (Exp Bit) -> (Exp Bit, Exp Bit)
halfAdd a b = (Xor a b,And a b) 
\end{verbatim} 

We can give meaning to such a data structure by writing a meaning
or evaluation function for values of type {\tt (Exp t)}. Note
how the meaning of a value of type {\tt (Exp t)} is a value of type {\tt t}.
\begin{verbatim}
data Bit = Low | High

eval :: Exp t -> t
eval O = Low
eval I = High
eval (And x y) = andB (eval x) (eval y)
eval (Or x y) = orB (eval x) (eval y)
eval (Xor x y) = xorB (eval x) (eval y)
eval (Not x) = notB (eval x)
eval (Lit n) = n
eval (Plus x y) = (eval x) + (eval y)
eval (Minus x y) = (eval x) - (eval y)
eval (Times x y) = (eval x) * (eval y)
eval (Equal x y) = if (eval x)==(eval y) then High else Low
\end{verbatim}

Note how the types of the constructor functions of {\tt Exp} play a role when
they are analyzed as well when they are constructed. The types are not
phantom here. Type-indexed ADTs are an instance of a more general mechanism
called Generalized Algebraic Datatypes (GADTs). The typing rules for GADT's
are beyond the scope of this short paper, but they are well
understood\cite{wobbly,Sheard:2004:LF}.

\subsection{Composite Data at the Circuit Level}

The function
{\tt halfAdd} produces a pair of circuits of type {\tt Bit} from two
circuits of type {\tt Bit}. Its type is \verb+halfAdd :: Exp Bit -> Exp Bit -> (Exp Bit, Exp Bit)+.
Notice how it returns a tuple of two circuits. Can we build circuits which
encode tuples and other data structures inside the circuit? Such a function might have
type: \verb+Exp Bit -> Exp Bit -> Exp(Bit,Bit)+. This would allow designers
to build structured circuits. To do this we would just enrich the 
GADT describing circuits.
\begin{verbatim}
data Vec:: *0 ~> Nat ~> *0 where
  Empty:: Vec a Z
  Add:: a -> Vec a m -> Vec a (S m)

data Exp ::  *0 ~> *0 where
  . . .
  Pair   :: Exp a -> Exp b -> Exp (a,b)
  CEmpty :: Exp (Vec a Z)
  CAdd   :: Exp a -> Exp (Vec a m) -> Exp (Vec a (S m))
\end{verbatim}
Now we can build circuits of four
different kinds of data: bits, integers, heterogeneous products, and
homogeneous vectors of fixed width. Here are some examples
\begin{verbatim}
#2         :: Nat' #2           -- (S (S Z))
I          :: Exp Bit           -- I
(Lit 5)    :: Exp Int           -- (Lit 5)
#(I,Lit 5) :: Exp (Bit,Int)     -- (Pair I (Lit 5))
#[I,O,I]   :: Exp (Vec Bit #3)  -- (CAdd I (CAdd O (CAdd I CEmpty)))
\end{verbatim}

In addition to the new constructors for \verb+Exp+ ({\tt Pair}, {\tt CAdd},
and {\tt CEmpty}) we are also illustrating a considerable amount of syntactic
sugar as well. The syntax on the left expands to the calls to the
constructor function expressions after the comment marks ({\tt --}) on the right. Notice
how we use syntactic sugar to model object-level numbers, tuples, and
vectors using the meta-level syntax, just by prefixing them with the object level
marker hash (\verb+#+). This is just another thing we couldn't do in an embedded
approach. Here
is second version of the half-adder that returns a single circuit
representing a pair of wires.

\begin{verbatim} 
halfAdd2 :: (Exp Bit) -> (Exp Bit) -> Exp(Bit,Bit)
halfAdd2 a b = #(Xor a b,And a b) 
\end{verbatim}

We can also mix and match between meta-level and object-level data.
Consider constructing a full adder from a n-bit adder

\begin{verbatim}
bitAdder :: Exp Bit -> Vec (Exp Bit) n -> (Vec (Exp Bit) n, Exp Bit)
bitAdder carry_in Empty = (Empty,carry_in) 
bitAdder carry_in (Add x xs) = (Add y ys,carry_out)
  where #(y,carry) = halfAdd2 carry_in x
        (ys,carry_out) = bitAdder carry xs
\end{verbatim}
Note how the {\tt halfAdd2} returns a structured circuit, which we take apart
using pattern matching, and then reassemble into a meta-level pair. Note also that the size of the
adder is indicated in the type of the function that constructs it!

\subsection{Binding and Naming Circuits}
Reconsider the half-adder function.
\begin{verbatim} 
halfAdd :: (Exp Bit) -> (Exp Bit) -> (Exp Bit, Exp Bit)
halfAdd a b = (Xor a b,And a b) 
\end{verbatim} 
Note how the inputs {\tt a} and {\tt b} are duplicated by the function.
If they represented large circuits the size of the output might be
larger than one might naively expect. Of course duplicating circuits
is not always bad. One might want to duplicate a small circuit to improve
the delay in a larger one, of which it is a part. The point is, duplication
is neither inherently good nor bad. What the designer needs is control over
naming and duplication.

Introducing naming into the object-level means the object-level data
structure must have binding constructs. Binding constructs are notoriously
hard to use if exposed naively to the user. The study of meta-programming
systems has provided a number of excellent solutions to this problem. These
solutions require some syntactic sugar, and are thus not available in an
embedded approach. Fortunately, we have avoided that trap by our initial
design choices. Consider
\begin{verbatim} 
halfAdd :: (Exp Bit) -> (Exp Bit) -> Exp(Bit,Bit)
halfAdd a b = circuit #(carry,sum) 
  where carry = Xor a b
        sum = And a b
\end{verbatim} 
Here we introduce the {\tt circuit} syntactic sugar for introducing
named circuits. It looks and behaves a like a meta-level {\tt where}
clause, except it builds an {\tt Exp} data-structure and names 
some of its sub-expressions. Further uses
of the named sub-expressions will mention the name of the structure
not their value, so duplication can be controlled.
To implement this we again need to enrich the GADT that represents circuits
by adding variables, patterns, and binding constructs. 

\begin{verbatim} 
data Pat :: *0 ~> *0 where
  PVar :: String -> (Rep t) -> Pat t
  PPair :: Pat a -> Pat b -> Pat (a,b)
  
data Exp ::  *0 ~> *0 where
  . . .
  Var    :: String -> Rep t -> Exp t
  Let    :: Decs t -> Exp t
  
Decs :: *0 ~> *0 
data Decs t 
   = In (exp t) 
   | exists s . Decl (Pat s) (Exp s) (Decs t)
\end{verbatim}
We introduce the {\tt Pat} type, and add two new constructors to the {\tt Exp} datatype ({\tt Var}, and {\tt
Let}), and a new auxiliary datatype {\tt Dec} to represent declarations. So
the definition of the circuit in the half adder function above
might expand into the data structure:
\begin{verbatim}
Let (Decl (PVar "carry" Bit) (Xor a b)
    (Decl (PVar "sum" Bit) (And a b)
    (In #(Var "carry" Bit,Var "sum" Bit))))
\end{verbatim}
Because we want our variables to be typed, we must introduce the 
representation for types {\tt Rep}.
\begin{verbatim}
data Rep:: *0 ~> *0 where
  Bit:: Rep Bit
  Int:: Rep Int
  Prod:: (Rep a) -> (Rep b) -> Rep (a,b)
  Vec:: Rep a -> Nat' n -> Rep(Vec a n)
\end{verbatim}
A value of type {\tt (Rep t)} is a value isomorphic in
shape to the type {\tt (t)}, but is a value rather than a type.
Like {\tt Nat'}, {\tt Rep} is a singleton type, because there is exactly one value
with type {\tt Rep t} and that value is isomorphic in shape to the type {\tt t}.
For example:
\begin{verbatim}
Bit :: Rep Bit
Prod Bit Int :: Rep (Bit,Int)
Vec Int #1 :: Rep (Vec Int #1)
\end{verbatim}
This allows users to create run-time values that encode
types. By storing these singletons in {\tt Var} and {\tt PVar} objects
the user can encode the type of the variables they are representing.
Again, this uses the trick of naming the constructors of the {\tt Rep}
values with the same name as the types they represent, by exploiting
the distinction between the value and type name space.

Patterns are a subset of expressions, essentially those expressions
constructed from only variables and products. Many times it is useful
to convert between patterns and this subset of expressions.

\begin{verbatim}
exp2pat :: Exp t -> Pat t
exp2pat (Var nm t) = PVar nm t
exp2pat (Pair x y) = PPair (exp2pat x) (exp2pat y)

pat2exp :: Pat t -> Exp t  
pat2exp (PVar nm t) = Var nm t
pat2exp (PPair x y) = Pair (pat2exp x) (pat2exp y)
\end{verbatim}

Including {\tt Rep}s inside of variables and patterns makes it possible
to compute the type of any expression. The function {\tt typeOf} is
an example of a function that analyzes an {\tt Exp} to compute some
property of that expression.

\begin{verbatim}
typeOf :: Exp t -> Rep t
typeOf (Var s t) = t
typeOf (If x y z) = typeOf y
typeOf O = Bit
typeOf I = Bit
typeOf (And x y) = Bit
typeOf (Or x y) = Bit
typeOf (Xor x y) = Bit
typeOf (Not x) = Bit
typeOf (Lit n) = Int
typeOf (Plus x y) = Int
typeOf (Minus x y) = Int
typeOf (Times x y) = Int
typeOf (Equal x y) = Bit
typeOf (Pair  x y) = Prod (typeOf x) (typeOf y)
typeOf (CAdd x y) = case typeOf  y of Vec _ n -> Vec (typeOf x) (S n)
typeOf (Let (In e)) = typeOf e 
typeOf (Let (Decl p e ds)) = typeOf (Let ds) 
typeOf (RLet ds) = typeOf (last ds)
  where last (Decl p e ds) = last ds
        last (In e) = e
\end{verbatim}

Note, that because all {\tt Exp}s are well formed, in some instance
it is not necessary to compute the type of a an expression's
sub-expressions to compute its type.

\subsection{Sequential Circuits}

Feed back loops capture recursion inside the circuit and are semantically
different than the recursion we used in {\tt bitAdder}. There the recursion
indicated that the circuit producing function was being called repeatedly,
not that the circuit being built had a feed back loop. Sequential circuits
also require several new primitives to capture the semantics of a circuit as
an infinite sequence of values one for each clock cycle. These primitive
include the {\tt (Const n)} operator that returns a sequential circuit that
returns {\tt n} on all clock cycles, and the {\tt (Delay n x)}, which returns
{\tt n} in the first clock cycle, and then returns the values returned by {\tt x},
only delayed by one clock cycle. To do this, we again enrich the {\tt Exp}
GADT.

\begin{verbatim}
 data Exp ::  *0 ~> *0 where
   . . .
   Delay  :: a -> Exp a -> Exp a
   Const  :: a -> Exp  a
   RLet   :: Decs t -> Exp t       -- Like "Let" but denotes recursive binding
\end{verbatim}

We can indicate feed back loops by expanding the
use of named circuits. If we allow the definition of names
inside of a {\tt circuit} to be mutually recursive, then feed back loops
become possible with no additional syntax. Consider:
\begin{verbatim}
toggle change = 
  circuit  out2  where
     out1 = Delay Low out2
     out2 = Xor change out1
\end{verbatim}

This is transformed by the compiler into a definition that replaces the
recursive equations by an expression built by using the constructor
functions of the {\tt Exp} datatype. These constructor functions
make explicit the recursion implicit in the equations. There
is considerable analysis going on here, and this is one reason
why an embedded approach comes up short. The expanded declaration
produced is as follows:

\begin{verbatim}
toggle change = 
  (RLet (Decl (PVar "out1" Bit) (Delay Low (Var "out2" Bit)) 
        (Decl (PVar "out2" Bit) (Xor (Input "x" Bit) (Var "out1" Bit)) 
        (In (Var "out2" Bit)))))
\end{verbatim}

The translation process is the same as for named circuits, except here the 
naming is mutually recursive, as expressed by the use of {\tt RLet}
instead of {\tt Let}. The analysis discovers the recursion and
inserts the {\tt RLet}. Note also, that the {\tt Rep} types inside of
{\tt Var} and {\tt PVar}, which is only implicit in the unexpanded 
{\tt circuit} syntactic sugar, is made explicit by the expansion process.
The type checker reconstructs these so that the user does not have to.
It would be possible to have only one kind of
naming, where all naming is potentially recursive. We intend to
explore that design decision in the near future.

\subsection{The Importance of Syntactic Sugar}

We wish to emphasize that users of the HDL need know nothing about the constructors
for declarations ({\tt In} and {\tt Decl}), variables ({\tt Var} and {\tt PVar}),
and binding ({\tt Let} and {\tt RLet}). These constructors are
completely abstracted by the {\tt circuit} syntactic sugar. In an embedded
approach none of this would be possible.

\subsection{The Truth Revealed}\label{truth}

The story of named circuits is plausible, but it doesn't quite stand
up to inspection. First, it is susceptible to inadvertent name capture.
Second, it doesn't really solve the duplication problem. To illustrate
the first point consider.
\begin{verbatim}
f x = circuit (halfAdd carry x)
       where carry = I
\end{verbatim}
This expression uses nested {\tt circuit} syntax. It expands to nested
{\tt Let} expressions, but they both employ the same local name {\tt carry}.
\begin{verbatim}
f x = 
  Let (Decl (PVar "carry" Bit) I
      (In (Let (Decl (PVar "carry" Bit) (Xor (Var "carry" Bit) x)
               (Decl (PVar "sum" Bit) (And (Var "carry" Bit) x)
               (In #(Var "carry" Bit,Var "sum" Bit)))))))
\end{verbatim}
Note how the second {\tt Decl} for {\tt carry} hides the first one.
To illustrate the second problem consider.
\begin{verbatim}
f x = And x x
g x y = f (halfAdd x y)
\end{verbatim}

This just duplicates the {\tt Let} expression in each argument of the {\tt
And} constructor. What we need is to push the {\tt And x x} inside the body
of the {\tt Let} expression. Fortunately, the meta-programming community has
developed standard solutions to both of these problems.  The first is solved
by a generative approach to naming. I.e. every time a {\tt circuit}
expression is expanded, a fresh version for each name is generated. This
approach has been well studied in both MetaML\cite{MetaOCaml-paper,TS00,T00} and Template
Haskell\cite{sheard02template}. There are even techniques for proving properties of programs
generated in this way\cite{Taha99}. The second is solved by the use of
continuations. The great thing about both these approaches is that users can
be completely unaware of the implementation mechanism. Using these approaches
the first example expands to:

\begin{verbatim}
f x = 
  Let (Decl (PVar "carry_1" Bit) I
      (In (Let (Decl (PVar "carry_2" Bit) (Xor (Var "carry_1" Bit) x)
               (Decl (PVar "sum_3" Bit) (And (Var "carry_1" Bit) x)
               (In #(Var "carry_2" Bit,Var "sum_3" Bit)))))))
\end{verbatim}
And the second: \verb+f (halfAdd x y)+  expands to:
\begin{verbatim}
Let (Decl (PVar "carry_1" Bit) (Xor x y)
    (Decl (PVar "sum_2" Bit) (And x y)
    (In (And (Var "carry_1" Bit) (Var "sum_2" Bit)))))
\end{verbatim}
We discuss the implementation details of a generative approach to name generation
and the use of continuations in more detail in Section \ref{implement}

\section{The Signal Datatype}\label{implement}
To handle possible inadvertent name capture and to control duplication, we
introduce the {\tt Sig} type constructor. A signal is just an abstraction
which produces an {\tt Exp} datatype, but which hides a few details. These
details solve the problems discussed in Section \ref{truth}.
Signals employ a combination of two techniques: the state monad, and the
continuation monad. It is useful to discuss each technique in isolation.

\paragraph{The state monad for name generation.} The generative approach
to naming must generate new, unique names when a named signal is created.
To do this we need a little bit of state from which to generate new names.
For example, if we were only interested in name generation we might
define {\tt Sig} as follows:
\begin{verbatim}
data Sig t = Sig (Int -> (Exp t,Int))

and :: Sig Bit -> Sig Bit -> Sig Bit
and (Sig f) (Sig g) = Sig h  
    where h n = let (x,n1) = f n
                    (y,n2) = g n1
                in (And x y,n2)
\end{verbatim}
Each of the operators of primitive circuits ({\tt And}, {\tt Or}, {\tt Xor}, etc.)
are easily lifted to the signal type as illustrated for the {\tt And}
case. Each signal (implicit in the function {\tt f} and {\tt g}), is
called, possibly generating a new integer ({\tt n1} and {\tt n2}) which
is propagated unchanged inside the definition of {\tt And} because it
does no naming. To illustrate how generative naming is performed
consider the simple named circuit.
\begin{verbatim}
f :: Sig Bit -> Sig Bit -> Sig Bit
f x y = circuit z where z = and x y
\end{verbatim}
This might be expanded to 
\begin{verbatim}
f x y = Sig h
  where h n = let (z,zpat) = gensym n "z" Bit
                  (ans,n2) = runSig (and x y) (n+1)
              in (Let (Decl zpat (In ans)),n2)
                            
runSig (Sig f) n = f n

gensym :: Int -> String -> Rep t -> (Exp t,Pat t)
gensym s t = (Var name t,PVar name t)
  where name = s++show n              
\end{verbatim}
The current integer ({\tt n}) is used to generate a new name, and the
body of the {\tt circuit} is ``run" with the next integer ({\tt n+1}). The resulting
{\tt Exp} ({\tt ans}), and the ultimate integer ({\tt n2}) are used
to assemble the result.

\paragraph{The continuation monad.} To avoid duplication, naming a 
sub-circuit must propagate the name of the circuit to all possible
future circuits. To do this we use continuations.
For example, if we were only interested in name propagation we might
define {\tt Sig} as follows:
\begin{verbatim}
data Sig a = Sig (forall c . (Exp a -> Decs c) -> Decs c)

and :: Sig Bit -> Sig Bit -> Sig Bit
and (Sig f) (Sig g) = Sig h  
    where h k = f (\ x -> 
                g (\ y -> k(And x y)))
\end{verbatim}
Here we lift each primitive into the {\tt Sig} type as illustrated
by the definition of {\tt and}. Note how the continuation {\tt k}
is pushed inside on the result of anding together the result of
running the two input computations. 
Now,  to propagate new names to all future computations we proceed
by using the new {\tt name} operation.
\begin{verbatim}
name :: Pat a -> Sig a -> Sig a
name p (x@(Sig f)) = Sig h
  where h k = Decl p (Let (f In)) (k (pat2exp p))
\end{verbatim}
Given a name {\tt p} and a computation {\tt x} we want to name in all
future circuit expressions, we start by building a new declaration
for {\tt p}. Note how all future signals will be in the scope
of this name since these signals are implicit in the continuation {\tt k},
and {\tt k} expands in the body of the new declaration. The tricky part
is to run {\tt x} in the simplest continuation {\tt In}, and then
use the resulting declarations to build an {\tt Exp} whose
value will be bound to the name {\tt p}. Note that when {\tt f} is applied
to a continuation, it  always expands into its own sequence of {\tt Decl}s.
Thus the shape of the result will be something like.

\begin{verbatim}
Decl p (Let (Decl_1 p_1 e_1 ... (Decl_n p_n e_n (In f_result))))
       (k (Var nm ty))
\end{verbatim}
In many interpretations of {\tt Exp} this is equivalent to the following
expression which normalizes the form of declarations:
\begin{verbatim}
(Decl_1 p_1 e_1 ... 
(Decl_n p_n e_n
(Decl p f_result
(In (k (Var nm ty))))))
\end{verbatim}
so it might be ok to define a {\tt name} function that builds declarations
in normal form.
\begin{verbatim}
name :: Pat a -> Sig2 a -> Sig2 a
name p (Sig2 f) = Sig2 h
  where h k = f (\ res -> Decl p res (k (pat2exp p)))
\end{verbatim}

\paragraph{The combined approach.} So while this propagates names, it doesn't
generate fresh ones. To do this we need to combine name generation and
continuations. Thus we might consider defining a signal as follows
\begin{verbatim}
Sig a = (forall c . Int -> (Int -> Exp a -> Decs c) -> Decs c)
\end{verbatim}
It turns out that it is more convenient to define a slightly more
generic type, which is a monad, and then to define a signal as an
instance of this monadic type.
\begin{verbatim}
data M a = M (forall c . Int -> (Int -> a -> Decs c) -> Decs c)
unM (M x) = x
runM (M f) = f 0 (\ s -> \x -> x)

return x = M h where h s k = k s x
bind (M a) f = M h where h s k = a s (\ s' b -> unM (f b) s' k)
fail s = error s
monadM = Monad return bind fail
next s = M h where h n k = k (n+1) (s++"_"++show n) 

data Sig a = Sig (M (Exp a))
unSig (Sig x) = x
\end{verbatim}
All of the primitive circuit constructors are easily lifted to this new
type. We follow the convention that the lifted operators (except in a few
cases) use the same name as the {\tt Exp} constructor functions,
but don't use an uppercase letter as the first character in their name.
\begin{verbatim}
lift0 :: Exp a -> Sig a
lift0 x = Sig(return x)

lift1 :: (Exp a -> Exp b) -> (Sig a -> Sig b)
lift1 f (Sig x) = 
  Sig(do { a <- x; return(f a)}) where monad monadM

lift2 :: (Exp a -> Exp b -> Exp c) -> (Sig a -> Sig b -> Sig c)
lift2 f (Sig x) (Sig y) = 
  Sig(do { a <- x; b <- y; return(f a b)}) where monad monadM
 
mux = lift3 If
pair = lift2 Pair
cempty = lift0 CEmpty
cadd = lift2 CAdd
o = lift0 O
i = lift0 I
and = lift2 And
or = lift2 Or
xor = lift2 Xor
not1 = lift1 Not
xor2 = lift2 Xor
lit n = lift0(Lit n)
equal = lift2 Equal
plus = lift2 Plus
minus  = lift2 Minus
times  = lift2 Times
delay a = lift1 (Delay a)
const a = lift0 (Const a)
\end{verbatim}
Because the new definition of signals includes a mechanism for generating
fresh names, we can define a {\tt name} operation that does not need
a name as input, but generates a fresh one on demand.
\begin{verbatim}
name (Sig (M f)) = Sig(M h) where
  h n k = f n (\ n2 res -> 
                 let t = (typeOf res)
                     name = ("x"++"_"++show n2)
                 in Decl (PVar name t) res (k (n2+1) (Var name t)))
\end{verbatim}
Now the user simply uses the lifted versions of the primitive operators.
All the naming machinery is still hidden inside the {\tt circuit}
syntactic sugar. 
\begin{verbatim}
toggle3 change = 
  circuit out2 where
     out1 = delay Low out2
     out2 = xor change out1
\end{verbatim}

\subsection{Obtaining Expressions from Signals}
Now suppose we wish to test one of our signal description
generators. We will need two things: (1) a way of generating
abstract inputs, and (2) a way of turning {\tt (Sig t)} into {\tt (Exp t)}.
Both are easy. First enrich the {\tt Exp} datatype with a notion of
an abstract input with a new constructor {\tt Input}, and
then define the function {\tt go} that turns a signal into a data
structure and then prints out the result.
\begin{verbatim}
data Exp ::  *0 ~> *0 where
  . . .
  Input :: String -> Rep t -> Exp t

arg :: String -> (Rep a) -> Sig a
arg s t = lift0 (Input s t)

go :: Sig a -> String
go (Sig (M f)) = showC(RLet (f 0 (\ n x -> In x)))
\end{verbatim}
The function {\tt arg} lifts abstract inputs from the {\tt Exp} level
to the {\tt Sig} level. The function {\tt go} forces the computation
inside a signal by applying it to an initial integer {\tt 0}, and
an initial continuation {\tt In}, and then prints the result.
We can now test one of our circuit descriptions
by evaluating \verb+go (toggle3 (arg "A" Bit))+
which produces the following string as output
\begin{verbatim}
"circuit out2_1 where
   out1_0::Bit = (Delay Low out2_1)
   out2_1::Bit = (Xor A out1_0)" 
\end{verbatim}   
If we want to analyze a circuit description, we simply skip the use
of the printing function {\tt showC}. Note that expressions are
still typed, and the types help us ensure that are analysis
are correct.


\section{Static Analyses}\label{static}
A static analysis computes a static property of a value or function.
The property can be determined by looking at the program without running it.
Type systems are one type of static analysis. An easy way to specify
a static analysis is by using an inference system. The inference system
states how to build the property of a large program from the
properties of its smaller sub-programs. We have already seen a simple
static property of {\tt Exp}. That property is its type index. It states
that {\it every} expression with type {\tt (Exp t)} denotes a circuit
with type {\tt t}. Note that functions over {\tt Exp} propagate
this property. For example: \verb+halfAdd2 :: (Exp Bit) -> (Exp Bit) -> Exp(Bit,Bit)+
states that given two expressions with static property {\tt Bit} it
creates an expression with static property {\tt (Bit,Bit)}. 

We specified the property for the constructors of {\tt Exp} using the {\tt
data} declaration and then we inferred (or checked) the property for every
function. There is no reason we cannot duplicate this process. Give rules
for inferring the property for each constructor, then infer (or check)
the rules for every function. For example, a circuit built without
using the {\tt RLet}, {\tt Const}, or {\tt Delay} constructor
can always be given an interpretation as a combinational circuit. Can we track
this property statically? The answer is yes.

\begin{verbatim}
kind Ctype = Comb | Seq 

property circuitType of 
  Pat :: Ctype ~> *0 where
    PVar :: String -> (Rep t) -> Pat ct
    PPair :: Pat ct -> Pat ct -> Pat ct
  
  Exp ::  Ctype ~> *0 where
    Var    :: String -> Rep t -> Exp ct
    If     :: Exp ct -> Exp ct -> Exp ct -> Exp ct
    Let    :: Decs ct -> Exp ct
    Pair   :: Exp ct -> Exp ct -> Exp ct
    CEmpty :: Exp ct
    CAdd   :: Exp ct -> Exp ct -> Exp ct
    O      :: Exp ct
    I      :: Exp ct
    And    :: Exp ct -> Exp ct -> Exp ct
    Or     :: Exp ct -> Exp ct -> Exp ct
    Not    :: Exp ct -> Exp ct
    Xor    :: Exp ct -> Exp ct -> Exp ct
    Lit    :: ct -> Exp ct
    Equal  :: Exp ct -> Exp ct -> Exp ct
    Plus   :: Exp ct -> Exp ct -> Exp ct
    Minus  :: Exp ct -> Exp ct -> Exp ct
    Times  :: Exp ct -> Exp ct -> Exp ct
    Delay  :: a -> Exp ct -> Exp Seq
    Const  :: a -> Exp  Seq
    RLet   :: Decs t -> Exp Seq
    Input :: String -> Rep t -> Exp ct
  
    
  Decs :: Ctype ~> *0  where
    In :: Exp t -> Decs ct
    Decl :: Pat ct -> Exp ct -> Decs ct -> Decs ct
\end{verbatim}
To introduce a new property, define a new kind ({\tt Ctype}) whose
structure is the carrier of the property. Here we have two
kinds of circuits: combinatorial ({\tt Comb}), and sequential ({\tt Seq}).
Then for each datatype involved ({\tt Pat}, {\tt Exp}, and {\tt Decs}),
give an inference system where the new property is an index of
the datatype. The similarity between {\tt data} declaration and
{\tt property} declaration is intentional. The property defines an alternate
type system for the constructors. In the property {\tt circuitType} most
of the constructors are polymorphic in their {\tt Ctype} index. Only
{\tt RLet}, {\tt Const}, and {\tt Delay} force the introduction
of the {\tt Seq} type index. Thus an expression without one of these
constructors will have a polymorphic type in the alternate {\tt circuitType}
inference system.

When we define a new function we can give multiple prototype declarations.
One for its type, and one for each static property we want to check. For example:

\begin{verbatim}
circuitType bitAdder :: Exp a -> Vec (Exp a) n -> (Vec (Exp a) n, Exp a)

bitAdder :: Exp Bit -> Vec (Exp Bit) n -> (Vec (Exp Bit) n, Exp Bit)

bitAdder carry_in Empty = (Empty,carry_in) 
bitAdder carry_in (Add x xs) = (Add y ys,carry_out)
  where #(y,carry) = halfAdd2 carry_in x
        (ys,carry_out) = bitAdder carry xs
\end{verbatim}
Here we state that {\tt bitAdder} does not necessarily force
its output to be a sequential circuit, but propagates the {\tt Ctype}
property of its inputs. The compiler checks this property by
type checking the body twice. First checking that it has the normal
type, and second by using the alternate {\tt circuitType} types for
the constructors ({\tt Add}), and helper functions ({\tt halfAdd2}). An error
in either check causes a compile-time error.


\section{Multiple Interpretations}\label{multinterp}

Once a circuit description is built, users want to interpret it
in many ways. It should be possible to do this, in fact this is
why we build descriptions in the first place. We've already
seen several interpretations of {\tt Exp}: {\tt eval} and {\tt typeOf}.
Other examples of
interpretations are:
\begin{verbatim}
eval1 :: Exp t -> Env1 -> [t]             -- an infinite sequence of t values
                                          -- one for each clock cycle
                                        
eval2 :: Exp t -> Env2 -> (Time -> t)     -- A function from Time to t

eval3 :: Exp t -> Env3 -> Code(Time -> t) -- The Haskell code of a function 
                                          -- from Time to t

numGates :: Exp t -> Int                  -- Number of primitive gates

depth :: Exp t -> Int                     -- depth of the circuit
\end{verbatim}
In the first three examples we need an environment to handle
free variables in expressions. The structure of the environment
will differ, depending on what the function is doing. For space
reasons we illustrate only {\tt eval2}, and we only show some
of the cases. The missing cases are all similar to the {\tt And}
and {\tt Plus} cases.
\begin{verbatim}
type Time = Int
data Entry2 = exists t . E2 String (Rep t) (Time -> t)
type Env2 = [Entry2]

find2 :: String -> Rep t -> Env2 -> (Time -> t)
find2 s t [] = error ("Not found: "++show s)
find2 s t ((E2 nm z v):xs) = case (eqStr s nm,testRep t z) of
                               (True,Just Eq) -> v
                                _ -> find2 s t xs

eval2 :: Exp t -> Env2 -> (Time -> t)
eval2 (Var s t) env = find2 s t env 
eval2 (And x y) env = \ t -> andB (eval2 x env t) (eval2 y env t)      
eval2 (Plus x y) env = \ t -> (eval2 x env t) + (eval2 y env t)
eval2 (Pair x y) env = \ t -> (eval2 x env t, eval2 y env t)
eval2 (Const n) env = \ t -> n
eval2 (Delay n x) env = \ t -> if t==0 then n else (eval2 x env (t-1))
eval2 (Let (In e)) env = eval2 e env
eval2 (Let (Decl (PVar nm t) e ds)) env = 
  eval2 (Let ds) ((E2 nm t (eval2 e env)):env)
eval2 (RLet ds) env = eval2 (last ds) env2
  where env2 = map ds
        map (Decl p e ds) =  bindPat2 p (lazy(eval2 e env2)) (map ds)
        map (In e) = env      
        last (Decl p e ds) = last ds
        last (In e) = e

bindPat2 :: Pat t -> (Time -> t) -> Env2 -> Env2
bindPat2 (PVar nm t) v env = (E2 nm t v):env
\end{verbatim}

\section{Smart Circuits}
Recently Mary Sheeran has described how to use clever circuits\cite{clever}
to build not-quite regular structures. The key idea here is to tag each
circuit with a value that circuit building combinators can observe
to direct how they should combine the smaller circuits into
larger ones. We can accommodate this by adding an extra slot
in each expression to store the tag.

\begin{verbatim}
data TagExp ::  *0 ~> *0 ~> *0 where
  Var    :: tag -> String -> Rep t -> Exp tag t
  If     :: tag -> Exp tag Bit -> Exp tag b -> Exp tag b -> Exp tag b
  Let    :: tag -> Decs tag t -> Exp tag t
  O      :: tag -> Exp tag Bit
  I      :: tag -> Exp tag Bit
  And    :: tag Exp Bit -> Exp tag Bit -> Exp tag Bit
\end{verbatim}
Of course, each combinator that does something special with the
tags will have to be individually encoded.



\section{Hardware Description and Domain Specific Languages}

A Hardware description language (HDL) is a domain specific language. As
such it can have a number of advantages over using a
general purpose language for hardware design. In this section we try
to open a discussion about the future of DSL's for hardware design. I am
deeply indebted to John O'Leary for many of the ideas discussed
in this section. First let's take stock of the current state of affairs.

\begin{itemize}

\item A HDL should adopt an existing notation,
and allow designers to create large designs with little or
no programming experience. The HDL language syntax is often
different than that of existing notation because high-level HDL's
attempt to do things existing notations were not designed to do.

While syntactical issues are important, a bigger problem is achieving a
smooth {\tt semantic} integration between a high-level HDL and an
existing notation. Ideally one should not have to switch mind sets when
switching between ``existing notation mode" and ``meta mode". This is a
challenge in any meta-programming system. 


\item A HDL program provides a common language that allows different
database users to share precisely the description of the hardware they are
interested in. An HDL communicates ideas. 

\item A HDL program relieves the designer from needing to know about
various aspects of the hardware design process, such as the data
structures that store the low level description, or the layout process
that maps the design onto silicon. These issues must eventually be
addressed, but they can be abstracted away in early steps of the
design. 

\item A HDL program encodes constraints. Constraints could include area,
timing, and power requirements. It's not clear how analysis of these
properties would fit with type checking, or even whether it should. In
general, design constraints get handed down from management or marketing
(e.g. it has to be an x86, consume less than 100 watts of power, run at 3 GHz,
and fit on a die so big). As the implementation is developed the
constraints need to be maintained and pushed down to sub-blocks, but
there's often not enough good information to say for certain whether a
block fulfills its constraints. This is because the most accurate
estimates come bottom-up from the detailed levels of design, and you
don't get there until the final few months of a project.

Also, some things are quite difficult to estimate statically - we might
be interested not in worst-case power but in average power over some
benchmark suite. In this case there's no substitute for dynamic
simulation. 

\item A HDL program can take advantage of domain specific properties.
Especially in the logic synthesis process. Given a Boolean
function and a set of constraints (e.g. area, power, delay), generate a
gate-level implementation meeting the constraints. The constraints (and
physics) are what make it more than just simplification, and design
principles used here would be hard to encode in a general
purpose programming language.  There's a
whole research community built around ``physically aware synthesis",
and an HDL should incorporate this knowledge.

\item A HDL program is a rapid prototype. An HDL description should be
easy to write and lets designer experiment with different solutions. If
the results of a design does not have the properties intended, the
designer is free to throw the original design away and try another
without excessive losses in time or resources. 

\item An HDL design should integrate with a number of existing contexts
or environments, such as timing analysis, and layout systems. 

\item An HDL design should support a natural notion of refinement.
For example, 
may design a sequential circuit over integers, and later refine integer
to a fixed 32-bit, twos-complement bus. The vision is that one
writes high level models of various blocks, validates that they work
together correctly, and then uses these models as a test harness for more
detailed designs (e.g. by substituting a gate-level block for one of the
high level blocks and running simulation again).

But what is missing from this picture is a formal account of refinement -
when can a block be said to refine another? Does refinement preserve
properties that were verified on the abstract model?  Besides data
refinement designers are interested in refinement in several other dimensions.
\begin{enumerate}
 \item{\bf Temporal refinement.} An abstract model of a pipeline might have an
    incompletely-specified delay from input to output (e.g. between four
    and six cycles). A more concrete model may have the delays nailed
    down.
    
 \item{\bf Interface refinement.} An abstract model might communicate via
    CSP-style channels:\\

        c!v   - send value v along channel c\\
        c?x   - receive a value on channel c and bind it to variable x\\
        
    A more concrete model would implement some sort of handshaking
    scheme. 
    There are several interesting things here. At the abstract level, the
    sender views a channel as an output and the receiver views it as an
    input. At the concrete level, channel\_imp has a mixture of inputs and
    outputs - to the sender, request and data are outputs while
    acknowledge is an input (and vice versa for the receiver). A type
    system supporting interface abstraction had better account for this.
    Another interesting thing is that the abstract level has atomic send
    and receive operations (! and ?). The concrete level implements
    these by a protocol involving a sequence of events. What does
    refinement mean here?
\end{enumerate}    



\item A HDL design is an efficient division of labor. Programming
language experts design the HDL  language, algorithm experts design the
algorithms used, and domain
experts design the devices. 
\begin{enumerate}
  \item HDLs facilitate division of labor in a design project because a
    small set of architects can develop the overall product design as an
    interconnection of major blocks with each block modeled abstractly,
    and farm the detailed design of each block to specialists. This also
    occurs within major blocks as they are partitioned into sub-blocks.

  \item HDLs facilitate division of labor in CAD tool development. The
    intermediate data format or "data model" is more important.
    The data model is the object code for the HDL compiler, and should
    preserve all the content of the original HDL description - logical
    functionality, constraints, etc.  With a perfect data
    model experts in the backend tasks of synthesis, optimization, layout
    can all do their thing without having to understand the whole
    problem. 
\end{enumerate}    


\end{itemize}


\subsection{Domain-Specific Language Evolution}

The life cycle of a DSL may be thought of as having three stages, each
calling for different implementation approaches.

\paragraph{Infancy.} 

In this stage, language design ideas are in flux and domain 
abstractions are not well understood.  While some domains have 
existing conceptual frameworks and notations that can be used for 
problem specification, many domains must be analyzed and studied to 
identify the key abstractions.  At this stage, modeling flexibility is 
at a premium, performance is less important, and embedded 
implementations are likely to be cost-effective. {\em I believe
we have passed this point in the life-cycle of HDLs.}

\paragraph{Adolescence.} 

In this stage, domain abstractions stabilize and become better-understood
as the language begins to be exercised by end-users. Implementation 
flexibility is still necessary.  Actual use uncovers the limits of the
implementation, which leads to new features, abstractions, and
implementation techniques. In this stage users begin to develop 
domain-specific logics for their DSLs.  As more users
exercise the system, the need for specialized language syntax, type
discipline, and error reporting become more apparent. These are
difficult or impossible to provide in a conventional host language.
Because most host languages lack the ability to reflect on their own
code -- \textit{i.e.}, to analyze it as well as execute it -- they make
it very difficult to enhance an embedded DSL implementation with
validation or optimization passes. {\em I believe this is exactly where
we currently stand in the HDL life cycle. Consider the work on
monads\cite{ruby}, observable sharing\cite{koen2001}, and labeling\cite{odonnel}, as
attempts to over come this restriction.}



\paragraph{Maturity.}

In this stage, domain abstractions are well-understood, and the DSL
design is fixed.  Features that enhance usability -- such as type systems,
debuggers, and programming environments -- become important as the DSL
gains adherents.  Performance of the DSL implementation itself may
become a problem as DSL program size and complexity increase.

A stand-alone implementation for a DSL can have its own syntax and type system
appropriate for just that domain. The DSL can be ``restricted" to enforce
constraints on what can be expressed, and it can have its own optimizer that
relies on domain-specific knowledge so that performance bottlenecks
can be addressed. If investing in a stand-alone DSL implementation ever makes
sense, it will be at this phase, when the language has become relatively
stable and user tools are important. {\em This is where we are headed, but we need to rethink
a few of our assumptions if we're ever to get here.}


\bibliographystyle{plain}
\bibliography{proposal}

\end{document}
